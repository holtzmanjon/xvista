\chapter{Input and Output of Images and Spectra}
\begin{rawhtml}
<!-- linkto input.html -->
<!-- linkto output.html -->
\end{rawhtml}

%
% keywords and function checked against the current sources
% added missing help for FIXHEAD
% moved SETUP to chapter 14 (spectroscopy), as it didn't really belong here
% and is primarily used for spectral extinction correction (put after
% the help entry for EXTINCT).
% moved SETDIR and CD from chapter 7 (they are needed here before there)
%
% rwp/osu 1998 Aug 6
%

VISTA operates on images or spectra stored in internal memory buffers.
Before you are able to analyze an image or a spectrum, you need to load
your images into one of the buffers from files on disk.  After an image has
been loaded into memory, you can copy it to other buffers, or write the
data to disk.  Below, we will use the word ``image'' to connote both 2-D
images and 1-D spectra, since functionally a 1-D spectrum is treated as a
1-D image with only one row of data.

NOTE: Version 5.0 of VISTA does not provide any commands for magnetic tape
handling, as magnetic tape as a random access medium has largely become
obsolete, having given way to mass disk storage (traditional magnetic
disks, optical disks, and CD-ROM media) with multi-gigabyte capacity.

These two commands are used to read and write images:
\begin{example}
  \item[RD\hfill]{read an image from a disk file}
  \item[WD\hfill]{write an image to a disk file}
\end{example}

These commands set the default data directories used by VISTA:
\begin{example}
   \item[SETDIR\hfill]{set the VISTA default directories \& file extensions}
   \item[CD\hfill]{change the current working directory}
\end{example}

These commands are used to work on image buffers:
\begin{example}
  \item[COPY\hfill]{copy an image between buffers}
  \item[BUFFERS\hfill]{list the contents of the image buffers}
  \item[DISPOSE\hfill]{clear (delete) image buffers}
  \item[CHANGE\hfill]{renames objects}
  \item[CREATE\hfill]{create a blank image buffer}
\end{example}

These commands work on the headers of images stored in memory:
\begin{example}
  \item[FITS\hfill]{insert/edit FITS header cards}
  \item[UNFIT\hfill]{delete FITS header cards}
  \item[HEDIT\hfill]{edit FITS headers}
  \item[FIXHEAD\hfill]{fix FITS headers}
\end{example}

\section{RD: Read an Image from a Disk File}
\begin{rawhtml}
<!-- linkto rd.html -->
\end{rawhtml}

\index{Disk!Read image or spectrum from}
\index{Image!Read from disk}
\index{Spectrum!Read from disk}
\begin{command}
  \item[\textbf{Form: } RD buf filename {[WFPC]} {[DOM]} {[SDAS]}
       {[SPEC]} {[OLD]} {[HEADONLY]}\hfill]{}
  \item[buf]{is the buffer in which the image will be stored, and}
  \item[filename]{(character string) is the name of the diskfile holding
       the spectrum.}
  \item[WFPC,DOM,SDAS]{read WFPC format file with appropriate filename}
  \item[SPEC]{read the image from the spectrum directory, instead of from
       the image directory (archaic).}
  \item[OLD]{read old style VISTA images}
  \item[HEADONLY]{read only the image header, not the image data.}
\end{command}
This command reads images from the disk into a buffer. Images with only one
dimension (i.e. spectra) are converted to two-dimensional images when read
from the disk.

VISTA (starting with v4.2) has adopted the FITS format as the default image
format for input and output.  FITS images are organized (in their simplest
form) into header and data records.  The FITS format has the great
advantage that is it defined independently of your machine type and is a
nearly universally accepted and well-documented standard for data
interchange in the astronomical community.  Consequently, it is possible to
exchange images between different machines and/or operating systems without
having to use special translation software.  For more information on the
FITS format, see the FITS Support Office Home Page at the NASA Goddard
Space Flight Center
(\htmladdnormallink{http://fits.gsfc.nasa.gov/}{http://fits.gsfc.nasa.gov/}).

If the specified image cannot be found, the program will attempt to find
a ``gzip''ed version of the file by appending a .gz to the file name. If
RD finds a file name with a .gz extension, it will copy the file to the
/tmp directory, uncompress it, read it, and then delete the temporary
uncompressed verions.

The optional keywords WFPC, DOM, and SDAS may be used to specify that the
file to be read is a WF/PC team format file (in DEC byte order). If WFPC is
specified, the names are taken to be filename.IMG and filename.HDR; for
SDAS, the filenames are filename.HHD and filename.HHH; for DOM, the file is
specified by a number, W00000n.IMG and W00000n.HDR. If any of these
extensions are explicitly mentioned in the filename, the WFPC switch is
automatically turned on.

If the directory is not explicitly given in the filename, the program tries
to read from the image directory (or from the spectrum directory if the
word SPEC is given, although this usage is now considered archaic, see the
notes below).  If a filename extension is not given in the filename, the
program puts on a default extension from the list of extensions.  You can
show the default directories and extensions with the command PRINT
DIRECTORIES.

In the examples below, assume that the default directories 
/vista/ccd/ for images and /vista/spectra/ for spectra:
\begin{example}
  \item[RD 1 m92\hfill]{reads /vista/ccd/m92.fits  to buffer 1.}
  \item[RD 7 demo/m92\hfill]{reads demo/m92.fits to buffer 7.}
  \item[RD 3 m92.xyz\hfill]{reads /vista/ccd/m92.xyz  to buffer 3.}
  \item[RD \$N m92\hfill]{reads /vista/ccd/m92.fits  to buffer N, where
       N is a variable.}
  \item[RD 12 HD183543 SPEC\hfill]{reads /vista/spectra/HD183143.fits to
buffer 5}
\end{example}

{\bf A Note About Images, Spectra, and Directories:}

We wish to point out that in general there is no requirement within VISTA
to treat images and spectra separately, as both are treated equally as 2-D
or 1-D FITS format images, respectively.  The old segregation was an
artifact of earlier versions of VISTA in which 2-D images and 1-D spectra
had completely separate data formats (images were .ccd files, spectra .spc
files), and separate commands associated with them (e.g., WD for images, WS
for spectra, separate image and spectral arithmetic routines, etc.).  With
Version 3, this illogical separation was eliminated, but the old SPEC
syntax was retained for backward compatibility.  It often confuses new
users, and in general we suggest that new users ignore the SPEC keyword and
V\_SPECDIR environment variables altogether.  Old users with bad habits are
on their own.

The use of special image and/or spectrum directories is of course an
artifact of the old days when computer systems had large scratch disks
(``large'' at that time meant 300Mb units the size of a large-capacity
washing machine) and very small user disk areas, and hence it was necessary
to farm out large images to centralized data disks assigned by the system
administrator.  In this day and age of multi-gigabyte disks that fit in the
palm of your hand, this no longer makes any sense, but it has been retained
to ensure some measure of backwards compatibility (note that IRAF carries
the convention out as well).  In general, we find that many VISTA users
define their default image directory to be \verb+./+, the current working
directory, and ignore the spectrum directory altogether.  The typical
exceptions are when using VISTA for quick-look data analysis at the
telescope, when many users find it convenient to define the default image
directory to be the incoming raw data file system.

\section{WD: Write an Image to a Disk File}
\begin{rawhtml}
<!-- linkto wd.html -->
\end{rawhtml}

\index{Disk!Write image or spectrum to}
\index{Image!Write to disk}
\index{Spectrum!Write to disk}
\begin{command}
  \item[\textbf{Form: } WD source filename {[FULL]} {[ZERO=z
       SCALE=s]} {[SPEC]} {[WFPC]} {[DOM]} {[SDAS]}\hfill]{}

  \item[source]{is the buffer holding the image to be written}

  \item[filename]{(character string) is the name of the file into which the
                 image will be written.}

  \item[FITS=]{specifies the number of bits to output for FITS integer images.}

  \item[FULL]{write the pixels as 32-bit floating point numbers rather than
       16-bit integers.}

  \item[ZERO=z]{Adjusts the zero level of the image before writing.}

  \item[SCALE=s]{Adjusts the range of the image before writing.}

  \item[SPEC]{write the image to the spectrum directory instead of the
       image directory (archaic)}

  \item[WFPC,DOM,SDAS]{Specifies that the file be written in WF/PC team format}
\end{command}

WD writes the image contained in buffer 'buf' to the specified file.
Unless you specify otherwise, the image will be written in FITS format to
the default directory for images with the current default file extension.
By default, WD writes the images in FITS format using 16-bit signed
integers (BITPIX=16).  You can alternatively specify 32-bit signed integers
(BITPIX=32) by using the FITS=32 keyword, or 32-bit floating point values
(BITPIX=-32) using the FULL keyword.  Because VISTA stores all image data
internally as 32-bit floating-point arrays, the FULL option is often the
fastest as it requires no data conversion/scaling computations to convert
the data arrays into the specified integer formats.  Also note that
conversion of data to 16-bit integers might incur precision degradation (or
loss of dynamic range), especially for heavily processed data comprised of
a combination of many individual images.

The optional keywords WFPC, DOM, and SDAS may be used to specify that the
file is to be written in proprietary formats used by the WF/PC-1 team (in
DEC byte order). IF WFPC is specified, the names are taken to be
filename.IMG and filename,HDR; for SDAS, the filenames are filename.HHD and
filename.HHH; for DOM, the file is specified by a number, W00000n.IMG and
W00000n.HDR. If any of these extensions are explicitly mentioned in the
filename, the WFPC switch is automatically turned on.  General users are
advised to ignore these options and write all data in FITS format, as FITS
data are generally portable and should be understood by all other data
analysis programs.

The SPEC word tells WD to write to the spectrum directory and to
append the default file extension for spectra [NB: This is an archaic usage
that is artifact of an earlier version of VISTA where images and spectra
were treated as separate entities with separate commands, and is no longer
required.  You can treat images and spectra identically in the current
version of VISTA.  See the notes at the end of the RD command for a further
discussion].

The command PRINT DIRECTORIES shows the current directories and
extensions. If you do not give a filename, the program will ask you for it.

FULL is used to write the pixels in 32-bit floating point format.  This
preserves the full precision of the pixels in the buffers.  Note that the
WF/PC standard is currently defined to be 16 bit integers, so FULL cannot
be used with WF/PC files.  If the data are to be written as integers
(either 16- or 32-bit formats), WD will first compute the maximum and
minimum pixel values, and then scale the pixels so that the maximum and
minimum values will be packed into the full allowed data range for that bit
precision (+/-32767 for 16-bit integers, and +/-2147483647 for 32-bit
integers).  The scaling parameters are printed on the terminal, and stored
in the FITS header as the BZERO and BSCALE keywords, which will be used by
subsequent FITS file readers to de-scale the data according to the formula:
\begin{hanging}
  \item{true = BZERO + data*BSCALE}
\end{hanging}
The scaling parameters can be overridden by the
ZERO=z and SCALE=s keywords, following the formula:
\begin{hanging}
  \item{data = (true - ZERO ) / SCALE}
\end{hanging}
but playing with data scaling is ill-advised unless you are absolutely
certain of what you are doing.

The command RD will use these parameters to restore the pixel values when
the image is read from the disk.  This is similar to the default option of
the WT command.

NOTE: This scaling convention conforms to FITS standard.  In versions 3
and earlier of VISTA, a backwards convention was used such that if for
some reason you encounter old FITS images written with VISTA before about
1985, you could find data scaling problems.  If you invert the BSCALE 
variable in the header and make BZERO negative, it should come out OK.

In the examples below, assume the default image directory and extension are
/vista/ccd/ and .fits, and the default spectrum directory and extension are
/vista/spectra and .fits.
\begin{example}
  \item[WD 1 m92\hfill]{writes buffer 1 to /vista/ccd/m92.fits.}
  \item[WD 7 demo/m92\hfill]{writes buffer 7 to demo/m92.CCD.}
  \item[WD 3 m92.xyz\hfill]{writes buffer 3 to /vista/ccd/m92.xyz.}
  \item[WD \$N m92\hfill]{writes buffer N to /vista/ccd/m92.CCD, where
        N is a variable.}
  \item[WD 5 mydir/junk SPEC\hfill]{writes buffer 5 to mydir/junk.fits}
  \item[WD 4 ./junk FULL\hfill]{writes buffer 4 to ./junk.fits, using
        full 32-bit precision.  The disk file will be twice as large this way.}
  \item[WD 3 file SPEC\hfill]{writes buffer 3 to /vista/spectra/file.fits}
\end{example}

In earlier versions of VISTA, the 32-bit per pixel format was the default.
The FULL keyword is new to version 3.

See the notes at the end of the RD command for an explanation of the
archaic custom of separating images and spectra.

\section{SETDIR: Set the VISTA Default Directories and File Extensions}
\index{Image!Set default directory}
\index{Spectra!Set default directory}
\index{Directories!Set default directories}
\index{Extensions!Set default for files}
\index{Files!Default directories and extensions}
\index{Procedures!Set default directory}
\begin{rawhtml}
<!-- linkto setdir.html -->
\end{rawhtml}
\begin{command}
  \item[\textbf{Form: } SETDIR code {[DIR=directory\_name]} 
       {[EXT=extension]}\hfill]{}
  \item[code]{specifies which directory is being set or changed}
  \item[DIR= ]{   specifies a directory for the type of object
       indicated by the code.}
  \item[EXT=]{gives the extension for files in the default directory}
\end{command}

SETDIR sets the default directories and extensions of files storing images,
spectra, color maps, etc.  You can see the default values with the command
PRINT DIRECTORIES.  See the section FILES (type HELP FILES if you're on a
terminal) for information about default directories and extensions.  See
also the command CD to change the current working directory (./) on a UNIX
system.

The DIR word gives the default directory for the type of object specified
by the code, and the EXT word gives the extension for that type of object.
An example of a default extension is that for '.fits' for images.  An
example of a default directory is 'ccd/spec' for spectra.  You must specify
either the directory or the extension or both with SETDIR. If the extension
is not blank, it must include a period as its first character: For example,
'.xyz' is a valid extension, while 'flk' is not.

The 'code' gives the directory which is being set or changed. The code is
derived from the type of object in the directory you are specifying.  You
must type at least the first two letters of the code:

\begin{center}
\begin{tabular}{llll}
Directory&Code&Abbrev&Notes\\
\hline
Images&IMAGES&IM&(user defined, usually ./)\\
Spectra&SPECTRA&SP&(archaic, use IMAGES)\\
Procedures&PROCEDURES&PR&(user defined)\\
Data Files&DATA&DA&(user defined, usually ./)\\
DAOPHOT files&PHOTOMETRY&PH&(user defined)\\
Flux calibration files&FLUX&FL&(assigned by system)\\
Wavelength files&WAVE&WA&(assigned by system)\\
Color maps&COLOR&CO&(assigned by system)\\
\hline
\end{tabular}
\end{center}

Examples: Suppose you see with PRINT DIR that the default directory for CCD
spectra is ccd/spec and the default extension is '.fits'
\begin{example}
  \item[SETDIR SP DIR=mydir/spec\hfill]{changes the default directory
       to mydir/spec}
  \item[SETDIR SP EXT=.xyz\hfill]{changes the default extension to '.xyz'}
  \item[SETDIR SP EXT=.XYZ DIR=mydir/spec\hfill]{changes both the
       directory and extension at one time.}
\end{example}

\section{CD: Change the Current Working Directory}
\begin{rawhtml}
<!-- linkto cd.html -->
\end{rawhtml}
\begin{command} 
  \item[\textbf{Form: } CD path\_name\hfill]{}
  \item[path\_name]{any valid Unix directory path}
\end{command}
 
CD will change the current working directory (./) of VISTA to be
path\_name.  This working directory will remain current until either you
issue CD again, or you exit VISTA.  On exiting VISTA, you will be in the
*original* working directory from which you executed VISTA.
 
CD is most useful when you have defined the default image directory to the
current working directory, (./), allowing you to navigate among different
directories containing data of interest without having to redefine the
default image directory each time with the SETDIR command.  It also defines
the current working directory for shell commands (\$).
 
\section{COPY: Copy an Image between Buffers}
\begin{rawhtml}
<!-- linkto copy.html -->
\end{rawhtml}

\index{Image!Copy}
\index{Spectrum!Copy}
\begin{command}
  \item[\textbf{Form: } COPY dest source {[BOX=n]}\hfill]{}
  \item[dest]{is the buffer where the new image will be stored.}
  \item[source]{is the original copy.}
  \item[BOX=n]{tells the program to copy only the part of the
       image that is in box 'n'.}
\end{command}

COPY copies the image in buffer 'source' to the buffer 'dest'. The header
associated with the source image is also copied.  NOTE THAT THE DESTINATION
BUFFER COMES FIRST ON THE COMMAND LINE!  The syntax of the copy command is:
\begin{hanging}
  \item{COPY into (dest) from image (source).}
\end{hanging}

\noindent{Examples:}
\begin{example}
  \item[COPY 2 1\hfill]{copies the image in buffer 1 to
       buffer 2, along with buffer 1's header
       and label.}
  \item[COPY \$B \$A\hfill]{copies the image image in buffer A to 
       buffer B. A are B are variables.}
  \item[COPY 1 2 BOX=7\hfill]{copies the segment of image 2 that is
       in box 7 into image buffer 1.  The
       size of the new image will be the size
       of the box.  See BOX for more details.}
\end{example}

The related command, WINDOW, is used to trim (``crop'') an image or
spectrum.

If you copy wavelength-calibrated spectra, you should copy the
\textit{entire} spectrum.  You should not use BOX= with wavelength-calibrated 
spectra.  If you do, the wavelength scale of the spectrum after the COPY
could end up incorrect.

\section{BUFFERS: List the Contents of the Image Buffers}
\begin{rawhtml}
<!-- linkto buffer.html -->
<!-- linkto buffers.html -->
\end{rawhtml}

\index{Image!Header}
\index{Spectrum!Header}
\index{Buffers!List of images in}
\begin{command}
  \item[\textbf{Form: } BUFFERS {[bufs]} {[FULL]} {[FITS{[=param]}]}
       (redirection)\hfill]{}
  \item[bufs]{ are integers which specify which buffers are to be displayed
       by the program}
  \item[FULL]{produce a long listing of the buffer information}
  \item[FITS]{list the FITS parameters for the image}
  \item[FITS=param]{list the individual fits parameter 'param'}
\end{command}
This command shows a brief summary of the header information for the images
contained in the buffers.  If specific buffers are listed, the program
lists information for only those buffers.  Otherwise, all buffers with
images in them are listed. If the keyword FULL is specified, a longer
listing is given for each buffer.  If the keyword FITS=param is given then
the literal text of the FITS header is printed for the FITS header
parameter 'param'.  If you use just the keyword FITS (no specified
parameter) AND you use the FULL keyword then all of the FITS header
parameters will be shown.

\noindent{Examples:}
\begin{example}
  \item[BUFFERS\hfill]{Give a brief list of all buffers containing images.}
  \item[BUFFERS FULL\hfill]{Give a long listing of the image buffers.}
  \item[BUFFERS 3 4 8 FULL\hfill]{Give a long listing of buffers 3, 4, and 8.}
  \item[BUFFERS 3 FULL FITS\hfill]{List all the individual FITS header 
       parameters and values for buffer 3.}
  \item[BUFFERS 3 FITS=AIRMASS\hfill]{Prints the FITS card beginning with
       'AIRMASS' (if it exists).}
\end{example}

\section{DISPOSE: Clear (Delete) an Image Buffer}
\begin{rawhtml}
<!-- linkto dispose.html -->
\end{rawhtml}

\index{Image!Delete from buffer}
\index{Spectrum!Delete from buffer}
\index{Virtual memory!Define}
\index{Virtual memory!What to do when not enough}
\begin{command}
  \item[\textbf{Form: } DISPOSE {[ALL]} {[buf]} {[buf2]} {[buf3]}
       {[...]} \hfill]{}
  \item[ALL]{delete the contents of all image buffers.}
  \item[buf]{is the image buffer to be deleted.}
  \item[buf2]{is another image buffer to be deleted ...}
\end{command}
This command will delete image buffer number 'buf' and discard its
contents.  The memory needed to hold an image or spectrum is not
pre-assigned when the VISTA program is compiled, instead it is allocated
dynamically by the system when a new image is read in or created.  By
allocating ``virtual memory'', VISTA can work efficiently by only using as
much system memory at one time as required.  The DISPOSE command is used to
free this memory when the image is no longer needed or when VISTA has run
out of memory for assigning images to other buffers.  If you want to save
the image, it should be written to disk or tape first with the WD command
before using DISPOSE.

You can delete several image buffers at once by putting their numbers on
the command line, up to a maximum of 20.  The DISPOSE ALL command will
clear out the contents of all image buffers, releasing all of the
dynamically allocated memory.

\noindent{ Examples:}
\begin{example}
  \item[DISPOSE 7\hfill]{deletes image 7}
  \item[DISPOSE 1 3 5 7 9\hfill]{deletes images 1, 3, 5, 7, and 9.}
  \item[DISPOSE \$Q\hfill]{deletes image Q, where Q is a variable.
        (This form is helpful in procedures.)}
  \item[DISPOSE ALL\hfill]{deletes all images and spectra.}
\end{example}

VISTA may sometimes respond with the message 'No virtual memory available'
when trying to load or create an image.  To make room for the new image,
use DISPOSE to delete an unused image buffer or two.  This frees up more
memory, allowing you to proceed.

If you DISPOSE of images that you will no longer need, the VISTA program
may run faster, especially if there are many users on the system or if you
are working with large (>2048x2048 images).

\section{CREATE: Create a Blank Image}
\begin{rawhtml}
<!-- linkto create.html -->
\end{rawhtml}

\begin{command}
  \item[\textbf{Form: } CREATE buf {[BOX=b]} {[SR=sr]} {[SC=sc]} {[NR=nr]} 
       {[NC=nc]} {[CONST=c]} {[N=n]} {[V=v]} {[HEADBUF=oldbuf]} \hfill]{}
  \item[buf]{is the buffer holding the new image}
  \item[BOX=b]{create an image with the size and orientation of box 'b'}
  \item[SR=sr]{specify the start row of the new image}
  \item[SC=sc]{specify the start column}
  \item[CR=sr]{specify the center row of the new image}
  \item[CC=sc]{specify the center column}
  \item[NR=nr]{specify the number of rows}
  \item[NC=nc]{specify the number of columns}
  \item[N=n]{specifies number of rows and columns (square image)}
  \item[V=n]{loads center row and column for new image using
       coordinates from VISTA variables Rn and Cn}
  \item[CONST=c]{fill the image pixels with value 'c'}
  \item[HEADBUF=oldbuf]{fill the new header with all cards from buffer 'oldbuf'}
\end{command}

CREATE creates a new blank image, giving all the pixels in that image some
constant value (the default is 0.0).  Any image that is in the buffer will
be destroyed.  Use CONST=c to set the initial values of all of the pixels
to 'c'.

The BOX=b keyword is used to define the image size and location.  This
makes the new image have the size and location of bux 'b'.  See BOX for
instructions for defining boxes. You may also give the size and location of
the new image with the keywords SR, SC, NR, NC, or N.  NR and NC, or N, are
required.  If SR, SC , CR, and CC are not given, the default origin is
(0,0).

If you wish to have the header cards in the new buffer populated with cards
from an existing buffer (except for cards having to do with image size,
origin, etc.), use the HEADBUF=oldbuf keyword to copy cards from buffer
\textit{oldbuf}.

\noindent{Examples:}
\begin{example}
  \item[CREATE 1 BOX=5 \hfill]{creates an image in buffer 1 having 
       the size and orientation of box 5.  the image is filled with zeroes.}
  \item[CREATE 1 BOX=5 CONST=100.0\hfill]{does the same as the first example, 
       but fills the image with value 100.0}
  \item[CREATE 5 SR=5 SC=10 NR=25 NC=35\hfill]{creates an image in buffer 5.  
       The start (row, column) is (5,10) and the size of the image is 25 rows by
       35 columns.}
  \item[CREATE 1 N=100\hfill]{creates an 100 by 100 image in buffer 1.
       The start row and column are both 0.}
\end{example}

\section{CHANGE: Change the Object Name of an Image}
\begin{rawhtml}
<!-- linkto change.html -->
\end{rawhtml}

\index{Image!Change object name}
\index{Spectrum!Change object name}
\index{Name!Change name of object image}
\begin{command} 
  \item[\textbf{Form: } CHANGE buf 'new\_name'\hfill]{}
  \item[buf]{is the number of the buffer holding the 
       image which is having its name changed.}
  \item[new\_name]{is the new label.}
\end{command}
This command allows you the change the object name of the image or spectrum
held in buffer number 'buf'.  The object name of an image is the value of
its OBJECT keyword in the FITS header.  The new image name can be given on
the command line or it can be given in response to the prompt if CHANGE
is issued with no arguments.

If the new name is given on the command line, it must be enclosed in quotes
if it is more than one word long.  

If the new name is not given on the command line, VISTA will type the
current object name and prompt you for the new name.  If you respond with
only a carriage return then the old name is left unchanged.  To force a
blank object name you must enter at least one space character before the
carriage return.  

\noindent{Examples: }
\begin{example}
  \item[CHANGE 1 HD183143\hfill]{changes the name of image 1 to 'HD183143'. }
  \item[CHANGE 7 'Test Image 2'\hfill]{changes the name of image 7 to 
       'Test Image 2'.}
  \item[CHANGE \$R 'new Label'\hfill]{changes the name of image R (where R
       is a variable) to 'new Label'.}
  \item[CHANGE 1\hfill]{changes the name of image 1.  The old 
       name is printed, and the program asks you for the new name.}
\end{example}

\section{FITS: Insert/Edit a FITS Header Card}
\begin{rawhtml}
<!-- linkto fits.html -->
\end{rawhtml}

\index{Image!Insert a FITS header card}
\index{Image!Edit a FITS header card}
\index{Spectrum!Insert a FITS header card}
\index{Spectrum!Edit a FITS header card}
\begin{command}
  \item[\textbf{Form: } FITS buf {[FLOAT=name]} float\_value\hfill]{}
  \item[FITS buf {[INT=name]} integer\_value\hfill]{}
  \item[FITS buf {[CHAR=name]} 'character string'\hfill]{}
  \item[FITS PROF {[FLOAT=name float]} {[INT=name in]} {[CHAR=name char]}\hfill]{}
\end{command}
Use one of these forms to insert a new FITS header card into the image
header, or to change the value of an existing FITS header card.  If you
specify the optional PROF keyword rather than a buffer number, the new
header card will be inserted into the internal PROFILE header (see PROFILE
and SAVE PROF).  The FITS keyword name may be no longer than 8 characters.

Except for two special cases, a FITS card already in the header with the
same name will be replaced.  The two exceptions are CHAR=COMMENT and
CHAR=HISTORY, which will always add a new COMMENT or HISTORY card,
respectively, to the header.  

The float\_value for the FLOAT keyword should be a floating point constant
or an arithmetic expression.  The integer\_value for the INT keyword should
be an integer constant or an arithmetic expression whose result will be
rounded to the nearest integer.

\noindent{Examples:}
\begin{example}
  \item[FITS 1 FLOAT=AIRMASS 1.3\hfill]{Inserts a FITS header card named
        AIRMASS with value 1.3 into the header of the image in buffer 1.}
  \item[FITS 3 CHAR=COMMENT 'My best observation ever'\hfill]{
        Inserts a new COMMENT card into he header of the image in buffer 3.}
\end{example}

\section{UNFIT: Delete a FITS Header Card}
\begin{rawhtml}
<!-- linkto unfit.html -->
\end{rawhtml}

\index{Image!Delete/Remove a FITS header card}
\index{Spectrum!Delete/Remove a FITS header card}
\begin{command}
  \item[\textbf{Form: } UNFIT buf {[CARD=name]} {[PROF]}\hfill]{}
\end{command}

Use UNFIT to delete the specified FITS header card from the header of image
in the specified VISTA buffer.  This image will remove only the first
instance of the card, so if you want to remove multiple COMMENT cards, for
example, the command needs to be issued many times.

You can delete cards from the internal PROFILE header by using the PROF
keyword, rather than specifying a buffer number.

\section{HEDIT: Edit FITS Headers}
\begin{rawhtml}
<!-- linkto hedit.html -->
\end{rawhtml}

\index{Image!Edit FITS header}
\index{Spectrum!Edit FITS header}
\begin{command} 
  \item[\textbf{Form: } HEDIT buf\hfill]{}
  \item[buf]{is the number of the buffer holding the 
image that is having its FITS header edited.}
\end{command}

VISTA writes the FITS header associated with buffer 'buf' into a temporary
disk file and then invokes the default editor to edit the file.  When you
exit the editor the modified version of the FITS header replaces the old
version.

The default editor is vi, although you can specify any other editor by
setting the environment variable VISUAL before running VISTA to be the name
of the editor you wish to use (e.g. to use emacs, setenv VISUAL
/path/to/emacs, where ``/path/to/'' is the full path to the emacs
executable on your system).

%
% HEADCON commented out as a potentially defunct command
%
%\section{HEADCON: Decode FITS Headers from Various Observatories}
%\begin{rawhtml}
%<!-- linkto headcon.html -->
%\end{rawhtml}

%\begin{command}
%  \item[\textbf{Form: } HEADCON imno {[MCD]} {[LAPALMA]} {[CTIO]} {[KPNO]} 
%       {[PALOMAR]}\hfill]{}
%  \item[imno]{is the image buffer with the image whose FITS header is
%       to be decoded.}
%  \item[only ONE of the following:\hfill]{}
%  \item[MCD]{decode McDonald Observatory FITS Headers}
%  \item[LAPALMA]{decode La Palma Observatory FITS Headers - NOT INSTALLED}
%  \item[CTIO]{decode CTIO FITS Headers- NOT INSTALLED}
%  \item[KPNO]{decode Kitt Peak FITS Headers- NOT INSTALLED}
%  \item[PALOMAR]{decode Palomar Observatory FITS Headers. - NOT INSTALLED}
%  \item[ESO]{decode European Southern Observatory FITS Headers - NOT INSTALLED}
%\end{command}
%
%Only a handful of FITS header cards are strictly defined in the FITS
%standard, and the remainder are used at the discretion of the observatory.
%While the FITS standard makes recommendations for names of other header
%cards, in practice different sites have different conventions of encoding
%information like exposure times, UT date/time, etc. Since there is no
%accepted standard yet (perhaps someday by near-universal acceptance of IRAF
%or some similar mega-package), these different header cards must be decoded
%into a form VISTA can understand on a case-by-case basis.
%
%VISTA follows from the FITS header convention used at Lick Observatory,
%which pretty much follows the recommendations of the standard (sort of) and
%so Lick FITS images are default.  The data which need to be decoded are:
%\begin{hanging}      
%  \item{EXPOSURE TIME}
%  \item{UT DATE and TIME at start of exposure}
%  \item{RA and DEC of TELESCOPE (image field)}
%  \item{HA at start of exposure}
%\end{hanging}
%All of these are essential for flux calibration and extinction correction.
%
%Other cards may be converted, mostly for consistency with the VISTA format:
%\begin{hanging}
%  \item{TAPENUM -  the number of the image on the data tape}
%  \item{OBSNUM  -  the observation number (assigned differently or not at all)}
%  \item{STATUS  -  a specific comment giving the image status }
%\end{hanging}
%New formats are always arising, and so users should be sensitive to
%peculiarities that might be explained if one is using data from an
%observatory that is "foreign" to your installation.  Addition of new
%conversion modules is relatively straightforward, but will require
%programming intervention by your local custodian.

\section{FIXHEAD: Fix FITS Headers}
\begin{rawhtml}
<!-- linkto fixhead.html -->
\end{rawhtml}

\index{Image!Fix buggy FITS headers}
\index{Image!Fix FITS header coordinate cards}
\begin{command} 
  \item[\textbf{Form: } FIXHEAD imbuf {[ORIGIN]} {[RORIGIN]} 
       {[CORIGIN]}\hfill]{}
  \item{{[WFPC]} {[GROUP]} {[BLANK]} }
  \item[imbuf]{image buffer with the FITS header to be fixed}
  \item[ORIGIN]{reset the FITS coordinate system cards.}
  \item[RORIGIN]{reset the FITS coordinates along rows only.}
  \item[CORIGIN]{reset the FITS coordinates along columns only.}
  \item[WFPC]{remove WF/PC1 image header cards}
  \item[GROUP]{remove all FITS group cards}
  \item[BLANK]{remove any blank cards from the header}
\end{command}

FIXHEAD is used to reset or fix FITS headers with various pathologies.

ORIGIN and the related RORIGIN and CORIGIN cards are used to reset the FITS
coordinate system cards (CRVALn, CTYPEn, CRPIXn, CDELTn, and CNPIXn).  This
is useful if the coordinate cards have been altered by programs (like IRAF
or STSDAS) in ways that cause problems with various VISTA image
manipulation programs, especially SHIFT.  FIXHEAD ORIGIN can also be used
to strip out wavelength calibration information (but you cannot uncalibrate
a spectrum if it has been, for example, resampled onto a linear wavelength
scale).  ORIGIN resets the coordinate info along both axes, CORIGIN and
RORIGIN only reset coordinates along the Columns and Rows axis,
respectively.

BLANK is used to remove any empty (blank) header cards.  While blank cards
are allowed under the standard FITS definition for headers, they can be a
nuisance as they eat up available header memory.

GROUP removes various header cards related to the ``group'' structure
of complex FITS files.  Since the current version of VISTA does not
recognize the group structure, these cards are eating up header memory.

WFPC is used to remove header cards in WF/PC1 imaging data.  Again, the
purpose is to try to free up header memory, especially in old WF/PC1 image
which have huge image headers.


\chapter{Fourier Transforms and Complex Arithmetic}
\begin{rawhtml}
<!-- linkto fourier.html -->
\end{rawhtml}

%
% minor editing & reformatting 
% checked keywords & function against current sources. 
% removed entries for unimplemented commands (that JJGG et al. probably
%  never will get around to - nice ideas, though, but they don't need
%  entries in the help files)
% rwp/osu 98Jul27
%

VISTA provides a limited capability for doing Fourier operations on images.
VISTA internally can have "Fourier images" which have 2 pixels associated
with each grid location, one for the real and one for the imaginary
part. For the VISTA Fourier routines, the data in the spatial domain IS
ASSUMED TO BE REAL. This assumption allows data in the frequency domain to
be stored in almost the same size image as the spatial domain data. Because
the Fourier transform of a real function is hermitian (symmetric real part
and antisymmetric imaginary part), only the Fourier amplitudes of
non-negative frequencies are stored to minimize redundancy and space.  The
Fourier-transformed image has the same size as the input image (except
padded up to a power of 2), plus 2 extra columns, one each for the real and
imaginary parts of the Nyquist frequency.

The VISTA Fourier Transform and Complex Arithmetic commands are:
\begin{example}
  \item[FFT]{Compute a Forward Fast-Fourier Transform}
  \item[IFFT]{Compute an Inverse FFT}
  \item[POWERS]{Generate a Power Spectrum (Periodogram Estimate)}
  \item[CABS]{Absolute Value of a Complex Image}
  \item[CMUL/CDIV]{Complex Image Multiplication}
  \item[CMPLX/IMAG/REAL/CONJ]{Complex Image Operators}
\end{example}

The command FFT can be used to generate the direct fourier transform of a
real image, the result is a complex image. IFFT would do the inverse
fourier transform from a complex image into a real image. Both
transformations can be perform in one or two dimensions. Command POWERS can
be used after FFT to generate the periodogram estimation of the power
spectrum.  

\section{FFT: Compute a Forward Fast-Fourier Transform}
\begin{rawhtml}
<!-- linkto fft.html -->
\end{rawhtml}
\begin{command}
  \item[Form: FFT  {[dest]} source {[ONEDIM]}\hfill]{}
  \item[source]{buffer holding the real image to be transformed.}
  \item[dest]{buffer holding the (complex) Fourier transform. If dest
       is not specified, 'source' will hold its own transform.}
  \item[ONEDIM]{perform a 1D direct transform for each row of 'source'
       rather than the 2D transform of the whole image.}
\end{command}

FFT generates the 1D or 2D direct fourier transform of a real image. If the
keyword ONEDIM is specified, an independent 1D direct fourier transform
will be generated for each row of the source image. For one-row images
(spectra) ONEDIM is the default.

The FFT basic routine works on data arrays whose size is an integer power
of two. If the number of columns of the original image is not an integer
power of two, the data is padded with zeros to the next power of two before
transforming. In the case of 2D transforms, also the rows will be
zero-padded to the next integer power of two.

No filtering of any kind (like the popular cosine-bell) is applied before
transforming (the user can create its own filters). It is up to the user to
manipulate the data before the FFT.

The output image is a complex image, that is to say that the real and
imaginary parts of each fourier frequency are stored in consecutive
columns. The number of columns in the transform (output image) is one more
than the power of two closest, but larger or equal, to the number of
columns in the original image. In the case of a 2D transform, the number of
rows is the a power of two closer, but larger equal, to the number of rows
in the original image. This image size accommodates all of the independent
fourier frequencies to reconstruct the original image.

BE AWARE that most VISTA commands treat complex images as real images
(except for commands FFT IFFT POWERS CMPLX CABS CMUL CDIV CONJ), some
operations on the transform-image can make it loose its meaning as "a
fourier transform of a real image" (things like rotations, column flips,
and boxing are particular dangerous). VISTA provides basic complex
arithmetic (CMUL, CDIV, CABS, etc) that can be safely used in fourier
images. If you want to make extensive manipulations of fourier transforms,
it may be safe to separate first the real and imaginary parts (with VISTA
commands REAL and IMAG), and perform identical operation on both parts
before recombining them again into a complex image (See command CMPLX).

Convolution and deconvolution in fourier space can be perform with the help
of complex arithmetic commands CMUL and CDIV. Estimations of the power
spectrum from fourier images can be made with command POWERS or by
squaring the modulus of the transform (command CABS).

\noindent{Examples:}
\begin{example}
  \item[FFT 2 3\hfill]{Perform the 2D direct-fourier-transform of 
       image in buffer 3 and save the transform image in buffer 2}
  \item[FFT 5\hfill]{Replace the image in buffer 5 by its fourier transform. 
       If image 5 has more than one row a 2D transform, 1d otherwise).}
  \item[FFT 1 ONEDIM\hfill]{Replace each row of buffer 1 by its 1D fourier 
       transform.}
\end{example}

\section{IFFT: Compute an Inverse FFT}
\begin{rawhtml}
<!-- linkto ifft.html -->
\end{rawhtml}
\begin{command}
  \item[Form: IFFT  {[dest]} source {[ONEDIM]}\hfill]{}
  \item[source]{buffer holding the fourier transform of a real image.}
  \item[dest]{will hold the inverse fourier transform of 'source'. If
       'dest' is not specified, 'source' will be substituted by its
       transform.}
  \item[ONEDIM]{perform a 1D inverse transform for each row of 'source'
       rather than the 2D inverse transform of the whole image.}
\end{command}

IFFT computes an inverse Fourier transform.

\section{POWERS: Generate a Power Spectrum (Periodogram Estimate)}
\begin{rawhtml}
<!-- linkto powers.html -->
\end{rawhtml}
\begin{command}
  \item[Form: POWERS dest source\hfill]{}
  \item[source]{buffer with the Fourier transform of a real image.}
  \item[dest]{image buffer to hold the power spectrum}
\end{command}

POWERS generates the periodogram power spectrum estimate from the 1D or 2D
fourier transform of a real image. The periodogram estimate is just a way
to normalize the power spectrum such that Parseval's theorem holds
(i.e. the sum of all the power terms equals the mean squared amplitude of
the transform).

POWERS only operates on fourier transforms generated with VISTA command
FFT. If you want to generate the power spectrum from a more general
complex image (p.e. an image synthesized with command CMPLX) which is
supposed to be the fourier transform of a real image, use command CABS to
get the modulus of the transform and multiply it by itself (with MULT not
CMULT since the modulus of a complex image is real). The periodogram
estimate and the modulus-squared are basically identical (they may only
differ in the 0th and Nyquist frequencies).

\section{CABS: Absolute Value of a Complex Image}
\begin{rawhtml}
<!-- linkto cabs.html -->
\end{rawhtml}
\begin{command}
  \item[Form: CABS dest source\hfill]{}
\end{command}

CABS Computes the absolute value of a complex image.

\section{Complex Image Arithmetic and Operators}
\begin{rawhtml}
<!-- linkto complex.html -->
\end{rawhtml}
\begin{command}
  \item[CMUL im1 im2\hfill]{Complex multiplication of images.}
  \item[CDIV im1 im2\hfill]{Complex division of images.}
  \item[CABS dest source   \hfill]{Complex Modulus of an image.}
  \item[IMAG dest source   \hfill]{Extract imaginary part from an image.}
  \item[REAL dest source   \hfill]{Extract real part from an image.}
  \item[CMPLX dest im2 im3 \hfill]{Synthesize a complex image}.
  \item[CONJ dest source   \hfill]{Complex Conjugate of an image.}
\end{command}

\section{CMUL/CDIV: Complex Image Multiplication}
\begin{rawhtml}
<!-- linkto cmul.html -->
<!-- linkto cdiv.html -->
\end{rawhtml}
\begin{command}
  \item[Form: CMUL dest source\hfill]{}
  \item[CDIV dest source\hfill]{}
\end{command}

CMUL and CDIV perform complex multiplication and division on complex
(Fourier) images.

\section{CMPLX/IMAG/REAL/CONJ: Complex Image Operators}
\begin{rawhtml}
<!-- linkto cmplx.html -->
<!-- linkto imag.html -->
<!-- linkto real.html -->
<!-- linkto conj.html -->
\end{rawhtml}
\begin{command}
  \item[Form: CMPLX dest real imag\hfill]{}
  \item[IMAG dest source\hfill]{}
  \item[REAL dest source\hfill]{}
  \item[CONJ dest source\hfill]{}
\end{command}

CMPLX synthesizes a complex image from a real and a imaginary buffers.  

IMAG extracts the imaginary part of a complex image.  

REAL extracts the real part of a complex image.  

CONJ take the complex conjugate of the complex image.

The 'source' buffer does not have to be a complex image for commands IMAG,
REAL and CONJ to operate. If the 'source' image is not complex, REAL will
extract the odd columns, IMAG will extract the even rows, and CONJ will
simply multiply the even columns by -1.









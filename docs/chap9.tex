\chapter{Marking Image Subsections or Pixels}

\section{Selecting Image Subsections for Computation}
\begin{rawhtml}
<!-- linkto marking.html -->
\end{rawhtml}

%
% all sections checked for keywords and function
% rwp/osu 98Aug2
%

VISTA provides various programs for designating that certain subsections of
an image are to be ignored or included in calculations.
\begin{example}
  \item[BOX\hfill]{define a rectangular region of an image --
       a set of contiguous rows and columns.}
  \item[MASK\hfill]{ignore specified pixels or regions}
  \item[UNMASK\hfill]{stop ignoring masked pixels}
  \item[MASKTOIM\hfill]{create a map of masked pixels.}
\end{example}

\section{BOX: Define a Box or Image Subsection}
\index{Box!Defining}
\begin{rawhtml}
<!-- linkto box.html -->
\end{rawhtml}
\begin{command}
  \item[\textbf{Form:}  BOX box\_num {[NC=n]} {[NR=n]} {[N=n]} {[CR=n]} 
       {[CC=n]} {[SR=n]} {[SC=n]} {[INT]}\hfill]{}
  \item{   {[CENT]} {[V=n]}}
  \item[box\_num]{(integer) is the number of the box being defined,}
  \item[NC]{defines the number of columns in the box,}
  \item[NR]{defines the number of rows in the box,}
  \item[N  ]{ defines the number of rows and columns (square box),}
  \item[CR]{defines the center row,}
  \item[CC]{defines the center column,}
  \item[SR]{defines the starting row, and}
  \item[SC]{defines the starting column.}
  \item[INT]{lets you define the box interactively on the TV}
  \item[CENT]{toggles back and forth between origin and center based system}
  \item[V=n  ]{defines the center row and column using the
       VISTA variables Rn and Cn}
\end{command}

VISTA can store the specifications for up to 20 boxes defining image
sub-sections.  These parameters can be used by other commands (such as TV,
WINDOW and PRINT) by including 'BOX=' in the command line, modifying these
commands so they operate only on the designated subsection.

When a box is initially defined, the size of the box in both dimensions
center must be specified; use both the NR and NC keywords, or for the case
of a square box, the N= keyword may also be used.  If the origin or center
is not given, the box is assumed to start at row 0 and column 0.  You may
change the origin and size of the box later by entering only the changes on
the command line.  All parameters not given on the command line are left
unchanged.

By default, the VISTA boxes have fixed origins, so if you only change the
box size, then the origin with be preserved. By using the CENT keyword,
however, you can change this to a center based system, where the center of
the box will be preserved. The CENT keyword acts as a toggle switch between
the two modes - once it is set in one mode, it remains in that mode until
explicitly changed by the user.

The INT command allows you to define boxes interactive on the video
display. Follow the instructions given.

\noindent{Examples:}
\begin{example}
  \item[BOX 1 NR=45 NC=56\hfill]{defines box 1.  Box 1 has origin at row 0 and 
       column 0 and has 45 rows and 56 columns.}
  \item[BOX 1 INT\hfill]{defines box 1. The locations of the upper left and 
       lower right corners are specified using the cursor on the TV.}
  \item[BOX 1 SC=100 NC=100 SR=0 NR=100\hfill]{defines box 1 as columns
       100 to 199 and rows 0 to 99.}
  \item[BOX 2 CC=100 CR=100 NR=13 NC=13\hfill]{    defines a box having 13 rows
       and columns, centered on row=100 and column=100.}
\end{example}

\noindent{Examples of changing the location of boxes.
Suppose we have defined box 1 as in example 1 above.}
\begin{example}
  \item[BOX 1 SR=50\hfill]{moves the box so that the starting row is row 50. 
       The starting column, and the number of rows and columns is unchanged.}
  \item[BOX 1 CC=50 CR=66\hfill]{moves the box so that the center of the box 
       is at row 50 and column 66.  The size of the box is unchanged.}
\end{example}

The locations of the defined boxes can be found with the PRINT BOXES
command.

All spectra have start row 0 and 1 row.  Thus, to define a box for use with
spectra, use
\begin{hanging}
  \item{BOX n SR=0 NR=1 ...}
\end{hanging}

You can display the boundaries of defined boxes on the television. See the
command TVBOX.

\section{MASK/UNMASK: Ignore (Mask) Specified Pixels}
\begin{rawhtml}
<!-- linkto mask.html -->
<!-- linkto unmask.html -->
\end{rawhtml}
\index{Pixel!Ignoring}
\index{Pixel!Mask}
\begin{command}
  \item[\textbf{Form:} MASK {[R=r1,r2]} {[C=c1,c2]} {[BOX=N]} 
       {[PIX=r,c]}\hfill]{}
  \item[UNMASK  {[R=r1,r2]} {[C=c1,c2]} {[BOX=N]} {[PIX=r,c]}\hfill]{}
  \item[R=]{Ignore all pixels in rows.}
  \item[C=]{Ignore all pixels in columns.}
  \item[BOX=b]{Ignore all pixels in box b.}
  \item[PIX=r,c]{Ignore the pixel at row 'r' and col 'c'.}
\end{command}

MASK tells VISTA programs to ignore the specified pixels when doing
computations.

The command UNMASK is the opposite of MASK; it tells the programs to stop
ignoring the specified pixels.  UNMASK can be used without options to
unmask all pixels.

You can type UNMASK without any options to tell the routines to use all the
pixels.  Remember that rows and columns are often numbered from ZERO.  The
row or column numbers specified in MASK are the same as those given by the
'D' option in the ITV command, but remember that if the TV image is
compressed, the 'D' option will not give the exact location of certain
features in the image.  Use the BOX option in the TV command to show
smaller sections of an image when finding features you want to mask.

The mask may be saved on the disk with SAVE, and retrieved later with GET.

Not all VISTA commands recognize masked pixels.  The ones that do are: ABX,
AUTOMARK, MN, CROSS, INTERP, MASH, and SURFACE.

\noindent{Examples:}
\begin{example}
  \item[MASK C=234\hfill]{Masks column 234.}
  \item[UNMASK C=234\hfill]{Removes the mask from column 234.}
  \item[MASK PIX=(120,100)\hfill]{Masks the single pixel at
column 100 and row 125.}
  \item[MASK R=20,40\hfill]{Masks pixels in rows 20 to 40.}
  \item[UNMASK BOX=5\hfill]{Unmasks the pixels in box 5.}
\end{example}

\section{MASKTOIM: Create an Image Showing Masked Pixels}
\begin{rawhtml}
<!-- linkto masktoim.html -->
\end{rawhtml}
\begin{command}
  \item[\textbf{Form:}  MASKTOIM buf {[BOX=b]} {[SR=sr]} {[SC=sc]} {[NR=nr]} {[NC=nc]}\hfill]{}
  \item[buf]{is the buffer holding the new image}
  \item[BOX=b]{create an image with the size and orientation of box 'b'}
  \item[SR=sr]{specify the start row of the new image}
  \item[SC=sc]{specify the start column}
  \item[NR=nr]{specify the number of rows}
  \item[NC=nc]{specify the number of columns}
\end{command}

MASKTOIM creates an image, showing which pixels are masked.  Masked pixels
are set to 0 in the image, while unmasked pixels are set to 1.

Use BOX=b to define the image size and location.  This makes the new image
have the size and location of box 'b'.  See BOX for instructions for
defining boxes.

You may also give the size and location of the new image with the keywords
SR, SC, NR, and NC.  NR and NC are required.  If SR or SC are not given,
they default to zero.

Any image already in the destination buffer is destroyed.

\noindent{Examples:}
\begin{example}
  \item[MASKTOIM 1 BOX=5\hfill]{creates an image in buffer 1 having the
       size and orientation of box 5.  The image is filled with zeroes.}

  \item[MASKTOIM 1 BOX=5 CONST=100.0\hfill]{does the same as the first
       example, but fills the image with value 100.0}

  \item[MASKTOIM 5 SR=5 SC=10 NR=25 NC=35\hfill]{creates an image in buffer
       5.  The start (row, column) is (5,10) and the size of the image is
       25 rows by 35 columns.}

  \item[MASKTOIM 1 NR=100 NC=100\hfill]{creates an 100 by 100 image in
       buffer 1.  The start row and column are both 0.}
\end{example}

\batchmode


\usepackage[dvips]{color}


\pagecolor[gray]{.7}

\usepackage[latin1]{inputenc}



\makeatletter

\makeatletter
\count@=\the\catcode`\_ \catcode`\_=8 
\newenvironment{tex2html_wrap}{}{}%
\catcode`\<=12\catcode`\_=\count@
\newcommand{\providedcommand}[1]{\expandafter\providecommand\csname #1\endcsname}%
\newcommand{\renewedcommand}[1]{\expandafter\providecommand\csname #1\endcsname{}%
  \expandafter\renewcommand\csname #1\endcsname}%
\newcommand{\newedenvironment}[1]{\newenvironment{#1}{}{}\renewenvironment{#1}}%
\let\newedcommand\renewedcommand
\let\renewedenvironment\newedenvironment
\makeatother
\let\mathon=$
\let\mathoff=$
\ifx\AtBeginDocument\undefined \newcommand{\AtBeginDocument}[1]{}\fi
\newbox\sizebox
\setlength{\hoffset}{0pt}\setlength{\voffset}{0pt}
\addtolength{\textheight}{\footskip}\setlength{\footskip}{0pt}
\addtolength{\textheight}{\topmargin}\setlength{\topmargin}{0pt}
\addtolength{\textheight}{\headheight}\setlength{\headheight}{0pt}
\addtolength{\textheight}{\headsep}\setlength{\headsep}{0pt}
\setlength{\textwidth}{349pt}
\newwrite\lthtmlwrite
\makeatletter
\let\realnormalsize=\normalsize
\global\topskip=2sp
\def\preveqno{}\let\real@float=\@float \let\realend@float=\end@float
\def\@float{\let\@savefreelist\@freelist\real@float}
\def\liih@math{\ifmmode$\else\bad@math\fi}
\def\end@float{\realend@float\global\let\@freelist\@savefreelist}
\let\real@dbflt=\@dbflt \let\end@dblfloat=\end@float
\let\@largefloatcheck=\relax
\let\if@boxedmulticols=\iftrue
\def\@dbflt{\let\@savefreelist\@freelist\real@dbflt}
\def\adjustnormalsize{\def\normalsize{\mathsurround=0pt \realnormalsize
 \parindent=0pt\abovedisplayskip=0pt\belowdisplayskip=0pt}%
 \def\phantompar{\csname par\endcsname}\normalsize}%
\def\lthtmltypeout#1{{\let\protect\string \immediate\write\lthtmlwrite{#1}}}%
\newcommand\lthtmlhboxmathA{\adjustnormalsize\setbox\sizebox=\hbox\bgroup\kern.05em }%
\newcommand\lthtmlhboxmathB{\adjustnormalsize\setbox\sizebox=\hbox to\hsize\bgroup\hfill }%
\newcommand\lthtmlvboxmathA{\adjustnormalsize\setbox\sizebox=\vbox\bgroup %
 \let\ifinner=\iffalse \let\)\liih@math }%
\newcommand\lthtmlboxmathZ{\@next\next\@currlist{}{\def\next{\voidb@x}}%
 \expandafter\box\next\egroup}%
\newcommand\lthtmlmathtype[1]{\gdef\lthtmlmathenv{#1}}%
\newcommand\lthtmllogmath{\dimen0\ht\sizebox \advance\dimen0\dp\sizebox
  \ifdim\dimen0>.95\vsize
   \lthtmltypeout{%
*** image for \lthtmlmathenv\space is too tall at \the\dimen0, reducing to .95 vsize ***}%
   \ht\sizebox.95\vsize \dp\sizebox\z@ \fi
  \lthtmltypeout{l2hSize %
:\lthtmlmathenv:\the\ht\sizebox::\the\dp\sizebox::\the\wd\sizebox.\preveqno}}%
\newcommand\lthtmlfigureA[1]{\let\@savefreelist\@freelist
       \lthtmlmathtype{#1}\lthtmlvboxmathA}%
\newcommand\lthtmlpictureA{\bgroup\catcode`\_=8 \lthtmlpictureB}%
\newcommand\lthtmlpictureB[1]{\lthtmlmathtype{#1}\egroup
       \let\@savefreelist\@freelist \lthtmlhboxmathB}%
\newcommand\lthtmlpictureZ[1]{\hfill\lthtmlfigureZ}%
\newcommand\lthtmlfigureZ{\lthtmlboxmathZ\lthtmllogmath\copy\sizebox
       \global\let\@freelist\@savefreelist}%
\newcommand\lthtmldisplayA{\bgroup\catcode`\_=8 \lthtmldisplayAi}%
\newcommand\lthtmldisplayAi[1]{\lthtmlmathtype{#1}\egroup\lthtmlvboxmathA}%
\newcommand\lthtmldisplayB[1]{\edef\preveqno{(\theequation)}%
  \lthtmldisplayA{#1}\let\@eqnnum\relax}%
\newcommand\lthtmldisplayZ{\lthtmlboxmathZ\lthtmllogmath\lthtmlsetmath}%
\newcommand\lthtmlinlinemathA{\bgroup\catcode`\_=8 \lthtmlinlinemathB}
\newcommand\lthtmlinlinemathB[1]{\lthtmlmathtype{#1}\egroup\lthtmlhboxmathA
  \vrule height1.5ex width0pt }%
\newcommand\lthtmlinlineA{\bgroup\catcode`\_=8 \lthtmlinlineB}%
\newcommand\lthtmlinlineB[1]{\lthtmlmathtype{#1}\egroup\lthtmlhboxmathA}%
\newcommand\lthtmlinlineZ{\egroup\expandafter\ifdim\dp\sizebox>0pt %
  \expandafter\centerinlinemath\fi\lthtmllogmath\lthtmlsetinline}
\newcommand\lthtmlinlinemathZ{\egroup\expandafter\ifdim\dp\sizebox>0pt %
  \expandafter\centerinlinemath\fi\lthtmllogmath\lthtmlsetmath}
\newcommand\lthtmlindisplaymathZ{\egroup %
  \centerinlinemath\lthtmllogmath\lthtmlsetmath}
\def\lthtmlsetinline{\hbox{\vrule width.1em \vtop{\vbox{%
  \kern.1em\copy\sizebox}\ifdim\dp\sizebox>0pt\kern.1em\else\kern.3pt\fi
  \ifdim\hsize>\wd\sizebox \hrule depth1pt\fi}}}
\def\lthtmlsetmath{\hbox{\vrule width.1em\kern-.05em\vtop{\vbox{%
  \kern.1em\kern0.8 pt\hbox{\hglue.17em\copy\sizebox\hglue0.8 pt}}\kern.3pt%
  \ifdim\dp\sizebox>0pt\kern.1em\fi \kern0.8 pt%
  \ifdim\hsize>\wd\sizebox \hrule depth1pt\fi}}}
\def\centerinlinemath{%
  \dimen1=\ifdim\ht\sizebox<\dp\sizebox \dp\sizebox\else\ht\sizebox\fi
  \advance\dimen1by.5pt \vrule width0pt height\dimen1 depth\dimen1 
 \dp\sizebox=\dimen1\ht\sizebox=\dimen1\relax}

\def\lthtmlcheckvsize{\ifdim\ht\sizebox<\vsize 
  \ifdim\wd\sizebox<\hsize\expandafter\hfill\fi \expandafter\vfill
  \else\expandafter\vss\fi}%
\providecommand{\selectlanguage}[1]{}%
\makeatletter \tracingstats = 1 


\begin{document}
\pagestyle{empty}\thispagestyle{empty}\lthtmltypeout{}%
\lthtmltypeout{latex2htmlLength hsize=\the\hsize}\lthtmltypeout{}%
\lthtmltypeout{latex2htmlLength vsize=\the\vsize}\lthtmltypeout{}%
\lthtmltypeout{latex2htmlLength hoffset=\the\hoffset}\lthtmltypeout{}%
\lthtmltypeout{latex2htmlLength voffset=\the\voffset}\lthtmltypeout{}%
\lthtmltypeout{latex2htmlLength topmargin=\the\topmargin}\lthtmltypeout{}%
\lthtmltypeout{latex2htmlLength topskip=\the\topskip}\lthtmltypeout{}%
\lthtmltypeout{latex2htmlLength headheight=\the\headheight}\lthtmltypeout{}%
\lthtmltypeout{latex2htmlLength headsep=\the\headsep}\lthtmltypeout{}%
\lthtmltypeout{latex2htmlLength parskip=\the\parskip}\lthtmltypeout{}%
\lthtmltypeout{latex2htmlLength oddsidemargin=\the\oddsidemargin}\lthtmltypeout{}%
\makeatletter
\if@twoside\lthtmltypeout{latex2htmlLength evensidemargin=\the\evensidemargin}%
\else\lthtmltypeout{latex2htmlLength evensidemargin=\the\oddsidemargin}\fi%
\lthtmltypeout{}%
\makeatother
\setcounter{page}{1}
\onecolumn

% !!! IMAGES START HERE !!!

\stepcounter{chapter}
\stepcounter{section}
\stepcounter{section}
\stepcounter{section}
\stepcounter{section}
{\newpage\clearpage
\lthtmlfigureA{hanging110}%
\begin{hanging}
   \item{BOX 2 CC=100 CR=123 NC=100 NR=25}
\end{hanging}%
\lthtmlfigureZ
\lthtmlcheckvsize\clearpage}

{\newpage\clearpage
\lthtmlfigureA{hanging113}%
\begin{hanging}
   \item{COPY 3 1; TV 3 BOX=1; CONTOUR 3 BOX=1}
\end{hanging}%
\lthtmlfigureZ
\lthtmlcheckvsize\clearpage}

{\newpage\clearpage
\lthtmlfigureA{hanging117}%
\begin{hanging}
   \item{a=b+3}
\end{hanging}%
\lthtmlfigureZ
\lthtmlcheckvsize\clearpage}

{\newpage\clearpage
\lthtmlinlinemathA{tex2html_wrap_inline225}%
$|$%
\lthtmlinlinemathZ
\lthtmlcheckvsize\clearpage}

{\newpage\clearpage
\lthtmlfigureA{hanging120}%
\begin{hanging}
   \item{SUBTRACT 1 CONST=40.3  ! Subtract 40.3 from image 1.}
\end{hanging}%
\lthtmlfigureZ
\lthtmlcheckvsize\clearpage}

\stepcounter{section}
{\newpage\clearpage
\lthtmlfigureA{example140}%
\begin{example}
   \item[WD 2 m15\hfill]{}
\end{example}%
\lthtmlfigureZ
\lthtmlcheckvsize\clearpage}

{\newpage\clearpage
\lthtmlfigureA{example143}%
\begin{example}
   \item[WD 2 /demo/m15\hfill]{writes to /demo/m15/m15.fits}
   \item[WD 2 m15.xyz\hfill]{writes to: /vista/ccd/m15.xyz}
   \item[WD 2 /demo/m15.xyz\hfill]{writes to: /demo/m15.xyz}
\end{example}%
\lthtmlfigureZ
\lthtmlcheckvsize\clearpage}

\stepcounter{section}
{\newpage\clearpage
\lthtmlfigureA{example157}%
\begin{example}
   \item[xvista\hfill]{}
\end{example}%
\lthtmlfigureZ
\lthtmlcheckvsize\clearpage}

\stepcounter{section}
{\newpage\clearpage
\lthtmlfigureA{command167}%
\begin{command}
   \item[\textbf{Form: } HELP {[subjects]} {[output redirection]}\hfill]{}
   \item[subjects]{are the subjects for which information is requested}
\end{command}%
\lthtmlfigureZ
\lthtmlcheckvsize\clearpage}

{\newpage\clearpage
\lthtmlfigureA{example174}%
\begin{example}
   \item[HELP SUBTRACT \hfill]{information on the command SUBTRACT}
   \item[HELP ZAP FITSTAR \hfill]{information on ZAP, AND FITSTAR.}
   \item[HELP Photometry \hfill]{information on photometry.}
\end{example}%
\lthtmlfigureZ
\lthtmlcheckvsize\clearpage}

{\newpage\clearpage
\lthtmlfigureA{example180}%
\begin{example}
   \item[HELP MASH\hfill]{Sends the information on 'MASH'
        to your terminal.}
   \item[HELP MASH $>$MASH.XXX\hfill]{Sends the information on 'MASH'
        to the file MASH.XXX}
\end{example}%
\lthtmlfigureZ
\lthtmlcheckvsize\clearpage}

{\newpage\clearpage
\lthtmlfigureA{hanging184}%
\begin{hanging}
   \item{HELP CO}
\end{hanging}%
\lthtmlfigureZ
\lthtmlcheckvsize\clearpage}

\stepcounter{section}
{\newpage\clearpage
\lthtmlfigureA{example190}%
\begin{example}
   \item[WD\hfill]{write an image to disk}
   \item[RD\hfill]{read an image from disk}
   \item[BUF\hfill]{show which images are in the buffers}
   \item[MN\hfill]{compute average of an image}
   \item[TV\hfill]{display an image on the television}
   \item[PLOT\hfill]{plot a row, column or spectrum}
   \item[CONTOUR\hfill]{produce a contour plot of an image}
   \item[MASH\hfill]{extract a spectrum from an image}
\end{example}%
\lthtmlfigureZ
\lthtmlcheckvsize\clearpage}

\stepcounter{section}
{\newpage\clearpage
\lthtmlfigureA{command203}%
\begin{command}
   \item[\textbf{Form: } QUIT\hfill]{}
\end{command}%
\lthtmlfigureZ
\lthtmlcheckvsize\clearpage}

\stepcounter{section}
{\newpage\clearpage
\lthtmlfigureA{command209}%
\begin{command}
  \item[\textbf{Form: } NEWS\hfill]{}
\end{command}%
\lthtmlfigureZ
\lthtmlcheckvsize\clearpage}

\stepcounter{section}
{\newpage\clearpage
\lthtmlfigureA{command216}%
\begin{command}
   \item[\textbf{Form: } TERM {[TERMINAL=vterm]} {[HARDCOPY=vhard]}\hfill]{}
\par
\item[(none)]{Prompt user for graphics display terminal type to use.
        Hardcopy devices must be changed explicitly using HARDCOPY= below.}
\par
\item[TERMINAL=vterm]{Makes default graphics display terminal for plots
        the device corresponding to LickMongo "vterm" device code}
\par
\item[HARDCOPY=vhard]{Makes default hardcopy device for plots the device
        corresponding to on of the VHARD device codes.  Typing HARDCOPY=0
        will give the user the hardcopy code menu and prompt for the
        desired option.}
\end{command}%
\lthtmlfigureZ
\lthtmlcheckvsize\clearpage}

\stepcounter{chapter}
\stepcounter{section}
{\newpage\clearpage
\lthtmlfigureA{example404}%
\begin{example}
  \item[?A\hfill]{List all commands beginning with A}
  \item[?MA\hfill]{List all commands beginning with MA}
\end{example}%
\lthtmlfigureZ
\lthtmlcheckvsize\clearpage}

{\newpage\clearpage
\lthtmlfigureA{example408}%
\begin{example}
  \item[? MASH\hfill]{Print help on the command MASH}
  \item[? HELP\hfill]{Print help on the command HELP}
  \item[?\hfill]{Help on everything.}
\end{example}%
\lthtmlfigureZ
\lthtmlcheckvsize\clearpage}

\stepcounter{section}
{\newpage\clearpage
\lthtmlinlinemathA{tex2html_wrap_inline548}%
$>$%
\lthtmlinlinemathZ
\lthtmlcheckvsize\clearpage}

{\newpage\clearpage
\lthtmlinlinemathA{tex2html_wrap_inline550}%
$>>$%
\lthtmlinlinemathZ
\lthtmlcheckvsize\clearpage}

{\newpage\clearpage
\lthtmlfigureA{example418}%
\begin{example}
  \item[PRINT PHOT $>$first.lis\hfill]{Prints the contents of a 
       photometry file into the file 'first.lis'.  The file will be
       located in the current directory.}
\par
\item[HELP MASH $>>$help.xxx\hfill]{Appends the help information for
        MASH to the end of file 'help.xxx', if it exists.  If it does not,
        the file is created.}
\end{example}%
\lthtmlfigureZ
\lthtmlcheckvsize\clearpage}

\stepcounter{section}
{\newpage\clearpage
\lthtmlfigureA{command424}%
\begin{command}
  \item[\textbf{Form: }HISTORY (output redirection)]{}
\end{command}%
\lthtmlfigureZ
\lthtmlcheckvsize\clearpage}

{\newpage\clearpage
\lthtmlfigureA{hanging428}%
\begin{hanging}
  \item{HISTORY $>$history.txt}
\end{hanging}%
\lthtmlfigureZ
\lthtmlcheckvsize\clearpage}

\stepcounter{section}
{\newpage\clearpage
\lthtmlfigureA{example440}%
\begin{example}
   \item[\%\hfill]{repeats command 15 (the last command):  PLOT 4}
   \item[\%10\hfill]{repeats command 10:  RD 1 [MYDIR]HD183143}
   \item[\%MASH \hfill]{repeats command 14 (the last MASH command): 
       MASH 4 1 SP=100,103}
   \item[\%M\hfill]{repeats command 14:  MASH 4 1 SP=100,103}
\end{example}%
\lthtmlfigureZ
\lthtmlcheckvsize\clearpage}

{\newpage\clearpage
\lthtmlfigureA{hanging451}%
\begin{hanging}
  \item{PLOT 1 R=50}
\end{hanging}%
\lthtmlfigureZ
\lthtmlcheckvsize\clearpage}

{\newpage\clearpage
\lthtmlfigureA{example454}%
\begin{example}
  \item[\%10 HARD\hfill]{does PLOT 1 R=50 HARD}
  \item[\%10 R=55\hfill]{does PLOT R=55}
  \item[\%10 R=55 HARD\hfill]{does PLOT R=55 HARD}
  \item[\%10 XS=100 XE=200\hfill]{does PLOT 1 R=50 XS=100 XE=100}
\end{example}%
\lthtmlfigureZ
\lthtmlcheckvsize\clearpage}

{\newpage\clearpage
\lthtmlfigureA{hanging460}%
\begin{hanging}
  \item{PLOT 3 PIXEL HARD XS=100 XE=300 MAX=4096.0}
\end{hanging}%
\lthtmlfigureZ
\lthtmlcheckvsize\clearpage}

{\newpage\clearpage
\lthtmlfigureA{example463}%
\begin{example}
  \item[\%58 -PIXEL -HARD\hfill]{does PLOT 3 XS=100 XE=300 MAX=4096.0}
\end{example}%
\lthtmlfigureZ
\lthtmlcheckvsize\clearpage}

\stepcounter{section}
{\newpage\clearpage
\lthtmlfigureA{hanging468}%
\begin{hanging}
  \item{ RD 1 /mydirectory/$|$}
  \item{really\_long\_file\_name SPEC}
\end{hanging}%
\lthtmlfigureZ
\lthtmlcheckvsize\clearpage}

\stepcounter{section}
{\newpage\clearpage
\lthtmlfigureA{command476}%
\begin{command}
  \item[\textbf{Form: }ALIAS {[synonym]} {[command]} {[output
       redirection]}\hfill]{} 
  \item[\textbf{Form: }UNALIAS {[synonym]}\hfill]{}
\end{command}%
\lthtmlfigureZ
\lthtmlcheckvsize\clearpage}

{\newpage\clearpage
\lthtmlfigureA{hanging488}%
\begin{hanging}
  \item{ALIAS T 'TV 1 1234.0 CF=NEWTHREE'}
\end{hanging}%
\lthtmlfigureZ
\lthtmlcheckvsize\clearpage}

{\newpage\clearpage
\lthtmlfigureA{hanging491}%
\begin{hanging}
  \item{T BOX=1}
\end{hanging}%
\lthtmlfigureZ
\lthtmlcheckvsize\clearpage}

{\newpage\clearpage
\lthtmlfigureA{hanging494}%
\begin{hanging}
  \item{ALIAS H 'HISTOGRAM'}
\end{hanging}%
\lthtmlfigureZ
\lthtmlcheckvsize\clearpage}

{\newpage\clearpage
\lthtmlfigureA{hanging497}%
\begin{hanging}
  \item{H 3\hfill}
\end{hanging}%
\lthtmlfigureZ
\lthtmlcheckvsize\clearpage}

{\newpage\clearpage
\lthtmlfigureA{hanging502}%
\begin{hanging}
  \item{ALIAS ALL 'RD 1 ./a.fits; MN 1; PLOT 1 R=50; BUF'}
\end{hanging}%
\lthtmlfigureZ
\lthtmlcheckvsize\clearpage}

\stepcounter{section}
{\newpage\clearpage
\lthtmlfigureA{command508}%
\begin{command}
  \item[\textbf{Form: }EDIT\hfill]{}
\end{command}%
\lthtmlfigureZ
\lthtmlcheckvsize\clearpage}

\stepcounter{section}
{\newpage\clearpage
\lthtmlfigureA{command523}%
\begin{command}
  \item[\textbf{Form: } SETDIR code {[DIR=directory\_name]} 
       {[EXT=extension]}\hfill]{}
  \item[code]{specifies which directory is being set or changed}
  \item[DIR= ]{   specifies a directory for the type of object
       indicated by the code.}
  \item[EXT=]{gives the extension for files in the default directory}
\end{command}%
\lthtmlfigureZ
\lthtmlcheckvsize\clearpage}

{\newpage\clearpage
\lthtmlfigureA{example537}%
\begin{example}
  \item[SETDIR SP DIR=mydir/spec\hfill]{changes the default directory
       to mydir/spec}
  \item[SETDIR SP EXT=.xyz\hfill]{changes the default extension to '.xyz'}
  \item[SETDIR SP EXT=.XYZ DIR=mydir/spec\hfill]{changes both the
       directory and extension at one time.}
\end{example}%
\lthtmlfigureZ
\lthtmlcheckvsize\clearpage}

\stepcounter{section}
{\newpage\clearpage
\lthtmlfigureA{command543}%
\begin{command} 
  \item[\textbf{Form: } CD path\_name\hfill]{}
  \item[path\_name]{any valid Unix directory path}
\end{command}%
\lthtmlfigureZ
\lthtmlcheckvsize\clearpage}

\stepcounter{chapter}
{\newpage\clearpage
\lthtmlfigureA{example704}%
\begin{example}
  \item[SET\hfill]{sets the value of a variable, either directly or in
       terms of arithmetic operations on other variables.}
\par
\item[TYPE\hfill]{displays the value of a variable or any arithmetic
       expression.}
\par
\item[ASK\hfill]{asks for information to be entered at the terminal.}
\par
\item[PRINTF\hfill]{formatted printing of variable values and character
       strings.}
\par
\item[STRING\hfill]{definition of string variables.}
\end{example}%
\lthtmlfigureZ
\lthtmlcheckvsize\clearpage}

\stepcounter{section}
{\newpage\clearpage
\lthtmlinlinemathA{tex2html_wrap_inline990}%
$<$%
\lthtmlinlinemathZ
\lthtmlcheckvsize\clearpage}

{\newpage\clearpage
\lthtmlfigureA{hanging732}%
\begin{hanging}
  \item{(B+0.53)*10\^{ }(45.6/(A+5))}
\end{hanging}%
\lthtmlfigureZ
\lthtmlcheckvsize\clearpage}

{\newpage\clearpage
\lthtmlfigureA{example737}%
\begin{example}
  \item[INT{[E]}\hfill]{nearest integer to the expression E}
  \item[ABS{[E]}\hfill]{absolute value of E}
  \item[MOD{[E,I]}\hfill]{E modulo I}
  \item[IFIX{[E]}\hfill]{integer part of E (truncation)}
  \item[MAX{[E,F]}\hfill]{the larger of E or F}
  \item[MIN{[E,F]}\hfill]{the smaller of E of F}
  \item[LOG10{[E]}\hfill]{log to the base 10 of E}
  \item[LOGE{[E]}\hfill]{log to the base e of E}
  \item[EXP{[E]}\hfill]{e raised to the power E (Use \^ 
       for all other exponentiations)}
  \item[SQRT{[E]}\hfill]{square root of absolute value of E}
  \item[RAN{[A,B]}\hfill]{returns a random number between A and B}
\end{example}%
\lthtmlfigureZ
\lthtmlcheckvsize\clearpage}

{\newpage\clearpage
\lthtmlfigureA{example762}%
\begin{example}
  \item[SIN{[E]}\hfill]{sine of E (E in radians)}
  \item[SIND{[E]}\hfill]{sine of E (E in degrees)}
  \item[COS{[E]}\hfill]{cosine of E (E in radians)}
  \item[COSD{[E]}\hfill]{cosine of E (E in degrees)}
  \item[ARCTAN{[E]}\hfill]{arctan of E, producing radians}
  \item[ARCTAND{[E]}\hfill]{arctan of E, producing degrees}
  \item[ARCCOS{[E]}\hfill]{arccos of E, producing radians}
  \item[ARCCOSD{[E]}\hfill]{arccos of E, producing degrees}
\end{example}%
\lthtmlfigureZ
\lthtmlcheckvsize\clearpage}

{\newpage\clearpage
\lthtmlfigureA{example781}%
\begin{example}
  \item[NR{[B]}\hfill]{number of rows of the object in buffer B.}
  \item[NC{[B]}\hfill]{number of columns of ...}
  \item[SR{[B]}\hfill]{start row ...}
  \item[SC{[B]}\hfill]{start column ...}
  \item[EXPOS{[B]}\hfill]{exposure time ...}
  \item[RA{[B]}\hfill]{right ascension in sec of time of ...}
  \item[DEC{[B]}\hfill]{declination in sec of arc ...}
  \item[ZENITH{[B]}\hfill]{zenith distance in radians ...}
  \item[UT{[B]}\hfill]{universal time of mid-exposure in hours ...}
\end{example}%
\lthtmlfigureZ
\lthtmlcheckvsize\clearpage}

{\newpage\clearpage
\lthtmlfigureA{example802}%
\begin{example}
  \item[GETVAL{[I,R,C]}\hfill]{returns the value of the pixel at row R and
       column C in image I.}
  \item[SETVAL{[I,R,C,V]}\hfill]{returns the value of the pixel at row R and
       column C in image I, then sets the value of that pixel to V.}
  \item[WL{[I,P]}\hfill]{returns the wavelength of pixel P in image I.}
  \item[PIX{[I,W]}\hfill]{returns the pixel corresponding to
       wavelength W in image I.}
\end{example}%
\lthtmlfigureZ
\lthtmlcheckvsize\clearpage}

{\newpage\clearpage
\lthtmlfigureA{hanging813}%
\begin{hanging}
  \item{PLOT 4 R=200 XS=X XE=X+50}
\end{hanging}%
\lthtmlfigureZ
\lthtmlcheckvsize\clearpage}

{\newpage\clearpage
\lthtmlfigureA{hanging816}%
\begin{hanging}
  \item{PLOT 4 R=200 XS=X XE=XLAST=X+50}
\end{hanging}%
\lthtmlfigureZ
\lthtmlcheckvsize\clearpage}

\stepcounter{section}
{\newpage\clearpage
\lthtmlfigureA{command821}%
\begin{command}
  \item[\textbf{Form: } SET var\_name=value {[var\_name=value]}\hfill]{}
  \item[var\_name]{the name of the variable being defined.}
  \item[value]{its value}
\end{command}%
\lthtmlfigureZ
\lthtmlcheckvsize\clearpage}

{\newpage\clearpage
\lthtmlfigureA{example828}%
\begin{example}
  \item[SET Q=6\hfill]{Sets Q to have the value 6}
  \item[SET A=1 B=3 C=D=6\hfill]{Sets several variables at once}
  \item[SET V=SIND{[45]}\hfill]{Functions may be used}
  \item[SET B=3.1415926\^0.5+4\hfill]{Any arithmetic expression may be used.}
  \item[SET C=LOG10{[@FILE.1]}\hfill]{References to data from files may be used.}
\end{example}%
\lthtmlfigureZ
\lthtmlcheckvsize\clearpage}

{\newpage\clearpage
\lthtmlfigureA{example838}%
\begin{example}
  \item[A=5\hfill]{defines A to be 5}
  \item[Q=SIND{[45]}\hfill]{defines A to be sine of 45 degrees.}
\end{example}%
\lthtmlfigureZ
\lthtmlcheckvsize\clearpage}

\stepcounter{section}
{\newpage\clearpage
\lthtmlfigureA{command846}%
\begin{command}
  \item[\textbf{Form: } TYPE expression {[expression]} {[expression]} ...\hfill]{}
\end{command}%
\lthtmlfigureZ
\lthtmlcheckvsize\clearpage}

{\newpage\clearpage
\lthtmlfigureA{example852}%
\begin{example}
  \item[TYPE X\hfill]{Evaluates X (a variable) and shows
the value of X on the screen.}
  \item[TYPE X+0.5\^3.4\hfill]{Evaluates the expression shows and
prints the value.}
\end{example}%
\lthtmlfigureZ
\lthtmlcheckvsize\clearpage}

{\newpage\clearpage
\lthtmlfigureA{hanging856}%
\begin{hanging}
  \item{TYPE 6.3}
\end{hanging}%
\lthtmlfigureZ
\lthtmlcheckvsize\clearpage}

\stepcounter{section}
{\newpage\clearpage
\lthtmlfigureA{command860}%
\begin{command}
  \item[\textbf{Form: } ASK {['An optional prompt in quotes']} var\_name {[DEFAULT=def]}\hfill]{}
\end{command}%
\lthtmlfigureZ
\lthtmlcheckvsize\clearpage}

{\newpage\clearpage
\lthtmlfigureA{example866}%
\begin{example}
  \item[ASK BCKGND\hfill]{will print 'ENTER BCKGND : on your screen.  When
       you enter an expression and hit RETURN, the value of BCKGND will be
       set to the number you specified.}
\par
\item[ASK 'Enter an estimate for the background $>>$\  ' BCKGND \hfill]
       {will type the prompt 'Enter an estimate for the background $>>$\  ' on
       your terminal, and wait for you to enter an expression; the value of
       BCKGND is the value of that expression.}
\par
\item[ASK 'Enter your favorite number: ' FAVORITE DEFAULT=7\hfill]
       { will type the prompt 'Enter you favorite number: ' and wait for
       you to enter an expression; if you hit a single carriage return, the
       value 7 will be assigned to the variable FAVORITE.}
\par
\end{example}%
\lthtmlfigureZ
\lthtmlcheckvsize\clearpage}

\stepcounter{section}
{\newpage\clearpage
\lthtmlfigureA{example873}%
\begin{example}
  \item[MN \$J\hfill]{mean of buffer J}
  \item[ADD \$BUF CONST=5\hfill]{add 5 to buffer BUF}
\end{example}%
\lthtmlfigureZ
\lthtmlcheckvsize\clearpage}

\stepcounter{section}
{\newpage\clearpage
\lthtmlfigureA{command879}%
\begin{command}
  \item[\textbf{Form: } PRINTF 'Format string' {[expressions]} {[redirection]}\hfill]{}
\end{command}%
\lthtmlfigureZ
\lthtmlcheckvsize\clearpage}

{\newpage\clearpage
\lthtmlfigureA{example885}%
\begin{example}
  \item[PRINTF HELLO\hfill]{prints HELLO}
  \item[PRINTF 'Hello, world'\hfill]{prints Hello, world}
\end{example}%
\lthtmlfigureZ
\lthtmlcheckvsize\clearpage}

{\newpage\clearpage
\lthtmlfigureA{example891}%
\begin{example}
  \item[PRINTF '\%F4.1 \%F9.4' A PI\hfill]{ prints ' 1.0    3.1416'}
  \item[PRINTF '\%I6 and \%F9.5' A PI\hfill]{ prints '     1 and   3.14159'}
  \item[PRINTF 'The value of pi is \%F9.7' A PI\hfill]{ prints 
'The value of pi is 3.1415900'}
\end{example}%
\lthtmlfigureZ
\lthtmlcheckvsize\clearpage}

{\newpage\clearpage
\lthtmlfigureA{example896}%
\begin{example}
  \item[A=5.1234\hfill]{}
  \item[PRINTF 'The value of A is \\n    \%F9.3' A\hfill]{prints
The value of A is}
  \item{5.123}
\end{example}%
\lthtmlfigureZ
\lthtmlcheckvsize\clearpage}

\stepcounter{section}
{\newpage\clearpage
\lthtmlfigureA{command903}%
\begin{command}
  \item[\textbf{Form: } STRING name {['format string']} {[expressions]}\hfill]{}
  \item[\textbf{Form: } STRING name '?query']{}
\end{command}%
\lthtmlfigureZ
\lthtmlcheckvsize\clearpage}

{\newpage\clearpage
\lthtmlfigureA{example911}%
\begin{example}
  \item[STRING EXPR  'This is a string with seven words.'\hfill]{}
  \item[STRING HELLO 'Hello, world'\hfill]{}
\end{example}%
\lthtmlfigureZ
\lthtmlcheckvsize\clearpage}

{\newpage\clearpage
\lthtmlfigureA{hanging915}%
\begin{hanging}
  \item{STRING NAME ' '}
\end{hanging}%
\lthtmlfigureZ
\lthtmlcheckvsize\clearpage}

{\newpage\clearpage
\lthtmlfigureA{hanging918}%
\begin{hanging}
  \item{J=7}
  \item{STRING NAME 'FILE\%I3.3' J}
\end{hanging}%
\lthtmlfigureZ
\lthtmlcheckvsize\clearpage}

{\newpage\clearpage
\lthtmlfigureA{example922}%
\begin{example}
  \item[STRING FILE '?Enter a filename for this image. $>>$\  '\hfill]{ will
   print 'Enter a filename for this image. $>>$\  ' then pause while you
   enter a name.  Your reply will be loaded into the character string
   FILE.}
\par
\item[STRING HEADER ?\hfill]{ will print 'Enter HEADER ' then accept a
        string.}
\end{example}%
\lthtmlfigureZ
\lthtmlcheckvsize\clearpage}

{\newpage\clearpage
\lthtmlfigureA{hanging926}%
\begin{hanging}
  \item{STRING FILE '?Enter a file for image \%I2 ' J}
\end{hanging}%
\lthtmlfigureZ
\lthtmlcheckvsize\clearpage}

\stepcounter{section}
{\newpage\clearpage
\lthtmlfigureA{hanging934}%
\begin{hanging}
  \item{RD buf filename}
\end{hanging}%
\lthtmlfigureZ
\lthtmlcheckvsize\clearpage}

{\newpage\clearpage
\lthtmlfigureA{hanging937}%
\begin{hanging}
  \item{RD 2 \{FNAME\}}
\end{hanging}%
\lthtmlfigureZ
\lthtmlcheckvsize\clearpage}

{\newpage\clearpage
\lthtmlfigureA{hanging940}%
\begin{hanging}
  \item{RD 2 MYDIR/MYFILE}
\end{hanging}%
\lthtmlfigureZ
\lthtmlcheckvsize\clearpage}

{\newpage\clearpage
\lthtmlfigureA{hanging950}%
\begin{hanging}
  \item{\{?BUFFER:CARDNAME\}}
\end{hanging}%
\lthtmlfigureZ
\lthtmlcheckvsize\clearpage}

{\newpage\clearpage
\lthtmlfigureA{example953}%
\begin{example}
  \item[STRING OBJ '{?23:OBJECT}'\hfill]{Loads the name of the object
       in buffer 23 into string {OBJ}.}
  \item[A={?1:FOCUS}\hfill]{Gets the value of the FOCUS
       card (a number) and loads the numerical value into A.}
\end{example}%
\lthtmlfigureZ
\lthtmlcheckvsize\clearpage}

{\newpage\clearpage
\lthtmlfigureA{hanging963}%
\begin{hanging}
    \item {RD 1 ./mydir/hd183143}
  \end{hanging}%
\lthtmlfigureZ
\lthtmlcheckvsize\clearpage}

{\newpage\clearpage
\lthtmlfigureA{hanging966}%
\begin{hanging}
    \item {WD 1 ./mydir/hd183143}
  \end{hanging}%
\lthtmlfigureZ
\lthtmlcheckvsize\clearpage}

\stepcounter{chapter}
\stepcounter{section}
{\newpage\clearpage
\lthtmlfigureA{example1195}%
\begin{example}
  \item[PEDIT\hfill]{edits the current procedure buffer.}
  \item[WP\hfill]{stores the procedure buffer on the disk.}
  \item[RP\hfill]{reads the procedure from disk.}
  \item[SHOW\hfill]{displays the procedure buffer}
  \item[GO\hfill]{begins execution of the procedure.}
\end{example}%
\lthtmlfigureZ
\lthtmlcheckvsize\clearpage}

{\newpage\clearpage
\lthtmlfigureA{example1202}%
\begin{example}
  \item[VERIFY\hfill]{executes a procedure line by line, to aid in debugging.}
  \item[PAUSE\hfill]{pauses during execution of a procedure.}
  \item[CALL\hfill]{runs a procedure as a subroutine.}
  \item[RETURN\hfill]{returns from a procedure used as a subroutine.}
  \item[DO, END\_DO\hfill]{define a loop in a program for execution a given
       number of times,}
  \item[GOTO\hfill]{jumps to another place in the procedure.}
  \item[:\hfill]{defines a place to jump to in the procedure.}
  \item[IF, END\_IF \hfill]{define a block of commands that are executed
       only under certain conditions.}
  \item[ELSE, ELSE\_IF\hfill]{control branching for branching that has many
       options.}
\end{example}%
\lthtmlfigureZ
\lthtmlcheckvsize\clearpage}

\stepcounter{section}
{\newpage\clearpage
\lthtmlfigureA{command1220}%
\begin{command}
  \item[\textbf{Form: } PEDIT\hfill]{}
\end{command}%
\lthtmlfigureZ
\lthtmlcheckvsize\clearpage}

\stepcounter{section}
{\newpage\clearpage
\lthtmlfigureA{command1230}%
\begin{command}
  \item[\textbf{Form: } STOP {['A message']}\hfill]{}
\end{command}%
\lthtmlfigureZ
\lthtmlcheckvsize\clearpage}

\stepcounter{section}
{\newpage\clearpage
\lthtmlfigureA{command1238}%
\begin{command} 
  \item[\textbf{Form: } END\hfill]{}
\end{command}%
\lthtmlfigureZ
\lthtmlcheckvsize\clearpage}

\stepcounter{section}
{\newpage\clearpage
\lthtmlfigureA{command1245}%
\begin{command}
  \item[\textbf{Form: } SHOW {[output redirection]}\hfill]{}
\end{command}%
\lthtmlfigureZ
\lthtmlcheckvsize\clearpage}

{\newpage\clearpage
\lthtmlfigureA{hanging1250}%
\begin{hanging}
  \item{SHOW $>$LP:}
\end{hanging}%
\lthtmlfigureZ
\lthtmlcheckvsize\clearpage}

\stepcounter{section}
{\newpage\clearpage
\lthtmlfigureA{command1256}%
\begin{command}
  \item[\textbf{Form: } WP filename\hfill]{}
  \item[filename]{is the name of the file that will store the current procedure.}
\end{command}%
\lthtmlfigureZ
\lthtmlcheckvsize\clearpage}

{\newpage\clearpage
\lthtmlfigureA{example1261}%
\begin{example}
  \item[WP medfly\hfill]{writes the current procedure to the file
       vista/procedure/medfly.pro}
  \item[WP /demo/medfly\hfill]{writes to /demo/medfly.pro}
  \item[WP medfly.xyz\hfill]{writes to vista/procedure/medfly.xyz}
\end{example}%
\lthtmlfigureZ
\lthtmlcheckvsize\clearpage}

\stepcounter{section}
{\newpage\clearpage
\lthtmlfigureA{command1269}%
\begin{command}
  \item[\textbf{Form: } RP filename\hfill]{}
  \item[filename]{is the name of the file that holds the desired procedure.}
\end{command}%
\lthtmlfigureZ
\lthtmlcheckvsize\clearpage}

{\newpage\clearpage
\lthtmlfigureA{example1275}%
\begin{example}
  \item[RP medfly\hfill]{reads the contents of /vista/procedure/medfly.pro
       into the procedure buffer.}
  \item[RP /demo/medfly\hfill]{reads from /demo/medfly.pro}
  \item[RP medfly.xyz\hfill]{reads from /vista/procedure/medfly.xyz}
\end{example}%
\lthtmlfigureZ
\lthtmlcheckvsize\clearpage}

\stepcounter{section}
{\newpage\clearpage
\lthtmlfigureA{command1283}%
\begin{command}
  \item[\textbf{Form: } GO {[parameter1]} {[parameter2]} ...]{}
  \item[parameter1,2,...]{are parameters passed to the procedure.}
\end{command}%
\lthtmlfigureZ
\lthtmlcheckvsize\clearpage}

{\newpage\clearpage
\lthtmlfigureA{example1291}%
\begin{example}
  \item[GO 10\hfill]{executes the procedure in the buffer,
       passing the numeric parameter 10 to the procedure.}
  \item[GO mydir/image\hfill]{execute the procedure, passing the
        string parameter to the procedure.}
\end{example}%
\lthtmlfigureZ
\lthtmlcheckvsize\clearpage}

\stepcounter{section}
{\newpage\clearpage
\lthtmlfigureA{command1299}%
\begin{command}
  \item[\textbf{Form: } CALL procedure\_filename {[parameter1]} {[parameter2]} ...\hfill]{}
  \item[procedure\_filename]{is the name of a file holding a procedure.}
  \item[parameter1,2,...]{are optional parameters passed to the called procedure.}
\end{command}%
\lthtmlfigureZ
\lthtmlcheckvsize\clearpage}

{\newpage\clearpage
\lthtmlfigureA{example1307}%
\begin{example}
  \item[CALL medfly\hfill]{executes /vista/procedure/medfly.pro}
  \item[CALL /mydir/medfly\hfill]{executes /mydir/medfly.pro}
  \item[CALL medfly.txt\hfill]{executes /vista/procedure/medfly.txt}
  \item[CALL medfly kill\hfill]{executes /vista/procedure/medfly.pro
passing it the parameter KILL}
\end{example}%
\lthtmlfigureZ
\lthtmlcheckvsize\clearpage}

\stepcounter{section}
{\newpage\clearpage
\lthtmlfigureA{command1315}%
\begin{command}
  \item[\textbf{Form: } RETURN\hfill]{}
\end{command}%
\lthtmlfigureZ
\lthtmlcheckvsize\clearpage}

\stepcounter{section}
{\newpage\clearpage
\lthtmlfigureA{command1330}%
\begin{command}
  \item[\textbf{Form: } PARAMETER {[varname]} {[varname]} {[STRING=string\_var]} ...\hfill]{}
\end{command}%
\lthtmlfigureZ
\lthtmlcheckvsize\clearpage}

{\newpage\clearpage
\lthtmlfigureA{hanging1337}%
\begin{hanging}
  \item{CALL TEST 2 X IMAGEFILE}
\end{hanging}%
\lthtmlfigureZ
\lthtmlcheckvsize\clearpage}

{\newpage\clearpage
\lthtmlfigureA{hanging1340}%
\begin{hanging}
  \item{PARAMETER BUFNUM FACTOR STRING=FILENAME}
\end{hanging}%
\lthtmlfigureZ
\lthtmlcheckvsize\clearpage}

{\newpage\clearpage
\lthtmlfigureA{hanging1343}%
\begin{hanging}
  \item{STRING FILENAME IMAGENAME}
\end{hanging}%
\lthtmlfigureZ
\lthtmlcheckvsize\clearpage}

\stepcounter{section}
{\newpage\clearpage
\lthtmlfigureA{command1349}%
\begin{command}
  \item[\textbf{Form: } VERIFY Y or VERIFY N\hfill]{}
\end{command}%
\lthtmlfigureZ
\lthtmlcheckvsize\clearpage}

\stepcounter{section}
{\newpage\clearpage
\lthtmlfigureA{command1355}%
\begin{command} 
  \item[\textbf{Form: } DEF [line\_number]\hfill]{}
  \item[line\_number]{is the [optional] line number of the
beginning of the new definition.}
\end{command}%
\lthtmlfigureZ
\lthtmlcheckvsize\clearpage}

{\newpage\clearpage
\lthtmlfigureA{example1361}%
\begin{example}
  \item[DEF\hfill]{begins a procedure definition on line 1.}
  \item[DEF 10\hfill]{begins a procedure definition on line 10.  
       Commands on lines 1 through 9, if there are any, are preserved.}
\end{example}%
\lthtmlfigureZ
\lthtmlcheckvsize\clearpage}

\stepcounter{section}
{\newpage\clearpage
\lthtmlfigureA{command1367}%
\begin{command} 
  \item[\textbf{Form: } SAME\hfill]{}
\end{command}%
\lthtmlfigureZ
\lthtmlcheckvsize\clearpage}

\stepcounter{chapter}
\stepcounter{section}
{\newpage\clearpage
\lthtmlfigureA{command1498}%
\begin{command}
\item[\textbf{Form: } PAUSE 'prompt message'\hfill]{}
\item[ctrl-C\hfill]{}
\end{command}%
\lthtmlfigureZ
\lthtmlcheckvsize\clearpage}

\stepcounter{section}
{\newpage\clearpage
\lthtmlfigureA{command1506}%
\begin{command}
  \item[\textbf{Form: } CONTINUE\hfill]{}
\end{command}%
\lthtmlfigureZ
\lthtmlcheckvsize\clearpage}

\stepcounter{section}
{\newpage\clearpage
\lthtmlfigureA{command1513}%
\begin{command}
  \item[\textbf{Form: } GOTO label\_name\hfill]{}
  \item[label\_name]{is a label defined somewhere else in the procedure.}
\end{command}%
\lthtmlfigureZ
\lthtmlcheckvsize\clearpage}

\stepcounter{section}
{\newpage\clearpage
\lthtmlfigureA{command1527}%
\begin{command}
  \item[\textbf{Form: } DO var=N1,N2,{[N3]}\hfill]{}
  \item[\textbf{Form: } \{any vista commands\}\hfill]{}
  \item[\textbf{Form: } END\_DO\hfill]{}
  \item[var]{is a variable name,}
  \item[N1]{is the initial value of the variable,}
  \item[N2]{is the final value,}
  \item[N3]{is the increment by which N1 is adjusted in
       each pass through the DO loop.}
\end{command}%
\lthtmlfigureZ
\lthtmlcheckvsize\clearpage}

\stepcounter{section}
{\newpage\clearpage
\lthtmlfigureA{example1557}%
\begin{example}
  \item[IF A$>$B\hfill]{Test A greater than B}
  \item[IF A$>$=B\hfill]{Test A greater than or equal to B}
  \item[IF A==B\hfill]{Test A equal to B}
  \item[IF A~=B\hfill]{Test A not equal to B}
  \item[IF A$<$=B\hfill]{Test A less than or equal to B}
  \item[IF A$<$B\hfill]{Test A less than B}
\end{example}%
\lthtmlfigureZ
\lthtmlcheckvsize\clearpage}

{\newpage\clearpage
\lthtmlfigureA{example1565}%
\begin{example}
  \item[IF (A$>$B)\&(A==C)\hfill]{Test A $>$\  B and A = C}
  \item[IF ((A==B)|(C$<$D))\&(C==B)\hfill]{Test (A = B or C $<$\  D) and C = B}
\end{example}%
\lthtmlfigureZ
\lthtmlcheckvsize\clearpage}

\stepcounter{section}
{\newpage\clearpage
\lthtmlfigureA{command1581}%
\begin{command}
  \item[\textbf{Form: } ERROR  VISTA\_command\hfill]{}
  \item[VISTA\_command]{is any valid VISTA command.}
\end{command}%
\lthtmlfigureZ
\lthtmlcheckvsize\clearpage}

\stepcounter{section}
{\newpage\clearpage
\lthtmlfigureA{command1595}%
\begin{command}
  \item[\textbf{Form: } EOF  VISTA\_command\hfill]{}
  \item[VISTA\_command]{is any valid VISTA command.}
\end{command}%
\lthtmlfigureZ
\lthtmlcheckvsize\clearpage}

\stepcounter{chapter}
{\newpage\clearpage
\lthtmlfigureA{example1712}%
\begin{example}
  \item[RD\hfill]{read an image from a disk file}
  \item[WD\hfill]{write an image to a disk file}
\end{example}%
\lthtmlfigureZ
\lthtmlcheckvsize\clearpage}

{\newpage\clearpage
\lthtmlfigureA{example1716}%
\begin{example}
   \item[SETDIR\hfill]{set the VISTA default directories \& file extensions}
   \item[CD\hfill]{change the current working directory}
\end{example}%
\lthtmlfigureZ
\lthtmlcheckvsize\clearpage}

{\newpage\clearpage
\lthtmlfigureA{example1720}%
\begin{example}
  \item[COPY\hfill]{copy an image between buffers}
  \item[BUFFERS\hfill]{list the contents of the image buffers}
  \item[DISPOSE\hfill]{clear (delete) image buffers}
  \item[CHANGE\hfill]{renames objects}
  \item[CREATE\hfill]{create a blank image buffer}
\end{example}%
\lthtmlfigureZ
\lthtmlcheckvsize\clearpage}

{\newpage\clearpage
\lthtmlfigureA{example1727}%
\begin{example}
  \item[FITS\hfill]{insert/edit FITS header cards}
  \item[UNFIT\hfill]{delete FITS header cards}
  \item[HEDIT\hfill]{edit FITS headers}
  \item[FIXHEAD\hfill]{fix FITS headers}
\end{example}%
\lthtmlfigureZ
\lthtmlcheckvsize\clearpage}

\stepcounter{section}
{\newpage\clearpage
\lthtmlfigureA{command1737}%
\begin{command}
  \item[\textbf{Form: } RD buf filename {[WFPC]} {[DOM]} {[SDAS]}
       {[SPEC]} {[OLD]} {[HEADONLY]}\hfill]{}
  \item[buf]{is the buffer in which the image will be stored, and}
  \item[filename]{(character string) is the name of the diskfile holding
       the spectrum.}
  \item[WFPC,DOM,SDAS]{read WFPC format file with appropriate filename}
  \item[SPEC]{read the image from the spectrum directory, instead of from
       the image directory (archaic).}
  \item[OLD]{read old style VISTA images}
  \item[HEADONLY]{read only the image header, not the image data.}
\end{command}%
\lthtmlfigureZ
\lthtmlcheckvsize\clearpage}

{\newpage\clearpage
\lthtmlfigureA{example1755}%
\begin{example}
  \item[RD 1 m92\hfill]{reads /vista/ccd/m92.fits  to buffer 1.}
  \item[RD 7 demo/m92\hfill]{reads demo/m92.fits to buffer 7.}
  \item[RD 3 m92.xyz\hfill]{reads /vista/ccd/m92.xyz  to buffer 3.}
  \item[RD \$N m92\hfill]{reads /vista/ccd/m92.fits  to buffer N, where
       N is a variable.}
  \item[RD 12 HD183543 SPEC\hfill]{reads /vista/spectra/HD183143.fits to
buffer 5}
\end{example}%
\lthtmlfigureZ
\lthtmlcheckvsize\clearpage}

\stepcounter{section}
{\newpage\clearpage
\lthtmlfigureA{command1767}%
\begin{command}
  \item[\textbf{Form: } WD source filename {[FULL]} {[ZERO=z
       SCALE=s]} {[SPEC]} {[WFPC]} {[DOM]} {[SDAS]}\hfill]{}
\par
\item[source]{is the buffer holding the image to be written}
\par
\item[filename]{(character string) is the name of the file into which the
                 image will be written.}
\par
\item[FITS=]{specifies the number of bits to output for FITS integer images.}
\par
\item[FULL]{write the pixels as 32-bit floating point numbers rather than
       16-bit integers.}
\par
\item[ZERO=z]{Adjusts the zero level of the image before writing.}
\par
\item[SCALE=s]{Adjusts the range of the image before writing.}
\par
\item[SPEC]{write the image to the spectrum directory instead of the
       image directory (archaic)}
\par
\item[WFPC,DOM,SDAS]{Specifies that the file be written in WF/PC team format}
\end{command}%
\lthtmlfigureZ
\lthtmlcheckvsize\clearpage}

{\newpage\clearpage
\lthtmlfigureA{hanging1785}%
\begin{hanging}
  \item{true = BZERO + data*BSCALE}
\end{hanging}%
\lthtmlfigureZ
\lthtmlcheckvsize\clearpage}

{\newpage\clearpage
\lthtmlfigureA{hanging1788}%
\begin{hanging}
  \item{data = (true - ZERO ) / SCALE}
\end{hanging}%
\lthtmlfigureZ
\lthtmlcheckvsize\clearpage}

{\newpage\clearpage
\lthtmlfigureA{example1791}%
\begin{example}
  \item[WD 1 m92\hfill]{writes buffer 1 to /vista/ccd/m92.fits.}
  \item[WD 7 demo/m92\hfill]{writes buffer 7 to demo/m92.CCD.}
  \item[WD 3 m92.xyz\hfill]{writes buffer 3 to /vista/ccd/m92.xyz.}
  \item[WD \$N m92\hfill]{writes buffer N to /vista/ccd/m92.CCD, where
        N is a variable.}
  \item[WD 5 mydir/junk SPEC\hfill]{writes buffer 5 to mydir/junk.fits}
  \item[WD 4 ./junk FULL\hfill]{writes buffer 4 to ./junk.fits, using
        full 32-bit precision.  The disk file will be twice as large this way.}
  \item[WD 3 file SPEC\hfill]{writes buffer 3 to /vista/spectra/file.fits}
\end{example}%
\lthtmlfigureZ
\lthtmlcheckvsize\clearpage}

\stepcounter{section}
\stepcounter{section}
\stepcounter{section}
{\newpage\clearpage
\lthtmlfigureA{command1835}%
\begin{command}
  \item[\textbf{Form: } COPY dest source {[BOX=n]}\hfill]{}
  \item[dest]{is the buffer where the new image will be stored.}
  \item[source]{is the original copy.}
  \item[BOX=n]{tells the program to copy only the part of the
       image that is in box 'n'.}
\end{command}%
\lthtmlfigureZ
\lthtmlcheckvsize\clearpage}

{\newpage\clearpage
\lthtmlfigureA{hanging1843}%
\begin{hanging}
  \item{COPY into (dest) from image (source).}
\end{hanging}%
\lthtmlfigureZ
\lthtmlcheckvsize\clearpage}

{\newpage\clearpage
\lthtmlfigureA{example1847}%
\begin{example}
  \item[COPY 2 1\hfill]{copies the image in buffer 1 to
       buffer 2, along with buffer 1's header
       and label.}
  \item[COPY \$B \$A\hfill]{copies the image image in buffer A to 
       buffer B. A are B are variables.}
  \item[COPY 1 2 BOX=7\hfill]{copies the segment of image 2 that is
       in box 7 into image buffer 1.  The
       size of the new image will be the size
       of the box.  See BOX for more details.}
\end{example}%
\lthtmlfigureZ
\lthtmlcheckvsize\clearpage}

\stepcounter{section}
{\newpage\clearpage
\lthtmlfigureA{command1857}%
\begin{command}
  \item[\textbf{Form: } BUFFERS {[bufs]} {[FULL]} {[FITS{[=param]}]}
       (redirection)\hfill]{}
  \item[bufs]{ are integers which specify which buffers are to be displayed
       by the program}
  \item[FULL]{produce a long listing of the buffer information}
  \item[FITS]{list the FITS parameters for the image}
  \item[FITS=param]{list the individual fits parameter 'param'}
\end{command}%
\lthtmlfigureZ
\lthtmlcheckvsize\clearpage}

{\newpage\clearpage
\lthtmlfigureA{example1869}%
\begin{example}
  \item[BUFFERS\hfill]{Give a brief list of all buffers containing images.}
  \item[BUFFERS FULL\hfill]{Give a long listing of the image buffers.}
  \item[BUFFERS 3 4 8 FULL\hfill]{Give a long listing of buffers 3, 4, and 8.}
  \item[BUFFERS 3 FULL FITS\hfill]{List all the individual FITS header 
       parameters and values for buffer 3.}
  \item[BUFFERS 3 FITS=AIRMASS\hfill]{Prints the FITS card beginning with
       'AIRMASS' (if it exists).}
\end{example}%
\lthtmlfigureZ
\lthtmlcheckvsize\clearpage}

\stepcounter{section}
{\newpage\clearpage
\lthtmlfigureA{command1881}%
\begin{command}
  \item[\textbf{Form: } DISPOSE {[ALL]} {[buf]} {[buf2]} {[buf3]}
       {[...]} \hfill]{}
  \item[ALL]{delete the contents of all image buffers.}
  \item[buf]{is the image buffer to be deleted.}
  \item[buf2]{is another image buffer to be deleted ...}
\end{command}%
\lthtmlfigureZ
\lthtmlcheckvsize\clearpage}

{\newpage\clearpage
\lthtmlfigureA{example1894}%
\begin{example}
  \item[DISPOSE 7\hfill]{deletes image 7}
  \item[DISPOSE 1 3 5 7 9\hfill]{deletes images 1, 3, 5, 7, and 9.}
  \item[DISPOSE \$Q\hfill]{deletes image Q, where Q is a variable.
        (This form is helpful in procedures.)}
  \item[DISPOSE ALL\hfill]{deletes all images and spectra.}
\end{example}%
\lthtmlfigureZ
\lthtmlcheckvsize\clearpage}

\stepcounter{section}
{\newpage\clearpage
\lthtmlfigureA{command1901}%
\begin{command}
  \item[\textbf{Form: } CREATE buf {[BOX=b]} {[SR=sr]} {[SC=sc]} {[NR=nr]} 
       {[NC=nc]} {[CONST=c]} {[N=n]} {[V=v]} {[HEADBUF=oldbuf]} \hfill]{}
  \item[buf]{is the buffer holding the new image}
  \item[BOX=b]{create an image with the size and orientation of box 'b'}
  \item[SR=sr]{specify the start row of the new image}
  \item[SC=sc]{specify the start column}
  \item[CR=sr]{specify the center row of the new image}
  \item[CC=sc]{specify the center column}
  \item[NR=nr]{specify the number of rows}
  \item[NC=nc]{specify the number of columns}
  \item[N=n]{specifies number of rows and columns (square image)}
  \item[V=n]{loads center row and column for new image using
       coordinates from VISTA variables Rn and Cn}
  \item[CONST=c]{fill the image pixels with value 'c'}
  \item[HEADBUF=oldbuf]{fill the new header with all cards from buffer 'oldbuf'}
\end{command}%
\lthtmlfigureZ
\lthtmlcheckvsize\clearpage}

{\newpage\clearpage
\lthtmlfigureA{example1928}%
\begin{example}
  \item[CREATE 1 BOX=5 \hfill]{creates an image in buffer 1 having 
       the size and orientation of box 5.  the image is filled with zeroes.}
  \item[CREATE 1 BOX=5 CONST=100.0\hfill]{does the same as the first example, 
       but fills the image with value 100.0}
  \item[CREATE 5 SR=5 SC=10 NR=25 NC=35\hfill]{creates an image in buffer 5.  
       The start (row, column) is (5,10) and the size of the image is 25 rows by
       35 columns.}
  \item[CREATE 1 N=100\hfill]{creates an 100 by 100 image in buffer 1.
       The start row and column are both 0.}
\end{example}%
\lthtmlfigureZ
\lthtmlcheckvsize\clearpage}

\stepcounter{section}
{\newpage\clearpage
\lthtmlfigureA{command1938}%
\begin{command} 
  \item[\textbf{Form: } CHANGE buf 'new\_name'\hfill]{}
  \item[buf]{is the number of the buffer holding the 
       image which is having its name changed.}
  \item[new\_name]{is the new label.}
\end{command}%
\lthtmlfigureZ
\lthtmlcheckvsize\clearpage}

{\newpage\clearpage
\lthtmlfigureA{example1945}%
\begin{example}
  \item[CHANGE 1 HD183143\hfill]{changes the name of image 1 to 'HD183143'. }
  \item[CHANGE 7 'Test Image 2'\hfill]{changes the name of image 7 to 
       'Test Image 2'.}
  \item[CHANGE \$R 'new Label'\hfill]{changes the name of image R (where R
       is a variable) to 'new Label'.}
  \item[CHANGE 1\hfill]{changes the name of image 1.  The old 
       name is printed, and the program asks you for the new name.}
\end{example}%
\lthtmlfigureZ
\lthtmlcheckvsize\clearpage}

\stepcounter{section}
{\newpage\clearpage
\lthtmlfigureA{command1956}%
\begin{command}
  \item[\textbf{Form: } FITS buf {[FLOAT=name]} float\_value\hfill]{}
  \item[FITS buf {[INT=name]} integer\_value\hfill]{}
  \item[FITS buf {[CHAR=name]} 'character string'\hfill]{}
  \item[FITS PROF {[FLOAT=name float]} {[INT=name in]} {[CHAR=name char]}\hfill]{}
\end{command}%
\lthtmlfigureZ
\lthtmlcheckvsize\clearpage}

{\newpage\clearpage
\lthtmlfigureA{example1970}%
\begin{example}
  \item[FITS 1 FLOAT=AIRMASS 1.3\hfill]{Inserts a FITS header card named
        AIRMASS with value 1.3 into the header of the image in buffer 1.}
  \item[FITS 3 CHAR=COMMENT 'My best observation ever'\hfill]{
        Inserts a new COMMENT card into he header of the image in buffer 3.}
\end{example}%
\lthtmlfigureZ
\lthtmlcheckvsize\clearpage}

\stepcounter{section}
{\newpage\clearpage
\lthtmlfigureA{command1977}%
\begin{command}
  \item[\textbf{Form: } UNFIT buf {[CARD=name]} {[PROF]}\hfill]{}
\end{command}%
\lthtmlfigureZ
\lthtmlcheckvsize\clearpage}

\stepcounter{section}
{\newpage\clearpage
\lthtmlfigureA{command1986}%
\begin{command} 
  \item[\textbf{Form: } HEDIT buf\hfill]{}
  \item[buf]{is the number of the buffer holding the 
image that is having its FITS header edited.}
\end{command}%
\lthtmlfigureZ
\lthtmlcheckvsize\clearpage}

\stepcounter{section}
{\newpage\clearpage
\lthtmlfigureA{command1994}%
\begin{command} 
  \item[\textbf{Form: } FIXHEAD imbuf {[ORIGIN]} {[RORIGIN]} 
       {[CORIGIN]}\hfill]{}
  \item{{[WFPC]} {[GROUP]} {[BLANK]} }
  \item[imbuf]{image buffer with the FITS header to be fixed}
  \item[ORIGIN]{reset the FITS coordinate system cards.}
  \item[RORIGIN]{reset the FITS coordinates along rows only.}
  \item[CORIGIN]{reset the FITS coordinates along columns only.}
  \item[WFPC]{remove WF/PC1 image header cards}
  \item[GROUP]{remove all FITS group cards}
  \item[BLANK]{remove any blank cards from the header}
\end{command}%
\lthtmlfigureZ
\lthtmlcheckvsize\clearpage}

\stepcounter{chapter}
\stepcounter{section}
{\newpage\clearpage
\lthtmlfigureA{example2200}%
\begin{example}
  \item[OPEN]{open an ASCII file}
  \item[CLOSE]{close an Open ASCII file}
  \item[READ]{read the next line of an ASCII file}
  \item[SKIP]{designate lines to skip in a file}
  \item[REWIND]{rewind an open ASCII file to the first line}
  \item[STAT]{compute statistics on data in an ASCII file}
\end{example}%
\lthtmlfigureZ
\lthtmlcheckvsize\clearpage}

\stepcounter{section}
{\newpage\clearpage
\lthtmlfigureA{command2211}%
\begin{command}
  \item[\textbf{Form: } SAVE data\_keyword=filename {[LOW=lowbad]} 
       {[HIGH=highbad]}\hfill]{}
  \item[GET  data\_keyword=filename\hfill]{}
  \item[data\_keyword]{is a word specifying the type of data
       being saved or get (APER, MASK, PHOT, COO or DAO, or PROFILE),}
  \item[filename]{is the name of the file that is holding
       the data (GET) or will hold the data (SAVE)}
\end{command}%
\lthtmlfigureZ
\lthtmlcheckvsize\clearpage}

{\newpage\clearpage
\lthtmlfigureA{command2220}%
\begin{command}
  \item[Keyword:\hfill]{}
  \item[APER=name]{Aperture photometry file 'name(.APR)'}
  \item[MASK=name]{Photometry mask file 'name(.MSK)'}
  \item[PHOT=name]{Stellar photometry file 'name(.PHO)'}
  \item[COO= or DAO=]{Reads DAOPHOT style ASCII file (\# x y ) into
       internal photometry file}
  \item[PROF=name]{Surface photometry file 'name(.PRF)'}
  \item[LINEID=name]{Saves the LINEID identifications.}
\end{command}%
\lthtmlfigureZ
\lthtmlcheckvsize\clearpage}

{\newpage\clearpage
\lthtmlfigureA{example2229}%
\begin{example}
  \item[GET PHOT=ORION\hfill]{loads the photometry file
       vista/data/ORION.PHO to VISTA}
  \item[SAVE MASK=MASK5\hfill]{writes the VISTA mask file
       to vista/data/MASK5.MSK}
\end{example}%
\lthtmlfigureZ
\lthtmlcheckvsize\clearpage}

\stepcounter{section}
{\newpage\clearpage
\lthtmlfigureA{command2246}%
\begin{command}
  \item[\textbf{Form: } PRINT {[data keywords]} {[output redirection]}\hfill]
  \item[data keywords]{specify which information is printed}
\end{command}%
\lthtmlfigureZ
\lthtmlcheckvsize\clearpage}

{\newpage\clearpage
\lthtmlfigureA{example2252}%
\begin{example}
  \item[object {[BOX=n]}\hfill]{ Print out the pixel values of object (in
       the subsection specified by box 'n').  If the object is a
       wavelength- calibrated spectrum, the wavelengths will be printed.
  \begin{hanging}
    \item{PRINT 1 BOX=2}
    \item{PRINT \$Q BOX=4 $>$imagesec.dat}
  \end{hanging}
}
\par
\item[BOXES\hfill]{Print out the sizes, centers, and origins of all boxes
       defined.
  \begin{hanging}
    \item{PRINT BOXES}
  \end{hanging}
}
\par
\item[PROF {[MEDIAN]} {[MAG]} {[SPIRAL]}\hfill]{ Print out the surface
       photometry profile.  If the MEDIAN keyword is specified, the median
       surface brightnesses, rather than the mean, will be printed. If the
       MAG keyword is specified, the surface brightnesses and total counts
       will be output in magnitude units. If the SPIRAL keyword is given
       (for SPIRAL galaxies), then some extra information, like the disk
       scale length, disk/tot, etc. will be printed - if these values have
       been computed using CORRECT.
  \begin{hanging}
    \item{PRINT PROFILE}
    \item{PRINT PROFILE $>$prof.dat}
  \end{hanging}
}
\par
\item[PHOT {[BRIEF]}\hfill]{ Print out the stellar photometry results.
       BRIEF produces a short listing.
  \begin{hanging}
    \item{PRINT PHOT }
    \item{PRINT PHOT BRIEF}
    \item{PRINT PHOT $>$photlist.dat}
  \end{hanging}
}
\par
\item[APER\hfill]{ Print out the aperture photometry results.
  \begin{hanging}
    \item{PRINT APER}
    \item{PRINT APER $>>$aperlist.dat}
  \end{hanging}
}
\par
\item[DIRECTORIES\hfill]{Print the default directories.
  \begin{hanging}
    \item{PRINT DIRECTORIES}
    \item{PRINT DIRECTORIES $>$dirs.dat}
  \end{hanging}
}
\par
\item[IMAGES or SPECTRA\hfill]{Print headers for images or spectra in the
       default directory.  (Use PRINT DIRECTORIES to see which directory
       this is)
  \begin{hanging}
    \item{PRINT IMAGES}
    \item{PRINT SPECTRA $>$imlist.dat}
  \end{hanging}
}
\par
\item[LINEID\hfill]{Print wavelength v. pixel identifications obtained
       with the LINEID command.
  \begin{hanging}
    \item{PRINT LINEID}
    \item{PRINT LINEID $>$lines.dat}
  \end{hanging}
}
\par
\item[STRINGS\hfill]{Print all defines strings.  See the command STRING
       to define strings.
  \begin{hanging}
    \item{PRINT STRINGS}
    \item{PRINT STRINGS $>$string.txt}
  \end{hanging}
}
\end{example}%
\lthtmlfigureZ
\lthtmlcheckvsize\clearpage}

\stepcounter{section}
{\newpage\clearpage
\lthtmlfigureA{command2297}%
\begin{command}
  \item[\textbf{Form: } OPEN logical\_name file\_name\hfill]{}
  \item[logical\_name]{Is the name you assign to the file}
  \item[file\_name]{Is the disk file name.  No default VISTA directories or
       extensions are applied to the name before it is opened.}
\end{command}%
\lthtmlfigureZ
\lthtmlcheckvsize\clearpage}

{\newpage\clearpage
\lthtmlfigureA{example2303}%
\begin{example}
  \item[OPEN DATA ./mydatafile.dat\hfill]{ Opens the file for reading
       and assigns it the logical name DATA.}
\end{example}%
\lthtmlfigureZ
\lthtmlcheckvsize\clearpage}

\stepcounter{section}
{\newpage\clearpage
\lthtmlfigureA{command2310}%
\begin{command}
  \item[\textbf{Form: }CLOSE logical\_name\hfill]{}
  \item[logical\_name]{is the name of a file previously
       opened for reading by the OPEN command.}
\end{command}%
\lthtmlfigureZ
\lthtmlcheckvsize\clearpage}

\stepcounter{section}
{\newpage\clearpage
\lthtmlfigureA{command2317}%
\begin{command}
  \item[\textbf{Form: }READ logical\_name\hfill]{}
  \item[logical\_name]{is the name of a file previously
       opened for reading by the OPEN command.}
\end{command}%
\lthtmlfigureZ
\lthtmlcheckvsize\clearpage}

\stepcounter{section}
{\newpage\clearpage
\lthtmlfigureA{command2335}%
\begin{command}
  \item[\textbf{Form: }SKIP logical\_name line line1,line2 ...\hfill]{}
  \item[logical\_name]{is the name of a file opened with the OPEN command.}
  \item[line]{is an arithmetic expression specifying one line in the file
       to skip.}  
  \item[line1,line2]{are two arithmetic expressions giving a range of lines
       to skip.}
\end{command}%
\lthtmlfigureZ
\lthtmlcheckvsize\clearpage}

{\newpage\clearpage
\lthtmlfigureA{example2343}%
\begin{example}
  \item[SKIP PHOT 1]{Marks line 1 of file PHOT to be skipped.}
  \item[SKIP PHOT 100,120]{Marks the range of lines from 100 to 120 to 
       be skipped.}
  \item[SKIP PHOT 1 100,120]{Does both.}
  \item[SKIP PHOT]{Prints the skip table for PHOT.}
\end{example}%
\lthtmlfigureZ
\lthtmlcheckvsize\clearpage}

\stepcounter{section}
{\newpage\clearpage
\lthtmlfigureA{command2351}%
\begin{command}
  \item[\textbf{Form: }REWIND logical\_name\hfill]{}
  \item[logical\_name]{is the name of an OPEN'ed file.}
\end{command}%
\lthtmlfigureZ
\lthtmlcheckvsize\clearpage}

\stepcounter{section}
{\newpage\clearpage
\lthtmlfigureA{command2358}%
\begin{command}
  \item[\textbf{Form: } STAT variable=function{[expression]}\hfill]{}
  \item[variable]{is a VISTA math variable in which the
       value of the statistic is stored.}
  \item[expression]{is an arithmetic expression which involves
       at least one reference to data in an OPEN'ed ASCII file.}
  \item[function]{is one of the following:}
  \item{MAX :Find the maximum value of the expression.}
  \item{MIN :Find the minimum value of the expression.}
  \item{FIRST :Finds the first value of the expression.}
  \item{LAST :Find the last value of the expression.}
  \item{COUNT :Counts the number of lines in the file.
        In this case 'expression' is a logical file name.}
  \item{LOAD   :  Loads the arithmetic expression from each
       line in the input file into a specified buffer
       using STAT N=LOAD{[buffer,expression]}}
\end{command}%
\lthtmlfigureZ
\lthtmlcheckvsize\clearpage}

{\newpage\clearpage
\lthtmlfigureA{example2373}%
\begin{example}
  \item[STAT LINES=COUNT{[DATAFILE]}\hfill]{ Set the variable LINES to the
       number of lines in the file DATAFILE.  DATAFILE must have been
       opened with the OPEN command.  SKIP'ed lines are not counted.}
\par
\item[STAT MAXVAL=MAX{[2.5*LOG10{[@PHOT.2]}]}\hfill]{ Evaluates the
       expression 2.5*LOG10{[@PHOT.2]} for each line in the file PHOT and
       sets MAXVAL to have the maximum value.  The file PHOT will be left
       repositioned to the beginning of the file after the STAT command
       completes.}
\par
\item[STAT N=LOAD{[1,@PHOT.1*@PHOT.2]}\hfill]{ Loads the product of the
       values in the first and second columns of the input file PHOT into
       VISTA buffer number 1.}
\end{example}%
\lthtmlfigureZ
\lthtmlcheckvsize\clearpage}

\stepcounter{chapter}
{\newpage\clearpage
\lthtmlfigureA{example2510}%
\begin{example} 
  \item[TV\hfill]{Load and image into the display}
  \item[IMPOST\hfill]{Make a PostScript hardcopy of an image}
  \item[TVRPLOT\hfill]{Make a Radial Intensity Plot of an object on the 
       TV Display}
  \item[TVBOX\hfill]{Overlay a VISTA box on the image display.}
  \item[TVPLOT \hfill]{Draw stuff like lines, boxes, text, and so on onto the 
       TV display}
  \item[PLOT\hfill]{Plot a row, column or spectrum}
  \item[RPLOT\hfill]{Make a radial intensity plot of an object in an image}
  \item[CONTOUR\hfill]{Make a contour plot of an image.}
  \item[PLOT3D\hfill]{Make a 3-d perspective (mesh) plot}
  \item[OVERLAY\hfill]{Draw contours over a greyscale image (PostScript)}
  \item[HISTOGRAM\hfill]{Display a Histogram of Image Pixel Values}
  \item[TEXT\hfill]{Embed Permanent Text on an Image}
\end{example}%
\lthtmlfigureZ
\lthtmlcheckvsize\clearpage}

\stepcounter{section}
{\newpage\clearpage
\lthtmlfigureA{example2533}%
\begin{example} 
  \item[TV\hfill]{Load and image into the display}
  \item[COLOR\hfill]{Reload the color map or define a new one.}
  \item[ITV\hfill]{Synchronize procedures and display interaction, mark
       features.}
  \item[TVBOX\hfill]{Overlay a VISTA box on the image display.}
  \item[TVPLOT \hfill]{Draw stuff like lines, boxes, text, and so on onto the 
       TV display}
\end{example}%
\lthtmlfigureZ
\lthtmlcheckvsize\clearpage}

{\newpage\clearpage
\lthtmlfigureA{example2553}%
\begin{example}
  \item[LOW CONTRAST]{Hold down the LEFT Mouse button, drag the left
       end of the color bar.}
\par
\item[HIGH CONTRAST]{Hold down the RIGHT Mouse button, drag the right
       end of the color bar.}
\par
\item[ROLL COLOR MAP]{Hold the MIDDLE Mouse button, "roll" the
       color bar left or right.}
\end{example}%
\lthtmlfigureZ
\lthtmlcheckvsize\clearpage}

\stepcounter{section}
{\newpage\clearpage
\lthtmlfigureA{command2574}%
\begin{command}
  \item[\textbf{Form: }TV buf {[span]} {[zero]} {[L=span]} {[Z=zero]} 
       {[BOX=n]} {[CF=xxx]} {[NOERASE]} {[BW]}\hfill]{}
  \item[{[CLIP]} {[FLIP]} {[OLD]} {[NCOLOR=]}\hfill]{}
  \item[buf]{is the image to be displayed,}
  \item[span or L=]{set the span level for the color map,}
  \item[zero or Z=]{set the zero level,}
  \item[OLD]{use zero and span levels from previous display}
  \item[BOX=n]{displays the part of the image in box 'n',}
  \item[CF]{specifies the color map,}
  \item[NOERASE]{prevents the TV from erasing the previous image
       when displaying a new one (used for blinking),}
  \item[BW]{displays the image in black and white.}
  \item[CLIP]{prevents roll-over of the color map.}
  \item[FLIP]{display origin in lower left rather than upper left}
  \item[NCOLOR=]{limit the number of colors displayed (X11)}
\end{command}%
\lthtmlfigureZ
\lthtmlcheckvsize\clearpage}

{\newpage\clearpage
\lthtmlfigureA{example2602}%
\begin{example}
  \item[CF=BW\hfill]{black and white}
  \item[CF=IBW\hfill]{inverse black and white}
  \item[CF=RAIN\hfill]{the color distribution in a rainbow}
  \item[CF=WRMB\hfill]{thought by many to be the most useful}
\end{example}%
\lthtmlfigureZ
\lthtmlcheckvsize\clearpage}

{\newpage\clearpage
\lthtmlinlinemathA{tex2html_wrap_inline3413}%
$+$%
\lthtmlinlinemathZ
\lthtmlcheckvsize\clearpage}

{\newpage\clearpage
\lthtmlinlinemathA{tex2html_wrap_inline3415}%
$-$%
\lthtmlinlinemathZ
\lthtmlcheckvsize\clearpage}

{\newpage\clearpage
\lthtmlinlinemathA{tex2html_wrap_inline3417}%
$+/-$%
\lthtmlinlinemathZ
\lthtmlcheckvsize\clearpage}

{\newpage\clearpage
\lthtmlfigureA{example2612}%
\begin{example}
  \item[{MN 1\hfill}]{}
  \item[{TV 1 CF=RAIN\hfill}]{loads image 1 into the display, setting the
       span to be 4 times the image mean.}
\end{example}%
\lthtmlfigureZ
\lthtmlcheckvsize\clearpage}

{\newpage\clearpage
\lthtmlfigureA{example2616}%
\begin{example}
  \item[{TV 3 100. 30.0\hfill}]{loads image 3 into the display, setting the
       span to be 100 and the zero to be 30.}
\end{example}%
\lthtmlfigureZ
\lthtmlcheckvsize\clearpage}

{\newpage\clearpage
\lthtmlfigureA{example2619}%
\begin{example}
  \item[{TV 3 L=100. Z=30.\hfill}]{does the same thing as example 2.}
\end{example}%
\lthtmlfigureZ
\lthtmlcheckvsize\clearpage}

{\newpage\clearpage
\lthtmlfigureA{example2622}%
\begin{example}
  \item[{TV 3 BOX=1\hfill}]{Displays that part of the image
       which is in box 1.}
\end{example}%
\lthtmlfigureZ
\lthtmlcheckvsize\clearpage}

\stepcounter{section}
{\newpage\clearpage
\lthtmlfigureA{command2629}%
\begin{command}
  \item[\textbf{Form: }ITV\hfill]{}
\end{command}%
\lthtmlfigureZ
\lthtmlcheckvsize\clearpage}

\stepcounter{section}
{\newpage\clearpage
\lthtmlfigureA{command2651}%
\begin{command}
  \item[\textbf{Form: }COLOR {[CF=filename]} {[BW]} {[INV]}\hfill]{}
  \item[CF=]{loads an already-defined map into the AED,}
  \item[BW]{loads a black and white map, and}
  \item[INV]{inverts the ordering of the map.}
\end{command}%
\lthtmlfigureZ
\lthtmlcheckvsize\clearpage}

{\newpage\clearpage
\lthtmlfigureA{hanging2661}%
\begin{hanging}
  \item{COLOR BW}
\end{hanging}%
\lthtmlfigureZ
\lthtmlcheckvsize\clearpage}

{\newpage\clearpage
\lthtmlfigureA{hanging2664}%
\begin{hanging}
  \item{COLOR INV}
\end{hanging}%
\lthtmlfigureZ
\lthtmlcheckvsize\clearpage}

{\newpage\clearpage
\lthtmlfigureA{example2667}%
\begin{example}
  \item[COLOR CF=WRMB\hfill]{looks for WRMB.CLR in the color directory}
  \item[COLOR CF=WRMB.XXX\hfill]{looks for WRMB.XXX in the color directory}
  \item[COLOR CF=./WRMB\hfill]{looks for ./WRMB.CLR}
\end{example}%
\lthtmlfigureZ
\lthtmlcheckvsize\clearpage}

\stepcounter{section}
{\newpage\clearpage
\lthtmlfigureA{command2677}%
\begin{command}
  \item[\textbf{Form: } TVBOX {[BOX=b]} {[SIZE=s PIX=r,c]}\hfill]{}
  \item[BOX=b]{tells the program to show the bounds
       of box 'b' on the television.}
  \item[SIZE=s PIX=r,c]{draws a box of size 's' centered at row 
       'r' and column 'c'}
\end{command}%
\lthtmlfigureZ
\lthtmlcheckvsize\clearpage}

{\newpage\clearpage
\lthtmlfigureA{example2686}%
\begin{example}
  \item[TVBOX BOX=3\hfill]{display the boundaries of box 3
       on the television.}
  \item[TVBOX SIZE=51 PIX=123,234\hfill]{display a box of size 51
       pixels, centered at row 123 and column 234.}
\end{example}%
\lthtmlfigureZ
\lthtmlcheckvsize\clearpage}

\stepcounter{section}
{\newpage\clearpage
\lthtmlfigureA{command2692}%
\begin{command}
  \item[\textbf{Form: }TVPLOT {[with keywords below as needed]}\hfill]{}
  \item[P=(r,c)]{         Enter a position (use twice as needed)}
  \item[L=l]{             Enter line length}
  \item[C=(r,c)]{         Enter center (row, column)}
  \item[TEXT=s]{          Plot text in quotes or string variable}
  \item[BOX=n]{           Plot image subsection box number n}
  \item[BOX]{             Plot boxes other than predefined image boxes}
  \item[W=w]{             Enter width of box}
  \item[H=h]{             Enter height of box}
  \item[CROSS]{           Plot cross}
  \item[CIRC=rad]{        Plot a circle}
  \item[COMPASS=rad]{     Plot a compass}
  \item[TICKS]{           Put ticks marks on line}
  \item[SCALE=s]{         Use pixel scale for all lengths}
  \item[PA=f]{            Position angle}
  \item[OFF=d]{           Offset of graphics from center}
\end{command}%
\lthtmlfigureZ
\lthtmlcheckvsize\clearpage}

{\newpage\clearpage
\lthtmlfigureA{example2713}%
\begin{example}
  \item[AXES 1 BOX=7\hfill]{Find the center of the object within
       the BOX 7 subsection of image 1.}
  \item[TVPLOT BOX=7\hfill]{Plot this same box on the TV.}
  \item[TVPLOT CIRC=10\hfill]{Draw a 10-pixel radius circle around it.}
  \item[TVPLOT CROSS\hfill]{Mark it with a cross.}
  \item[TVPLOT COMPASS=20\hfill]{a really cool map compass}
  \item[STRING BIGDEAL  'This is the object in BOX 7']{}
  \item[TVPLOT TEXT=BIGDEAL\hfill]{Label the stupid thing.}
\end{example}%
\lthtmlfigureZ
\lthtmlcheckvsize\clearpage}

{\newpage\clearpage
\lthtmlfigureA{example2722}%
\begin{example}
  \item[TVPLOT P=(101,237)\hfill]{This will draw a line from where ever
       the center is to row=101, col=237.}
  \item[TVPLOT P=(r1,c1) P=(r2,c2)\hfill]{OK, this is your standard line
       between any two points.}
  \item[TVPLOT BOX P=(r1,c1) P=(r2,c2)\hfill]{Makes a box instead of a line.}
  \item[TVPLOT TEXT=BIGDEAL P=(r,c)\hfill]{Stick a label anywhere.}
\end{example}%
\lthtmlfigureZ
\lthtmlcheckvsize\clearpage}

\stepcounter{section}
{\newpage\clearpage
\lthtmlfigureA{command2732}%
\begin{command}
  \item[\textbf{Form: } IMPOST imbuf {[BOX=b]} {[Z=zero]} {[L=span]} {[CLIP]} 
       {[POSITIVE]} {[TITLE]}\hfill]{}
  \item[{[HIST=xxx]} {[LAND]} {[FOUR]} {[AXES]} {[COMMENT]} {[COMMENT=xxx]}
        {[FILE=xxx]} {[OUT=xxx]}]{}
  \item[{[SCALE=s]} {[CEN=r,c]} {[FLIP]} {[BAR=xxx]} {[NOBAR]} {[WIND=w,h]} 
        {[ORIGIN=x,y]} {[PAGE=L,S]}]{}
  \item[{[COPIES=n]} {[LARGE]} {[INFO]} {[INT]} {[MACRO=]} {[EPS]}]{}
  \item[imbuf]{The buffer containing the image to be printed.}
  \item[BOX=b]{Only print the region within the given box.}
  \item[Z=zero ]{Zero point of the intensity mapping.}
  \item[L=span ]{Span of the intensity mapping.  }
  \item[CLIP]{Prevent roll-over of the intensity mapping.}
  \item[POSITIVE ]{Print the image intensities as White-on-Black.
                   The default mapping is Black-on-White.}
  \item[TITLE ]{Write the OBJECT name on the print (from the header).}
  \item[HIST ]{Map image intensities using flat histogram equalization.}
  \item[HIST=xxx]{Map image intensities using other histogram scalings
                  Options: (FLAT,SQRT,LOG).  "HIST" = "HIST=FLAT"}
  \item[LAND ]{Output will appear in LANDSCAPE mode}
  \item[FOUR ]{Use a 4 bit (16 level) gray scale rather than the
               default 8 bit (256 level) gray scale.}
  \item[AXES ]{Draw coordinates axes around the image.}
  \item[COMMENT='xxx']{Add a comment line to the print.  Comments may be
                       up to 64 characters long. }
  \item[COMMENT ]{Prompt the user interactively for the comment line.}
  \item[FILE=psfile]{Direct the output into a file named "psfile".}
  \item[OUT=psfile]{(same as FILE=psfile)}
  \item[SCALE=s ]{specify the pixel scale in units/pixel.}
  \item[CEN=(r,c)]{specify the image fiducial center in pixels for
                  scaled coordinate axes (see SCALE=).}
  \item[FLIP ]{flips image in rows (bottom-to-top raster order).}
  \item[BAR='xxx' ]{Label the gray scale intensity bar with the string
                    given.  Default bar label is "Intensity"}
  \item[NOBAR ]{Do not draw a gray scale intensity bar on the print.}
  \item[INFO ]{Write in a line of auxiliary info on the print.}
  \item[INT  ]{Puts user into interactive Mongo mode for labelling.}
  \item[MACRO=macrofile]{ Executes Mongo macro file macrofile after drawing}
  \item[Advanced Page Control\hfill]{}
  \item[EPS]{Make an Encapsulated PostScript file.}
  \item[COPIES=n ]{Specify the number of copies to print if more than 1.}
  \item[WIND=(w,h)]{Specify the maximum size in inches of the window in 
                    which IMPOST will try to print the image.}
  \item[LARGE ]{Alternative to WIND=, will make the window as large as 
                possible for the given paper size and orientation.}
  \item[ORIGIN=(x,y)]{Specify the origin of the plotting window in inches 
                      from the lower left-hand corner of the paper.}
  \item[PAGE=(L,S) ]{specify the physical size of the paper in inches.
                     "L" = long dimension, "S" = short dimension.}
\end{command}%
\lthtmlfigureZ
\lthtmlcheckvsize\clearpage}

{\newpage\clearpage
\lthtmlfigureA{hanging2865}%
\begin{hanging}
  \item{          INTENS(r,c) = {[IMG(r,c) - ZERO]}*(255/SPAN)}
\end{hanging}%
\lthtmlfigureZ
\lthtmlcheckvsize\clearpage}

{\newpage\clearpage
\lthtmlfigureA{example2869}%
\begin{example}
  \item[TITLE]{puts the object name of the image (stored in the header in
          the OBJECT FITS card).  The title appears above the upper
          left-hand corner of the image in large type.  Standard 
          Mongo escape sequences can be used to print special characters
          (like a greek alpha), or super- and sub-scripts.}
  \item[COMMENT]{appends a line of text below the image title in a slightly
          smaller font.  Comments may be specified on the command line
          by using the COMMENT='comment string' keyword.  You may also
          have IMPOST prompt the terminal for the comment by using
          the COMMENT keyword without an ='s sign.  Comments may be 
          single or multiple words up to 64 characters long.  Note that
          Mongo special characters may also be used, in addition to 
          super- and sub-scripts, but the extra escape sequence characters
          count against the 64 maximum.}
  \item[AXES]{In its simplest form, AXES will draw coordinate axes around
          the image and label the axes in pixel units.  Using the
          SCALE= and/or CEN= keywords (see below) will allow you
          to scale the axes in arcseconds.}
  \item[SCALE= ]{Sets the pixel scale in arcseconds/pixel for the AXES 
          keyword (above).  When used with AXES, the center of
          the image is taken to be the origin (0,0), and the axes
          are labeled accordingly.  To set a particular pixel to be
          the center, use the CEN= keyword.}
  \item[CEN=]{Sets the origin of the image to be the (r,c) location given.
          Two (2) arguments must be given, and need not lie inside the
          region actually plotted (useful if you wish to show a zoom of
          a region in a larger image while still preserving the original
          coordinate system).  If CEN is not used with SCALE, then the
          center of the region displayed is assumed to be the origin.
          (r,c) may be fractional pixel locations (say the centroid of
          a galaxy or planetary nebula central star).}
  \item[NOBAR]{Tells IMPOST not to draw in the gray intensity scale bar
          on the image. }
  \item[BAR=]{By default, IMPOST will label the gray intensity scale
          bar with the word "Intensity", but you may set it to 
          anything you like with the BAR= keyword.  For example,
          BAR='Log H\\ga Flux' will label the intensity bar as
          indicating the logarithm of the H-alpha flux.  Any valid
          Mongo special character escape sequence (including 
          super- and sub-scripts is allowed).}
  \item[INFO]{Puts a line of auxiliary information across the bottom of
          the print.  This provides a hardcopy of the parameters
          that went into the picture (zero point, span, clipping, etc).}
  \item[INT]{Puts the user into interactive Mongo after drawing image. The
          limits are the row and column limits. Here you can do your
          own labelling, overlays of contours, etc.}
  \item[MACRO=file]{Executes the Mongo macro file after drawing image. Like
          INT but uses predefined Mongo macro.}
  \item[FLIP]{This keyword "flips" the image orientation in the window
          vertically (about rows) so that the origin of the image is
          in the lower left-hand corner of the page rather than the
          upper left-hand corner as is default for VISTA.  This is
          the orientation used by IRAF and STSDAS.}
\end{example}%
\lthtmlfigureZ
\lthtmlcheckvsize\clearpage}

{\newpage\clearpage
\lthtmlfigureA{example2883}%
\begin{example}
  \item[LAND]{This keyword will cause the print to be oriented with the
         horizontal axis aligned with the long side of the paper 
         (so-called "landscape" mode).  By default, IMPOST prints are 
         made in "portrait" mode.}
  \item[COPIES=n]{If more than one copy is desired, then set the COPIES keyword.
         This option is very economical of processing time, as the
         printer only has to "draw" the image once and then knock off
         as many copies as required.  This is far more efficient than 
         printing the same file N-times.}
  \item[WIND=(w,h)]{This keyword specifies the maximum possible size of the
         plotting window, in inches.  The window is the largest
         region in which IMPOST will try to fit the image.
         By default, the plotting window is:
             Portrait (default):  6" x 6"
               Landscape (LAND):  8" x 5"
         Dimensions are specified as "width","height" in units of inches.
         These dimensions must be smaller than the paper dimensions 
         (see PAGE keyword).}
  \item[LARGE]{An alternative to WIND= is the "LARGE" keyword.  Will attempt
         to make the window as large as possible on the page, leaving only
         a 0.25-inch margin around the edge.  LARGE is equivalent to
            ORIGIN=0.25,0.25  WIND=8.0,10.5  PAGE=11.0,8.5}
  \item[ORIGIN=(dh,dv)]{Specifies the origin of the plotting window
         in inches measured from the lower left-hand corner of the 
         paper.  DH is the horizontal offset, and DV is the vertical
         offset.  By default, IMPOST tries to center the plotting 
         window on the page, and adds a 30pt "binding margin" to 
         Portrait mode prints.  ORIGIN overrides this binding margin.}
  \item[PAGE=(lng,sh)]{This keyword specifies the physical size of the paper in
         inches.  "lng" = long dimension, "sh" = short dimension.
         Note that which is horizontal or vertical depends on
         whether or not the LAND keyword was used.  For example, 
         PAGE=(11,8.5) is equivalent to the IMPOST default, and
         corresponds to the dimensions of standard U.S. letter
         paper.  PAGE should only be used if you have non-standard 
         sized paper in your local PostScript printer and do not
         wish to change the hardcoded values in the IMPOST fortran
         code itself.  European VISTA users should have their
         local custodians hardwire IMPOST to the appropriate local
         paper size.}
\end{example}%
\lthtmlfigureZ
\lthtmlcheckvsize\clearpage}

\stepcounter{section}
{\newpage\clearpage
\lthtmlfigureA{command2897}%
\begin{command}
  \item[\textbf{Form: } PLOT b1 b2 b3 .. {[R=n1,n2]} {[C=n1,n2]}
       {[RS=r1,r2]} {[CS=c1,c2]}\hfill]{}
  \item{{[XS=f]} {[XE=f]} {[MIN=f]} {[MAX=f]} {[XMIN=]} {[XMAX=]} 
        {[YMIN=]} {[YMAX=]}}
  \item{{[SEMILOG]} {[LOG]} {[R4]} {[NH=nh]} {[NV=nv]} {[OLD]}}
  \item{{[HIST]} {[POINTS=psty]} {[USER]} {[PIXEL]} {[GRID]} }
  \item{{[NOLABEL]} {[NOERASE]} {[INFO]} {[PROFILE]} {[MULT=]} }
  \item{{[INT]} {[MACRO=file]} {[HARD]} {[PORT]} {[PSFILE=file]} {[EPS]} 
       {[WAIT]} }
  \item[b1,b2,b3..]{the image buffers which will be plotted (up to 15).}
  \item[BOX=b]{limits the image 'buf' to box 'b'.}
  \item[R=n1,n2, C=n1,n2]{plots the selected row(s) or column(s).}
  \item[RS=r1,r2]{plots sum of selected rows.}
  \item[CS=c1,c2]{plots sum of selected columns.}
  \item[XS, XE]{select limits for x-axis of plot (also XMIN \& XMAX).}
  \item[MIN, MAX]{select limits for y-axis of plot (also YMIN \& YMAX).}
  \item[SEMILOG]{plots log(y) against x.}
  \item[LOG]{plots log(y) against log(x).}
  \item[R4      ]{plot x**0.25 instead of x.}
  \item[NH=nh, NV=nv ]{number of horizontal and vertical windows
       for multiple plots in multiple windows}
  \item[WIND=nx,ny,iw]{sets desired plotting window}
  \item[OLD]{uses parameters from previous plot.}
  \item[HIST]{produces a histogram style (step) plot instead of a line plot.}
  \item[POINTS=psty]{Plot points in the given Mongo style instead of a line
                     plot.}
  \item[USER]{uses axis scaling and labels stored in the FITS header.}
  \item[PIXEL]{force plot in pixels of wavelength calibrated spectra.}
  \item[GRID]{produces a full plot grid.}
  \item[NOLABEL]{suppresses all labelling of the plot.}
  \item[NOERASE]{do not erase the previous plot to allow overlapping
       plots (see WAIT for hardcopy).}
  \item[INFO]{puts additional information about the data on the plot.}
  \item[PROFILE ]{plots results from the PROFILE common block}
  \item[MULT=n ]{plots the nth galaxy from the internal multiple
        nucleus common block}
  \item[INT]{places the user in interactive Mongo after plotting.}
  \item[MACRO=file ]{executes the Mongo macro file 'file' after plotting.}
  \item[HARD]{sends the output to the hardcopy device specified by
       the TERM command.}
  \item[PORT]{make an portrait mode plot (default: landscape).}
  \item[WAIT]{keep the hardcopy channel open so subsequent plots will
              appear on the same image (or PostScript file).}
  \item[PSFILE=file]{directs PostScript output to the specified file,
       rather than sending it to the printer}
  \item[EPS]{make an Encapsulated PostScript file.}
\end{command}%
\lthtmlfigureZ
\lthtmlcheckvsize\clearpage}

{\newpage\clearpage
\lthtmlinlinemathA{tex2html_wrap_inline3419}%
$x^1/4$%
\lthtmlinlinemathZ
\lthtmlcheckvsize\clearpage}

{\newpage\clearpage
\lthtmlfigureA{hanging2966}%
\begin{hanging}
  \item{X = CRVAL1 + CDELT1 * FLOAT(COL - CRPIX1)}
\end{hanging}%
\lthtmlfigureZ
\lthtmlcheckvsize\clearpage}

{\newpage\clearpage
\lthtmlfigureA{hanging2969}%
\begin{hanging}
  \item{X = CRVAL2 + CDELT2 * FLOAT(ROW - CRPIX2)}
\end{hanging}%
\lthtmlfigureZ
\lthtmlcheckvsize\clearpage}

{\newpage\clearpage
\lthtmlfigureA{example2978}%
\begin{example}
  \item[PLOT 1 R=100]{Plots row 100 of image 1.}
  \item[PLOT 1 RS=100,120 MIN=0. MAX=100.]{  Plots sum of rows 100 to
       120 of image 1, with Y-axis running from 0. to 100.}
  \item[PLOT 4 XS=20 XE=45 R=4]{Plots row 4 of image 4,
       where the X-axis runs over columns 20 through 45.}
  \item[PLOT 4 OLD SEMILOG]{Plot the old graph (whatever it was)
       in semi-log format.}
  \item[PLOT 1]{Plots the spectrum in buffer 1.  No
       row or column specifers are necessary.}
  \item[PLOT 4 INT ]{After plotting the spectrum (note that 4
       must be a spectrum as no row or col was
       specified!) enter interactive Mongo mode
       to label and finish the plot.}
  \item[PLOT 3 R=50 HIST]{Draws a plot of row 50 in image 3.
       The plot is d}one in histogram style.
  \item[PLOT 3 MIN=0.0 MAX=1000. HIST]{Does the same as example
       8, but sets the Y-axis to run from 0.0 to 1000.0.}
  \item[PLOT 2 R=50 HARD USER]{Plots R=50 of image 2 on the Imagen laser
       printer and scales and labels the axes with
       the user coordinate info stored in the FITS header of image 2.}
\end{example}%
\lthtmlfigureZ
\lthtmlcheckvsize\clearpage}

\stepcounter{section}
{\newpage\clearpage
\lthtmlfigureA{command2990}%
\begin{command}
  \item[\textbf{Form: } RPLOT imbuf {[RAD=r{[,c]}]} {[CEN=r0,c0]} 
       {[SCALE=s]} {[HARD]}\hfill]{}
  \item{{[MIN=ymin]} {[MAX=ymax]} {[COLOR=c]} {[NOERASE]} {[LOG]}}
  \item[imbuf]{Specify the image buffer.  If no image buffer
       is given, the currently displayed image is assumed.}
  \item[RAD=r{[,c]}]{ Set the radius of the box in pixels in both rows 
        and columns.  A second argument sets a rectangular 
        box with the radius in Rows and Columns separately 
        defined.  (Default: RAD=5)}
  \item[CEN=r0,c0]{ Define the central pixel for the radial intensity
       plot.}
  \item[SCALE=s]{Define the pixel scale in Arcsec/Pixel for the plot
       (does not apply to the RAD= keyword values)}
  \item[HARD]{Make a hardcopy of the radial intensity plot }
  \item[MIN=, MAX=]{ Sets y-axis limits for plot}
  \item[COLOR=]{ Set the color to be used for the plot}
  \item[NOERASE]{ Do not erase between plots}
  \item[LOG]{ Plot intensities on a logarithmic scale}
\end{command}%
\lthtmlfigureZ
\lthtmlcheckvsize\clearpage}

{\newpage\clearpage
\lthtmlfigureA{example3024}%
\begin{example}
  \item[RPLOT 1 RAD=10 CEN=51,25\hfill]{
   Makes a radial intensity profile of the image in buffer 1 centered on
   the pixel at Row=51, Column=25, using all pixels within a 21x21 pixel
   wide box surrounding the center pixel.}
  \item[RPLOT 1 RAD=10 CEN=51.5,25.2\hfill]{
   Same as above, but the center of the plot is now at fractional pixel
   location (51.5,25.2)}
  \item[RPLOT\hfill]{
   Makes a radial intensity plot of the currently displayed image using
   the center computed by the previous AXES command (AXR and AXC variables).
   The box is 11x11 pixels (RAD=5) centered on (AXR,AXC).}
  \item[RPLOT SCALE=0.54 HARD\hfill]{
   Same as above, but scales the radii plotted to the pixel scale of 0.54
   arcseconds/pixel, and sends the plot to the hardcopy device.  The size
   of the box is still 11x11 pixels.}
\end{example}%
\lthtmlfigureZ
\lthtmlcheckvsize\clearpage}

\stepcounter{section}
{\newpage\clearpage
\lthtmlfigureA{command3031}%
\begin{command}
  \item[\textbf{Form: } TVRPLOT {[RAD=r{[,c]}]} {[SCALE=s]} {[MIN=ymin]} 
       {[MAX=ymax]}\hfill]{}
  \item[RAD=r{[,c]}]{ Set the radius of the box in pixels in both rows 
       and columns.  A second argument sets a rectangular 
       box with the radius in Rows and Columns separately 
       defined.  (Default: RAD=5)}
  \item[SCALE=s]{   Define the pixel scale in Arcsec/Pixel for the plot
       (does not apply to the RAD= keyword values)}
  \item[MIN=, MAX=]{ Sets y-axis limits for plot}
  \item[COLOR=]{ Set the color to be used for the plot}
  \item[NOERASE]{ Do not erase between plots}
  \item[LOG]{ Plot intensities on a logarithmic scale}
\end{command}%
\lthtmlfigureZ
\lthtmlcheckvsize\clearpage}

\stepcounter{section}
{\newpage\clearpage
\lthtmlfigureA{command3058}%
\begin{command} 
  \item[\textbf{Form:} CONTOUR im1 {[im2...]} {[BOX=b]} 
       {[LEVELS=(L1,L2,...)]} {[LOW=l]} {[RATIO=r]} ]{}
  \item[{[DIFF=d]} {[FID=l]} {[SCALE=s]} {[USER]} 
        {[TR=(X0,X1,X2,Y0,Y1,Y2)]} ]{}
  \item[{[TITLE]} {[DASH]} {[EXACT]} {[NOERASE]} {[NOLABEL]} {[NOAXES]}]{}
  \item[{[LWEIGHT=w]} {[LTYPE=n1,n2,...]} {[COLOR=c1,c2,...]} {[NV=n]} 
        {[NH=n]} {[NW=n]} {[SUBMAR=sx,sy]}]{}
  \item[{[HARD]} {[LAND]} {[FILE=psfile]} {[EPS]} {[NOPRINT]} {[FULL]} ]{}
  \item[im1 im2 ...]{buffer(s) with the images to be plotted (max of 15)}
  \item[BOX=b]{plots only the part of the image in box 'b'.}
  \item[\textbf{Contour Level Control}\hfill]{}
  \item[LEVELS=]{sets the levels of the plot explicitly.}
  \item[LOW=l]{gives the lowest contour level.}
  \item[RATIO=r]{says that contours above the lowest level will increase
       each time by the given ratio.}
  \item[DIFF=d]{says that the contours above the lowest level will
       increase each time by the given difference.}
  \item[FID=l]{establishes a fiducial contour level for the plot.  The
       fiducial contour is drawn in a heavier line style.
       Contours greater than this level are drawn as a lighter
       continuous line, and those below it are drawn as broken
       lines.  The default value is 0.0.}
  \item[\textbf{Contour Plot Scaling Keywords}\hfill]{}
  \item[SCALE=s]{Sets the image scale to "s" arcseconds/pixel on BOTH axes.}
  \item[USER]{Scales the contour axes based on the CRPIXn, CRVALn,
       and CDELTn FITS header cards, and labels the axes using
       the CTYPEn FITS cards.}
  \item[TR=()]{specify a general linear transformation matrix to scale
       the contour map (see below).}
  \item[\textbf{Contour Style Control}\hfill]{}
  \item[DASH]{draw contours as broken (dashed) lines.}
  \item[EXACT]{Uses a slower contour "following" algorithm than the
       default routine.}
  \item[LWEIGHT=w]{Sets the thickness of all line segments}
  \item[LTYPE=n1,n2...]{determines the line type of im1,im2,...}
  \item[COLOR=c1,c2...]{determines the color of im1,im2,...}
  \item[\textbf{Plot Title and Output Control}\hfill]{}
  \item[TITLE]{put object label on plot}
  \item[NOERASE]{do not erase screen.}
  \item[NOLABEL ]{do not label axex.}
  \item[FULL ]{makes contour plot fill up entire page, rather
       than preserving correct aspect ratio}
  \item[\textbf{Multiwindow Plots}\hfill]{}
  \item[NH=h]{Divide the plotting window into H windows horizontally}
  \item[NV=v]{Divide the plotting window into V windows vertically}
  \item[NW=n]{Put the contour plot into window N for single images 
       (mongo order)}
  \item[SUBMAR=dx,dy]{Change the default spacing (submargins) between windows}
  \item[\textbf{Hardcopy Output Control}\hfill]{}
  \item[HARD]{sends output to the default hardcopy device}
  \item[FILE=psfile]{Send output to the PostScript file named "psfile"}
  \item[EPS]{Make an Encapsulated PostScript file.}
  \item[NOPRINT]{do not send hardcopy to the printer (PostScript only).}
  \item[LAND]{Make the hardcopy in landscape mode (default: portrait)}
\end{command}%
\lthtmlfigureZ
\lthtmlcheckvsize\clearpage}

{\newpage\clearpage
\lthtmlfigureA{hanging3139}%
\begin{hanging} 
  \item{X = CRVAL1 + CDELT1*(C-CRPIX1)}
  \item{Y = CRVAL2 + CDELT2*(R-CRPIX2).}
\end{hanging}%
\lthtmlfigureZ
\lthtmlcheckvsize\clearpage}

{\newpage\clearpage
\lthtmlfigureA{hanging3143}%
\begin{hanging} 
  \item{TR=(X0,X1,X2,Y0,Y1.Y2)}
\end{hanging}%
\lthtmlfigureZ
\lthtmlcheckvsize\clearpage}

{\newpage\clearpage
\lthtmlfigureA{hanging3146}%
\begin{hanging} 
  \item{X = X0 + X1*C + X2*R}
  \item{Y = Y0 + Y1*C + Y2*R}
\end{hanging}%
\lthtmlfigureZ
\lthtmlcheckvsize\clearpage}

{\newpage\clearpage
\lthtmlfigureA{example3172}%
\begin{example}
  \item[CONTOUR 2\hfill]{
Plots all of image 2.  The contour levels are set in the program.}
  \item[CONTOUR 2 HARD\hfill]{
Does the same as example 1, but sends the output to the default
        hardcopy device}
  \item[CONTOUR 2 BOX=3\hfill]{
Does the same as example 1, but plots only the part of the image in
box 3.}
  \item[CONTOUR 1 LEVELS=10,20,30,40,50\hfill]{
An example of setting the levels explicitly.}
  \item[CONTOUR 1 LOW=10.0 DIFF=20.0\hfill]{
Plots image 1 with contours at 10, 30, 50, 70, ...}
  \item[CONTOUR 1 LOW=2 RATIO=2\hfill]{
Plots image 1 with contours levels of 2, 4, 8, 16, 32, ...}
  \item[CONTOUR 1 LOW=2. RATIO=2.5 HARD USER\hfill]{
Plots image 1 with contours spaced at 1 magnitude intervals on the
default hardcopy device, and scales and labels the axes using the
FITS header of image 1.}
  \item[CONTOUR 3 LOW=20. DIFF=10. TITLE\hfill]{
Plots image 3 with contours at levels of 20, 30, 40, ... , and puts
the object name at the top.}
  \item[CONTOUR 1 2 LOW=20. DIFF=10. FID=50. HARD\hfill]{
        Plots image 1 \& 2 over each other with contours at levels of 20, 30, 
        40, ..., where the levels below 50 are plotted as dashed lines, 
        levels above 50 as normal solid lines, and level 50 as a heavier 
        solid line. The result is sent the default hardcopy device.}
  \item[CONTOUR 1 2 LEVELS=1000,5000 LTYPE=0,1 \hfill]{
        Plots images 1 and 2 over each other with contour levels 1000 \& 5000.
        Image 1 is drawn as a solid line (LTYPE=0), and image 2 as a dotted 
        line (LTYPE=1).}
  \item[CONTOUR 1 2 LOW=20. DIFF=10. COLOR=1,2\hfill]{
Plots images 1 and 2 over each other, with contour levesl 20,30,...
        Image 1 is plotted as a white line, while images 2 is plotted as
        a red line (color 2) on a color device, otherwise both appear white.}
  \item[CONTOUR 1 2 LOW=20. DIFF=10. NH=2 NV=1\hfill]{
Divides the plotting window in half horizontally, plotting image
        1 in the first (left-hand) window, and image 2 in the second 
        (right-hand) window.  Both contours have levels 20,30,...}
  \item[CONTOUR 1 2 3 4 LOW=10 RATIO=2.5 NH=2 NV=2 HARD SUBMAR=0,0 NOAXES HARD PORT\hfill]{
Divides the the page in quarters (2 windows horizontal and 2 windows
        vertical), plotting images 1-4 in separate windows, starting from
        the lower left-hand window and ending up in the upper right-hand
        window.  Only the axis box is plotted without ticks or labels, and
        the margins are set to 0,0, so all windows are joined together.
        The plot is sent to the hardcopy device as a Portrait mode plot.
All contour maps have the SAME contour levels: 10,25,62.5,165.25....}
\end{example}%
\lthtmlfigureZ
\lthtmlcheckvsize\clearpage}

\stepcounter{section}
{\newpage\clearpage
\lthtmlfigureA{command3188}%
\begin{command}
  \item[\textbf{Form:} OVERLAY ibuf {[IBOX=b]} {[cbuf]} {[CBOX=b]} {[PROF]} 
       {[PROF=n]}\hfill]{}
  \item[{[Z=zero]} {[L=span]} {[CLIP]} {[POSITIVE]} {[NOBIN]}
        {[LEVELS=(c1,c2,c3,...)]} {[LOW=f]} {[RATIO=f]} {[DIFF=f]}]{}
  \item[{[NC=n]} {[FID=f]} {[CTHRESH=f]} {[DASH]} {[LTYPE=n]} {[LWEIGHT=f]} 
        {[COLOR=c]} {[COLOR=r,g,b]}]{}
  \item[{[EXACT]} {[TITLE]} {[BAR=xxx]} {[COMMENT]} {[COMMENT=xxx]}
        {[SCALE=s]} {[CEN=r,c]} {[LAND]} {[MACRO=xxx]} {[FILE=xxx]}]{}
  \item[{[NOAXES]} {[NOBAR]} {[LARGE]} {[INFO]} {[WIND=w,h]} {[ORIGIN=x,y]} 
        {[PAGE=L,S]} {[COPIES=n]}]{}
  \item[ibuf ]{the buffer containing the grayscale image}
  \item[IBOX=]{only consider the pixels of the grayscale image within IBOX}
  \item[cbuf]{ the buffer containing the contour image}
  \item[CBOX=]{only consider the pixels of the contour image within CBOX}
  \item[PROF]{ instead of a contour image, draw the current best fit 
isophotes in the PROFILE common block}
  \item[PROF=n]{Only draw every Nth isophote in the PROFILE common block}
  \item[Keywords controlling the grayscale image appearance:\hfill]{}
  \item[Z=zero ]{Zero point of the intensity mapping.  Default value is 0.0}
  \item[L=span]{Span of the intensity mapping.  If none is specified, the 
                default value will be taken to be 4 times the image mean.}
  \item[CLIP]{Prevent roll-over of the intensity mapping.  Default is 
              no clipping.}
  \item[POSITIVE]{Make the hardcopy White-on-Black background.  Default is 
                  Black-on-White (conventional photonegative).}
  \item[NOBIN]{Suppresses autobinning if the image is larger than about
               512x512 pixels.  Default is autobinning.}
  \item[Keywords controlling the contour map \hfill]{}
  \item[LEVELS=(c1,c2,c3,...)]{set the levels explicitly.  Up to 40 contours 
       may be given on the command line.}
  \item[LOW=c0]{Lowest contour to be drawn}
  \item[DIFF=f]{Difference between contour levels (equally spaced contours)}
  \item[RATIO=f]{Ratio between contours (for log-spaced contours).  RATIO
                 and DIFF are mutually exclusive}
  \item[NC=n]{Limit the number of contours drawn using LOW/DIFF or 
                  LOW/RATIO to N (up to a maximum of 100).}
  \item[FID=f]{Define a fiducial contour.  Levels above this value are
                  drawn with the default line type, below are dashed, and
                  the fiducial contour itself is drawn with a heavier
                  line weight.  Default fiducial is the zero contour.}
  \item[CTHRESH=f]{All contours with intensity above this level will be
                  drawn in reverse color (default: white).  Provides a means
                  for rudimentary contour "visibility" against saturated
                  colors on the grayscale.  Default is uniform color.}
  \item[DASH]{All contours are to be drawn as dashed (not just those below
              the fiducial level).  Default is solid lines above fiducial.}
  \item[LTYPE=n ]{Draw contours with LickMongo line type N (see table below)}
  \item[LWEIGHT=f]{Draw contours with line weight F (Default is 0.5)}
  \item[COLOR=c]{Draw contours with color C.  Default is black.  A color table 
                 is given below.}
  \item[COLOR=r,g,b]{ Draw contours with the RGB color given.  0 <= RGB <= 1.  
               Allows for arbitrary colors (or shades of gray).}
  \item[EXACT]{Contours are drawn using a slow contour-following algorithm.
               Default: fast rastering algorithm.}
  \item[Keywords affecting the axes of the plot\hfill]{}
  \item[TITLE  ]{Put a title (in FITS header OBJECT card for the grayscale 
                 image) on the plot.}
  \item[BAR=XXX ]{Label the intensity scaling bar with string "xxx" to 
                  indicate the units. Default label is "Intensity".}
  \item[COMMENT ]{Print a comment line on the plot.  Comment lines may}
  \item[COMMENT= ]{be up to 64 characters long. }
  \item[SCALE=s ]{Pixel scale in units/pixel for the image.  The axis ticks 
                  will be in these units rather than pixels (default).  
                  The origin is assumed to be the image center unless the 
                  CEN=(r,c) keyword is used.}
  \item[CEN=(r,c) ]{The center of the image in pixels for use with the SCALE=s 
                  keyword.  Default center is the image array center, unless 
                  PROF or PROF= was called, then the isophote center is used }
  \item[MACRO=xxx]{External file with LickMongo commands to be executed 
                  after the plot is completed.  Provides a facility for 
                  customizing axes and adding annotation.}
  \item[LAND ]{Orient the plot in landscape mode.  Default is Portrait.}
  \item[FILE=xxx]{Direct the PostScript Image into a file named xxx.  By 
                  default, OVERLAY writes the image into IMAGE.PS in the 
                  current working directory.}
  \item[NOAXES ]{Supress plotting of coordinates axes on the image.}
  \item[NOBAR ]{Supress plotting of the intensity scaling bar}
  \item[LARGE]{Expand the plot as large as possible for the given paper size
               while still preserving the 1:1 axis aspect ratio.}
  \item[INFO]{Write in a line of auxilliary info along the bottom of 
              the page in small type.}
  \item[Advanced Output Format Control\hfill]{}
  \item[WIND=(w,h)]{The maximum possible size of the plotting window in inches.
                  This defines the largest region that OVERLAY will try to 
                  fit the image into.  Defaults:\newline
                         Portrait:  6" x 6"\newline
                         Landscape: 8" x 5"\newline
                  Must be smaller than the paper dimensions (see PAGE).}
  \item[ORIGIN=(x,y)]{origin of the plotting window in inches, measured from 
                  the lower left-hand corner of the page.  X is horizontal, 
                  and Y is vertical.  Default centers the window.}
  \item[PAGE=(L,S)]{physical size of the paper in inches.  "L" = long 
                  dimension, "S" = short dimension.  Default: (11,8.5)
                  Used if making a plot for an odd-sized paper printer.}
  \item[COPIES=n]{Number of copies to print.  Default is 1 copy.  Makes the
                  generation of multiple 1st generation copies efficient.}
\end{command}%
\lthtmlfigureZ
\lthtmlcheckvsize\clearpage}

{\newpage\clearpage
\lthtmlfigureA{example3297}%
\begin{example}
  \item[OVERLAY 1 2 Z=0. L=100. CLIP LOW=30. RATIO=1.5 SCALE=0.32 FILE=n1068.ps\hfill]{
   Plots the image in buffer 1 as a grayscale map with values running from
   0. to 100.0 rendered as white-to-black, and draws on top of that figure
   the image in buffer 2 as a contour map with contours running from 30.0
   an spaced geometrically every factor of 1.5 in intensity.  The axes
   are scaled as 0.32 arcseconds/pixel referenced to buffer 1, with the
   origin (0,0) assumed to be the center of buffer 1.  The output is
   directed to a file called n1068.ps in the current working directory
   for later printing at the user's discretion (there is no autoprinting).}
  \item[OVERLAY 1 1 Z=0. L=100. CLIP LOW=20. DIFF=20. SCALE=0.32 FILE=graycon.ps\hfill]{
   This plots the image as a grayscale, with the given Z and L, and then
   plots on top of this the image intensity contours, starting at 20 and
   spaced every 20 intensity units.  The scale is 0.32 arcsec/pixel and
   the origin (0,0) is the image center.  Output is sent to a file
   called graycon.ps  (old AIPS users will recognize this kind of figure)}
  \item[OVERLAY 1 PROF=3 Z=0. L=1000. CLIP FILE=isofit.ps\hfill]{
   The image in buffer 1 is rendered as a grayscale plot between
   intensities (0:1000), and the current best-fit isophotes stored in
   the PROFILE common block are superimposed.  For clarity, only every 
   third isophote is  plotted (including the first and last isophote).
   The output file is called "isofit.ps".}
  \item[OVERLAY 1 PROF=3 Z=0. L=1000. CLIP CTHRESH=1000. FILE=isofit.ps\hfill]{}
  \item[\hfill]{Same as (3) above, but now all best-fit isophotes with a mean
   surface brightness >= 1000. are plotted as white lines to make
   them visible against the black image pixels in the center.}
  \item[OVERLAY 1 IBOX=2 4 CBOX=1 Z=0. L=1000. 
        CLIP LEVELS=(.1,.2,.3,.5,.7,.9)\hfill]{}
  \item[SCALE=0.51 CEN=(AXR,AXC) FILE=hacon.ps\hfill]{
   The parts of buffer 1 within box 2 are rendered as a grayscale map
   between intensities (0:1000), with the contour levels of the parts
   of buffer 4 within box 1 are superimposed.  The axes are scaled to 0.51
   arcseconds/pixel, with the image center given by the results of a recent
   AXES calculation stored in the AXR and AXC variables (see AXES).}
  \item[OVERLAY 1 IBOX=2 4 CBOX=1 Z=0. L=1000. 
        CLIP LEVELS=(.1,.2,.3,.5,.7,.9)\hfill]{}
  \item[NOAXES MACRO=hacon.mgo FILE=hacon.ps\hfill]{
   Same as (5) except now no axes are drawn by OVERLAY, and instead the
   external LickMongo macro file "hacon.mgo" in the current working 
   directory is used to draw the axes and put on custom labels.  Output
   is directed to the PostScript file hacon.ps.}
  \item[OVERLAY 1 PROF=3 Z=0. L=1000. CLIP FILE=isofit.ps COLOR=3\hfill]{}
  \item[LTYPE=2\hfill]{
   Same as (3) above, but this user has a color PostScript device, and
   is plotting the best-fit isophotes as dashed green lines (no accounting
   for taste).  }
  \item[OVERLAY 1 PROF=3 Z=0. L=1000. CLIP FILE=isofit.ps 
        COLOR=.7,.7,.7\hfill]{} 
  \item[LTYPE=2\hfill]{
   Having the plot made in (7) thrown back by his co-authors, the user
   now draws the best-fit isophotes as 70\% gray dashed lines.}
\end{example}%
\lthtmlfigureZ
\lthtmlcheckvsize\clearpage}

\stepcounter{section}
{\newpage\clearpage
\lthtmlfigureA{command3313}%
\begin{command}
  \item[\textbf{Form:}   HISTOGRAM source {[BOX=b]} {[NOLOG]} {[BIN=n]} {[XMIN=x1]} {[XMAX=x2]}\hfill]{}
  \item{{[YMIN=y1]} {[YMAX=y2]} {[HARD]} {[WIND=n]} {[BUF=buf]}}
  \item{{[NOERASE]} {[PORT]}}
  \item[source\hfill]{   specifies the image.}
  \item[BOX=b\hfill]{limits the computation to those pixels in box 'b'.}
  \item[NOLOG\hfill]{displays the number of pixels at each intensity,
rather than the logarithm.}
  \item[BIN=n\hfill]{   bins the image values by the specified factor.}
  \item[XMIN, XMAX\hfill]{  limits the computation to those pixels with values
between x1 and x2, inclusive.}
  \item[YMIN, YMAX   \hfill]{   limits the display of the histogram on the Y-axis
to be from y1 to y2.}
  \item[HARD\hfill]{sends the plot to the hardcopy device.}
  \item[WIND=n\hfill]{put the plot in window n of a 2x2 grid}
  \item[BUF=buf\hfill]{  load the histogram data into image buffer 'buf'}
  \item[   NOERASE   \hfill]{don't erase screen before plotting.}
  \item[PORT\hfill]{make hardcopy output in portrait mode (default: landscape)}
\end{command}%
\lthtmlfigureZ
\lthtmlcheckvsize\clearpage}

{\newpage\clearpage
\lthtmlfigureA{example3341}%
\begin{example}
  \item[HISTOGRAM 4 \hfill]{Shows the histogram for image 4}
  \item[HISTOGRAM \$Q BOX=3 \hfill]{Shows the histogram for image Q 
       (where Q is a variable) using only the pixels in box 3}
  \item[HISTOGRAM 2 XMIN=1000 XMAX=1999 \hfill]{
       Shows the log of number of pixels with values between 1000 and 1999.}
  \item[HISTOGRAM 4 NOLOG \hfill]{
       Shows the number of pixels (not the log) at each value in image 4.}
  \item[HISTOGRAM 3 HARD\hfill]{
       computes a histogram for image 3, sending it to hardcopy printer.}
  \item[HISTOGRAM 2 XMIN=1000 XMAX=1999 BUF=10\hfill]{ same as \#3 above,
       except that instead of plotting the histogram, it is loaded into image
       buffer 10 as a spectrum.}
\end{example}%
\lthtmlfigureZ
\lthtmlcheckvsize\clearpage}

\stepcounter{section}
{\newpage\clearpage
\lthtmlfigureA{command3350}%
\begin{command}
  \item[\textbf{Form: }PLOT3D image {[BOX=b]} ALT={[alt]} AZ={[az]} 
   ZFAC={[zfac]}\hfill]{}
  \item{ZOFF={[zoff]} NOERASE HARD}
  \item[image]{buffer to be plotted}
  \item[BOX=b]{only plot the image section within the given box}
  \item[ALT=]{viewing altitude}
  \item[AZ=]{viewing azimuth}
  \item[ZFAC=,ZOFF=]{Z-axis scale factor and offset}
  \item[SCALE=]{scale factor to apply to ZFAC and ZOFF}
  \item[NOERASE]{doesn't clear screen first}
  \item[HARD]{generates hardcopy}
\end{command}%
\lthtmlfigureZ
\lthtmlcheckvsize\clearpage}

\stepcounter{section}
{\newpage\clearpage
\lthtmlfigureA{command3370}%
\begin{command}
  \item[\textbf{Form: } CLEAR {[TEXT]} {[VEC]} {[IMAGES]}\hfill]{}
\end{command}%
\lthtmlfigureZ
\lthtmlcheckvsize\clearpage}

\stepcounter{section}
{\newpage\clearpage
\lthtmlfigureA{command3379}%
\begin{command}
  \item[\textbf{Form: }TEXT  imbuf {[COL=c ROW=r]} {[X0=x Y0=y]} 
       {[TEXT=s]}\hfill]{}
  \item[]{{[SCALE=s]} {[ROT=ang]} {[FILL=f]}}
  \item[imbuf]{   buffer with the image to be tagged.}
  \item[X0=,Y0=]{coordinates of the text string (upper left corner of
       the text box) in pixels (X0 \& Y0 are columsn \& rows, respectively).}
  \item[TEXT=s]{Text string to embed (can be a string variable)}
  \item[SCALE=f]{Text scale factor (default=1)}
  \item[ROT=ang]{Text rotation angle in degrees (default=0)}
  \item[FILL=]  {Text Fill (default=0=nofill, 1=fill with zeros)}
\end{command}%
\lthtmlfigureZ
\lthtmlcheckvsize\clearpage}

\stepcounter{chapter}
\stepcounter{section}
{\newpage\clearpage
\lthtmlfigureA{example3923}%
\begin{example}
  \item[BOX\hfill]{define a rectangular region of an image --
       a set of contiguous rows and columns.}
  \item[MASK\hfill]{ignore specified pixels or regions}
  \item[UNMASK\hfill]{stop ignoring masked pixels}
  \item[MASKTOIM\hfill]{create a map of masked pixels.}
\end{example}%
\lthtmlfigureZ
\lthtmlcheckvsize\clearpage}

\stepcounter{section}
{\newpage\clearpage
\lthtmlfigureA{command3931}%
\begin{command}
  \item[\textbf{Form:}  BOX box\_num {[NC=n]} {[NR=n]} {[N=n]} {[CR=n]} 
       {[CC=n]} {[SR=n]} {[SC=n]} {[INT]}\hfill]{}
  \item{   {[CENT]} {[V=n]}}
  \item[box\_num]{(integer) is the number of the box being defined,}
  \item[NC]{defines the number of columns in the box,}
  \item[NR]{defines the number of rows in the box,}
  \item[N  ]{ defines the number of rows and columns (square box),}
  \item[CR]{defines the center row,}
  \item[CC]{defines the center column,}
  \item[SR]{defines the starting row, and}
  \item[SC]{defines the starting column.}
  \item[INT]{lets you define the box interactively on the TV}
  \item[CENT]{toggles back and forth between origin and center based system}
  \item[V=n  ]{defines the center row and column using the
       VISTA variables Rn and Cn}
\end{command}%
\lthtmlfigureZ
\lthtmlcheckvsize\clearpage}

{\newpage\clearpage
\lthtmlfigureA{example3957}%
\begin{example}
  \item[BOX 1 NR=45 NC=56\hfill]{defines box 1.  Box 1 has origin at row 0 and 
       column 0 and has 45 rows and 56 columns.}
  \item[BOX 1 INT\hfill]{defines box 1. The locations of the upper left and 
       lower right corners are specified using the cursor on the TV.}
  \item[BOX 1 SC=100 NC=100 SR=0 NR=100\hfill]{defines box 1 as columns
       100 to 199 and rows 0 to 99.}
  \item[BOX 2 CC=100 CR=100 NR=13 NC=13\hfill]{    defines a box having 13 rows
       and columns, centered on row=100 and column=100.}
\end{example}%
\lthtmlfigureZ
\lthtmlcheckvsize\clearpage}

{\newpage\clearpage
\lthtmlfigureA{example3964}%
\begin{example}
  \item[BOX 1 SR=50\hfill]{moves the box so that the starting row is row 50. 
       The starting column, and the number of rows and columns is unchanged.}
  \item[BOX 1 CC=50 CR=66\hfill]{moves the box so that the center of the box 
       is at row 50 and column 66.  The size of the box is unchanged.}
\end{example}%
\lthtmlfigureZ
\lthtmlcheckvsize\clearpage}

{\newpage\clearpage
\lthtmlfigureA{hanging3968}%
\begin{hanging}
  \item{BOX n SR=0 NR=1 ...}
\end{hanging}%
\lthtmlfigureZ
\lthtmlcheckvsize\clearpage}

\stepcounter{section}
{\newpage\clearpage
\lthtmlfigureA{command3974}%
\begin{command}
  \item[\textbf{Form:} MASK {[R=r1,r2]} {[C=c1,c2]} {[BOX=N]} 
       {[PIX=r,c]}\hfill]{}
  \item[UNMASK  {[R=r1,r2]} {[C=c1,c2]} {[BOX=N]} {[PIX=r,c]}\hfill]{}
  \item[R=]{Ignore all pixels in rows.}
  \item[C=]{Ignore all pixels in columns.}
  \item[BOX=b]{Ignore all pixels in box b.}
  \item[PIX=r,c]{Ignore the pixel at row 'r' and col 'c'.}
\end{command}%
\lthtmlfigureZ
\lthtmlcheckvsize\clearpage}

{\newpage\clearpage
\lthtmlfigureA{example3992}%
\begin{example}
  \item[MASK C=234\hfill]{Masks column 234.}
  \item[UNMASK C=234\hfill]{Removes the mask from column 234.}
  \item[MASK PIX=(120,100)\hfill]{Masks the single pixel at
column 100 and row 125.}
  \item[MASK R=20,40\hfill]{Masks pixels in rows 20 to 40.}
  \item[UNMASK BOX=5\hfill]{Unmasks the pixels in box 5.}
\end{example}%
\lthtmlfigureZ
\lthtmlcheckvsize\clearpage}

\stepcounter{section}
{\newpage\clearpage
\lthtmlfigureA{command4000}%
\begin{command}
  \item[\textbf{Form:}  MASKTOIM buf {[BOX=b]} {[SR=sr]} {[SC=sc]} {[NR=nr]} {[NC=nc]}\hfill]{}
  \item[buf]{is the buffer holding the new image}
  \item[BOX=b]{create an image with the size and orientation of box 'b'}
  \item[SR=sr]{specify the start row of the new image}
  \item[SC=sc]{specify the start column}
  \item[NR=nr]{specify the number of rows}
  \item[NC=nc]{specify the number of columns}
\end{command}%
\lthtmlfigureZ
\lthtmlcheckvsize\clearpage}

{\newpage\clearpage
\lthtmlfigureA{example4016}%
\begin{example}
  \item[MASKTOIM 1 BOX=5\hfill]{creates an image in buffer 1 having the
       size and orientation of box 5.  The image is filled with zeroes.}
\par
\item[MASKTOIM 1 BOX=5 CONST=100.0\hfill]{does the same as the first
       example, but fills the image with value 100.0}
\par
\item[MASKTOIM 5 SR=5 SC=10 NR=25 NC=35\hfill]{creates an image in buffer
       5.  The start (row, column) is (5,10) and the size of the image is
       25 rows by 35 columns.}
\par
\item[MASKTOIM 1 NR=100 NC=100\hfill]{creates an 100 by 100 image in
       buffer 1.  The start row and column are both 0.}
\end{example}%
\lthtmlfigureZ
\lthtmlcheckvsize\clearpage}

\stepcounter{chapter}
\stepcounter{section}
{\newpage\clearpage
\lthtmlfigureA{example4081}%
\begin{example}
  \item[ADD\hfill]{add two images or an image and a constant.}
  \item[SUBTRACT\hfill]{subtract two images or subtract a constant and a 
       image.}
  \item[MULTIPLY\hfill]{multiply two images or an image and a constant}
  \item[DIVIDE\hfill]{divide two images or divide an image by a constant.}
  \item[SQRT\hfill]{square root}
  \item[LOG\hfill]{base-10 logarithm}
  \item[LN\hfill]{natural logarithm}
  \item[EXP\hfill]{exponentiate image (e\^image):  LOG INV does 10\^image}
  \item[TAN\hfill]{tangent}
  \item[ARCTAN\hfill]{inverse of TAN}
  \item[SIN\hfill]{sine}
  \item[COS\hfill]{cosine}
  \item[NINT\hfill]{nearest integer}
  \item[ONEOVER\hfill]{invert (1/x)}
\end{example}%
\lthtmlfigureZ
\lthtmlcheckvsize\clearpage}

\stepcounter{section}
{\newpage\clearpage
\lthtmlfigureA{command4106}%
\begin{command}
  \item[\textbf{Form:}ADD        dest {[other]} {[CONST=c]} {[BOX=B]} {[DR=dr]} {[DC=dc]}\hfill]{}
  \item[SUBTRACT   dest {[other]} {[CONST=c]} {[BOX=B]} {[DR=dr]} {[DC=dc]}\hfill]{}
  \item[MULTIPLY   dest {[other]} {[CONST=c]} {[BOX=B]} {[DR=dr]} {[DC=dc]}\hfill]{}
  \item[DIVIDE     dest {[other]} {[CONST=c]} {[BOX=B]} {[DR=dr]} {[DC=dc]} {[FLAT]}\hfill]{}
  \item[dest]{is the buffer where the result will be stored}
  \item[other]{is the other buffer in the calculation, if any.}
  \item[CONST=c]{performs arithmetic between a buffer and a constant}
  \item[BOX=B]{limits the arithmetic to those pixels
in the source which are in box B}
  \item[DR=dr]{Gives a row offset between the 'dest' and 'other' buffers.}
  \item[DC=dc]{Gives a column offset between the 'dest' and 'other' buffers.}
  \item[FLAT]{rescales the division by the mean of 'other', preserving the 
mean of the source.}
\end{command}%
\lthtmlfigureZ
\lthtmlcheckvsize\clearpage}

{\newpage\clearpage
\lthtmlfigureA{example4142}%
\begin{example}
  \item[ADD 1 2\hfill]{Add image in buffer 2 to image in buffer 1.}
  \item[SUBTRACT 3 4\hfill]{Subtract image in buffer 4 from image in buffer 3.}
\end{example}%
\lthtmlfigureZ
\lthtmlcheckvsize\clearpage}

{\newpage\clearpage
\lthtmlfigureA{example4148}%
\begin{example}
  \item[ADD 2 CONST=9.5\hfill]{Adds 9.5 to every pixel in buffer 2}
  \item[DIVIDE 3 CONST=10/3.1415\hfill]{  Divides buffer 3 by 10/3.14159}
\end{example}%
\lthtmlfigureZ
\lthtmlcheckvsize\clearpage}

{\newpage\clearpage
\lthtmlfigureA{example4153}%
\begin{example}
  \item[ADD 2 3 CONST=5\hfill]{Adds buffers 3 to 2, then adds 5.0
       to the result.}
  \item[MULTIPLY 1 \$J CONST=0.01\hfill]{ Multiplies buffer 1 by buffer J
       (J a variable), then multiplies the result by 0.01.}
\end{example}%
\lthtmlfigureZ
\lthtmlcheckvsize\clearpage}

\stepcounter{section}
{\newpage\clearpage
\lthtmlfigureA{command4160}%
\begin{command}
  \item[\textbf{Form:}SQRT source {[SGN]} {[SIGN]} {[NOABS]}\hfill]{}
\end{command}%
\lthtmlfigureZ
\lthtmlcheckvsize\clearpage}

\stepcounter{section}
{\newpage\clearpage
\lthtmlfigureA{command4169}%
\begin{command}
  \item[\textbf{Form:} LOG source {[INV]}\hfill]{}
\end{command}%
\lthtmlfigureZ
\lthtmlcheckvsize\clearpage}

{\newpage\clearpage
\lthtmlfigureA{example4175}%
\begin{example}
  \item[LOG 3\hfill]{take the log of each pixel in buffer 3.}
\end{example}%
\lthtmlfigureZ
\lthtmlcheckvsize\clearpage}

\stepcounter{section}
{\newpage\clearpage
\lthtmlfigureA{command4180}%
\begin{command}
  \item[\textbf{Form :} LN source {[INV]}\hfill]{}
\end{command}%
\lthtmlfigureZ
\lthtmlcheckvsize\clearpage}

{\newpage\clearpage
\lthtmlfigureA{example4186}%
\begin{example}
  \item[LN 3\hfill]{take the natural log of each pixel in buffer 3.}
\end{example}%
\lthtmlfigureZ
\lthtmlcheckvsize\clearpage}

\stepcounter{section}
{\newpage\clearpage
\lthtmlfigureA{command4191}%
\begin{command}
  \item[\textbf{Form:}EXP source\hfill]{}
\end{command}%
\lthtmlfigureZ
\lthtmlcheckvsize\clearpage}

{\newpage\clearpage
\lthtmlinlinemathA{tex2html_wrap_inline4232}%
$e^n$%
\lthtmlinlinemathZ
\lthtmlcheckvsize\clearpage}

\stepcounter{section}
{\newpage\clearpage
\lthtmlfigureA{command4197}%
\begin{command}
  \item[\textbf{Form:}  TAN source\hfill]{}
\end{command}%
\lthtmlfigureZ
\lthtmlcheckvsize\clearpage}

\stepcounter{section}
{\newpage\clearpage
\lthtmlfigureA{command4203}%
\begin{command}
  \item[\textbf{Form:} ARCTAN source {[0TO180]}\hfill]{}
\end{command}%
\lthtmlfigureZ
\lthtmlcheckvsize\clearpage}

\stepcounter{section}
{\newpage\clearpage
\lthtmlfigureA{command4209}%
\begin{command}
  \item[\textbf{Form:} COS source\hfill]{}
\end{command}%
\lthtmlfigureZ
\lthtmlcheckvsize\clearpage}

\stepcounter{section}
{\newpage\clearpage
\lthtmlfigureA{command4214}%
\begin{command}
  \item[\textbf{Form:} SIN source\hfill]{}
\end{command}%
\lthtmlfigureZ
\lthtmlcheckvsize\clearpage}

\stepcounter{section}
{\newpage\clearpage
\lthtmlfigureA{command4219}%
\begin{command}
  \item[\textbf{Form:} NINT source\hfill]{}
\end{command}%
\lthtmlfigureZ
\lthtmlcheckvsize\clearpage}

\stepcounter{section}
{\newpage\clearpage
\lthtmlfigureA{command4224}%
\begin{command}
  \item[\textbf{Form:} ONEOVER source\hfill]{}
\end{command}%
\lthtmlfigureZ
\lthtmlcheckvsize\clearpage}

\stepcounter{chapter}
{\newpage\clearpage
\lthtmlfigureA{example4307}%
\begin{example}
  \item[MN\hfill]{Compute the Mean of the Pixel Values}
  \item[SKY\hfill]{Compute the Modal Sky/Background Level in an Image}
  \item[ABX\hfill]{Analyze Pixel Statistics in an Image}
  \item[AXES\hfill]{Compute the Centroid of an Object in an Image}
\end{example}%
\lthtmlfigureZ
\lthtmlcheckvsize\clearpage}

\stepcounter{section}
{\newpage\clearpage
\lthtmlfigureA{command4315}%
\begin{command}
  \item[\textbf{Form:}MN source {[NOBL]} {[BOX=b]} {[NOZERO]} {[MASK]}
       {[PIX=p]} {[SILENT]} {[W=w1,w2]}\hfill]{}
  \item[source]{the image or spectrum for which the mean
       is to be computed}
  \item[NOBL]{ignore baseline (last) column in an image}
  \item[BOX=b]{find the mean in box b}
  \item[NOZERO]{ignore pixels with value zero}
  \item[MASK]{ignore masked pixels}
  \item[PIX=p]{use every p'th pixel (for speed)}
  \item[W=w1,w2]{find the mean in a wavelength calibrated
       spectrum in the interval from wavelength w1 to w2.}
  \item[SILENT]{do not print the results of the calculation.}
\end{command}%
\lthtmlfigureZ
\lthtmlcheckvsize\clearpage}

{\newpage\clearpage
\lthtmlfigureA{example4335}%
\begin{example}
  \item[MN 4\hfill]{finds the mean of buffer 4.  The
  value of the mean is printed on the terminal and is loaded into the 
  variables MEAN and M4.}
  \item[MN 2 PIX=5\hfill]{computes the mean using every 5th 
  pixel in every 5th row.  This makes the computation go very fast.}
  \item[MN 5 W=4000,5000\hfill]{finds the mean of the wavelength-
  calibrated spectrum in buffer 5 in the interval from 4000 to 5000 Angstroms.}
  \item[MN \$Q MASK\hfill]{finds the mean of the object in buffer
  Q (Q a variable), ignoring masked pixels.}
  \item[MN 4 BOX=6\hfill]{finds the mean of the object in
  buffer 4 using those pixels in box 6.}
\end{example}%
\lthtmlfigureZ
\lthtmlcheckvsize\clearpage}

\stepcounter{section}
{\newpage\clearpage
\lthtmlfigureA{command4344}%
\begin{command}
  \item[\textbf{Form:}SKY source {[BOX=n]} {[SILENT]} {[CORNERS]}
       {[MAX=c]}\hfill]{}
  \item[source]{is the number of the image that SKY works on.}
  \item[BOX=n]{tells SKY to work only box=n.}
  \item[SILENT]{do not print the result of the calculation.}
  \item[CORNERS]{tells SKY to use the corners of the image.}
  \item[MAX=c]{tells SKY to ignore pixels greater than 'c'.}
\end{command}%
\lthtmlfigureZ
\lthtmlcheckvsize\clearpage}

{\newpage\clearpage
\lthtmlfigureA{example4358}%
\begin{example}
  \item[SKY 5\hfill]{ finds the background in image 5, loading 
  its value into the variable SKY.}
  \item[SKY \$W BOX=5\hfill]{finds the sky level in image W
  (W a variable) in box 5.}
  \item[SKY 2 CORNERS SILENT\hfill]{finds the sky level in image 2 using
  only the corners of the image.  The value is loaded into SKY, but nothing
  is printed on the terminal.}
\end{example}%
\lthtmlfigureZ
\lthtmlcheckvsize\clearpage}

\stepcounter{section}
{\newpage\clearpage
\lthtmlfigureA{command4370}%
\begin{command}
  \item[\textbf{Form:}ABX source {[boxes]} {[ALL]} {[W=w1,w2]} {[SILENT]} 
       {[MASK]}\hfill]{}
  \item{{[TOTAL=var]} {[MEAN=var]} {[HIGH=var]}}
  \item{{[LOW=var]} {[HIGH\_ROW=var]} {[HIGH\_COL=var]}}
  \item{{[LOW\_ROW=var]} {[LOW\_COL=var]} {[SIGMA=var]} (redirection)}
  \item{{[AREA=farea]} {[P=var]}}
  \item[source]{specifies the object.}
  \item[boxes or BOX=b1,b2...]{list boxes to be used in the analysis.}
  \item[ALL]{tells the program to analyze the entire image or spectrum.}
  \item[W=w1,w2]{limits the analysis to the wavelength interval w1 to w2 for 
       wavelength-calibrated spectra.}
  \item[SILENT]{do not print output.} 
  \item[MASK]{ignore masked pixels.}
  \item[AREA=farea]{Determine the pixel position where the total reaches
       the value farea.}
  \item[var]{the name of a variable.}
\end{command}%
\lthtmlfigureZ
\lthtmlcheckvsize\clearpage}

{\newpage\clearpage
\lthtmlfigureA{example4398}%
\begin{example}
  \item[ABX 3\hfill]{Finds the properties of object 3 in the entire image}
  \item[ABX 3 1\hfill]{Finds the properties of object 3 in box 1}
  \item[ABX 2 3 4 5 6\hfill]{Analyzes object 2 in boxes 3, 4, 5, and 6}
  \item[ABX 2 BOX=3,4,5,6 \hfill]{Analyzes object 2 in boxes 3, 4, 5, and 6}
\end{example}%
\lthtmlfigureZ
\lthtmlcheckvsize\clearpage}

{\newpage\clearpage
\lthtmlfigureA{example4404}%
\begin{example}
  \item[TOTAL=var\hfill]{Stores the total count of all the pixels in 'var'}
  \item[MEAN=var\hfill]{Stores the average of the image}
  \item[HIGH=var\hfill]{Stores the VALUE of the pixel with the highest count}
  \item[LOW=var\hfill]{Stores the VALUE of the pixel with the lowest count}
  \item[HIGH\_ROW=var\hfill]{Stores the row number in which the highest-valued
       pixel is located}
  \item[HIGH\_COL=var\hfill]{Stores the column number in which the 
       highest-valued pixel is located}
  \item[LOW\_ROW=var\hfill]{Stores the row number in which the lowest-valued
       pixel is located.}
  \item[LOW\_COL=var\hfill]{Stores the column number in which the lowest-valued
       pixel is located.}
  \item[SIGMA=var\hfill]{Stores the standard deviation of the pixel values
       about the mean.}
  \item[P=var \hfill]{ Stores the pixel where the total reaches farea
       (using AREA=farea) keyword.}
\end{example}%
\lthtmlfigureZ
\lthtmlcheckvsize\clearpage}

{\newpage\clearpage
\lthtmlfigureA{example4417}%
\begin{example}
  \item[ABX 1 3 MEAN=M3 SIGMA=SIG3\hfill]{Analyzes image 1 in box 3, storing
       the mean in variable M3 and the standard deviation in SIG3.}
  \item[ABX 2 7 HIGH\_ROW=HR HIGH\_COL=HC\hfill]{ Analyzes image 2 in box 7, 
       storing the location of the highest-valued pixel in HR and HC}
\end{example}%
\lthtmlfigureZ
\lthtmlcheckvsize\clearpage}

\stepcounter{section}
\stepcounter{section}
{\newpage\clearpage
\lthtmlfigureA{command4461}%
\begin{command}
  \item[\textbf{Form:}AXES source {[BOX=n]} {[SKY=s]} {[W=w1,w2]} (redirection) {[SILENT]}\hfill]{}
  \item[source\hfill]{is the image that AXES uses.}
  \item[BOX=n\hfill]{specifies the section of the image used.}
  \item[SKY=s\hfill]{specifies the sky value used in the calculation.}
  \item[W=\hfill]{finds the centroid in a specified interval
of a wavelength-calibrated spectrum (for use
with emission spectra, NOT absorption spectra.)}
  \item[SILENT \hfill]{suppresses terminal output}
\end{command}%
\lthtmlfigureZ
\lthtmlcheckvsize\clearpage}

{\newpage\clearpage
\lthtmlfigureA{example4475}%
\begin{example}
  \item[AXES 3\hfill]{finds the centroid of the object in buffer 3.}
  \item[AXES 4 BOX=5\hfill]{finds the centroid of the part of
       object 4 that is in box 5.}
  \item[AXES 1 W=6550,6800\hfill]{finds the centroid of the part of
       the wavelength calibrated spectrum that lies between wavelengths 6550
       and 6800 Angstroms.}
  \item[AXES 3 >AXES.OUT\hfill]{does the same as the first example,
       but prints the result of the calculation on the file AXES.OUT, 
       instead of on the terminal.}
  \item[AXES 3 SILENT\hfill]{does the same as the first example,
       but does not print output.}
\end{example}%
\lthtmlfigureZ
\lthtmlcheckvsize\clearpage}

\stepcounter{chapter}
{\newpage\clearpage
\lthtmlfigureA{example4559}%
\begin{example}
  \item[MERGE\hfill]{Merge Overlapping Images or Spectra}
  \item[WINDOW\hfill]{Window (Crop) an Image to a Smaller Size}
  \item[ROTATE\hfill]{Rotate an Image}
  \item[FLIP\hfill]{Change the Orientation of an Image}
  \item[BIN\hfill]{Compress (bin) an Image or Spectrum}
  \item[SHIFT\hfill]{Shift an Image}
  \item[CLIP\hfill]{Replace Pixels Outside an Intensity Range}
  \item[COLFIX\hfill]{Fix Deferred Charge Columns in a CCD Image}
  \item[SMOOTH\hfill]{Smooth an Image or Spectrum}
  \item[ZAP\hfill]{Non-Interactive Image Median Filter}
  \item[TVZAP\hfill]{Interactively Remove Pixels by Median Filtering}
  \item[MEDIAN\hfill]{Compute Median of Several Images}
  \item[BIGMEDIAN\hfill]{Compute Median of a Large Number of FITS Images}
  \item[PICCRS\hfill]{Optimally Combine Frames with Outlier Rejection}
  \item[SURFACE\hfill]{Fit a Plane or Second-Order Surface to an Image}
  \item[SPLINE\hfill]{Replace Image or Spectrum by a Cubic Spline Fit}
  \item[CROSS\hfill]{Cross-Correlate Images or Spectra}
  \item[INTERP\hfill]{Interpolate Across Rows, Columns, or Masked Pixels}
  \item[REGISTAR\hfill]{Register Images Using Field Star Positions}
  \item[DSSCOORD\hfill]{Compute RA and Dec for Digitized Sky Survey Images}
  \item[ATODSIM/ATODFIX\hfill]{Simulate/Fix WF/PC1 A/D Conversion Errors}
  \item[LINCOMB\hfill]{Fit a Linear Combination of Images to an Image}
  \item[BL\hfill]{Correct an Image for Baseline Subtraction Noise}
\end{example}%
\lthtmlfigureZ
\lthtmlcheckvsize\clearpage}

\stepcounter{section}
{\newpage\clearpage
\lthtmlfigureA{command4587}%
\begin{command}
  \item[\textbf{Form: }MERGE im1 im2 im3 im4 ... {[NOMATCH]}\hfill]{}
  \item[im1]{is the first buffer to be used in the average,
       and is also the destination buffer,}
  \item[im2, im3, ...]{are the other images to be used in the average,}
  \item[NOMATCH]{allows the merging of images whose pixels do not
line-up exactly.}
\end{command}%
\lthtmlfigureZ
\lthtmlcheckvsize\clearpage}

\stepcounter{section}
{\newpage\clearpage
\lthtmlfigureA{command4599}%
\begin{command}
  \item[\textbf{Form: }WINDOW source BOX=n\hfill]{}
  \item[source]{is the image being made smaller.}
  \item[BOX]{tells VISTA what part of the old image to save.}
\end{command}%
\lthtmlfigureZ
\lthtmlcheckvsize\clearpage}

{\newpage\clearpage
\lthtmlfigureA{hanging4606}%
\begin{hanging}
  \item{BOX 1 SR=100 SC=200 NR=100 NC=100}
  \item{WIND 7 BOX=1}
\end{hanging}%
\lthtmlfigureZ
\lthtmlcheckvsize\clearpage}

\stepcounter{section}
{\newpage\clearpage
\lthtmlfigureA{command4612}%
\begin{command}
  \item[\textbf{Form: }   ROTATE source {[LEFT]} {[RIGHT]} {[UD]} 
       {[PA=degrees]} {[BOX=b]} \hfill]{}
  \item{{[TRANSPOSE]} {[SINC]}}
  \item[source]{the buffer containing the image to be rotated.}
  \item[LEFT]{rotates the image LEFT (counterclockwise).}
  \item[RIGHT]{rotates the image RIGHT (clockwise).}
  \item[UD]{turns the image upside-down.}
  \item[TRANSPOSE]{transposes the image.}
  \item[PA=degrees]{rotates the image by the specified number of degrees.}
  \item[BOX=b]{gives the size of the output image when the rotation is 
       specified in degrees.}
  \item[SINC]{uses 2-D sinc interpolation, rather than bilinear,
       for general rotation.}
\end{command}%
\lthtmlfigureZ
\lthtmlcheckvsize\clearpage}

{\newpage\clearpage
\lthtmlfigureA{example4631}%
\begin{example}
  \item[ROTATE 4 LEFT\hfill]{Rotates all of image 4 90 degrees
       in a counter-clockwise direction.}
  \item[ROTATE 1 BOX=6 PA=145\hfill]{Rotates image 1 by 145 degrees and
       keeps only the part of the rotated image which lies in box 6.}
\end{example}%
\lthtmlfigureZ
\lthtmlcheckvsize\clearpage}

\stepcounter{section}
{\newpage\clearpage
\lthtmlfigureA{command4637}%
\begin{command}
  \item[\textbf{Form: }FLIP source {[ROWS]} {[COLS]}\hfill]{}
\end{command}%
\lthtmlfigureZ
\lthtmlcheckvsize\clearpage}

{\newpage\clearpage
\lthtmlfigureA{example4644}%
\begin{example}
  \item[FLIP 3 ROWS\hfill]{Inverts image 3 top to bottom.}
  \item[FLIP 3 COLS\hfill]{Inverts image 3 left to right.}
\end{example}%
\lthtmlfigureZ
\lthtmlcheckvsize\clearpage}

{\newpage\clearpage
\lthtmlfigureA{example4649}%
\begin{example}
  \item[FLIP 5 ROWS\hfill]{Flip image 5 about the central row.}
  \item[FLIP 3 COLS\hfill]{Flip image 3 around the central column.}
\end{example}%
\lthtmlfigureZ
\lthtmlcheckvsize\clearpage}

\stepcounter{section}
{\newpage\clearpage
\lthtmlfigureA{command4656}%
\begin{command}
  \item[\textbf{Form: }BIN source {[BIN=b]} {[BINR=br]} {[BINC=bc]} \hfill]{}
  \item{{[SR=sr]} {[SC=sc]} {[NORM]}}
  \item[source]{is the image being compressed}
  \item[BIN=b]{compress rows and columns by integer factor 'b'}
  \item[BINR=br]{specify row compression}
  \item[BINC=bc]{specify column compression}
  \item[SR=sr]{give starting row of 'source' to locate region
       being compressed}
  \item[SC=sc]{give starting column in 'source'}
  \item[NORM]{output is average of pixel values instead of sum.}
\end{command}%
\lthtmlfigureZ
\lthtmlcheckvsize\clearpage}

{\newpage\clearpage
\lthtmlfigureA{example4673}%
\begin{example}
  \item[BINR\hfill]{gives the amount by which rows or compressed.  This
       shortens each row, thus producing an image with
       fewer columns in it.}
  \item[BINC\hfill]{gives the amount by which columns are compressed.
       This shortens each column, thus producing an image
       with fewer rows in it.}
\end{example}%
\lthtmlfigureZ
\lthtmlcheckvsize\clearpage}

{\newpage\clearpage
\lthtmlfigureA{example4678}%
\begin{example}
  \item[BIN 4 BIN=2\hfill]{compresses image 4 by a factor of 2
       in rows and columns.  Adjacent pixels
       are added.  The mean of the image will
       be about 4 times the mean before
       compression.}
  \item[BIN 3 BINR=6\hfill]{shortens each row in image 3 by a 
       factor of 6.}
  \item[BIN 4 BIN=5 SR=200 SC=400\hfill]{
       takes the part of the image running 
       from row 200 to the end and from
       column 400 to the end, and compresses
       it by a factor of 2.  The pixels 
       before row 200 and column 400 are
       dropped from the image.}
\end{example}%
\lthtmlfigureZ
\lthtmlcheckvsize\clearpage}

\stepcounter{section}
{\newpage\clearpage
\lthtmlfigureA{command4685}%
\begin{command}
  \item[\textbf{Form: }SHIFT source {[DC=f]} {[DR=f]} {[SINC]} {[NORM]} 
       {[RMODEL=i]}\hfill]{}
  \item[]{{[CMODEL=i]} {[MEAN]} {[BILINEAR]}}
  \item[source]{tells SHIFT what image to work on,}
  \item[DC=f]{shifts the image by f columns, and}
  \item[DR=f]{shifts the image by f rows.}
  \item[BILINEAR]{enables the bilinear interpolation option}
  \item[SINC]{enables the sinc interpolation option}
  \item[NORM]{attempts to conserve counts.}
  \item[RMODEL=]{specifies a model for the row shift.}
  \item[CMODEL=]{specifies a model for the column shift.}
  \item[MEAN]{forces the mean shift of a model to be zero.}
\end{command}%
\lthtmlfigureZ
\lthtmlcheckvsize\clearpage}

{\newpage\clearpage
\lthtmlfigureA{example4707}%
\begin{example}
  \item[SHIFT 1 DR=0.5\hfill]{shifts image 1 by 0.5 rows using
     bilinear interpolation.}
  \item[SHIFT \$IM DR=-0.2 DC=14.3\hfill]{shifts image IM (IM a variable)
     by -0.2 rows, 14.3 columns.}
  \item[SHIFT 4 RMODEL=3 MEAN  \hfill]{shifts image 4 according to the
     model contained in spectrum 3.
     The mean shift is forced to be zero.}
\end{example}%
\lthtmlfigureZ
\lthtmlcheckvsize\clearpage}

\stepcounter{section}
{\newpage\clearpage
\lthtmlfigureA{command4717}%
\begin{command}
  \item[\textbf{Form: } CLIP source {[MAX=f]} {[MIN=f]} {[VMAX=f]} {[VMIN=f]} 
       {[BOX=n1,n2,...]}\hfill]{}
  \item[]{{[MASK]} {[MASKONLY]} {[RAD=r]} {[PHOT=r]} {[VMASK=f]}}
  \item[source]{the image or spectrum that CLIP works on.}
  \item[MAX=n]{sets the level above which pixels are adjusted.}
  \item[MIN=n]{sets the level below which pixels are adjusted.}
  \item[VMAX=n]{replaces all pixels above MAX by 'n'.}
  \item[VMIN=n]{replaces all pixels below MIN by 'n'.}
  \item[MASK]{Performs the MASK function on the clipped pixels.}
  \item[MASKONLY]{Performs the MASK function on the pixels which
       would be clipped without actually clipping them.}
  \item[VMASK=n]{replaces all masked pixels with n}
  \item[RAD=r]{replaces all pixels within r pixels of a clipped pixel}
  \item[PHOT=r]{replaces all pixels within r pixels of the location
       on the VISTA internal photometry list}
  \item[BOX=n1,n2...]{clips within boxes n1,n2, etc.}
\end{command}%
\lthtmlfigureZ
\lthtmlcheckvsize\clearpage}

{\newpage\clearpage
\lthtmlfigureA{example4744}%
\begin{example}
  \item[CLIP 1 MAX=110. VMAX=100.\hfill]{replace all pixels in image 1
       that are above 110 by 100.}
  \item[CLIP 1 MIN=SKY VMIN=0.0\hfill]{replace all pixels below SKY
       by 0.0.  SKY is a previously-defined variable.}
  \item[CLIP 1 MAX=110. VMAX=100. MIN=SKY VMIN=0.0\hfill]{
       does the same as examples 1 and 2 simultaneously.}
  \item[CLIP 1 VMAX=100.\hfill]{replaces all pixels above 100 by 100.0}
  \item[CLIP 1 VMAX=100. BOX=6\hfill]{does the same as example 4,
       but only in box 6.}
  \item[CLIP 1\hfill]{sets all negative pixels in image 1 to zero.}
  \item[CLIP 1 MIN=40.\hfill]{sets all pixels below 40 with 0.0}
  \item[CLIP 1 VMAX=1024.0 MASK\hfill]{clips pixels above 1024.0 to
       1024.0 and loads the positions of the pixels onto the mask list.}
  \item[CLIP 1 VMAX=1024.0 MASKONLY\hfill]{masks those pixels with values
       above 1024.  Does not modify the image.}
\end{example}%
\lthtmlfigureZ
\lthtmlcheckvsize\clearpage}

\stepcounter{section}
{\newpage\clearpage
\lthtmlfigureA{command4756}%
\begin{command}
  \item[\textbf{Form: }  COLFIX source {[BOX=n]} {[SPLINE]} {[ROWS]} 
       {[MAX=f]} {[SMOOTH]} {[MODE]}\hfill]{}
  \item[source]{tells VISTA what image to work on}
  \item[BOX=n]{only use the image within box n}
  \item[SPLINE]{fit spline to deferred charge buffer}
  \item[ROWS]{fix rows instead of columns}
  \item[MAX=f]{Ignore pixels above this level}
  \item[SMOOTH]{Median filter buffer}
  \item[MODE]{Use modal estimator}
\end{command}%
\lthtmlfigureZ
\lthtmlcheckvsize\clearpage}

\stepcounter{section}
{\newpage\clearpage
\lthtmlfigureA{command4775}%
\begin{command}
  \item[\textbf{Form: }SMOOTH source {[FW=f]} {[FWC=f]} {[FWR=f]} 
       {[BOXCAR]} {[RUNMEAN]} {[WID=]}\hfill]{}
  \item[source]{is the object being smoothed,}
  \item[FW=f]{sets the full width of the (Gaussian or boxcar)
       smoothing function to be f pixels, and}
  \item[FWC= and FWR=]{set the full width for the (Gaussian or boxcar)
       smoothing function to have a different width in rows or columns.}
  \item[BOXCAR]{convolve with a box function instead of a Gaussian.}
  \item[RUNMEAN]{convolve with a box function, but using a
       running-mean algorithm that is faster than the BOXCAR option. 
       The results are identical for odd-sized boxes.}
\end{command}%
\lthtmlfigureZ
\lthtmlcheckvsize\clearpage}

{\newpage\clearpage
\lthtmlfigureA{example4791}%
\begin{example}
  \item[SMOOTH 2 FW=8.5\hfill]{Smoothes image 2 with a Gaussian having
       full width 8.5 pixels.  The width is
       the same in the row and column directions.}
  \item[SMOOTH 2 FWR=8.5 FWC=8.0\hfill]{Smoothes image 2 with a Gaussian
       having full width = 8.5 rows and 8.0 columns.}
  \item[SMOOTH 5 FWR=6.2\hfill]{Smoothes each column individually with a
       Gaussian having full width 6.2 rows.}
  \item[SMOOTH 2 FW=5 BOXCAR\hfill]{Convolve with a box of size 5 pixels.}
\end{example}%
\lthtmlfigureZ
\lthtmlcheckvsize\clearpage}

\stepcounter{section}
{\newpage\clearpage
\lthtmlfigureA{command4800}%
\begin{command}
  \item[\textbf{Form: }ZAP source {[SIG=f]} {[SIZE=s]} {[SIZE=r,c]} 
       {[BOX=n]} {[TTY]} {[MASK]} {[SURGICAL]} {[COLS]} \hfill]{}
  \item[source]{is the} image being worked on.
  \item[SIZE=s]{sets the size (full width) of a square median
       window around each pixel to be used in the median filter
       computation.  Default is a 5x5 pixel square box.}
  \item[SIZE=r,c]{sets the size of a rectangular median window
       around each pixel to be 'r' rows by 'c' columns wide.  Default
       is a 5x5 pixel square box}
  \item[SIG=f]{specifies the rejection threshold in units
       of the standard deviation within SIZE.  Default is 5}.
  \item[BOX=n]{tells ZAP to work only on those parts of the image
       within the specified box}.
  \item[TTY]{print verbose output while zapping}
  \item[MASK]{masks pixels rather than replacing them with median}
  \item[SURGICAL]{only edit/mask the central pixel of the running
       median window.  Default edits all pixels within the box that
       meet exceed the filtering threshold.}
  \item[COLS]{will ZAP each row, replacing all pixels on that row with the median}
\end{command}%
\lthtmlfigureZ
\lthtmlcheckvsize\clearpage}

{\newpage\clearpage
\lthtmlfigureA{example4823}%
\begin{example}
  \item[ZAP 1\hfill]{Do the filtering with 5 by 5 square,
       adjusting pixels that are 5 sigma away from the median.}
  \item[ZAP 1 BOX=4\hfill]{does the same as example 1, but only in the
       region defined by box 4.}
  \item[ZAP 1 SIZE=7\hfill]{considers a 7 by 7 box at each pixel.}
  \item[ZAP 5 SIZE=1,15\hfill]{considers a box that is 1 row by 15 columns
       in size.}
  \item[ZAP 1 TTY\hfill]{does the same as example 1, but shows the
       coordinates of the zapped pixels on the screen.}
\end{example}%
\lthtmlfigureZ
\lthtmlcheckvsize\clearpage}

\stepcounter{section}
{\newpage\clearpage
\lthtmlfigureA{command4834}%
\begin{command}
  \item[\textbf{Form: }TVZAP {[SIG=f]} {[SIZE=s]} {[SIZE=r,c]} 
       {[SEARCH=s]} {[TTY]} {[MASK]} {[SURGICAL]}\hfill]{}
  \item[SIG=f]{specifies the rejection threshold in units
       of the standard deviation within SIZE.  Default is 5}.
  \item[SIZE=s]{sets the size (full width) of a square median
       window around each pixel to be used in the median filter
       computation.  Default is a 5x5 pixel square box.}
  \item[SIZE=r,c]{sets the size of a rectangular median window
       around each pixel to be 'r' rows by 'c' columns wide.  Default
       is a 5x5 pixel square box}
  \item[TTY]{print verbose output while zapping}
  \item[MASK]{masks pixels rather than replacing them with median}
  \item[SURGICAL]{only edit/mask the central pixel of the running
       median window.  Default edits all pixels within the box that
       meet exceed the filtering threshold.}
\end{command}%
\lthtmlfigureZ
\lthtmlcheckvsize\clearpage}

\stepcounter{section}
{\newpage\clearpage
\lthtmlfigureA{command4858}%
\begin{command}
  \item[\textbf{Form: }MEDIAN dest im1 im2 im3 {[im4 im5 ...]} {[TTY]} 
       {[NOMEAN]}\hfill]{}
  \item[dest]{is the buffer holding the median}
  \item[im1 im2 ...]{are the input images for computing the median}
  \item[TTY]{provide output of the progress of the computation.}
  \item[NOMEAN]{tells the program not to scale each image by
       its mean before taking the median}
\end{command}%
\lthtmlfigureZ
\lthtmlcheckvsize\clearpage}

{\newpage\clearpage
\lthtmlfigureA{example4870}%
\begin{example}
  \item[MEDIAN 11 1 2 3 4 5 TTY\hfill]{Loads into buffer 11 the median 
       of the images in buffers 1 through 5.}
\end{example}%
\lthtmlfigureZ
\lthtmlcheckvsize\clearpage}

\stepcounter{section}
{\newpage\clearpage
\lthtmlfigureA{command4876}%
\begin{command}
  \item[\textbf{Form: }BIGMEDIAN dest LIST=file {[BIAS=file]} 
       {[NOMEAN]}\hfill]{}
  \item[dest]{is the buffer which will hold the median}
  \item[LIST=file]{specifies the filename of a file which contains
       names of individual FITS files to median. Optionally,
       this file can contain normalization constants and
       zero points for each individual file}
  \item[BIAS=file]{specifies a FITS file with a bias frame to subtract
       from each input image before performing the median}
  \item[NOMEAN]{tells the program not to scale each image 
       before taking the median}
\end{command}%
\lthtmlfigureZ
\lthtmlcheckvsize\clearpage}

{\newpage\clearpage
\lthtmlfigureA{hanging4886}%
\begin{hanging}
  \item{image001.fits}
  \item{image002.fits}
  \item{image003.fits}
\end{hanging}%
\lthtmlfigureZ
\lthtmlcheckvsize\clearpage}

{\newpage\clearpage
\lthtmlfigureA{hanging4891}%
\begin{hanging}
  \item{image001.fits    1.   0.}
  \item{image002.fits   20.   0.}
  \item{image003.fits   20.   5.}
\end{hanging}%
\lthtmlfigureZ
\lthtmlcheckvsize\clearpage}

\stepcounter{section}
{\newpage\clearpage
\lthtmlfigureA{command4898}%
\begin{command}
  \item[\textbf{Form: }PICCRS dest LIST=file {[BIAS=file]} {[RN=]} {[GAIN=]} 
       {[TP=]} {[TN=]} {[SKY]}\hfill]{}
  \item{{[TD=]} {[F=]} {[BSAT=]} {[BLANK=]} }
  \item{{[NEG]} {[SIG]} {[FITS]} {[MASK]} {[MEDIAN]} {[MIN=]}}
  \item[dest]{is the buffer which will hold the median}
  \item[LIST=file]{specifies the filename of a file which contains
       names of individual FITS files to median. Optionally,
       this file can contain normalization constants and
       zero points for each individual file}
  \item[BIAS=file]{specifies a FITS file with a bias frame to subtract
       from each input image before performing the median}
  \item[RN=]{gives the readout noise to use in the noise computation}
  \item[GAIN=]{gives the gain to use in the noise computation}
  \item[TP=,TN=,TD=]{specify the thresholding parameter for primary events,
                        adjacent neighbors, and diagonal neighbors}
  \item[F=]{gives the fraction error parameter (see below) }
  \item[BSAT=]{specifies the saturation value}
  \item[BLANK=]{specifies the value of BLANK to use in output image}
  \item[NEG]{tells PICCRS to test pixels with largest ABSOLUTE
       deviation from mean first, rather than largest
       POSITIVE deviation from mean}
  \item[SIG]{has PICCRS load the output variance image instead of 
       the mean image}
  \item[WFPC]{specifies input files are WFPC, not FITS format}
  \item[FITS]{specifies input files are FITS (default), not WFPC }
  \item[MASK]{tells PICCRS to evaluate all masked pixels as if
                        they were neighbor pixels}
  \item[MEDIAN]{tells PICCRS to take the median of the stack rather
                        than the weighted mean for the CR rejection ONLY}
  \item[MIN=]{specifies the lowest good data value. NOTE: default=0,
                        i.e. all negative pixels are considered to be bad}
  \item[SKY]{tells PICCRS to read separate sky frames and subtract these
            before combining, but uses sky to calculate variances}
\end{command}%
\lthtmlfigureZ
\lthtmlcheckvsize\clearpage}

\stepcounter{section}
{\newpage\clearpage
\lthtmlfigureA{command4940}%
\begin{command}
  \item[\textbf{Form: }SURFACE source {[BOX=n]} {[PLANE]} {[SUB]} 
       {[LOAD]}\hfill]{}
  \item{{[DIV]} {[MASK]} {[NOZERO]} {[PIX=N]} (redirection)}
  \item[source]{is the image to which the surface is being fit,}
  \item[BOX=n]{tells VISTA to do the fit only in box 'n',}
  \item[PLANE]{fits a plane, rather than a second order surface,}
  \item[SUB]{has the program subtract the best-fit surface
       from the original image,}
  \item[DIV]{replaces the original image with itself divided by the 
       best-fit surface,}
  \item[MASK]{tells VISTA to ignore masked pixels, and}
  \item[NOZERO]{suppresses rejection of pixels with zero value.}
  \item[PIX=n]{uses every n'th pixel for speed}
  \item[LOAD]{load variables with the surface fit}
\end{command}%
\lthtmlfigureZ
\lthtmlcheckvsize\clearpage}

{\newpage\clearpage
\lthtmlfigureA{example4963}%
\begin{example}
  \item[SURFACE 1\hfill]{replaces image 1 by the best-fitting
       polynomial.  Pixels with value zero are not included in the fit.}
  \item[SURFACE 1 NOZERO\hfill]{does the same as example 1, but this
       time ALL pixels are included in the fit.}
  \item[SURFACE 1 BOX=2\hfill]{does the fit only in BOX 2.}
  \item[SURFACE 1 MASK\hfill]{fits the best second-order surface, ignoring
       masked pixels.}
  \item[SURFACE 1 SUB\hfill]{subtracts the best fitting surface from image 1.}
  \item[SURFACE 1 PIX=5\hfill]{does the fit using every 25th pixel.}
  \item[SURFACE 1 PIX=5 LOAD\hfill]{does the same as example 6, loading the
       coefficients of the fit into variables.}
\end{example}%
\lthtmlfigureZ
\lthtmlcheckvsize\clearpage}

\stepcounter{section}
{\newpage\clearpage
\lthtmlfigureA{command4977}%
\begin{command}
  \item[\textbf{Form: } SPLINE source {[R=r1,r2,...]} {[C=c1,c2...]} 
       {[W=w1,w2,...]}\hfill]{}
  \item{{[AVG=a]} {[SUB]} {[DIV]}}
  \item[source]{is the image or spectrum SPLINE works on}
  \item[R=r1,r2,...]{replaces each column in the image 'source'
       with the best spline through the knot points at rows r1,r2,...}
  \item[C=c1,c2,...]{replaces each row in the image (or uncalibrated 
       spectrum) 'source' with the best spline 
       through the knot points at columns c1,c2,...}
  \item[W=w1,w2,...]{replaces a wavelength calibrated spectrum
       'source' by the best-fitting spline with knot
       points at wavelengths w1,w2,...}
  \item[AVG=a]{averages a pixels or Angstroms around the knots.}
  \item[SUB]{subtracts the best fitting spline(s) from the 
       rows or columns, instead of replacing by the spline.}
  \item[DIV]{divides each row/column by the best fitting
       spline, instead of replacing by the spline.}
\end{command}%
\lthtmlfigureZ
\lthtmlcheckvsize\clearpage}

{\newpage\clearpage
\lthtmlfigureA{example4995}%
\begin{example}
  \item[SPLINE 11 W=5000,5500,6000,7000\hfill]{ replaces spectrum 11 with
       the best-fitting spline.  Knot points are at 5000, 5500, 6000, 7000
       Angstroms.}
\par
\item[SPLINE 11 W=5000,5500,6000,7000 AVG=40\hfill]{ does the same as
       example 1, but takes the average in 40 Angstrom bins around the
       designated knot points.  The averages are used in the spline fit.}
\par
\item[SPLINE 11 W=5000,5500,6000,7000 DIV\hfill]{ does the same ax
       example 1, but divides the spectrum by the best fitting spline
       instead of replacing by the spline.}
\par
\item[SPLINE 11 C=20,40,100,300,500 AVG=10\hfill]{ this form is used for
       spectra that are not wavelength calibrated.  It averages the
       spectrum in 10 channel bins around columns 20, 40, ..., and replaces
       the spectrum by the best fitting spline.}
\end{example}%
\lthtmlfigureZ
\lthtmlcheckvsize\clearpage}

{\newpage\clearpage
\lthtmlfigureA{example5001}%
\begin{example}
  \item[SPLINE 11 C=20,40,100,300,500 AVG=10\hfill]{ this command is the
       same as example 4 above.  When applied to an image, it replaces each
       row in the image by the spline which best fits that row. The knot
       points are taken to be averages of the image intensity in 10 channel
       bins around columns 20, 40,...}
\end{example}%
\lthtmlfigureZ
\lthtmlcheckvsize\clearpage}

\stepcounter{section}
{\newpage\clearpage
\lthtmlfigureA{command5006}%
\begin{command}
  \item[\textbf{Form: }CROSS dest source1 source2 {[BOX=n]} {[RAD=r]} {[RADR=r]} {[RADC=c]} {[CORR]}\hfill]{}
  \item[dest]{is the buffer in which the cross-correlation will be stored,}
  \item[source1,source2]{are the images to cross-correlate,}
  \item[BOX=n]{tells VISTA to use in box 'n',}
  \item[RAD=r]{defines the radius of the resulting cross-correlation image 
       (the image will run from row -r to row +r and from col -r to col +r).}
  \item[RADC= RADR=]{To specify two different radii for the cross-
       correlation.}
  \item[CORR]{Compute instead of the cross-correlation (XC=<AB>), 
              the correlation coefficient: (R=(XC-<A><B>)/SIG(A)*SIG(B))}
\end{command}%
\lthtmlfigureZ
\lthtmlcheckvsize\clearpage}

{\newpage\clearpage
\lthtmlfigureA{hanging5021}%
\begin{hanging}
  \item{CROSS 3 1 2 RAD=3}
  \item{ABX 3 HIGH\_COL=HC HIGH\_ROW=HR}
  \item{BOX 1 CC=HC CR=HR N=3  (or possibly N=5 or larger if cross correlation
             function is well behaved)}
  \item{WINDOW 3 BOX=1}
  \item{SURFACE 3 LOAD}
  \item{DENOM=4*COEFFC2*COEFFR2-COEFFRC*COEFFRC}
  \item{DELR=MIDR+(COEFFC*COEFFRC-2*COEFFR*COEFFC2)/DENOM}
  \item{DELC=MIDC+(COEFFR*COEFFRC-2*COEFFR2*COEFFC)/DENOM}
  \item{SHIFT 2 DR=DELR DC=DELC SINC}
\end{hanging}%
\lthtmlfigureZ
\lthtmlcheckvsize\clearpage}

\stepcounter{section}
{\newpage\clearpage
\lthtmlfigureA{command5036}%
\begin{command}
  \item[\textbf{Form: }INTERP imno {[BOX=b1,b2,...]} {[COL]} {[ROW]} {[ORD=n]} {[AVE=a]} {[MASK]}\hfill]{}
  \item[imno]{is the buffer on which interpolation will be performed,}
  \item[BOX=b1,b2,...]{is a list of boxes to be interpolated across,}
  \item[COL]{interpolates across bad columns,}
  \item[ROW]{interpolates across bad rows,}
  \item[ORD=n]{specifies order of polynomial interpolation,}
  \item[AVE=a]{specifies how many pixels on either side are to be averaged 
       to provide the interpolation,}
  \item[MASK]{interpolates across all masked pixels.}
\end{command}%
\lthtmlfigureZ
\lthtmlcheckvsize\clearpage}

{\newpage\clearpage
\lthtmlfigureA{example5054}%
\begin{example}
  \item[INTERP 4 BOX=2,5,3 COL\hfill]{Boxes 2, 3, and 5 have been defined
       to contain bad columns or sets of columns.  These boxes in image 4
       will be replaced by the average of the their two adjacent columns.}
\par
\item[INTERP 2 MASK ROW ORD=1\hfill]{Bad rows and/or pixels in image 2
       which have been masked are to be removed by linear interpolation.}
\par
\item[INTERP 17 BOX=3 COL AVE=4\hfill]{Bad columns in box 3 are to be
       removed from image 17, at each row the bad columns will be replaced
       by the average of the eight adjacent columns (four on either side).}
\par
\item[INTERP 17 BOX=3 COL AVE=4 ORD=2\hfill]{Same as above but the bad
       pixels will be replaced by a 2nd-order polynomial fit to the eight
       adjacent columns at each row.}
\end{example}%
\lthtmlfigureZ
\lthtmlcheckvsize\clearpage}

\stepcounter{section}
{\newpage\clearpage
\lthtmlfigureA{command5062}%
\begin{command}
  \item[\textbf{Form: } REGISTAR imbuf {[RADIUS=n]} {[DR=r]} {[DC=c]} 
       {[RSHIFT=rs]} {[CSHIFT=cs]}\hfill]{}
       \item{{[SINC]} {[REJECT=sigma]} {[LOG=xxx]} {[INT]}}
  \item[imbuf]{buffer with image to be registered ("target image").}
  \item[RADIUS]{size of the region used in the multiple centroiding.
       (DEFAULT: fit on a 3 by 3 box)}
  \item[DC, DR]{offsets to apply to the positions of the stars in the
       reference list before calculating the relative shift.}
  \item[RSHIFT,CSHIFT]{give limits on how far the centroid can deviate 
       from the original position.}
  \item[REJECT]{the threshold sigma level at which to reject stars. The
       default is 4-sigma.}
  \item[SINC]{use sinc interpolation. Default is 4-th order Lagrangian}
  \item[LOG=xxx]{The name of a registration log file to open.  If it exists,
       the results are appended.  }
  \item[INT]{interactively control the iterations.}
\end{command}%
\lthtmlfigureZ
\lthtmlcheckvsize\clearpage}

{\newpage\clearpage
\lthtmlfigureA{example5084}%
\begin{example}
  \item[REGISTAR 2 RADIUS=2\hfill]{Register the image in buffer 2 with
       respect to stars in the current photometry list, and restrict the
       search for the stars to a 5x5 region around the fiducial centers.}
\par
\item[REGISTAR 2 LOG=CIRCINUS \hfill]{Register the image in buffer 2 wrt
       stars in the current photometry list and record the results in a
       file called CIRCINUS.REG in the default data directory.}
\par
\item[GET ORIONHA.PHO ; REGISTAR 3 REJECT=3\hfill]{ Read in the
       previously defined photometry list called ORIONHA.PHO, and register
       the image in buffer 3 relative to the stars in this file, adopting a
       rejection threshold of 3 sigma for outlying centroid values.}
\par
\item[REGISTAR 3 DR=4. INT\hfill]{Registar the image in buffer 3 wrt to
       the stars in the current photometry list.  Image 3 is known to be
       shifted at least 4 pixels relative to this list already.  Monitor
       the iterations by hand.}
\end{example}%
\lthtmlfigureZ
\lthtmlcheckvsize\clearpage}

\stepcounter{section}
{\newpage\clearpage
\lthtmlfigureA{command5091}%
\begin{command}
  \item[\textbf{Form: } DSSCOORD imbuf {[PHOT]} {[P=(c,r)]} {[V=n]} {[TTY]} 
       {[STANDARD]} {[redirection]}\hfill]{}
  \item[imbuf]{buffer with the Digitized Sky Survey Image}
  \item[PHOT]{use star coordinates in the currently active
       photometry file (e.g., from MARKSTAR or GET PHOT='}
  \item[P=(c,r)]{pixel coordinates on the image to be converted to
       (RA,Dec)}
  \item[V=n]{use Column and Row coordinates using VISTA variable
       Cn and Rn (e.g., from the TV display cursor)}
  \item[TTY]{Prompt for user input of pixel coordinates}
  \item[STANDARD]{Also print standard coordinates (xi,eta)}
\end{command}%
\lthtmlfigureZ
\lthtmlcheckvsize\clearpage}

\stepcounter{section}
{\newpage\clearpage
\lthtmlfigureA{command5109}%
\begin{command}
  \item[\textbf{Form: }ATODSIM imno FILE=name BIAS=bias\hfill]{}
  \item[ATODFIX imno FILE=name BIAS=bias\hfill]{}
  \item[imno]{is the image buffer}
  \item[FILE=name]{Gives the name of the A/D bit error file.}
  \item[BIAS=bias]{Specifies bias level to add (subtract)}
\end{command}%
\lthtmlfigureZ
\lthtmlcheckvsize\clearpage}

\stepcounter{section}
{\newpage\clearpage
\lthtmlfigureA{command5121}%
\begin{command}
  \item[\textbf{Form: } LINCOMB source BUF=i1,i2,... {[BOX=box]} 
       {[WBUF=wbuf]} {[LOAD]}\hfill]{}
  \item{{[DIV]} {[SUB]} {[CONST=c]} {[SILENT]}}
  \item[source]{is the image to fit}
  \item[BUF=i1,i2,...]{gives the buffer numbers of the images to form
       the linear combination}
  \item[BOX=box]{operates only in BOX box}
  \item[WBUF=buf]{uses buffer buf as weights for the fit}
  \item[SUB]{subtracts the fit from the input image}
  \item[DIV]{divides the image by the fit}
  \item[LOAD]{load the fit coefficients into VISTA variables
       L0, L1, L2, ...., errors into DL0, DL1, ..., and
       unbiased standard deviation from fit into STDDEV}
  \item[SILENT]{suppress terminal output}
  \item[CONST=c]{do not fit the constant term, use c instead}
\end{command}%
\lthtmlfigureZ
\lthtmlcheckvsize\clearpage}

\stepcounter{section}
{\newpage\clearpage
\lthtmlfigureA{command5144}%
\begin{command}
  \item[\textbf{Form: }BL source {[JUMP]} {[KEEP]} {[INVERT]}\hfill]{}
  \item[source]{tells VISTA what image to work on.}
  \item[JUMP]{implements detection of jumps in the baseline column.}
  \item[KEEP]{leaves baseline column unchanged. Otherwise
       baseline will be replaced by best fitting baseline.}
  \item[INVERT]{reverse baseline process (if you goofed), only
                will work properly if KEEP was specified during
                original BL command.}
\end{command}%
\lthtmlfigureZ
\lthtmlcheckvsize\clearpage}

\stepcounter{chapter}
{\newpage\clearpage
\lthtmlfigureA{example5510}%
\begin{example}
  \item[FFT]{Compute a Forward Fast-Fourier Transform}
  \item[IFFT]{Compute an Inverse FFT}
  \item[POWERS]{Generate a Power Spectrum (Periodogram Estimate)}
  \item[CABS]{Absolute Value of a Complex Image}
  \item[CMUL/CDIV]{Complex Image Multiplication}
  \item[CMPLX/IMAG/REAL/CONJ]{Complex Image Operators}
\end{example}%
\lthtmlfigureZ
\lthtmlcheckvsize\clearpage}

\stepcounter{section}
{\newpage\clearpage
\lthtmlfigureA{command5519}%
\begin{command}
  \item[Form: FFT  {[dest]} source {[ONEDIM]}\hfill]{}
  \item[source]{buffer holding the real image to be transformed.}
  \item[dest]{buffer holding the (complex) Fourier transform. If dest
       is not specified, 'source' will hold its own transform.}
  \item[ONEDIM]{perform a 1D direct transform for each row of 'source'
       rather than the 2D transform of the whole image.}
\end{command}%
\lthtmlfigureZ
\lthtmlcheckvsize\clearpage}

{\newpage\clearpage
\lthtmlfigureA{example5528}%
\begin{example}
  \item[FFT 2 3\hfill]{Perform the 2D direct-fourier-transform of 
       image in buffer 3 and save the transform image in buffer 2}
  \item[FFT 5\hfill]{Replace the image in buffer 5 by its fourier transform. 
       If image 5 has more than one row a 2D transform, 1d otherwise).}
  \item[FFT 1 ONEDIM\hfill]{Replace each row of buffer 1 by its 1D fourier 
       transform.}
\end{example}%
\lthtmlfigureZ
\lthtmlcheckvsize\clearpage}

\stepcounter{section}
{\newpage\clearpage
\lthtmlfigureA{command5534}%
\begin{command}
  \item[Form: IFFT  {[dest]} source {[ONEDIM]}\hfill]{}
  \item[source]{buffer holding the fourier transform of a real image.}
  \item[dest]{will hold the inverse fourier transform of 'source'. If
       'dest' is not specified, 'source' will be substituted by its
       transform.}
  \item[ONEDIM]{perform a 1D inverse transform for each row of 'source'
       rather than the 2D inverse transform of the whole image.}
\end{command}%
\lthtmlfigureZ
\lthtmlcheckvsize\clearpage}

\stepcounter{section}
{\newpage\clearpage
\lthtmlfigureA{command5543}%
\begin{command}
  \item[Form: POWERS dest source\hfill]{}
  \item[source]{buffer with the Fourier transform of a real image.}
  \item[dest]{image buffer to hold the power spectrum}
\end{command}%
\lthtmlfigureZ
\lthtmlcheckvsize\clearpage}

\stepcounter{section}
{\newpage\clearpage
\lthtmlfigureA{command5549}%
\begin{command}
  \item[Form: CABS dest source\hfill]{}
\end{command}%
\lthtmlfigureZ
\lthtmlcheckvsize\clearpage}

\stepcounter{section}
{\newpage\clearpage
\lthtmlfigureA{command5553}%
\begin{command}
  \item[CMUL im1 im2\hfill]{Complex multiplication of images.}
  \item[CDIV im1 im2\hfill]{Complex division of images.}
  \item[CABS dest source   \hfill]{Complex Modulus of an image.}
  \item[IMAG dest source   \hfill]{Extract imaginary part from an image.}
  \item[REAL dest source   \hfill]{Extract real part from an image.}
  \item[CMPLX dest im2 im3 \hfill]{Synthesize a complex image}.
  \item[CONJ dest source   \hfill]{Complex Conjugate of an image.}
\end{command}%
\lthtmlfigureZ
\lthtmlcheckvsize\clearpage}

\stepcounter{section}
{\newpage\clearpage
\lthtmlfigureA{command5563}%
\begin{command}
  \item[Form: CMUL dest source\hfill]{}
  \item[CDIV dest source\hfill]{}
\end{command}%
\lthtmlfigureZ
\lthtmlcheckvsize\clearpage}

\stepcounter{section}
{\newpage\clearpage
\lthtmlfigureA{command5568}%
\begin{command}
  \item[Form: CMPLX dest real imag\hfill]{}
  \item[IMAG dest source\hfill]{}
  \item[REAL dest source\hfill]{}
  \item[CONJ dest source\hfill]{}
\end{command}%
\lthtmlfigureZ
\lthtmlcheckvsize\clearpage}

\stepcounter{chapter}
\stepcounter{section}
{\newpage\clearpage
\lthtmlfigureA{example5632}%
\begin{example}
  \item[MASH\hfill]{extract a spectrum from an image by summing (mashing)
       adjacent image rows or columns.}
  \item[SPECTROID\hfill]{extract a spectrum that curves across an image 
       (also used to map curved extraction apertures).}
  \item[EXTRACT\hfill]{optimal extraction of the spectra of faint or
       moderately faint point-sources (Horne Algorithm).}
  \item[LINEID\hfill]{identify lines in a wavelength calibration spectrum.}
  \item[WSCALE\hfill]{compute wavelength calibration solutions from
       identified lamp lines.}
  \item[COPW\hfill]{apply (copy) a wavelength scale to a spectrum.}
  \item[ALIGN\hfill]{re-sample a spectrum onto a linearized wavelength
       scale (or adjust an existing spectrum).}
  \item[SKYLINE\hfill]{re-calibrate wavelengths using night sky lines.}
  \item[EXTINCT\hfill]{correct a spectrum for atmospheric extinction.}
  \item[SETUP\hfill]{set the Longitude and Latitude for extinction correction}
  \item[FLUXSTAR\hfill]{compute flux calibration curves from observations 
       of flux standard stars.}
  \item[FLUX\hfill]{apply flux calibration to a spectrum.}
  \item[POLY\hfill]{fit a polynomial to a spectrum.}
  \item[STRETCH\hfill]{expand (stretch) and spectrum into a 2-D image.}
  \item[ISPLINE\hfill]{interactively fit a spline to a spectrum.}
  \item[ROWFIT\hfill]{fit a polynomial to each row of an image.}
  \item[FINDSHIFT\hfill]{find relative shifts between two spectra.}
  \item[FINDPEAK\hfill]{find the position of a spectrum along rows or 
       columns.}
  \item[SPINDEX\hfill]{measures equivalent widths at each row of an image}
  \item[FQUO\hfill]{fits a broadening function to Fourier quotient spectra}
  \item[EXTSPEC\hfill]{extract a given order from a wavelength-calibrated
       echelle file.}
  \item[EWAVE\hfill]{identify lines in an arc spectrum obtained with the
        Hamilton spectrograph and compute wavelength scale from line
        identifications}
\end{example}%
\lthtmlfigureZ
\lthtmlcheckvsize\clearpage}

\stepcounter{section}
{\newpage\clearpage
\lthtmlfigureA{command5659}%
\begin{command}
  \item[Form:MASH dest source SP=i1,i2 {[BK=b1,b2]} {[COLS]} {[COL]} 
       {[COL=c1,c2]} {[ROW=r1,r2]}\hfill]{}
  \item{{[SKY=s]} {[NORM]} {[REFLAT]} {[SUB]} {[MASK]}}
  \item[dest\hfill]{is the buffer which will hold the resulting spectrum.}
  \item[source\hfill]{is the image from which the spectrum is being made.}
  \item[COLS or COL\hfill]{says that the spectrum runs down columns.}
  \item[SP=\hfill]{delimits the rows or columns in the image 
        'source' which are used to make the spectrum.}
  \item[BK=\hfill]{delimits rows (columns) used for the background}
  \item[COL=\hfill]{takes the spectrum from the specified columns.}
  \item[ROW=\hfill]{takes the spectrum from the specified rows.}
  \item[SKY=\hfill]{saves the sky rows (from BK= keyword) as a
        spectrum in buffer s.}
  \item[NORM\hfill]{averages the added rows.}
  \item[REFLAT\hfill]{fits each background column (row) with a parabola, 
        and uses this as the background.  Alters source buffer.}
  \item[SUB\hfill]{subtracts the average background spectrum from
        the original image.}
  \item[MASK\hfill]{uses the image mask defined with the MASK
       command to ignore bad pixels and/or image}
       segments.  This works only with REFLAT option.
\end{command}%
\lthtmlfigureZ
\lthtmlcheckvsize\clearpage}

{\newpage\clearpage
\lthtmlfigureA{example5686}%
\begin{example}
  \item[MASH 2 1 SP=50,55 BK=20,30 BK=75,80\hfill]{ produces spectrum 2
       from image 1.  The spectrum is taken from rows 50-55.  The
       background is rows 20-30 and 75-80.}
\par
\item[MASH 2 1 SP=50,55 BK=20,30 BK=75,80 COL=100,200\hfill]{ does the
       same as example 1, but takes the spectrum only from columns 100 to
       200.}
\par
\item[MASH 2 1 SP=50,55 BK=20,30 BK=75,80 REFLAT\hfill]{ A polynomial is
       fit down each column, using the rows selected for the background.
       Each column in the image is divided by the polynomial which was fit
       in that column.  The MASH then proceeds as in example 1.}
\end{example}%
\lthtmlfigureZ
\lthtmlcheckvsize\clearpage}

\stepcounter{section}
{\newpage\clearpage
\lthtmlfigureA{command5694}%
\begin{command}
  \item[Form:SPECTROID dest source {[SP=s1,s2]} {[BK=b1,b2]} 
       {[SPW=ds]}\hfill]{}
  \item{{[BKW=db]} {[MODEL=m]} {[DLOC=d]} {[LOC=r]} {[NOSHIFT]} {[FIT=p1,p2]}}
  \item{{[TRACE]} {[SELF]} {[LOAD]} {[NOMASH]} {[TAGALONG]} {[TAG]} 
       {[MOMENTS]}}
  \item[dest]{is the buffer which will hold the map (which is a spectrum)}
  \item[source]{is the image from which the map is being made,}
  \item[SP=]{delimits the rows in the image 'source' which
       are used to make the spectrum,}
  \item[SPW=]{alternatively, specifies the width of a
       symmetric spectrum window centered on the centroid,}
  \item[BK=]{delimits rows (or columns) used to determine the background,}
  \item[BKW=]{alternatively, specifies the width of the
       symmetric background window,}
  \item[LOC=]{specifies the starting row (or column) for calculating
       the centroid}
  \item[DLOC=]{alternatively, specifies a relative shift in
       rows (or columns) from the centroid model (see MODEL=),}
  \item[MODEL=]{specifies a buffer containing a model to
       use as an approximation to the location of the spectrum.}
  \item[NOMASH]{inhibits the mashing feature -- the destination
       buffer will then contain the centroids themselves.}
  \item[NOSHIFT]{specifies that no shift from the model will be calculated,}
  \item[FIT=]{specifies the range in pixels over which a
       median shift between the model and the calculated centroid will 
       be determined.}
  \item[TRACE]{allows the tracing of the centroiding procedure
       for debugging purposes.}
  \item[SELF]{tells SPECTROID to do the mashing without a model.}
  \item[LOAD]{loads the VISTA variable SHIFT with the calculated value.}
  \item[TAGALONG or TAG]{implements the "following" algorithm in which a
       filtered mean of the last 15 centroids is used as the starting guess
       for the next centroid.  (A centroid is rejected from the mean if it
       differs by more than 0.5 pixels from the previous mean.)}
  \item[MOMENTS]{calculate the first five moments of the spectrum,
       including both the area and the centroid.}
\end{command}%
\lthtmlfigureZ
\lthtmlcheckvsize\clearpage}

{\newpage\clearpage
\lthtmlfigureA{example5731}%
\begin{example}
  \item[SPECTROID 1 2 SPW=10 BKW=20 LOC=250 NOMASH\hfill]{ SPECTROID in its
       simplest form, which loads the centroid as a function of column
       number into buffer 1. The spectrum window is ten pixels wide, and
       the background window twenty. Row 250 was specified as an estimate
       of the centroid.}
\par
\item[SPECTROID 1 2 SPW=9.4 BK=-6,9 MODEL=5 TRACE\hfill]{ Since the NOMASH
       keyword was not specified the result of this command will be the
       production of a total intensity spectrum.  Note that the spectrum
       window is not an integral number of pixels wide, and the BK=
       background has been used to specify an asymmetric (with respect to
       the centroid) background window.  Also note that the BK= keyword
       uses relative row numbers, not absolute.  Spectrum buffer 5 will be
       used as a model for the curvature of the spectrum, and a shift
       between this model and the calculated centroid will be determined
       and used.  The TRACE keyword will cause a (possibly lengthy) tracing
       of the centroiding process to be printed out.}
\par
\item[SPECTROID 6 2 MODEL=5 SPW=3 BK=-25,-20 BK=20,25 NOMASH\hfill]
  \item{This command uses model 5 as the starting estimate for the
       centroid, but recalculates the centroid using the new spectrum and
       background window specifications.  Again, the NOMASH keyword
       inhibits the mashing feature.}
\end{example}%
\lthtmlfigureZ
\lthtmlcheckvsize\clearpage}

\stepcounter{section}
{\newpage\clearpage
\lthtmlfigureA{command5738}%
\begin{command}
  \item[Form:EXTRACT spec image SP=s1,s2 BK=b1,b2 BK=b3,b4 {[SKY=s]} 
       {[VAR=v]} {[SUB]}\hfill]{}
  \item[{[SORDER=sord]} {[PORDER=pord]}{[RONOISE=r]} 
        {[EPERDN=eperdn]} \hfill]{}
  \item[spec\hfill]{is the buffer to contain the resulting spectrum,}
  \item[image\hfill]{is the image from which the spectrum is created,}
  \item[SP=s1,s2\hfill]{specifies the range of spectrum rows,}
  \item[BK=b1,b2\hfill]{specifies the background rows (several background
       regions may be specified),}
  \item[SKY=s\hfill]{places the sky spectrum (at the position of the
       middle of the spectrum) into buffer s,}
  \item[VAR=v\hfill]{places the estimated variance of the extracted
       spectrum into buffer v,}
  \item[SUB\hfill]{subtracts the fitted sky from the original image,}
  \item[SORDER=sord\hfill]{specifies the order of polynomial to be fit to the
       sky columns (default=2, parabolic fit),}
  \item[PORDER=pord\hfill]{specifies the order of polynomial to be fit to the
       PSF profile (default=2, parabolic),}
  \item[RONOISE=r\hfill]{changes the value used for the read-out noise of the
       detector in electrons (default=7),}
  \item[EPERDN=eperdn\hfill]{changes the inverse gain (in electrons per DN) 
       used for the detector (default=2.5 electrons/DN).}
\end{command}%
\lthtmlfigureZ
\lthtmlcheckvsize\clearpage}

{\newpage\clearpage
\lthtmlfigureA{example5763}%
\begin{example}
  \item[EXTRACT 10 1 SP=134,138 BK=120,130 BK=142,152\hfill]{ extracts a
       spectrum from rows 134-138 in image 1 and places the result in
       buffer 10.  Rows 120-130 and 142-152 are used as the background
       rows.}
\par
\item[EXTRACT 14 4 SP=100,104 BK=90,114 SUB SKY=34 VAR=24
       SORDER=3\hfill]{ extracts a spectrum from rows 100-104 of image 4.
       Note that in this case the BK= specifications straddle the spectrum
       rows (this will be handled correctly, as it is in MASH).  The sky is
       fit with cubic polynomials (SORDER=3) and is the fit is subtracted
       from the image 4.  The object spectrum is placed into buffer 14, the
       sky spectrum into 34, and the variance of the object spectrum into
       buffer 24.}
\end{example}%
\lthtmlfigureZ
\lthtmlcheckvsize\clearpage}

\stepcounter{section}
{\newpage\clearpage
\lthtmlfigureA{command5770}%
\begin{command}
  \item[Form: LINEID source {[FILE=xxx]} {[ADD]} {[TTY]} {[INT]} {[CEN=]} 
       {[DISP=]}\hfill]{}
  \item{{[LAMBDA=wave,pix{[,uncer]}]} {[redirection]}}
  \item[source]{selects the spectrum used in the calibration,}
  \item[FILE=]{is a file of line identifications,}
  \item[ADD]{will add newly identified lines to a list
       of lines from a previous execution of LINEID,}
  \item[TTY]{sends more extensive output to the terminal,}
  \item[INT]{allows interactive selection of lines,}
  \item[CEN=]{allows the user to guess a central wavelength,}
  \item[DISP=]{allows the user to guess a dispersion.}
  \item[LAMBDA=]{allow the user to preset a (lambda,pixel) pair}
\end{command}%
\lthtmlfigureZ
\lthtmlcheckvsize\clearpage}

{\newpage\clearpage
\lthtmlfigureA{example5789}%
\begin{example}
  \item[LINEID 4 FILE=NEON\hfill]{Matches lines in spectrum 4 with the
       wavelengths supplied in ccd/spec/NEON.WAV.  The resulting
       identifications replace any previously saved identifications in the
       common block.}
\par
\item[LINEID 4 FILE=NEON $>$id.txt\hfill]{Does the same as example 1, but
       this time sending the output to the file id.txt in the current
       working directory.}
\par
\item[LINEID 4 INT FILE=NEON ; LINEID 5 TTY FILE=MERCURY ADD\hfill]{The
       first command does the same as example 1, but allows the user to
       make additional line identifications after the command has made its
       best attempt.  The second command works on a separate mercury lamp
       spectrum in buffer 5.  The identifications are added to those for
       the neon spectrum.}
\par
\item[LINEID 10 FILE=THAR CEN=6715.4 DISP=0.134\hfill]{Attempts a
       wavelength calibration using the line list THAR.WAV and starting
       guesses for the central wavelength of 6715.4 Angstroms and for the
       dispersion of 0.134 Angstroms.}
\end{example}%
\lthtmlfigureZ
\lthtmlcheckvsize\clearpage}

\stepcounter{section}
{\newpage\clearpage
\lthtmlfigureA{command5801}%
\begin{command}
  \item[Form: WSCALE dest {[ORD=n]} {[TTY]} {[INT]} {[redirection]}\hfill]{}
  \item[dest]{selects the spectrum for which the wavelength scale is to apply,}
  \item[ORD=]{determines the order of the polynomial 
       expressing the relation between channel number and wavelength,}
  \item[TTY]{sends more extensive output to the terminal,}
  \item[INT]{allows interactive selection of lines.  INT implies TTY.}
\end{command}%
\lthtmlfigureZ
\lthtmlcheckvsize\clearpage}

{\newpage\clearpage
\lthtmlfigureA{example5813}%
\begin{example}
  \item[WSCALE 4 ORD=2\hfill]{ Computes the wavelength scale for spectrum
       4, doing a second-order fit to wavelength versus channel number. The
       resulting wavelength scale is printed on the terminal.}
\par
\item[WSCALE 4 ORD=2 $>$fit.txt:\hfill]{ Does the same as example 1, but
       this time sending the output to the file fit.txt in the
       current working directory.}
\par
\item[WSCALE 4 ORD=1 INT\hfill]{ Does a first order fit to wavelength
       versus channel number for spectrum 4.  The user can interactively
       give weights to the identified lines being fit (0 weight discards a
       line).}
\end{example}%
\lthtmlfigureZ
\lthtmlcheckvsize\clearpage}

\stepcounter{section}
{\newpage\clearpage
\lthtmlfigureA{command5821}%
\begin{command}
  \item[Form:COPW dest source {[source2]}\hfill]{}
  \item[dest]{is the destination spectrum to which a wavelength scale is to
       be copied.}
  \item[source]{is the source spectrum from which the wavelength scale is
       to be copied}
  \item[source2]{is a second source spectrum to be used for wavelength
       interpolation}
\end{command}%
\lthtmlfigureZ
\lthtmlcheckvsize\clearpage}

{\newpage\clearpage
\lthtmlfigureA{example5829}%
\begin{example}
  \item[COPW 3 2\hfill]{transfer the wavelength scale of spectrum 3 to
       spectrum 2}
\par
\item[COPW 1 3 11\hfill]{transfer interpolate the wavelength scale
       between spectra 3 and 11.  Transfer this interpolated scale to
       spectrum 1.}
\end{example}%
\lthtmlfigureZ
\lthtmlcheckvsize\clearpage}

\stepcounter{section}
{\newpage\clearpage
\lthtmlfigureA{command5837}%
\begin{command}
  \item[Form: ALIGN source {[DSP=disp]} {[W=l,p]} {[NW=n]} {[LOG]} 
       {[FLIP]} {[MS=n]}\hfill]{}
  \item{{[NOFLUX]} {[V=v]} {[Z=z]} {[RED=z]} {[DERED=z]} {[DP=dp]} {[DS=dp]}}
  \item{{[LCMOD=n]} {[SINC]} {[LGI]} {[SILENT]}}
  \item[source]{is the image that the program works on.}
  \item[DSP=disp]{set the dispersion of the new scale (p.e. Angs/pixel),}
  \item[W=l,p]{sets zero-point of new scale (lambda 'l' at pixel 'p',}
  \item[NW=n]{makes final scale cover the range of n pixels.}
  \item[LOG]{transforms to a logarithmic scale (linear otherwise),}
  \item[FLIP]{reverse the direction of the dispersion,}
  \item[MS=n]{set the final pixel scale to be that of image 'n',}
  \item[NOFLUX]{prevent conversion to counts/sec/Angstrom.}
  \item[V=v]{remove velocity shift of 'v' km/sec,}
  \item[Z=z, DERED=z]{remove redshift z (i.e. deredshifts the image rows),}
  \item[RED=z]{redshift the image by z.}
  \item[DP=dp]{remove a shift of dp pixels from the image,}
  \item[DS=ds]{remove a stretch of ds relative-units from the image,}
  \item[LCMOD=n]{remove the polynomial distortion given in buffer 'n',}
  \item[SILENT]{inhibit printout.}
\end{command}%
\lthtmlfigureZ
\lthtmlcheckvsize\clearpage}

{\newpage\clearpage
\lthtmlfigureA{example5880}%
\begin{example} 
  \item[LOG\hfill]{Puts the image on a logarithmic wavelength scale such
       that
       \begin{verbatim}

  (lambda) = Ln(Lambda0) + b * pix,\end{verbatim}

       and the DSP=disp keyword specifies the dispersion at the wavelength
       specified by the W=l,p keyword.}
\par
\item[NW=n \hfill]{ALIGN generates an output image covering the
       wavelength range of the input image. NW=n specifies the number of
       columns of the output to cover larger (filled with zeros) or shorter
       wavelength ranges.}
\par
\item[MS=n\hfill]{Forces the new scale to match that of a previously
       aligned image in buffer 'n'.}
\par
\item[FLIP\hfill]{will reverse the order of the dispersion by flipping it
       about the center of the image.}
\par
\item[SILENT\hfill]{Prevents printout of the new scale.}
\end{example}%
\lthtmlfigureZ
\lthtmlcheckvsize\clearpage}

{\newpage\clearpage
\lthtmlfigureA{example5889}%
\begin{example} 
  \item[ALIGN 3 DSP=5.75 W=6000,1\hfill]{re-bins the row of image 3 into a
       linear wavelength scale with a linear dispersion of 5.75
       Angstroms/pixel, and with 6000 Angstroms at pixel 1.}
\par
\item[ALIGN 3 DSP=5.75 W=6000,1 SINC\hfill]{Same as above, but sampling
       only at the center of each new pixel.}
\par
\item[ALIGN 3 DSP=5.0 W=6000,5 LOG\hfill]{Transform image 3 to a
       logarithmic wavelength scale in which the dispersion is 5.0
       Angstroms/pixel at 6000 Angstroms.  Pixel 5 corresponds to 6000
       Angstroms.}
\par
\item[ALIGN 1 DSP=8.0 W=(4300,1) Z=0.0035\hfill]{ Removes a redshift of
       0.0035 from spectrum 1 and creates a final de-redshifted spectrum
       which has a dispersion of 8 Angstroms/pixel and begins at 4300
       Angstroms in pixel 1.}
\end{example}%
\lthtmlfigureZ
\lthtmlcheckvsize\clearpage}

\stepcounter{section}
{\newpage\clearpage
\lthtmlfigureA{command5898}%
\begin{command}
  \item[Form:SKYLINE s1 {[s2]} {[s3]} ... {[s15]} {[INT]}\hfill]{}
  \item[s1]{is the buffer number of the sky spectrum.}
  \item[s2 ... s15]{are buffer numbers whose wavelength
       scales are to be re-calibrated using the night sky spectrum s1.}
  \item[INT]{requests interactive deletion of lines
       from the list of identified night sky lines.}
\end{command}%
\lthtmlfigureZ
\lthtmlcheckvsize\clearpage}

{\newpage\clearpage
\lthtmlfigureA{example5909}%
\begin{example}
  \item[MASH 1 8 SP=30,40 BK=10,20 BK=50,60 SKY=2\hfill]{}
  \item[COPW 1 20\hfill]{! Copy wavelength parameters}
  \item[COPW 2 20\hfill]{}
  \item[ALIGN 1 DSP=7.0\hfill]{! Xform to linear wavelength}
  \item[ALIGN 2 DSP=7.0\hfill]{}
  \item[SKYLINE 2 1 INT\hfill]{! Correct zero-point}
\end{example}%
\lthtmlfigureZ
\lthtmlcheckvsize\clearpage}

\stepcounter{section}
{\newpage\clearpage
\lthtmlfigureA{command5921}%
\begin{command}
  \item[Form: EXTINCT source {[CTIO]} {[KPNO]} {[LICK]}\hfill]{}
\end{command}%
\lthtmlfigureZ
\lthtmlcheckvsize\clearpage}

\stepcounter{section}
{\newpage\clearpage
\lthtmlfigureA{command5933}%
\begin{command}
  \item[\textbf{Form: } SETUP {[LONG=longitude]} {[LAT=latitude]}\hfill]{}
  \item{{[MMT]} {[KPNO]} {[CTIO]} {[LOWELL]} {[APO]} {[LICK]} {[AAT]} {[SSO]}}
  \item{{[ESO]} {[CFHT]} {[CHECKAIR]} {[COMPUTEAIR]} {[USEAIR]}} 
\end{command}%
\lthtmlfigureZ
\lthtmlcheckvsize\clearpage}

\stepcounter{section}
{\newpage\clearpage
\lthtmlfigureA{command5955}%
\begin{command}
  \item[Form: FLUXSTAR source {[standard]} {[AVE]} {[WT=w]} {[SYSA]} {[SYSC]}
       {[POLY=n]} {[TTY]}\hfill]{}
  \item[source]{is the spectrum used to determine the flux curve,}
  \item[standard]{(character string) is a file 
       containing the flux calibration for the 'source' spectrum,}
  \item[AVE]{averages the current spectrum with previous results, and}
  \item[WT=w]{specifies weighting for the averaging.}
  \item[SYSA]{produces a "System A" (IDS nomenclature) response curve,}
  \item[SYSC]{produces a "System C" (Compromise between A and B) response 
       curve.}
  \item[POLY=n]{use a nth degree polynomial fit rather than a spline}
  \item[TTY]{extra screen output}
\end{command}%
\lthtmlfigureZ
\lthtmlcheckvsize\clearpage}

{\newpage\clearpage
\lthtmlfigureA{example5977}%
\begin{example}
  \item[FLUX 5 FILE=mydir/L1020\hfill]{takes the spectrum in buffer 5, and
       computes a flux curve from it, using the calibration contained in
       the file mydir/L1020.FLX.  The spectrum is replaced by the flux
       curve.}
\par
\item[FLUX 6 FILE=mydir/L1040 AVE\hfill]{does the same as the first
       example with buffer 6, using the file mydir/L1020.FLX.  The spectrum
       is replaced by the average of the flux curves computed by these two
       spectra.}
\par
\item[FLUX 33 SYSC FILE=JUNK\hfill]{begins a new flux curve using
       spectrum 33.  The file is jUNK.FLX in the default spectrum
       directory.  System-C fluxes is used.}
\end{example}%
\lthtmlfigureZ
\lthtmlcheckvsize\clearpage}

\stepcounter{section}
{\newpage\clearpage
\lthtmlfigureA{command5984}%
\begin{command}
  \item[Form: FLUX source\hfill]{}
  \item[source]{is the spectrum being fluxed.}
\end{command}%
\lthtmlfigureZ
\lthtmlcheckvsize\clearpage}

{\newpage\clearpage
\lthtmlinlinemathA{tex2html_wrap_inline6277}%
$^2$%
\lthtmlinlinemathZ
\lthtmlcheckvsize\clearpage}

\stepcounter{section}
{\newpage\clearpage
\lthtmlfigureA{command5990}%
\begin{command}
  \item[Form:POLY source ORD=n {[SUB]} {[DIV]} {[LOAD]} {[NOZERO]} 
       {[SILENT]}\hfill]{}
  \item[source]{is the number of the spectrum to fit,}
  \item[ORD=]{is the order of the polynomial to fit (<7),}
  \item[SUB]{subtracts the polynomial from the data,}
  \item[DIV]{divides the polynomial by the data,}
  \item[LOAD]{loads the VISTA variables COEFF0, COEFF1, ...,
       COEFFn with the coefficients of the fit, and
       DCOEFF0, DCOEFF1,..., with their estimated error.}
  \item[NOZERO]{ignore pixels with value= 0 in the fit}
  \item[SILENT]{suppresses terminal output}
\end{command}%
\lthtmlfigureZ
\lthtmlcheckvsize\clearpage}

{\newpage\clearpage
\lthtmlfigureA{example6008}%
\begin{example}
  \item[POLY 14 ORD=3]{Fits a cubic polynomial to the spectrum in buffer
       14.  The spectrum is replaced by the polynomial fit.}
\par
\item[POLY 3 ORD=2 SUB]{Buffer 3 is fit with a parabola, which is then
       subtracted from the original data.}
\par
\item[POLY 5 ORD=3 LOAD]{Buffer 5 is fit with a third-order (cubic)
       polynomial which then replaces the original data in the buffer.  The
       four VISTA variables COEFF0, COEFF1, COEFF2, and COEFF3 are loaded
       with the fit coefficients, and variables DCOEFF0, DCOEFF1, DCOEFF2,
       and DCOEFF3 with the fit coefficient errors.}
\end{example}%
\lthtmlfigureZ
\lthtmlcheckvsize\clearpage}

\stepcounter{section}
{\newpage\clearpage
\lthtmlfigureA{command6016}%
\begin{command}
  \item[Form:STRETCH dest source {[VERT]} {[HORIZ]} {[SIZE=]} 
       {[START=]}\hfill]{}
  \item[dest]{is the number of the image produced by the operation,}
  \item[source]{is the number of the spectrum which is being stretched,}
  \item[VERT]{specifies that the spectrum will be stretched vertically,}
  \item[HORIZ]{specifies that the spectrum will be stretched horizontally,}
  \item[SIZE=]{specifies the length of the stretch,}
  \item[START=]{specifies the start column or row (for HORIZ or
       VERT respectively) of the output image.}
\end{command}%
\lthtmlfigureZ
\lthtmlcheckvsize\clearpage}

{\newpage\clearpage
\lthtmlfigureA{example6030}%
\begin{example}
  \item[STRETCH 1 3\hfill]{Stretches spectrum 3 vertically into image 1.
       If image 1 does not exist previous to this command, or if it has a
       number of columns not equal to the length of spectrum 3, the user
       will be prompted for the number of rows to stretch into image 1.
       The vertical stretch will make image 1 constant down each column.}
\par
\item[STRETCH 2 5 HORIZ SIZE=180\hfill]{Stretches spectrum 5 horizontally
       into image 2, and makes image 2 with 180 columns.  Image 2 will
       therefore be constant across each row.}
\end{example}%
\lthtmlfigureZ
\lthtmlcheckvsize\clearpage}

\stepcounter{section}
{\newpage\clearpage
\lthtmlfigureA{command6036}%
\begin{command} 
  \item[Form: ISPLINE source {[XY]} {[AVG=n]} {[SUB]} {[DIV]} {[HIST]}\hfill]{}
  \item[source]{the SPECTRUM to which a spline is to be fitted.}
  \item[XY]{Use cursor's absolute (X,Y) position to select
       knot points for the spline fit.}
  \item[AVG=n]{Knot point is determined as the average of the
       `n' nearest pixels to the cursor's X axis location. (default is AVG=1)}
  \item[SUB]{Subtract the spline fit from the spectrum.}
  \item[DIV]{Divide the spectrum by the spline fit.}
  \item[HIST]{Plot the spectrum as bins (histogram).}
\end{command}%
\lthtmlfigureZ
\lthtmlcheckvsize\clearpage}

{\newpage\clearpage
\lthtmlfigureA{example6063}%
\begin{example} 
  \item[ISPLINE 1 \hfill]{Fit a spline to the spectrum in buffer 1, using
       the nearest pixel to the cursor for the knot points, and replacing
       the spectrum by the spline fit.}
\par
\item[ISPLINE 1 SUB\hfill]{Same as (1), except that the spline fit is
       subtracted from the spectrum and the result is stored in buffer 1.}
\par
\item[ISPLINE 1 DIV \hfill]{Same as (1), except that the spectrum is
       divided by the the spline fit and the result is stored in buffer 1.}
\par
\item[ISPLINE 2 SUB HIST\hfill]{Same as (2), except that the spectrum is
       plotted as a "histogram".}
\par
\item[ISPLINE 3 XY\hfill]{Fit a spline to the spectrum in buffer 3, using
       the absolute (X,Y) position of the cursor to select the knot points.
       The spectrum will be replaced by the spline fit.}
\par
\item[ISPLINE 4 AVG=5 SUB\hfill]{Fit a spline using the average of 5
       pixels around the X-axis position of the cursor as the knot points,
       and replace the contents of buffer 4 by the spectrum minus the
       spline fit.}
\end{example}%
\lthtmlfigureZ
\lthtmlcheckvsize\clearpage}

\stepcounter{section}
{\newpage\clearpage
\lthtmlfigureA{command6074}%
\begin{command}
  \item[Form: ROWFIT source {[BOX=b]} {[FIT=cs,ce]} {[NCOEF=m]} {[LOAD=bp]} 
       {[UNLOAD=bp]}\hfill]{}
  \item{{[SUB]} {[DIV]} {[POLY]} {[FOUR]} {[CLIP=f]} {[FITONLY]} {[XBUF=buf]}}
  \item[BOX=b]{Box within which the pixels will be substituted.}
  \item[FIT=cs,ce]{Column limits of up to 4 fitting regions.}
  \item[NCOEF=m]{Number of coefficients to fit, default 1.}
  \item[LOAD=bp]{Loads in buffer bp the coefficients, the errors,
       and the standard deviation from the fit.}
  \item[UNLOAD=bp]{Uses the pre-loaded coefficients in buffer bp,  
       to evaluate the fit and operate on the image.}
  \item[CLIP=f]{Clips out pixels that deviate more than f sigmas from
       the fit.}
  \item[SUB]{Subtract the fit from the image.}
  \item[DIV]{Divide the image by the fit.}
  \item[XBUF=buf]{Use as abscissae not the column number but the spectrum  
       in buffer 'buf'.}
  \item[WBUF=buf]{Uses the spectrum in buffer 'buf' to weight individual
   pixels in the fit.}
  \item[POLY]{Fit polynomial of order NCOEF-1.}
  \item[FOUR]{Fit a Fourier series up to order (NCOEF-1)/2.}
  \item[FITONLY]{Does not substitute fit in image. If CLIP is specified,
       only clipped pixels are replaced.}
\end{command}%
\lthtmlfigureZ
\lthtmlcheckvsize\clearpage}

{\newpage\clearpage
\lthtmlfigureA{example6105}%
\begin{example}
  \item[ROWFIT 1 NCEOF=2 LOAD=3 FIT=100,150 FIT=300,350 SUB \hfill]{ Fits a
       line to each row of the image on buffer 1 using only columns 100 to
       150 and 300 to 350, subtracts the fits from the whole image and
       saves the fit parameters in buffer 3.}
\par
\item[ROWFIT 5 UNLOAD=3 DIV\hfill]{ Divides buffer 5 by the fit generated
       in example 1).}
\par
\item[ROWFIT 10 XBUF=11 NCOEF=2 CLIP=5\hfill]{ Fit buffer 10 with a
       constant plus a scaled spectrum (buffer 11) but do not take into
       account pixels that deviate more than 5 sigmas from the fit.}
\par
\item[ROWFIT 5 BOX=10 CLIP=5 NCOEF=5 FITONLY\hfill]{ Fitting a 4th-order
       polynomial to each row of buffer 5, substitute by the fit only those
       pixels that deviate more than 5 sigmas, do so only in the defined by
       box 10. (a way to identify and CLIP ion hits or bad columns).}
\end{example}%
\lthtmlfigureZ
\lthtmlcheckvsize\clearpage}

\stepcounter{section}
{\newpage\clearpage
\lthtmlfigureA{command6112}%
\begin{command}
  \item[Form: FINDSHIFT s1 s2 {[RAD=r]} {[SHIFT=s]} {[XS=i]} {[XE=j]} 
       {[ROW=k]} {[LOAD]} {[SILENT]}\hfill]{}
  \item[RAD=R]{Search Radius for the Chi-squared minimum.
               The default is 1/4 of the number of columns.}
  \item[SHIFT=]{A guess of SHIFT, i.e. the column shift on the
                second spectrum required to match the template.}
  \item[XS=I]{First column, on template, of the spectral
              region for Chi-squared computation.}
  \item[XE=J]{Last column of the selected region.}
  \item[ROW=k]{Estimate shift in row k (when images are given).}
  \item[LOAD]{Loads variables SHIFT, DSHIFT, and SHIFTVAR with
              the relative shift, an estimation of its error
              and the unbiased mean variance of the spectra.}
  \item[SILENT]{Loads variables but does not print results.}
\end{command}%
\lthtmlfigureZ
\lthtmlcheckvsize\clearpage}

\stepcounter{section}
{\newpage\clearpage
\lthtmlfigureA{command6131}%
\begin{command}
  \item[Form: FINDPEAK buf {[MODEL=b]} {[LOC=r1,r2]} {[SPW=w]} {[SP=s1,s2]}
       {[DLOC=dp]} {[INT]} {[COLS]}\hfill]{}
  \item[MODEL=b]{Specifies a spectrum buffer where a model for the
       curvature of the spectrum will be found.}
  \item[LOC=r1,r2]{specifies the approximate row upon which the peak lies.
       If only one row is specified that row will be used as
       the model for the peak search. If two rows are specified 
       a linear model will be used.}
  \item[SPW=w]{To specify a width for the box within which the peak
       will be searched. See also SP=.}
  \item[SP=s1,s2]{specify a window (possibly asymmetric) around the model,
       to find the peak. Two numbers row numbers must be given,
       and are assumed to be relative to the model.}
  \item[DLOC=dp]{specifies a shift of d rows between model and peak}
  \item[INT]{Find the peak location only within a pixel. Otherwise,
       the fractional location of the peak will be estimated
       fitting a parabola around the peak.}
  \item[MEDIAN]{Finds the median row, rather than the peak, inside
       the given limits.}
  \item[AREA=fa]{Rather than the peak or the median, finds the row at 
       which the cumulative counts reach a fraction fa of the 
       total (fa=.5 is equivalent to MEDIAN). This keyword is
       analogous to the one in command ABX.}
  \item[COLS]{Finds the location of the peak or median at each row
       rather than at each column}
\end{command}%
\lthtmlfigureZ
\lthtmlcheckvsize\clearpage}

\stepcounter{section}
{\newpage\clearpage
\lthtmlfigureA{command6151}%
\begin{command}
  \item[Form: SPINDEX dest source FILE=file {[DP=dp]} {[DLAM=dlam]} 
       {[V=v]}\hfill]{}
  \item{{[Z=z]} {[ROT=buf]} {[VAR=var]} {[TTY]} {[output-redirection]}}
  \item[source]{the input image (with a wavelength scale)}
  \item[dest]{the destination buffer}
  \item[FILE=file]{ASCII file containing a list of wavelength intervals
       to be used in the line-strength.}
  \item[DP=dp]{take into account a shift of dp pixels}
  \item[DLAM=dl]{take into account a shift of dl angstroms}
  \item[V=v]{take into account a radial velocity of v km/sec}
  \item[Z=z]{take into account a redshift of z}
  \item[ROT=buf]{take into account the rotation profile (km/sec)
       in buffer 'buf' (for images with more than one row).}
  \item[VAR=buf]{estimate errors in the spectral indices using the
       variances image in buffer 'buf' (for images with more
       than one row)}
  \item[TTY]{Type results on screen or in output-redirection file.}
\end{command}%
\lthtmlfigureZ
\lthtmlcheckvsize\clearpage}

\stepcounter{section}
{\newpage\clearpage
\lthtmlfigureA{command6180}%
\begin{command}
  \item[Form: FQUO fdata ftemplate {[G=g(,mn,mx)]} {[S=s(,mn,mx)]} 
       {[V=v(,mn,mx)]}\hfill]{}
  \item{{[GUESS=g,s,v]} {[GUESS=buf]} {[FIXG]} {[FIXS]} {[FIXV]} {[KS=i]} 
       {[KE=j]}}
  \item{{[WBUF=w]} {[ROW=k]} {[MODE=1]} {[LOAD]} {[LOAD=b]} {[INTER]} 
       {[SILENT]}}
  \item[fdata]{is the 1-D Fourier transform of the data.}
  \item[ftemplate]{is the 1-D Fourier transform of the template.}
  \item[G=g(,mn,mx)]{Guess (and optional limits) for relative line-strength
       relative to the template (gamma-parameter).}
  \item[S=s(,mn,mx)]{First guess (limits) for the velocity dispersion (km/s).}
  \item[V=v(,mn,mx)]{First guess (limits) for the relative-velocity (km/s).}
  \item[GUESS=g,s,v]{Another way of giving first guess for the parameters.}
  \item[GUESS=buf]{Use values in buffer 'b' as parameter's first guess.}
  \item[FIXG]{Fix (do not fit) gamma-parameter at value g.}
  \item[FIXS]{Fix (do not fit) velocity dispersion at value s km/s.}
  \item[FIXV]{Fix (do not fit) velocity at the guessed value v km/s.}
  \item[KS=I]{First wavenumber of the fitting region.}
  \item[KE=J]{Last wavenumber of the selected region.}
  \item[WBUF=w]{Use buffer w as weights (not errors) for the fit.}
  \item[LOAD]{Loads variables G, S, V, DG, DS, DV and DEV with the
       gamma-parameter, velocity-dispersion, relative velocity,
       their estimated errors and the unbiased-standard-deviation
       from the fit. If GUESS=b keyword is used, these values
       are also loaded in the input buffer b.}
  \item[LOAD=p]{Loads the fit-parameters in buffer 'p'.
       format of VISTA commands ROWFIT and SPINDEX.}
  \item[ROW=r]{Fit only in row r (when A has more than 1 row).}
  \item[MODE=1]{Fits the Gaussian to the quotient A/B (old way).}
  \item[MODE=2]{Fits B*Gaussian to A (default and a nicer way).}
  \item[INTER]{Stops after each iteration for user to inspect results.}
  \item[SILENT]{To prevent printing results on the screen.}
\end{command}%
\lthtmlfigureZ
\lthtmlcheckvsize\clearpage}

\stepcounter{section}
{\newpage\clearpage
\lthtmlfigureA{command6223}%
\begin{command}
  \item[Form:EWAVE source {[XOFF=x0]} {[PORD=nw]} {[MORD=nm]} {[REJ=rej]} 
       {[TTY]} {[TRACE]} \hfill]{}
  \item[source]{is the buffer to be wavelength calibrated.}
  \item[XOFF=]{x-offset (column offset) from the nominal 
       pattern center. It is positive when the CCD
       is to the left of the nominal pattern center.}
  \item[PORD=]{order of the polynomial to be fit in the x coordinate.}
  \item[MORD=]{order of the polynomial to be fit in the 1/m
       (1/order number) coordinate.}
  \item[REJ=]{hard limit in mA determining the maximum residual
       allowed between the wavelength of a line and the
       wavelength determined from the fit.}
  \item[TTY]{gives a detailed printout of the lines identified and 
       their residuals.}
  \item[TRACE]{traces the fit parameters, printing them out after each 
       iteration.}
\end{command}%
\lthtmlfigureZ
\lthtmlcheckvsize\clearpage}

{\newpage\clearpage
\lthtmlfigureA{example6240}%
\begin{example}
  \item[RD 1 THAR.EXT; EWAVE 1; WD 1 THAR.WAV FULL\hfill]{ Wavelength
       calibrates an echelle image centered on the nominal pattern center
       using a two-D polynomial of order 2 for the column number and of
       order 1 for 1/m.  The input image called "THAR.CCD" was read from
       disk and is written back to disk after calibration.}
\par
\item[RD 1 THAR\_100.EXT\hfill]{}
\par
\item[EWAVE 1 XOFF=100 PORD=2 MORD=1 REJ=75 TTY TRACE\hfill]{}
\par
\item[WD 1 THAR\_100.WAV FULL\hfill]{ Wavelength calibrates an echelle
       image shifted by 100 pixels to the left from the nominal pattern
       center using again a two-D polynomial of order 2 for the column
       number and of order 1 for 1/m.  All the lines having a residual
       larger than 75 mA will be rejected after the second iteration.  The
       fit parameters, the lines identified and their residuals will be
       printed out out after each iteration.}
\end{example}%
\lthtmlfigureZ
\lthtmlcheckvsize\clearpage}

\stepcounter{section}
{\newpage\clearpage
\lthtmlfigureA{command6248}%
\begin{command}
  \item[Form: EXTSPEC dest source ORD=nord\hfill]{}
  \item[dest]{is the buffer which will hold the spectrum.}
  \item[source]{is the buffer containing the extracted
       echelle wavelength-calibrated image.}
  \item[ORD=]{is the number of the order to be transferred.}
\end{command}%
\lthtmlfigureZ
\lthtmlcheckvsize\clearpage}

{\newpage\clearpage
\lthtmlfigureA{example6255}%
\begin{example}
  \item[EXTSPEC 10 1 ORD=87\hfill]{ Copy order 87 from a extracted echelle
       wavelength-calibrated image into buffer 10.}
\end{example}%
\lthtmlfigureZ
\lthtmlcheckvsize\clearpage}

\stepcounter{chapter}
{\newpage\clearpage
\lthtmlfigureA{example6690}%
\begin{example}
  \item[CUT\hfill]{extract a brightness profile cut along any position angle}
  \item[PROFILE\hfill]{compute surface brightness profile by elliptical
       isophote fitting}
  \item[ANNULUS\hfill]{compute a radial profile by azimuthal averaging}
  \item[RECON\hfill]{reconstruct an image from a surface photometry profile}
  \item[APER\hfill]{aperture photometry of an extended object}
  \item[ROUND\hfill]{deproject spirals or elliptical images}
  \item[POLAR\hfill]{regrid an image into polar coordinates}
  \item[EGAL\hfill]{reconstruct a galaxy from its profile}
  \item[SNUC\hfill]{surface photometry of multiple galaxy systems}
  \item[RENUC\hfill]{reconstruct galaxies from multiple galaxy photometry}
  \item[SECTOR\hfill]{surface photometry of star clusters or complex
       objects} 
  \item[AEDIT\hfill]{edit values in selected image regions}
  \item[CLPROF\hfill]{clear the contents of the profile common block}
  \item[CPROF\hfill]{correct the results of a profile calculation}
  \item[EMAG\hfill]{aperture photometry with elliptical apertures}
  \item[EMARK\hfill]{interactively specify ellipse parameters for EMAG} 
  \item[RMARK\hfill]{interactively set maximum radius for EMAG}
  \item[ELLMAG\hfill]{compute elliptical magnitudes for PROFILE ellipses}
  \item[TVPROF\hfill]{display PROFILE results on the image display}
  \item[OPREP\hfill]{open a batch surface photometry preparation file}
  \item[CLPREP\hfill]{close a batch surface photometry preparation file}
  \item[RPREP\hfill]{read a record from a photometry prep-file}
  \item[WPREP\hfill]{write a record into photometry prep-file}
\end{example}%
\lthtmlfigureZ
\lthtmlcheckvsize\clearpage}

\stepcounter{section}
{\newpage\clearpage
\lthtmlfigureA{command6717}%
\begin{command}
  \item[Form: CUT dest source {[PA=n]} {[C=(r,c)]} {[W=w]} {[NORM]} {[L=h]}
       {[OFF=f]} {[TV]} {[PROF]} {[DPA=dpa]} \hfill]{}
  \item[dest]{(integer or \$ construct) is an image which
       will hold the calculated cut profile,}
  \item[source]{(integer or \$ construct) is the image (object)
       from which the cut is to be measured,}
  \item[PA=n]{sets the position angle of radial slice through the 
       extended object,}
  \item[C=(r,c)]{resets the center of the slice from the AXES values,}
  \item[L=l]{Sets the length of the cut in pixels,}
  \item[W=w]{Sets the cut width in pixels,}
  \item[NORM]{Normalizes by the cut width,}
  \item[OFF=f]{Offsets the cut from the center in the width direction.}
  \item[TV]{Shows the area of the cut on the TV.}
  \item[PROF]{Plots using information from PROFILE common block (center, 
        position angle, sky}
  \item[DPA=dpa]{Using PROF option, plots at an angle dpa degrees from
        PROFILE pa}
\end{command}%
\lthtmlfigureZ
\lthtmlcheckvsize\clearpage}

\stepcounter{section}
{\newpage\clearpage
\lthtmlfigureA{command6743}%
\begin{command}
  \item[Form: PROFILE dest source {[N=n]} {[ITER=n1,n2]} {[SCALE=f]} 
       {[CENTER]} {[PA=f]} \hfill]{}
  \item{{[INT]} {[FOUR]} {[SKY]} {[PSEUDO]} {[GPROF]} {[TTY]} {[NFINE=]} 
       {[RMAX]}}
  \item[dest]{is a spectrum which will hold the calculated profile.}
  \item[source]{is the image which contains the object which is
       having its profile measured.}
  \item[N=n or RMAX]{sets the number of steps in the iteration.}
  \item[ITER=n1,n2]{sets number of iterations for (n1) fast bilinear
       interpolation and (n2), slower sinc interpolation.}
  \item[CENTER]{solves for the contour centers.}
  \item[PA=f]{position angle for the top of the image.}
  \item[INT]{interactively iterate contour solution.}
  \item[FOUR]{include 4-theta terms in solution.}
  \item[SKY]{does sky subtraction using value in VISTA variable SKY}
  \item[NFINE=]{sets the number of contours (radial) for high accuracy 
                          sinc interpolation, defaults to 15}
  \item[PSEUDO]{uses pseudo median rather than median for derivatives}
  \item[GPROF]{sets all parameters to recover Berkeley GPROF routine}
  \item[TTY]{displays output from each iteration }
\end{command}%
\lthtmlfigureZ
\lthtmlcheckvsize\clearpage}

{\newpage\clearpage
\lthtmlfigureA{example6773}%
\begin{example}
  \item[PROFILE 11 1 SCALE=0.267 N=20 ITER=2,4 >PROFILE.OUT\hfill]{Create
       spectrum in buffer 11 with the radial profile of the object in image
       1.  The centroid of the object was already calculated with AXES.
       The profile is computed out to 20 pixels, with a scale of 0.267
       arcsec/pixel.  Four iterations will be done, the first two with
       quick bilinear interpolation.  The result is stored in the file
       PROFILE.OUT.}
\end{example}%
\lthtmlfigureZ
\lthtmlcheckvsize\clearpage}

\stepcounter{section}
{\newpage\clearpage
\lthtmlfigureA{command6779}%
\begin{command}
  \item[Form: ANNULUS dest source N=n {[STEP=dr]} {[PA=pa]} {[INC=i]} 
       {[CEN=r0,c0]} {[SCALE=s]} {[FAST]} {[RAD=r]} {[PROF]}\hfill]{}
  \item[dest]{Buffer to hold the average radial profile (spectrum)}
  \item[source]{Buffer with the source image.}
  \item[N=n]{Number of concentric annuli to average over.}
  \item[STEP=dr]{(optional) Spacing between annuli in either pixels or
       units of the "SCALE=" keyword.}
  \item[PA=pa]{(optional) Position Angle in degrees of the Major Axis 
       of elliptical annuli.}
  \item[INC=i]{(optional) Inclination Angle in degrees of elliptical 
       Annuli relative to the line-of-sight.}
  \item[CEN=r0,c0]{(optional) Center of annuli (in rows,columns on image).}
  \item[SCALE=s]{(optional) Image scale, typically expressed in units
       of arcseconds/pixel.}
  \item[FAST]{(optional) Use faster bi-linear interpolation in outer annuli.}
  \item[RAD=r]{(optional) Allows the user to compute an average along
       a single annulus.  No profile is calculated}
  \item[PROF]{(optional) Computes averages along isophotes defined by
       the PROFILE command.}
\end{command}%
\lthtmlfigureZ
\lthtmlcheckvsize\clearpage}

{\newpage\clearpage
\lthtmlfigureA{hanging6801}%
\begin{hanging}
  \item{AXES: REQUIRED unless the "CEN=" keyword is used.  This provides
        ANNULUS with the object center for the averaging annuli. }
\par
\item{PROFILE: Only required if you are using the PROF keyword. (see PROF
        below).}
\par
\item{GET: May be used in place of PROFILE before using the PROF keyword
        to load in a previous PROFILE fit saved in an external file with
        the SAVE command. }
\end{hanging}%
\lthtmlfigureZ
\lthtmlcheckvsize\clearpage}

{\newpage\clearpage
\lthtmlfigureA{example6821}%
\begin{example}
  \item[ANNULUS 15 1 N=50\hfill]{A radial profile of the object in buffer 1
       is computed as the azimuthal average over 50 circular annuli spaced
       in radius by 1 pixel and stored in buffer 15.}
\par
\item[ANNULUS 15 1 N=50 >RADPRF.DAT\hfill]{Same as (1), except that the
       output is redirected to an external file called "RADPRF.DAT" in the
       users current directory.}
\par
\item[ANNULUS 15 1 N=50 PA=17 INC=59 STEP=1 SCALE=0.5\hfill]{Compute a
       radial profile of the Image in buffer 1 as an average over 50
       concentric ellipses with PA=17 degrees and Inclined 59 degrees
       relative to the line-of-sight.  The annular spacing is 1 arcsecond,
       and the pixel scale is 0.5 "/pixel.  Store the results in buffer
       15.}
\par
\item[ANNULUS 15 2 RAD=12 CEN=102.3,237.8\hfill]{Compute the azimuthal
       average along a single circular annulus centered at Row=102.3,
       Column=237.8 with a radius of 12 pixels of the image in buffer 2.
       The 15 is a dummy argument, nothing is stored in buffer 15.}
\end{example}%
\lthtmlfigureZ
\lthtmlcheckvsize\clearpage}

\stepcounter{section}
{\newpage\clearpage
\lthtmlfigureA{command6836}%
\begin{command}
  \item[Form: RECON source {[CR=f]} {[CC=f]}\hfill]{}
  \item[source]{is the image that will contain the reconstructed object.}
  \item[CR=f]{is an optional redefinition of the center row.}
  \item[CC=f]{is an optional redefinition of the center column.}
\end{command}%
\lthtmlfigureZ
\lthtmlcheckvsize\clearpage}

{\newpage\clearpage
\lthtmlfigureA{example6845}%
\begin{example}
  \item[RECON 3\hfill]{Reconstruct the surface profile in image 3.}
\par
\item[RECON 3 CR=100 CC=101\hfill]{Do the same as the above example,
       but center the new object at row 100 and column 101.}
\end{example}%
\lthtmlfigureZ
\lthtmlcheckvsize\clearpage}

\stepcounter{section}
{\newpage\clearpage
\lthtmlfigureA{command6851}%
\begin{command}
  \item[Form: APER source {[RAD=r1,r2,...,r10]} {[MAG=M1,M2,...MN]} 
       {[STEP=size,n]} {[SCALE=f]} {[C=r,c]} {[SCALE=f]}\hfill]{}
  \item{{[GEO=first,fact]} {[OLD]} {[INT]} {[REF]} {[PLOT]} {[STORE]}}
  \item[source]{the image being measured}
  \item[RAD=]{specify radii in pixels (or arcsec)}
  \item[MAG=]{list observed magnitudes for various radii}
  \item[STEP=]{automatically set radii with arithmetic sequence}
  \item[GEO=]{automatically set radii with geometric sequence}
  \item[C]{center of object is at this position}
  \item[SCALE=f]{input scale of image (arcsec / pixel)}
  \item[OLD]{repeat command using old parameters}
  \item[INT]{interactively set radii}
  \item[REF]{inter reference number for object.}
  \item[PLOT]{plots apertures on TV (not all devices)}
  \item[STORE]{stores output in VISTA variables}
\end{command}%
\lthtmlfigureZ
\lthtmlcheckvsize\clearpage}

{\newpage\clearpage
\lthtmlfigureA{example6879}%
\begin{example}
  \item[SKY 3; SUBTRACT 3 CONST=SKY; APERTURE 3 RAD=1,3,5,7,9,13
       SCALE=0.267 C=50,35\hfill]{this sequence of commands computes and
       subtracts the mean background level for image 3.  It lists the total
       counts in the object centered on row 50, column 35, in apertures of
       1, 3, 5, 7, 9, and 13 arcsec.  The scale of the image is 0.267
       arcsec/pixel}
\end{example}%
\lthtmlfigureZ
\lthtmlcheckvsize\clearpage}

\stepcounter{section}
{\newpage\clearpage
\lthtmlfigureA{command6884}%
\begin{command}
  \item[Form: ROUND dest source {[PA=f]} {[E=e]} {[C=(r,c)]} {[NORM]} 
       {[SINC]}\hfill]{}
  \item[dest]{is the destination image buffer. It does not need
       to exist beforehand.}
  \item[source]{is the image buffer to transform}
  \item[PA]{is PA of object's major axis}
  \item[E]{gives its ellipticity = 1-b/a}
  \item[C]{gives its row and column center}
  \item[NORM]{renormalize pixel intensities}
  \item[SINC]{use high-accuracy sinc interpolation.}
\end{command}%
\lthtmlfigureZ
\lthtmlcheckvsize\clearpage}

\stepcounter{section}
{\newpage\clearpage
\lthtmlfigureA{command6900}%
\begin{command}
  \item[Form: POLAR dest source {[PA=f]} {[E=e]} {[C=(r,c)]} {[NORM]} 
       {[SINC]} {[R=(rmin,rmax)]}\hfill]{}
  \item[dest]{is the destination image buffer. It does not need
       to exist beforehand.}
  \item[source]{is the image buffer to transform}
  \item[PA]{is PA of object's major axis}
  \item[E]{gives its ellipticity = 1-b/a}
  \item[C]{gives its row and column center}
  \item[NORM]{renormalize pixel intensities}
  \item[SINC]{use high-resolution sinc interpolation.}
  \item[R]{gives the radial limits of the new grid}
\end{command}%
\lthtmlfigureZ
\lthtmlcheckvsize\clearpage}

\stepcounter{section}
{\newpage\clearpage
\lthtmlfigureA{command6919}%
\begin{command}
  \item[Form: EGAL source {[CR=f]} {[CC=f]} {[AVANG]} {[NODEV]}\hfill]{}
  \item[source]{is the image buffer}
  \item[CR=f]{set the row center to f}
  \item[CC=f]{set the column center}
  \item[AVANG]{average isophote PAs together}
  \item[NODEV]{do not fit r\^1/4 law to outer profile}
\end{command}%
\lthtmlfigureZ
\lthtmlcheckvsize\clearpage}

\stepcounter{section}
{\newpage\clearpage
\lthtmlfigureA{command6934}%
\begin{command} 
  \item[Form: SNUC buf {[SKY=f]} {[SCALE=f]} {[OLD]} {[NPASS=n]} 
       {[EX=(n1,n2,...)]}\hfill]{}
  \item{{[RECEN=n]} {[BOX=n]} {[CENTER=(n1,n2,...)]} {[DELT=f]} 
       {[LIST=filespec]} {[MASK]}}
  \item[buf]{Buffer number of image to be processed.}
  \item[SKY=]{DN value of any sky level subtracted from the image}
  \item[SCALE=]{Pixel scale (arcsec/pixel)}
  \item[OLD]{Continue iterating on previous photometry solution}
  \item[NPASS=]{Perform 'n' iterations to find the photometry parameters}
  \item[EX=]{With OLD option, recalulate solution excluding last 'n'
       isophotes from galaxies 1, 2, etc from old solution. If
       'n' is negative, then extend out 'n' more isophotes.}
  \item[RECEN=]{Find centroids of galaxies, using box 'n' pixels in radius}
  \item[BOX=]{Only fit the image within box 'n'}
  \item[CENTER=]{Allow the isophote centers for galaxies 'n1', 'n2', etc
       to be free parameters.}
  \item[DELT=]{Space isophotes geometrically by ratio 1+f}
  \item[LIST=]{Specify file with galaxy centers, radii, processing flags.
       (Format: ROW COL RADIUS FLAG). Full file pathname must be
       provided.}
  \item[MASK]{ignores MASKed pixels in the fit}
\end{command}%
\lthtmlfigureZ
\lthtmlcheckvsize\clearpage}

{\newpage\clearpage
\lthtmlfigureA{example6961}%
\begin{example}
  \item[SNUC buf SKY=back SCALE=0.334 RECEN=5\hfill]{Setup a
       decomposition, do three iterations and stop.}
\par
\item[SNUC buf SKY=back OLD EX=(-2,0,1) NPASS=6\hfill]{Continue this
       solution, but extend the primary galaxy by more isophotes and delete
       the outer isophote of galaxy 3.  Iterate for several passes.}
\par
\item[SNUC buf SKY=back SCALE=0.334 DELT=0.1\hfill]{Start a solution
       using a finer isophote spacing.}
\par
\item[SNUC buf SKY=back DELT=0.1 LIST=/mydir/galaxy.lis\hfill]{Use an
       input file to specify galaxy parameters.}
\end{example}%
\lthtmlfigureZ
\lthtmlcheckvsize\clearpage}

\stepcounter{section}
{\newpage\clearpage
\lthtmlfigureA{command6970}%
\begin{command} 
  \item[Form: RENUC source {[BOX=n]} {[EX=(n1,n2,...)]} {[LIM=(n1,n2,...]}
       {[INC=(n1,n2,...)]} {[CENTER=(n1,n2,...)]}\hfill]{}
  \item[source]{The image buffer which will hold the reconstruction}
  \item[BOX=]{Only reconstruct within BOX n.}
  \item[EX=]{Do not reconstruct galaxies n1, n2, etc.}
  \item[INC=]{Only reconstruct galaxies n1, n2, etc.}
  \item[LIM=]{Do not use the last n isophotes from galaxies 1, 2, etc.}
  \item[CENTER=]{Reconstruct using the individual isophote centers for
       galaxies n1, n2, etc.}
\end{command}%
\lthtmlfigureZ
\lthtmlcheckvsize\clearpage}

{\newpage\clearpage
\lthtmlfigureA{example6984}%
\begin{example}
  \item[RENUC 1\hfill]{Reconstruct the entire system fitted by SNUC (into
       image 1).}
\par
\item[RENUC 1 INC=(1) \hfill]{}
\par
\item[or RENUC 1 EX=(2,3,4)\hfill]{Only reconstruct galaxy 1 (four
       galaxies were fitted);}
\par
\item[RENUC 1 LIM=(0,2)\hfill]{Do not use the last isophote of galaxy 2.}
\end{example}%
\lthtmlfigureZ
\lthtmlcheckvsize\clearpage}

\stepcounter{section}
{\newpage\clearpage
\lthtmlfigureA{command6992}%
\begin{command}
  \item[Form: SECTOR source {[RAD=(lin,geo)]} {[C=(r,c)]} {[SCALE=f]} 
       {[PLOT]}\hfill]{}
  \item[source]{The image buffer}
  \item[RAD=]{Describe the isophote spacing. Isophotes are initially spaced
       linearly every `lin' pixels in radius. Isophotes further out are
       spaced geometrically by `geo' when this step is larger than `lin'.
       Defaults: lin=2 pixels, geo=0.15 (increase radius by 15\% per
       step).}
  \item[C=]{Give the object center.  Default is to use results from
       the AXES or AUTOCEN commands.}
  \item[SCALE=]{Give the scale in arcsec/pixel. Default is 0.55.}
  \item[PLOT]{Plot the isophotes on the TV.}
\end{command}%
\lthtmlfigureZ
\lthtmlcheckvsize\clearpage}

\stepcounter{section}
{\newpage\clearpage
\lthtmlfigureA{command7005}%
\begin{command}
  \item[Form: AEDIT source {[BOX=n]} {[INT]} {[SET=x]} {[MASK]} {[OLD]} 
       {[filename]}\hfill]{}
  \item[BOX=n]{specifies the BOX to use}
  \item[INT]{BOX will be specified interactively}
  \item[SET=x]{specifies the flag value}
  \item[MASK]{tells AEDIT to create a VISTA logical mask only (pixel values
       are unchanged)}
  \item[OLD]{tells AEDIT to edit pixels specified in the file specified on
       the command line}
  \item[filename]{tells AEDIT to write out the regions it edits into a
       formatted file (if INT is specified) or to read in the regions from
       the file (if OLD is specified)}
\end{command}%
\lthtmlfigureZ
\lthtmlcheckvsize\clearpage}

\stepcounter{section}
{\newpage\clearpage
\lthtmlfigureA{command7021}%
\begin{command}
  \item[Form: CLPROF {[RAD=r]}\hfill]{}
  \item[RAD=r]{Specifies the radius beyond which the prf common
       block is cleared.}
\end{command}%
\lthtmlfigureZ
\lthtmlcheckvsize\clearpage}

\stepcounter{section}
{\newpage\clearpage
\lthtmlfigureA{command7027}%
\begin{command}
  \item[Form: CPROF {[ORDER=n]} {[MEDIAN]} {[SQWEIGHT]}\hfill]{}
  \item[ORDER=n]{Specifies the order of the polynomial fit. Default is 2.}
  \item[MEDIAN]{Specifies that median surface brightness, rather
       than mean, will be used for exponential fit (if performed)}
  \item[SQWEIGHT]{Specifies that square root weighting, rather than 
       uniform weighting, is to be used during exponential fit}
\end{command}%
\lthtmlfigureZ
\lthtmlcheckvsize\clearpage}

\stepcounter{section}
{\newpage\clearpage
\lthtmlfigureA{command7037}%
\begin{command}
  \item[Form: EMAG source {[CENTER=(row,col)]} {[PA=theta]} {[ELL=eps]} {[N=n]}
     {[PROF=p]} {[RMAX]} {[ERAD]} {[APPEND]}\hfill]{}
  \item[source]{is the image EMAG works on}
  \item[CENTER=(row,col)]{allows the user to determine the center
       of the object being worked on.}
  \item[PA=theta]{allows the user to set the position angle of
       the elliptical apertures (theta in degrees.)}
  \item[ELL=eps]{allows the user to set the ellipticity of the
       apertures.}
  \item[N=n]{user-determined number of elliptical apertures;
       the outermost aperture has semimajor axis (n-1) pixels.}
  \item[PROF=p]{tells EMAG to take the ellipse parameters from the p-th
       aperture in the PRF common block--this works only if PROFILE have
       been previously run, or if their results have been read in by
       GPGET.}
  \item[RMAX]{tells EMAG to learn the maximum semimajor axis radius from
       the VISTA variable RMAX (this works only if EMARK or RMARK have been
       previously run.)}
  \item[ERAD]{tells EMAG to take the ellipse parameters from the PRF common
       block, at a radius given by the value of the VISTA variable ERAD,
       set by the routine EMARK.}
  \item[APPEND]{tells EMAG to compute not only elliptical totals, but also
       surface brightnesses in elliptical belts, for radii between the
       previous maximum radius in the PRF common block, and the new
       max. radius determined by the N=n or RMAX keywords; the fixed values
       of ellipticity and PA used for the EMAG calculation are also filled
       in.}
\end{command}%
\lthtmlfigureZ
\lthtmlcheckvsize\clearpage}

{\newpage\clearpage
\lthtmlfigureA{example7058}%
\begin{example}
  \item[EMAG 5 CENTER=(243,209) N=81 PA=32.5 ELL=0.71\hfill]{}
\par
\item[EMAG 5 ERAD RMAX APPEND\hfill]{Gets ellipse parameters from PRF
       common block at radius ERAD, works out to radius RMAX, and tacks on
       ellipticities, position angles, and surface brightnesses.}
\end{example}%
\lthtmlfigureZ
\lthtmlcheckvsize\clearpage}

\stepcounter{section}
{\newpage\clearpage
\lthtmlfigureA{command7063}%
\begin{command}
  \item[Form: EMARK\hfill]{}
\end{command}%
\lthtmlfigureZ
\lthtmlcheckvsize\clearpage}

\stepcounter{section}
{\newpage\clearpage
\lthtmlfigureA{command7067}%
\begin{command}
  \item[Form: RMARK\hfill]{}
\end{command}%
\lthtmlfigureZ
\lthtmlcheckvsize\clearpage}

\stepcounter{section}
{\newpage\clearpage
\lthtmlfigureA{command7071}%
\begin{command}
  \item[Form: ELLMAG {[MEDIAN]}\hfill]{}
  \item[MEDIAN]{specifies that median, rather than mean, surface
       brightnesses are to be used for computing magnitudes}
\end{command}%
\lthtmlfigureZ
\lthtmlcheckvsize\clearpage}

\stepcounter{section}
{\newpage\clearpage
\lthtmlfigureA{command7077}%
\begin{command}
  \item[Form: TVPROF N1=n1 SPACE=n2\hfill]{}
  \item[N1=n1]{first ellipse displayed has semimajor axis radius n1.}
  \item[SPACE=n2]{subsequent ellipse semimajor axis radii are incremented
       by n2 pixels, continuing out to the last entry in the PRF common
       block.}
\end{command}%
\lthtmlfigureZ
\lthtmlcheckvsize\clearpage}

\stepcounter{section}
{\newpage\clearpage
\lthtmlfigureA{command7083}%
\begin{command}
  \item[Form: OPREP {[R]} {[W]} {[filename]}\hfill]{}
  \item[R]{open to read}
  \item[W]{open to write}
  \item[filename]{prep-file name.}
\end{command}%
\lthtmlfigureZ
\lthtmlcheckvsize\clearpage}

\stepcounter{section}
{\newpage\clearpage
\lthtmlfigureA{command7100}%
\begin{command}
  \item[Form: CLPREP\hfill]{}
\end{command}%
\lthtmlfigureZ
\lthtmlcheckvsize\clearpage}

\stepcounter{section}
{\newpage\clearpage
\lthtmlfigureA{command7104}%
\begin{command}
  \item[Form: RPREP\hfill]{}
\end{command}%
\lthtmlfigureZ
\lthtmlcheckvsize\clearpage}

\stepcounter{section}
{\newpage\clearpage
\lthtmlfigureA{command7115}%
\begin{command}
  \item[Form: WPREP image\hfill]{}
\end{command}%
\lthtmlfigureZ
\lthtmlcheckvsize\clearpage}

\stepcounter{chapter}
{\newpage\clearpage
\lthtmlfigureA{example7475}%
\begin{example}
  \item[PRINT PHOT {[BRIEF]}\hfill]{prints the information contained in
       the photometry file.  The output of this can be redirected.}
  \item[SAVE PHOT=file\hfill]{}
\par
\item[GET PHOT=file\hfill]{saves (or retrieves) VISTA photometry
       files from the disk.}
\par
\item[SAVE DAO=file\hfill]{}
\par
\item[GET DAO=file \hfill]{saves (or retrieves) DAOPHOT style photometry
       files from the disk.}
\par
\item[MODPHOT\hfill]{allows you to enter a name and/or the
       celestial coordinates for entries in a photometry file.}
\end{example}%
\lthtmlfigureZ
\lthtmlcheckvsize\clearpage}

{\newpage\clearpage
\lthtmlfigureA{example7484}%
\begin{example}
  \item[COORDS\hfill]{does a coordinate solution for the stars
on a frame, producing right ascension and declination.}
  \item[PHOTONS\hfill]{which adds noise or artificial stars
to a frame, to calibrate your photometry
or for experiments on the accuracy of photometry.}
\end{example}%
\lthtmlfigureZ
\lthtmlcheckvsize\clearpage}

\stepcounter{section}
\stepcounter{section}
{\newpage\clearpage
\lthtmlfigureA{command7492}%
\begin{command}
  \item[Form: OPTIONS {[op=value]} \hfill]{}
  \item[op]{specifies the two letter code for the option}
  \item[im]{specifies the value to set the parameter to}
\end{command}%
\lthtmlfigureZ
\lthtmlcheckvsize\clearpage}

\stepcounter{section}
{\newpage\clearpage
\lthtmlfigureA{command7502}%
\begin{command}
  \item[Form: FIND dum im {[THRESH=th]} {[LOWBAD=low]} {[INT]}\hfill]{}
  \item[dum]{specifies a dummy buffer}
  \item[im]{specifies the buffer in which to find stars}
\end{command}%
\lthtmlfigureZ
\lthtmlcheckvsize\clearpage}

\stepcounter{section}
{\newpage\clearpage
\lthtmlfigureA{command7512}%
\begin{command}
  \item[Form: PHOTOMETRY im {[BATCH]} {[RAD=]} {[GAIN=]} {[RN=]} {[SKY=]} {[SKYINT]} {[MEAN]} {[SKYRAD=r1,r2]}\hfill]{}
  \item[im]{specifies the buffer number}
\end{command}%
\lthtmlfigureZ
\lthtmlcheckvsize\clearpage}

\stepcounter{section}
{\newpage\clearpage
\lthtmlfigureA{command7526}%
\begin{command}
  \item[Form: GROUP {[CRIT=crit]}\hfill]{}
  \item[GROUP]{Groups stars according to user specified PSF overlap}
  \item[OLDGROUP]{Groups stars in the old DAOPHOT way, just by distances
to neighbors}
\end{command}%
\lthtmlfigureZ
\lthtmlcheckvsize\clearpage}

\stepcounter{section}
{\newpage\clearpage
\lthtmlfigureA{command7533}%
\begin{command}
  \item[Form: PSF im {[STARS=s1,s2,...]} {[INT]}\hfill]{}
  \item[im]{specifies the buffer number}
\end{command}%
\lthtmlfigureZ
\lthtmlcheckvsize\clearpage}

\stepcounter{section}
{\newpage\clearpage
\lthtmlfigureA{command7541}%
\begin{command} 
  \item[Form: PEAK im\hfill]{} \end{command}%
\lthtmlfigureZ
\lthtmlcheckvsize\clearpage}

\stepcounter{section}
{\newpage\clearpage
\lthtmlfigureA{command7545}%
\begin{command}
  \item[Form: NSTAR im  {[CLIP=nclip]}\hfill]{}
\end{command}%
\lthtmlfigureZ
\lthtmlcheckvsize\clearpage}

\stepcounter{section}
{\newpage\clearpage
\lthtmlfigureA{command7550}%
\begin{command}
  \item[Form: SUB* im EXCLUDE=s1,s2,s3,...\hfill]{}
  \item[im]{specifies the buffer number for the image.}
  \item[EXCLUDE=s1,s2,... ]{  Will not subtract stars (s1,s2,s3,....) from
    the frame}
\end{command}%
\lthtmlfigureZ
\lthtmlcheckvsize\clearpage}

\stepcounter{section}
{\newpage\clearpage
\lthtmlfigureA{example7557}%
\begin{example}
  \item[Form: MONITOR\hfill]{Turns on display of various information}
  \item[Form: NOMONITOR\hfill]{Turns off this display}
\end{example}%
\lthtmlfigureZ
\lthtmlcheckvsize\clearpage}

\stepcounter{section}
{\newpage\clearpage
\lthtmlfigureA{command7562}%
\begin{command}
  \item[Form: SORT {[INDEX=ind]} {[RENUM]} {[NORENUM]}\hfill]{}
\end{command}%
\lthtmlfigureZ
\lthtmlcheckvsize\clearpage}

\stepcounter{section}
{\newpage\clearpage
\lthtmlfigureA{command7569}%
\begin{command}
  \item[Form: OFFSET {[im]} {[INV]} {[DX=dx DY=dy]}\hfill]{}
\end{command}%
\lthtmlfigureZ
\lthtmlcheckvsize\clearpage}

\stepcounter{section}
{\newpage\clearpage
\lthtmlfigureA{command7576}%
\begin{command}
  \item[Form: SELECT {[SIZE=s1,s2]}\hfill]{}
\end{command}%
\lthtmlfigureZ
\lthtmlcheckvsize\clearpage}

\stepcounter{section}
{\newpage\clearpage
\lthtmlfigureA{command7581}%
\begin{command}
  \item[Form: APPEND\hfill]{}
\end{command}%
\lthtmlfigureZ
\lthtmlcheckvsize\clearpage}

\stepcounter{section}
{\newpage\clearpage
\lthtmlfigureA{command7585}%
\begin{command}
  \item[Form: DAOSKY im {[BOX=b]} {[3SIG]}\hfill]{}
  \item[im]{is the image number on which sky is to be determined}
  \item[BOX=b]{specfies VISTA box to do calculation in}
\end{command}%
\lthtmlfigureZ
\lthtmlcheckvsize\clearpage}

\stepcounter{section}
{\newpage\clearpage
\lthtmlfigureA{command7597}%
\begin{command}
  \item[Form: DUMP\hfill]{}
\end{command}%
\lthtmlfigureZ
\lthtmlcheckvsize\clearpage}

\stepcounter{section}
{\newpage\clearpage
\lthtmlfigureA{command7603}%
\begin{command}
  \item[Form: AUTOMARK imno {[RADIUS=rad]} {[RANGE=low,high]} {[REJECT=rej]}\hfill]{}
  \item{{[BOX=b]} {[NEW]} {[MOMENT=n]} {[EDGE=n]} {[SILENT]}}
  \item{{[ID=]} {[OBSNUM=]} {[DMIN=]} {[FORCE]} {[NITER=n]}}
  \item[imno]{is the image number on which stars are
to be found}
  \item[RADIUS=rad]{computes a centroid to locate each star
using a square of half-size 'rad'}
  \item[RANGE=low,high]{picks stars with peak counts between 'low' and 'high'}
  \item[REJECT=rej]{ignore stars that come within 'rad' of
ANY masked pixel}
  \item[BOX=b]{find stars in box 'b'}
  \item[NEW]{create a new list.}
  \item[MOMENT=n]{Specifies a new moment for the centroiding routine}
  \item[EDGE=n]{Only takes stars more than n pixels away from the edge of the frame}
  \item[SILENT ]{suppresses terminal output}
  \item[ID=id]{specifies DAOPHOT ID number to be associated with object}
  \item[OBSNUM]{specifies observation number to code into ID number}
  \item[DMIN=d]{specifies minimum distance an object needs to
lie from another to be judged a separate star}
  \item[FORCE]{forces all peaks to be marked as stars,
regardless of any nearby objects}
  \item[NITER=n]{specifies maximum number of iterations to try for
centroid. Default is 6.}
\end{command}%
\lthtmlfigureZ
\lthtmlcheckvsize\clearpage}

{\newpage\clearpage
\lthtmlfigureA{example7639}%
\begin{example}
  \item[AUTOMARK 2 RADIUS=2 REJECT=7 RANGE=300,30000\hfill]{
finds stars which have peak heights between 300 and 30000.  It 
considers a 5 by 5 box around each star, computing a centroid
in that box to locate the center.  Stars closer than 7 pixels
from masked pixels are rejected.}
\end{example}%
\lthtmlfigureZ
\lthtmlcheckvsize\clearpage}

\stepcounter{section}
{\newpage\clearpage
\lthtmlfigureA{command7647}%
\begin{command}
  \item[Form: MARKSTAR {[NEW]} {[RADIUS=r]} {[NOBOX]} {[STAR=s1,s2,...]}\hfill]{}
  \item{{[AUTO]} {[DR=dr]} {[DC=dc]} {[RSHIFT=rs]} {[CSHIFT=cs]} }
  \item{{[BOX=n]} {[COMPLETE]} {[ANGLE=a]} {[AR=ar]} {[AC=ac]}}
  \item{{[QUICK]} {[QSAVE]} {[EXIT]} {[CLR]} {[MOMENT=m]}}
  \item[NEW]{creates a new photometry file.}
  \item[RADIUS=r]{computes centroid in a box of half-size 'r'}
  \item[NOBOX]{do not display locations of stars}
  \item[STAR=s1,s2,...]{show location of individual stars}
  \item[AUTO]{use previous photometry file to compute new
positions.}
  \item[DR=dr]{new positions are about 'dr' in rows away
from the old positions (with AUTO only)}
  \item[DC=dc]{new positions are about 'dc' in columns away
from the old positions (with AUTO only)}
  \item[RSHIFT=rs]{ignore stars which are further away in rows
than 'rs' from the old position (with AUTO)}
  \item[CSHIFT=cs]{ignore stars which are further away in columns
than 'cs' from the old position (with AUTO)}
  \item[BOX=n]{Specifies size of boxes to draw on the TV}
  \item[COMPLETE]{For use with AUTO option, to force AUTOMARK to
find a new star for every object on the old list}
  \item[ANGLE=angle]{     For use with the AUTO options, to specify an image}
  \item[AR=row]{rotation, by the angle specified, around the row}
  \item[AC=col]{and column specified}
  \item[QUICK, QSAVE]{I forget what these do. Something quick. Check the code.}
  \item[EXIT]{Exit immediately after displaying stars. Do not
wait for more entries.}
  \item[CLR]{Clear the overlay (on devices with 7+1 bit display)
for whole image or around specified stars}
  \item[MOMENT=n]{Specifies moment for centroiding routine}
\end{command}%
\lthtmlfigureZ
\lthtmlcheckvsize\clearpage}

{\newpage\clearpage
\lthtmlfigureA{example7696}%
\begin{example}
  \item[TV 6\hfill]{}
  \item[MARKSTAR NEW RADIUS=2\hfill]{}
  \item[SAVE PHOT=./FIELDA\hfill]{
loads the image in buffer 6 into the TV.  MARKSTAR
then works on this image.  A new list is created.
After the stars are marked, the photometry file is
saved in the file ./FIELDA.PHO}
  \item[MARKSTAR\hfill]{
displays the current positions of the stars in the
photometry file on the TV.}
  \item[TV 6 \hfill]{}
  \item[MARKSTAR NEW RADIUS=2\hfill]{}
  \item[SAVE PHOT=./BLUE\hfill]{}
  \item[TV 7\hfill]{}
  \item[MARKSTAR AUTO DR=-0.5 DC=0.5 RADIUS=2\hfill]{}
  \item[SAVE PHOT=./RED\hfill]{ This is an example of marking two exposures
       of the same field.  The first three commands find the positions of
       the stars on image 6, just as in the first example.  The positions
       are saved in the file ./BLUE.PHO.  Then image 7 is loaded into the
       TV.  MARKSTAR AUTO takes the positions in the current file, adds
       -0.5 to each row and 0.5 to each column. It then goes through these
       positions, computing the centroid for the stars there.  The
       resulting file is saved in ./RED.PHO.}
\par
\item[PRINT PHOT BRIEF $>$phot.txt\hfill]{}
\par
\item[MARKSTAR STAR=5,13,206,1107\hfill]{ which outputs a short version
       of the photometry file to the ASCII file phot.txt which may then be
       printed or edited (see HELP PRINT for information).  MARKSTAR STAR=
       then draws boxes around the star \# 5, \#13, etc.}
\end{example}%
\lthtmlfigureZ
\lthtmlcheckvsize\clearpage}

\stepcounter{section}
{\newpage\clearpage
\lthtmlfigureA{command7715}%
\begin{command}
  \item[Form: APERSTAR source STAR=rs SKY=r1,r2 {[SKY=NONE]} {[GAIN=g]} {[RONOISE=r]} {[REJECT=sig]}\hfill]{}
  \item[source]{is the image being measured}
  \item[STAR=rs]{measures the total counts for each star.
An aperture of radius rs pixels is used.}
  \item[SKY=r1,r2]{An annulus of inner radius r1 and outer
radius r2 is used to measure the sky brightness}
  \item[SKY=NONE]{assumes the sky is zero.}
  \item[GAIN=g]{specifies the gain of the CCD in photons/count}
  \item[RONOISE=r]{specifies the read-out noise of the CCD in rms COUNTS.}
\end{command}%
\lthtmlfigureZ
\lthtmlcheckvsize\clearpage}

{\newpage\clearpage
\lthtmlfigureA{example7731}%
\begin{example}
  \item[APERSTAR 5 STAR=6 SKY=8,10 GAIN=10.5 RONOISE=8.57\hfill]{
This finds the brightness of all stars on image 5 which
have been marked in the photometry file with MARKSTAR 
or AUTOMARK.  Each brightness is computed in a circle of
radius 6 pixels, with the sky being computed using those
pixels from 8 to 10 pixels from the center of each star.
The errors are computed using a gain of 10.5 photons/count,
with a readout noise of 8.57 counts rms.}
\end{example}%
\lthtmlfigureZ
\lthtmlcheckvsize\clearpage}

\stepcounter{section}
{\newpage\clearpage
\lthtmlfigureA{command7738}%
\begin{command}
  \item[Form: COORDS {[output redirection]}\hfill]{}
\end{command}%
\lthtmlfigureZ
\lthtmlcheckvsize\clearpage}

\stepcounter{section}
{\newpage\clearpage
\lthtmlfigureA{command7745}%
\begin{command}
  \item[Form: MODPHOT\hfill]{}
\end{command}%
\lthtmlfigureZ
\lthtmlcheckvsize\clearpage}

\stepcounter{section}
{\newpage\clearpage
\lthtmlfigureA{command7749}%
\begin{command}
  \item[Form: SHORTAP {[AP=nap]}\hfill]{}
\end{command}%
\lthtmlfigureZ
\lthtmlcheckvsize\clearpage}

\stepcounter{section}
{\newpage\clearpage
\lthtmlfigureA{command7754}%
\begin{command}
  \item[Form: COMBINE {[REF=file]} {[DAT=file]} {[CMB=file]} {[MER]} {[COMB]}\hfill]{}
  \item{{[SKY=sky]} {[RN=rn]} {[GAIN=gain]} {[NEFF=neff]}}
  \item{{[FACT=fact]} {[NORM=norm]} {[REFMAG]} {[SILENT]} {[PLOT]} {[PORT]}}
  \item{{[HARD]} {[MAG=dm]} {[DIST=dist1,dist2,...]} {[DR=dr]} {[DC=dc]}}
  \item{{[SIG=sig]} {[MEAN=mean]} {[NMAT=nmat]} {[REF]} {[COO]}}
\end{command}%
\lthtmlfigureZ
\lthtmlcheckvsize\clearpage}

\stepcounter{section}
{\newpage\clearpage
\lthtmlfigureA{command7788}%
\begin{command}
  \item[Form: MAGAVER {[NORM]} {[PLOT]} {[TTY]} {[NOSHIFT]} {[FILTER=]}\hfill]{}
\end{command}%
\lthtmlfigureZ
\lthtmlcheckvsize\clearpage}

\stepcounter{section}
{\newpage\clearpage
\lthtmlfigureA{command7797}%
\begin{command}
  \item[Form: REGISTER {[im1,im2,im3...]} {[LOAD]} {[HEADER]} {[NOMEAN]}\hfill]{}
\end{command}%
\lthtmlfigureZ
\lthtmlcheckvsize\clearpage}

\stepcounter{section}
{\newpage\clearpage
\lthtmlfigureA{command7807}%
\begin{command}
  \item[Form: FITSTAR or FIT* {[AIR=]} {[HJD=]} {[COL=]} {[NOPLOT]} {[HARD]} {[HARD=]}\hfill]{}
  \item{{[PS=]} {[TITLE]} {[TITLE=]} {[RES]} {[RES=]} }
  \item{{[OUT=]} {[LOCK=]} {[OBSNUM=]} {[OBSNUM2=]}}
  \item{{[NOOBSNUM]} {[ERRMIN=]} {[STNERR=]} {[AP=]} {[APCOR=]}}
  \item{{[STN=]} {[DAT=]} {[SCOL=]} {[BATCH]} {[NEW]} {[NOPLOT]} }
\end{command}%
\lthtmlfigureZ
\lthtmlcheckvsize\clearpage}

\stepcounter{section}
{\newpage\clearpage
\lthtmlfigureA{command7847}%
\begin{command}
  \item[Form: CORRECT  {[BIN=bin]} {[PROF]} {[TRN=trnfile]} {[MAG=magfile]}\hfill]{}
  \item{{[LINE=line]} {[COLOR=color]} {[COLCOL=c1,c2]}}
  \item[BIN=bin]{specifies if a binning factor was used
for observations but not for standards}
  \item[PROF]{corrects PROFILE common block rather than stellar photometry file}
  \item[TRN=trnfile]{specifies name of transformation file}
  \item[MAG=magfile]{ specifies name of file with magnitudes}
  \item[LINE=line ]{specifies which transformation line to use}
  \item[COLOR=color]{specifies standard color to use in
transformation equation if you only have observations in one color}
  \item[COLCOL=c1,c2  ]{  specifies the 2 observed color columns to 
use in transformation if you have observations in more than one color}
\end{command}%
\lthtmlfigureZ
\lthtmlcheckvsize\clearpage}

\stepcounter{section}
{\newpage\clearpage
\lthtmlfigureA{command7866}%
\begin{command}
  \item[Form: DAOFILES {[COO=file]} {[MAG=file]} {[PSF=file]} {[PRO=file]}\hfill]{}
  \item{{[GRP=file]} {[FILE=file]} {[FILE2=file]} {[FILE3=file]} {[NONE]}}
\end{command}%
\lthtmlfigureZ
\lthtmlcheckvsize\clearpage}

\stepcounter{section}
{\newpage\clearpage
\lthtmlfigureA{command7879}%
\begin{command}
  \item[Form: AUTOCEN buf {[N=n]} {[SIZE=n]} {[STEP=n]} {[C=(r,c)]} 
       {[PLOT]}\hfill]{}
\end{command}%
\lthtmlfigureZ
\lthtmlcheckvsize\clearpage}

\stepcounter{chapter}
{\newpage\clearpage
\lthtmlfigureA{example8211}%
\begin{example}
  \item[PHOTONS\hfill]{add artificial stars and/or noise to an image}
  \item[TEMPLATE\hfill]{generate an image from a user-supplied profile}
  \item[DEVAUC\hfill]{generate an image with a deVaucouleurs profile}
\end{example}%
\lthtmlfigureZ
\lthtmlcheckvsize\clearpage}

\stepcounter{section}
{\newpage\clearpage
\lthtmlfigureA{command8219}%
\begin{command}
  \item[Form: PHOTONS source [NSTARS=] [AT=r,c] [COUNTS=c1,c2] [MEAN=]\hfill]{}
  \item{[RN=] [GAIN=] [GAUSS] [FILE=] [FW=] [DAOPSF=]}
  \item{[PHOT] [NEW] [PSFLIB=] [STR=] [NDIV] [TRUNC]}
  \item[NSTARS=]{gives the number of stars to make.}
  \item[AT=R,C]{Put a single star at row R and column C
       This cannot be used with NSTARS.}
  \item[STR=file]{specifies a file with the positions and
       counts of the stars to be added}
  \item[COUNTS=a,b]{gives ranges in total counts for the
       stars. If only one number is given, range will be from 0 to that number}
  \item[MEAN=]{gives the mean level (counts) added to the image to mimic
       sky. If mean>=0, photon noise will be added properly given counts in
       each pixel.  Mean=-1 if unspecified. If mean<0 then no photon noise
       is added to image. }
  \item[RN=]{ gives the readout noise (electrons) 0 if unspecified. }
  \item[GAIN=]{gives the conversion between photons and counts.  This is in
       the units photons per count. Needed to do noise properly.}
  \item[GAUSS ]{tell program to make images from a sampled Gaussian, with
       FWHM specified by FW= keyword If not include, PSF profile will be
       read a a disk file specified by FILE= keyword}
  \item[FW=]{star FWHM in pixels for Gaussian, or half width for arbitrary
       PSF (in disk file)}
  \item[FILE=]{ specifies file with 250 numbers (arr(i),i=1,250)
       representing 1-D PSF: half-width (no. of pixels per 250 bins) must
       be specified with FW keyword.  If FILE is specified, GAUSS is
       ignored}
  \item[DAOPSF=file]{creates stars with PSF from a DAOPHOT .PSF file}
  \item[PSFLIB=file]{Specifies PSFLIB library file to use to look up PSF at
       various pixel centerings. Must be created with PSFLIB command and
       special image}
  \item[PHOT]{store the positions and total counts of added
       stars in a photometry file. }
  \item[NEW]{creates a new photometry file.  NEW implies PHOT}
  \item[NDIV]{allows user to change the pixel subdivisions
       used for the PSF integration over pixels}
  \item[TRUNC]{Makes PHOTONS integer truncate}
\end{command}%
\lthtmlfigureZ
\lthtmlcheckvsize\clearpage}

\stepcounter{section}
{\newpage\clearpage
\lthtmlfigureA{command8244}%
\begin{command}
  \item[Form: TEMPLATE dest source {[PA=n]} {[PAM=n]} {[E=n]} {[FIT=n1,n2]}
       {[SUB]} {[GAUSS]} {[EXP]} {[HUB]} {[DEV]}\hfill]{}
  \item[dest]{(integer or \$ construct) is an image which already 
       exists to which the template will be written,}
  \item[source]{(integer or \$ construct) is the spectrum that
       contains the profile used to generate the template,}
  \item[PA=n]{(re)sets the position angle of radial profile
       extended object,}
  \item[PAM=n]{(re)sets the position angle of the major axis
       of the original image,}
  \item[FIT=n1,n2]{requests the template be generated by fitting
       the profile spectrum over pixels n1 through n2
       with the appropriate surface brightness law,}
  \item[SUB]{requests that the template be subtracted from
       image already in image buffer specified,}
  \item[GAUSS]{generates a Gaussian template (point source),}
  \item[EXP]{generates an exponential disk template (spiral),}
  \item[HUB]{generates a Hubble template (elliptical),}
  \item[DEV]{generates a deVaucouleurs template (elliptical).}
\end{command}%
\lthtmlfigureZ
\lthtmlcheckvsize\clearpage}

{\newpage\clearpage
\lthtmlfigureA{example8267}%
\begin{example}
  \item[TEMPLATE 2 1 FIT=10,19 EXP\hfill]{Generates an exponential disk
       template from the profile in spectrum 1 (fit to pixels 10 through
       19) and writes the image to image buffer 2 (which already exists).}
\par
\item[TEMPLATE 2 1 FIT=10,19 DEV SUB \hfill]{same as above, but fits
       deVaucouleurs profile and subtracts fit from image already in 2}
\end{example}%
\lthtmlfigureZ
\lthtmlcheckvsize\clearpage}

\stepcounter{section}
{\newpage\clearpage
\lthtmlfigureA{command8272}%
\begin{command}
  \item[Form: DEVAUC source [REFF=r] [SEFF=s] [X0=x0] [Y0=y0] 
       [PIX=pix]\hfill]{}
  \item[source]{gives the image in which to create the simulation}
  \item[REFF=reff]{specifies the effective radius (arcmin)}
  \item[SEFF=seff]{specifies the surface brightness at reff (counts/pixel)}
  \item[X0=]{specifies the column center}
  \item[Y0=]{specifies the row center}
  \item[PIX=pix]{specifies the pixel scale ("/pixel)}
\end{command}%
\lthtmlfigureZ
\lthtmlcheckvsize\clearpage}

\stepcounter{chapter}
\stepcounter{section}
{\newpage\clearpage
\lthtmlfigureA{command8323}%
\begin{command}
  \item[Form: \$ Any valid shell command\hfill]{}
\end{command}%
\lthtmlfigureZ
\lthtmlcheckvsize\clearpage}

\stepcounter{section}
{\newpage\clearpage
\lthtmlfigureA{command8327}%
\begin{command}
  \item[Form: TIME cmd\hfill]{}
\end{command}%
\lthtmlfigureZ
\lthtmlcheckvsize\clearpage}

\stepcounter{section}
{\newpage\clearpage
\lthtmlfigureA{command8331}%
\begin{command}
  \item[Form: CLOCK\hfill]{}
\end{command}%
\lthtmlfigureZ
\lthtmlcheckvsize\clearpage}

\stepcounter{section}
{\newpage\clearpage
\lthtmlfigureA{command8335}%
\begin{command}
  \item[Forms: BELL Y, BELL N, or BELL R\hfill]{}
\end{command}%
\lthtmlfigureZ
\lthtmlcheckvsize\clearpage}

{\newpage\clearpage
\lthtmlfigureA{example8338}%
\begin{example}
  \item[N\hfill]{prevents the bell from ringing}
  \item[Y\hfill]{restores the bell with the prompt}
  \item[R\hfill]{rings the bell}
\end{example}%
\lthtmlfigureZ
\lthtmlcheckvsize\clearpage}


\end{document}

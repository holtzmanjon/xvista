%
%  VISTA Cookbook;  Chapter X.  Dirty Tricks
%
%  Began:  1988 August 9
%
%  Last Revision:  1988 August 31
%
%  Note:  Throughout this chapter, subsections are all labelled for
%         cross-referencing.  The base reference key is:
%
%                        sec:dirt
%


%\documentstyle {manual}
%\input sys$user:[rick.thesis]thesismacros.pogge
%
%\newenvironment{command}{\begin{center}
%\begin{list}{\tt GO:}{\setlength{\rightmargin}%
%{\leftmargin}}\tt\singlespace }{\end{list}\end{center}}
%
%\def \comm#1{{\tt #1\/}}
%
%\begin{document}

\chapter{Dirty Tricks}

The VISTA plotting routines are as general as we could make them, and as a
result, there is a wide range of possibilities.  Many of these take the form
of what we call ``dirty tricks'' --- little hooks that can enhance the
appearance of a plot, without trying to re-write huge sections of code to add
custom features.  These are some of our favorites.  Be warned, however, that
this is genuine ``black belt'' stuff.

\section{User Defined Axes for Plots}

The \comm{USER} keyword in the \comm{PLOT} command may be used to make custom
scalings and labelling of the user coordinates.  They work by way of the
coordinate cards provided in the FITS header.  These cards are as follows, one
set for each axis (row and columns):
\begin{verbatim}
      For Columns:  CRVAL1, CRPIX1, CDELT1, CTYPE1

         For Rows:  CRVAL2, CRPIX2, CDELT2, CTYPE2
\end{verbatim}
The internal coordinates of any image are the so-called ``pixel'' coordinates,
simply: Columns and Rows.  Images displayed by VISTA are always in pixel
coordinates by default (except for wavelength calibrated spectra).  The user
may define coordinates which have some physical meaning (like angular measure
in arcseconds or wavelength in angstroms).  These are called ``world
coordinates'' or ``physical coordinates'' (in the sense that their units have
some meaning in the world outside the machine).  The transformation between
``world'' and ``pixel'' coordinates is through the FITS header cards shown
above.  The tranformation equations are:
\begin{verbatim}
      X(world) = CRVAL1 + CDELT1*(Column - CRPIX1)

      Y(world) = CRVAL2 + CDELT2*(Row - CRPIX2)
\end{verbatim}
Where X and Y axes correspond to Columns and Rows respectively. In general,
the FITS coordinate cards correspond to:
\begin{verbatim}
      CRVALn  =  Origin along Axis n in World Coordinates
      CRPIXn  =  Origin along Axis n in Pixel Coordinates
      CDELTn  =  Scaling Factor giving relation between
                 Pixels and World Coordinates
      n=1 for columns
      n=2 for rows
\end{verbatim}

The units of the world coordinates are identified by the \comm{CTYPE1} and
\comm{CTYPE2} FITS header cards for Columns and Rows respectively.  These
contain character strings which give the units.  The units of the pixel data
itself (\eg\ pixel intensities) are given by the \comm{BUNIT} FITS header
card.

Confusing?  How about an example.  You have a 2-D image in which Columns are
oriented East towards West in the sky (RA), and Rows are along North towards
South (Dec).  After some processing, the image pixels measure flux in
millijanskys (mJy) at 2 microns.  You want to plot profiles of the image of
the object (a planetary nebula) through the center along both RA (by plotting
along Row 128), and Dec (by plotting along Column 67).  The pixel scale in
both rows and columns is 0\arcsecpoint54/pixel, and you want the origin to be
the central star of the planetary, which is at \comm{(C,R)=(67,128)}.  The
transformation equations between image pixels (C,R) and world coords (RA,Dec)
is then:
\begin{verbatim}
      RA = 0.0 - 0.54*(C - 67)
     Dec = 0.0 - 0.54*(R - 128)
\end{verbatim}
Note that because RA increases from West to East, and column number increases
in the opposite direction, the scale factor is $-0.54$ rather than $+0.54$.
Since the Row number increase North to South (top to bottom), the scale factor
is $-0.54$ as for columns.  This means that the necessary FITS coordinate
cards you must load are:
\begin{verbatim}
      CRVAL1 = 0.0        CRVAL2 = 0.0
      CRPIX1 = 67         CRPIX2 = 128
      CDELT1 = -0.54      CDELT2 = -0.54
\end{verbatim}
And, to get the units to be plotted right, you would load these units cards
into the header:
\begin{verbatim}
      CTYPE1 = 'Right Ascension (arcsec)'
      CTYPE2 = 'Declination (arcsec)'
      BUNIT = 'Flux (mJy)'
\end{verbatim}

Since the units cards are used for keeping track of the image internally
to VISTA, you need to proceed with a little caution.  Suppose the
image of interest is in Buffer 1.  We want to copy it into Buffer 2
and then start monkeying with the FITS cards to get our plots.  Use the
following procedure:

\begin{command}
      \item COPY 2 1
      \item FITS 2 FLOAT=CRVAL1 0.0
      \item FITS 2 FLOAT=CRPIX1 67.
      \item FITS 2 FLOAT=CDELT1 -0.54
      \item FITS 2 FLOAT=CRVAL2 0.0
      \item FITS 2 FLOAT=CRPIX2 128.
      \item FITS 2 FLOAT=CDELT2 -0.54
      \item FITS 2 CHAR=CTYPE1 'Right Ascension (arcsec)'
      \item FITS 2 CHAR=CTYPE2 'Declination (arcsec)'
      \item FITS 2 CHAR=BUNIT 'Flux (mJy)'
      \item COPY 2 2
\end{command}

The FITS cards for the pixel-to-world coordinate transformations have all been
loaded.  The command \comm{COPY 2 2} makes VISTA happy again, as it now knows
where the image is.  Now, make the plots, first along RA (Columns), then along
Dec (Rows):

\begin{command}
      \item PLOT 2 R=128 USER
      \item PLOT 2 C=67 USER
\end{command}

In order to be as general as possible, I chose an example in which both axes
of the image were changed.  If the image was a spectrum (1-D image), then you
would only have changed the \comm{CRVAL1}, \comm{CRPIX1}, \comm{CDELT1},
\comm{CTYPE1}, and \comm{BUNIT} cards in the FITS header. In the example
given, I've used familiar angular coordinates.  They could just as easily have
been microns along a slit and angstroms of wavelength perpendicular to it.
There are no limitations to the transformation or the units, except that the
transforms be linear in one and only one pixel coordinate (true, there is a
FITS card called \comm{CROTAn} which handles rotations of the given axis, but
VISTA doesn't handle them in its current incarnation).

It is important to note that VISTA uses the FITS header cards in a somewhat
non-standard way internally.  This is discussed in detail in Appendix D.  The
bottom line of this and related dirty tricks which deal with changing these
particular FITS cards is to be very circumspect.  A little lack of caution,
and VISTA could crash out from under you --- most often with a fatal virtual
memory error (VMS will tell you, Unix will just have it wink out of existance
with a ``bus error, core dumped'' message).

%------------------------------------------------------

\section{Putting Labels on Plots}

The \comm{INT} keyword of the \comm{PLOT} command puts you temporarily into
the interactive MONGO command environment, with the specified plotting data
loaded into the X and Y plotting arrays.  Thus, you can have the full power of
MONGO to make a custom plot at your disposal, without the intermediary of
having to write the data (somehow) into a 2 column text file of X and Y points
and then leaving VISTA.

The interactive MONGO prompt is an asterisk ($\ast$).  Once you have the MONGO
prompt, you can enter all valid interactive MONGO commands. To leave
interactive MONGO and return to VISTA, type the \comm{END} command.

The X and Y data are pre-loaded by the \comm{PLOT} subroutine.  Pre-prepared
MONGO macros may be read in using the MONGO command \comm{READ}; do not use
the \comm{INPUT} command.

There are a number of ways to put labels on a plot once you are in the
interactive MONGO command level.  The \comm{PUTLABEL} command is the best way,
specifying the location of the label with \comm{RELOCATE}. For example:  you
wish to label the H$\alpha$\ and H$\beta$\ emission lines in a spectrum of a
planetary nebula (with no redshift).  The spectrum is in buffer 10, and you
want the plot to run from 4000 to 7000\AA, with the lowest intensity at 0.0,
and the highest intensity autoscaled.  The labels are to be centered over the
lines.  To do this, you would issue the following sequence of VISTA and MONGO
commands.  The VISTA commands are indicated with the \comm{GO:} prompt, and
the MONGO commands with the \comm{*} prompt.

The first pass is to only plot on the screen to get the location of the
labels right.  The second pass, you repeat the relocate and putlabels
that gave the best results, this time blind (\ie\ you don't see a plot
or the results of the plot), and the hardcopy is generated.

\begin{verbatim}

            GO: PLOT 10 XS=4000 XE=7000 MIN=0. INT

            * RELOCATE 4861 1.5E-15
            * PUTLABEL 5 H\gb
            * RELOCATE 6563 5.0E-15
            * PUTLABEL 5 H\ga
              .
              .
         (iterate until the labels look right)
              .
              .
            * END

   (now, do it again, but make a hardcopy)

            GO: PLOT 10 XS=4000 XE=7000 MIN=0. INT HARD
            * RELOCATE 4861 1.5E-15
            * PUTLABEL 5 H\gb
            * RELOCATE 6563 5.0E-15
            * PUTLABEL 5 H\ga
            * END

            GO:

\end{verbatim}

While in interactive MONGO, you can erase the screen, re-scale the plot, put
down your own axis labels and so forth, as if you had read in the data from a
external file using the \comm{XCOLUMN} and \comm{YCOLUMN} commands in MONGO.
If you are on a terminal with an interactive cursor, then you can use the
\comm{CURS} command in MONGO to make placing the labels faster.  The
possibilites are virtually endless.

If you are unfamiliar with the MONGO package, a copy of the Lick MONGO User's
Manual has been provided with the VISTA distribution package, and should be
available from your VISTA custodian.

%------------------------------------------------------

\section{Plotting RA and DEC on Contour Maps}

The \comm{CONTOUR} command also has a \comm{USER} keyword, but since this
necessitates fooling with FITS header cards used as internal array index
pointers by VISTA, some care must be taken to avoid crashing VISTA.  The use
of the FITS header cards is identical to that described above for user defined
coordinates on plots.  Consider the planetary nebula example from above.  We
now wish to make a contour map with the origin on the central star centroid
which has been found to be at \comm{(C,R)=(67.2,127.9)}, and with contours
spaced every 2 mJy beginning at 1 mJy.  The image is in Buffer 1.  Copy it
into Buffer 2, and load the appropriate FITS cards as follows:
\begin{command}
      \item COPY 2 1
      \item FITS 2 FLOAT=CRVAL1 0.0
      \item FITS 2 FLOAT=CRPIX1 67.2
      \item FITS 2 FLOAT=CDELT1 -0.54
      \item FITS 2 FLOAT=CRVAL2 0.0
      \item FITS 2 FLOAT=CRPIX2 127.9
      \item FITS 2 FLOAT=CDELT2 -0.54
      \item FITS 2 CHAR=CTYPE1 'Right Ascension (arcsec)'
      \item FITS 2 CHAR=CTYPE2 'Declination (arcsec)'
      \item COPY 2 2
\end{command}
above, and then type the following command:
\begin{command}
      \item CONTOUR 2 LOW=1. DIFF=2. USER TITLE
\end{command}
to make the contour map.  An example of a contour map (not of a planetary
nebula) made this way with VISTA can be found in Pogge (1988, {\it Ap.J.},
{\bf 328}, 519).  A simple proof that a VISTA dirty trick is potentially
acceptable by the editor of the {\it Ap.J.}!

%------------------------------------------------------

\section{Preparing Images with Boxes and Scale Bars}

Color photography of the images displayed on the color monitor can make very
nice, impressive slides for presentation.  A few simple hooks were built-in to
the \comm{TV} command to give users some latitude in preparing images for
display.  Actual tricks for photographing off the monitor are not discussed as
we cannot seem to get a consensus among the photographers at Lick about how to
do it.  The only common points are: (a) all agree to use color slide film,
ASA100, and (b) to use a long focal length lens so that the curvature of the
monitor screen doesn't make the edges go out of focus.  Beyond that advice,
you're pretty much on your own as to technique.

Few things are more irritating than to have a speaker show slide after slide
of objects without some identifying label.  Great stuff, to be sure, but what
is the name of that object?  Also, most images don't have scale bars. So
someone has to interrupt with ``what is the image scale we're looking at?''.
The problem here is one of proper form.  You've been staring at these images
for months.  You know them by name and the image details intimately. But,
don't forget that (1) your audience has probably never seen these objects
before, (2) isn't necessarily up on their names, (3) doesn't know a thing
about your instrument (like image scale), and (4) isn't going to know any of
it because you told them before flashing up the slide.  Put it all up in front
of them so they can see it for themselves, and it will be greatly appreciated.

Built into VISTA's \comm{TV} command are two keyword called \comm{TITLE} and
\comm{TOP} which facilitate putting up image titles.  \comm{TITLE} puts the
object title (FITS header OBJECT card) on the TV display, without plotting the
tickmarks or numbers along the axes. \comm{TOP} puts the label over the image,
rather than below it (which is default).  It's nice to have a box (border)
around the image, but the axis ticks and labels giving pixel numbers cause too
much clutter.  Face it AEDs don't do so hot with axis labels despite our best
efforts unless the image is really big. The trick here is to have a border
around the image without the ticks.  This is accomplished using the
\comm{TVBOX} command.  How about an example.

Suppose you want to display the image in Buffer 1 with a border around it and
the title ``A Cool Galaxy'' centered over the image. Let's say that the image
zero and span are `0.0' and `1200.0' respectively, and you want to use the
inverse black and white color map (photonegative), and clip the colormap
roll-over.  You would do this as follows:
\begin{command}
      \item CH 1 'A Cool Galaxy'
      \item TV 1 Z=0.0 L=1200.0 CF=IBW CLIP TITLE TOP
      \item BOX 1 SR=SR[1] SC=SC[1] NR=NR[1] NC=NC[1]
      \item TVBOX BOX=1
\end{command}
We're using \comm{TVBOX} with a box which exactly surrounds the image
displayed to make our border for us.  The keywords \comm{TITLE TOP} on the
\comm{TV} command tell VISTA to display the image without axis ticks and
labels, with the image title centered over the top of the image.  Note that it
was necessary to change the object title to our desired title using the
\comm{CH} command before displaying the image.

A scale bar, say one indicating 10\arcsec\ on the sky would be nice too, it
gives your viewer and idea of the size.  In Appendix B, a short command
procedure called ``BAR'' has been given which will draw a scale bar on the
image for you.  Put this procedure into a file called ``BAR.PRO'' in your
procedure directory.  Since we are clipping the image color scale with the
\comm{TV} command at \comm{L=1200.0}, we will set the ``intensity'' of the bar
to be 1199.0 (1200--1).  Pick a place for the bar to start (it will be drawn
to the right of the place you pick) in rows and columns. I'm going to pick
ROW=300, COL=18.  For the image in question, 10 arcseconds on the sky
corresponds to 17 pixels on the image.  Use this procedure to draw the bar
into the image, and redisplay the image.  BAR will ask for the place to draw
the bar, its length, and its intensity.  NOTE:  BAR will change data values in
the image, not just draw on the screen.  Best thing to do is copy the image
into another buffer before hitting it with BAR.

\begin{verbatim}
      GO: COPY 2 1
      GO: CALL BAR
      IMAGE NUMBER ? : 2
      PUT BAR IN WHAT IMAGE ROW ? : 300
      STARTING COLUMN ? : 18
      LENGTH ? : 17
      INTENSITY VALUE ? : 1199.
      GO: TV 2 Z=0.0 L=1200.0 CF=IBW CLIP TITLE TOP
      GO: BOX 2 SR=SR[2] SC=SC[2] NR=NR[2] NC=NC[2]
      GO: TVBOX BOX=2
\end{verbatim}

Now, if this is what you want, get your camera and shoot away.

%------------------------------------------------------


%\end{document}
%\end

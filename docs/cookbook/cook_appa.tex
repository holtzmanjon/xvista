%
%  VISTA Cookbook;  VISTA Command Summary (Appendix A)
%
%  Began:  1988 July 10
%
%  Last Revision:  1988 August 9
%
%  Note:  Throughout this chapter, subsections are all labelled for
%         cross-referencing.  The base reference key is:
%
%                        sec:com
%

%\documentstyle {manual}
%\input sys$user:[rick.thesis]thesismacros.pogge
%
%\newenvironment{command}{\begin{center}
%\begin{list}{\tt GO:}{\setlength{\rightmargin}%
%{\leftmargin}}\tt\singlespace }{\end{list}\end{center}}
%
%\def \comm#1{{\tt #1\/}}
%\def \hitkey#1 {$\langle${\tt #1}$\rangle$\/}
%
%\begin{document}
%
%\appendix

\chapter{VISTA Command Summary}

The following is a summary of standard VISTA commands, intended to be used as
a quick reference guide.  Commands are listed by function.  Detailed
descriptions of each of the commands may be found in the VISTA Help Manual, or
by typing \verb+HELP <command name>+.

%---------------------------------------

\section{Stopping VISTA}

The end a VISTA session, type:
\begin{verbatim}
      QUIT
\end{verbatim}

%---------------------------------------

\section{VISTA Command Syntax}

The basic command syntax for all VISTA commands is as follows:
\begin{verbatim}
     COMMAND imbuf req#1 req#2 ... [opt#1] [opt#2] ...
\end{verbatim}
Where:
\begin{description}
      \item[{\tt COMMAND}] is the command name.  You do not need to type
            the full name.  Commands may be abbreviated to the shortest
            unambiguous form.
      \item[{\tt imbuf}]   is the image buffer the command is to operate
            on.  It is an integer.  In some cases, more than one image may
            be required.
      \item[{\tt req\#n}] are the required keywords.  These may NOT be
            abbreviated.
      \item[{\tt [opt\#n]}] are the optional keywords.  These may NOT
            be abbreviated either.
\end{description}

\noindent \underbar{Command Keywords}:

The keyword syntax used throughout is:
\begin{verbatim}
      uservalue1 KEY1 KEY2=uservalue2 KEY3=user3,user4
\end{verbatim}
Capitalized words are command keywords.  Lowercase words are numerical or
string values that the user must supply.

\noindent \underbar{Extended Commands}:

Commands can be extended to more than one line by ending the line with the
``\verb+|+'' character (the ``pipe'' character, not a lowercase l), hitting
\hitkey{RETURN}\ and entering the rest of the command on the next line.
Commands are limited to 256 characters in length.

\noindent \underbar{Multiple Commands}:

Multiple commands can be chained together on a command line separated commands
by semicolons (\comm{;}).

\noindent \underbar{Command History}:

The command history is a list of the last 20 VISTA commands.  It allows you to
see what you have typed in, and with the command recall facility (\comm{\%}
command below), you can recall any command listed by \comm{HISTORY}.

\begin{verbatim}
      HISTORY [output redirection]
\end{verbatim}

\noindent \underbar{Repeating and/or Modifying Previous Commands}:
\begin{verbatim}
   %            repeat last command
   % newkey     repeat last command with new keyword "newkey"
   %xyz         repeat last command beginning with pattern "xyz"
   %xyz newkey  as above, but add new keyword "newkey"
\end{verbatim}

\noindent \underbar{Editing Previous Commands}:
\begin{verbatim}
      EDIT
\end{verbatim}

\noindent \underbar{Defined a Command Synonym (Alias)}:
\begin{verbatim}
      ALIAS [synonym] ['command'] [output redirection]
\end{verbatim}

\noindent \underbar{Remove a Command Synonym}:
\begin{verbatim}
      UNALIAS synonym
\end{verbatim}

\noindent \underbar{On-Line Command Help Facilities}:
\begin{verbatim}
       HELP [subjects] [ALL] [output redirection]
   or
       ? command
\end{verbatim}

\noindent \underbar{Appropriate Command Search}:
\begin{verbatim}
      APROPOS topic_1 topic_2 ...
\end{verbatim}
\begin{quote}
      \comm{APROPOS} suggests possible VISTA \comm{HELP} topics based
      on a keyword search.   The topics are the principle sections of
      the help manual.  Similar to the Unix \comm{apropos} command.
\end{quote}

%-------------------

\section{VISTA User-Defined Variables}

\subsection{Defining VISTA variables}

\underbar{Definition of a VISTA variable}:
\begin{verbatim}
      SET variable_name=value
   or
      variable_name=value
\end{verbatim}

\noindent Note that there can be NO SPACES separating the \comm{=} from either
the variable name or the value.  Variable names in VISTA are insensitive to
case, so \comm{IGGY} is the same as \comm{iggy} is the same as \comm{IgGy},
and so forth.  Variable names may contain both characters and numbers.

\subsection{Arithmetic Operations among VISTA Variables}

The following table lists the arithmetic operations supported by VISTA.

\begin{center}
   \begin{tabular}{|c|l|ccc|} \hline
      Symbol&\multicolumn{1}{|c|}{Function}&
      \multicolumn{3}{c|}{Examples}\\ \hline
      {\tt +}&addition&{\tt B+C}&&{\tt 2.0+5}\\ \hline
      {\tt -}&subtraction&{\tt X-Y}&&{\tt 123-54}\\ \hline
      {\tt *}&multiplication&{\tt X1*X2}&&{\tt 56.4*60.0}\\ \hline
      {\tt /}&division&{\tt A/B}&&{\tt 125/0.54}\\ \hline
      {\tt -}&negative sign&{\tt -X}&&{\tt -12.3}\\ \hline
      {\verb+^+}&exponentiation&{\verb+A^0.5+}&&{\verb+12^2+}\\ \hline
      {\tt =}&equate&{\tt A=B}&&{\tt A=B=3.5}\\ \hline
   \end{tabular}
\end{center}

\noindent Note that there must be no spaces between the operator and the
operands.  For example:
\begin{verbatim}
      X=A+B-C        is valid
  but
      X = A + B - C    is not
\end{verbatim}

Expressions are evaluated in the same order as they are in Fortran. You may
change the order of evaluation using parentheses "()".  For example:
\begin{verbatim}
      X=(B+0.53)*10^(45.6/(A+5))
\end{verbatim}

\noindent Arithmetic expressions among VISTA functions may be of arbitrary
complexity.  The only requirements are that all variables appearing on the
right hand side of equations be previously defined and that parentheses be
balanced. NOTE:\ SPACES MAY NOT APPEAR ANYWHERE WITHIN AN ARITHMETIC
EXPRESSION.

\subsection{VISTA Functions}

VISTA supports the following functions.  Arguments of functions may be numbers
or previously defined VISTA variables (as appropriate).  Functions can appear
as arguments of other functions, and can contain expressions of arbitrary
complexity. Spaces are not allowed anywhere within a function.

\noindent \underbar{Arithmetic Functions}:
\begin{verbatim}
   INT[E]          nearest integer to the expression E
   ABS[E]          absolute value of E
   MOD[E,I]        E modulo I (remainder of E/I)
   IFIX[E]         integer part of E (truncation)
   MAX[E,F]        the larger of E or F
   MIN[E,F]        the smaller of E of F
   LOG10[E]        Log base 10 of E
   LOGE[E]         Log base e ("natural log") of E
   EXP[E]          e raised to the power E
                     (Use ^ for all other exponentiations)
   SQRT[E]         square root of absolute value of E
   RAN[A,B]        returns a random number between A and B
\end{verbatim}

\noindent \underbar{Trigonometric Functions}:
\begin{verbatim}
   SIN[E]         sine of E (E in radians)
   SIND[E]        sine of E (E in degrees)
   COS[E]         cosine of E (E in radians)
   COSD[E]        cosine of E (E in degrees)
   ARCTAN[E]      arctan of E, returns radians
   ARCTAND[E]     arctan of E, returns degrees
\end{verbatim}

\noindent \underbar{Functions to Extract Image FITS Header Information}:
\begin{verbatim}
   NR[B]       number of rows of the image in buffer B.
   NC[B]       number of columns of ...
   SR[B]       starting row of ...
   SC[B]       starting column of ...
   EXPOS[B]    exposure time of ...
   RA[B]       right ascension in seconds of time of ...
   DEC[B]      declination in seconds of arc of ...
   ZENITH[B]   zenith distance in radians of ...
   UT[B]       UT time of start of exposure in hours of ...
\end{verbatim}

\noindent \underbar{Advanced Image Pixel Functions}
\begin{verbatim}
   GETVAL[I,R,C]   returns the value of the pixel at
                   row R and column C in image I.
   SETVAL[I,R,C,V] returns the value of the pixel at
                   row R and column C in image I, then
                   sets the value of that pixel to be
                   equal to V.
   WL[I,P]         returns wavelength of pixel P in image
                   I (a wavelength calibrated spectrum).
   PIX[I,W]        returns the pixel number corresponding
                   to the wavelength W in image I.
\end{verbatim}

\subsection{Variable Related Commands}

\underbar{Define a Variable or Set of Variables}:
\begin{verbatim}
   SET var1=value1 [var2=value2] ... [var15=value15]
\end{verbatim}
Note that you may omit \comm{SET} if desired.

\noindent \underbar{Evaluate an Expression and Print the Result}:
\begin{verbatim}
   TYPE expression1 [expression2] ... [expression15]
\end{verbatim}

\subsection{String Variables}

\underbar{To Define String Variables}:
\begin{verbatim}
   STRING name ['format string'] [expressions]
   STRING name '?query'
\end{verbatim}

\noindent \underbar{Print List of All User-Defined Strings}:
\begin{verbatim}
   PRINT STRINGS [output redirection]
\end{verbatim}

\noindent \underbar{Substituting String Variables into a Command Line}:
\begin{verbatim}
   COMMAND {STRING} [rest of command]
\end{verbatim}

%-------------------

\section{Tape and Disk Input/Output}

\subsection{FITS Tape Input/Output}

\underbar{Mounting and Dismounting Tapes}:
\begin{verbatim}
   MOUNT [UNIT=n] [BPI=m]

   DISMOUNT [UNIT=n]
\end{verbatim}

\noindent \underbar{Listing of Tape Contents}:
\begin{verbatim}
   TDIR [comment] [UNIT=n] [BRIEF] [output redirection]
\end{verbatim}

\noindent \underbar{Reading and Writing Images}:
\begin{verbatim}
   RT imbuf tape_number [UNIT=n] [NOMEAN]

   WT imbuf [PDP8] [ZERO=z SCALE=s] [NOAUTO] [UNIT=n]
            [BITPIX=m]
\end{verbatim}

\noindent \underbar{Initialize or Edit a VISTA FITS Tape}:
\begin{verbatim}
   INT [firstimage#] [UNIT=n]
\end{verbatim}

\subsection{VISTA Disk Format Image Input/Output}

\underbar{List the Default Disk Directories}:
\begin{verbatim}
   PRINT DIRECTORIES
\end{verbatim}

\noindent{Change the Default Directory Assignment}:
\begin{verbatim}
   SETDIR IM [DIR=directory_name] [EXT=extension]
 and
   SETDIR SP [DIR=directory_name] [EXT=extension]
\end{verbatim}
\comm{IM} is for images, \comm{SP} is for spectra.

\noindent \underbar{Read an Image from Disk}:
\begin{verbatim}
   RD imbuf filename [SPEC]
\end{verbatim}

\noindent \underbar{Write an Image to Disk}:
\begin{verbatim}
   WD imbuf filename [SPEC] [FULL] [ZERO=z SCALE=s]
\end{verbatim}

\noindent \underbar{Write a Spectrum into an ASCII Text File on Disk}:
\begin{verbatim}
   PRINT spbuf SPEC >filename
\end{verbatim}

%-------------------

\section{VISTA Buffers}

\underbar{Listing of VISTA Buffers}:
\begin{verbatim}
   BUFFERS [bufs] [FULL] [FITS[=param]] [output redirection]
\end{verbatim}

\noindent \underbar{Copying Images among VISTA Buffers}:
\begin{verbatim}
   COPY imbuf source [BOX=b]
\end{verbatim}
NOTE:  The syntax of \comm{COPY} is:
\begin{verbatim}
   COPY <TO this buffer>  <FROM this buffer>
\end{verbatim}

\noindent \underbar{Deleting a VISTA Buffer}:
\begin{verbatim}
   DISPOSE [buf] [buf2] [buf3] [...] [ALL]
\end{verbatim}

\noindent \underbar{Create a Blank Image}:
\begin{verbatim}
   CREATE imbuf [BOX=b] [SR=sr] [SC=sc] [NR=nr] [NC=nc]
                [CONST=c]
\end{verbatim}

\noindent \underbar{Changing Name of an Image in a VISTA Buffer}:
\begin{verbatim}
   CHANGE buf 'new_name'
\end{verbatim}

\noindent \underbar{Change or Remove Individual FITS Header Cards}:
\begin{verbatim}
   FITS buf [FLOAT=name] float_value
 or
   FITS buf [INT=name] integer_value
 or
   FITS buf [CHAR=name] 'character string'
 or
   FITS buf [REMOVE=name]
\end{verbatim}

\noindent \underbar{Edit a FITS Header}:
\begin{verbatim}
   HEDIT buf
\end{verbatim}
You will have the FITS header placed in the default screen editor appropriate
to your installation.

%-------------------

\section{Display of Images and Spectra}

\subsection{Image Display and Interaction}

\underbar{Display an Image on a Color Monitor}:
\begin{verbatim}
   TV imbuf [span] [zero] [L=span] [Z=zero] [BOX=b] [BW]
            [CF=colormap] [NOLABEL] [LEFT] [RIGHT]
            [NOERASE] [CLIP]
\end{verbatim}

\noindent \underbar{Load or Define a Color Map}:
\begin{verbatim}
   COLOR [CF=filename] [BW] [INV]
\end{verbatim}

\noindent \underbar{Put an up Interactive Cursor on an Image Display}:
\begin{verbatim}
   ITV
\end{verbatim}
The operation of \comm{ITV} depends on your installation of VISTA.

\noindent \underbar{Overlay a \comm{BOX} on a Displayed Image}:
\begin{verbatim}
   TVBOX [BOX=b] [SIZE=s PIX=r,c]
\end{verbatim}

\subsection{Image Hardcopy}

\underbar{Grayscale Hardcopy on a Versatec}:
\begin{verbatim}
   VTEC imbuf [span] [zero] [L=span] [Z=zero] [BOX=b]
              [OLD] [INV] [CLIP]
\end{verbatim}

\noindent \underbar{Gray Halftone Hardcopy on a PostScript Device}:
\begin{verbatim}
   POSTIM imbuf [L=span] [Z=zero] [BOX=b] [CLIP] [INV]
                [TITLE] [PORT]
\end{verbatim}

\subsection{Line Graphics and Spectrum Plotting}

\underbar{Plot an Image Row, Column or a Spectrum}:
\begin{verbatim}
   PLOT imbuf [R=n] [C=n] [CS=c1,c2] [RS=r1,r2] [MIN=f]
              [MAX=f] [XS=f] [XE=f] [LOG] [SEMILOG] [GRID]
              [INFO] [HARD] [OLD] [SWAPXY] [PIXEL] [NOERASE]
              [NOPRINT] [HIST] [USER] [NOLABEL] [INT]
\end{verbatim}
Keywords (\comm{[R=n]} thru \comm{[RS=r1,r2]}) are omitted if \comm{imbuf}
contains a spectrum.

\noindent \underbar{Plot an Image Contour Map}:
\begin{verbatim}
   CONTOUR imbuf [BOX=b] [LEVELS=(L1,L2,...)] [LOW=l]
                 [RATIO=r] [DIFF=d] [FID=l] [SCALE=s] [USER]
                 [TITLE] [DASH] [TR=(X0,X1,X2,Y0,Y1,Y2)]
                 [EXACT] [HARD] [NOERASE] [NOPRINT] [FULL]
                 [NOLABEL]
\end{verbatim}

\noindent \underbar{Specify or Change the Default Graphics Devices}:
\begin{verbatim}
   TERM  (none)  [TERMINAL=vterm]  [HARDCOPY=vhard]
\end{verbatim}

%-------------------

\section{Marking Image Segments and Pixels}

\subsection{VISTA Image {\tt BOX}es}

\underbar{Define a \comm{BOX} or Image Subsection}:
\begin{verbatim}
   BOX box_num [NC=n] [NR=n] [CR=n] [CC=n] [SR=n] [SC=n]
\end{verbatim}

\noindent \underbar{Print a List of Current Boxes}:
\begin{verbatim}
   PRINT BOX
\end{verbatim}

\noindent \underbar{Overlay a \comm{BOX} on a Displayed Image}:
\begin{verbatim}
   TVBOX [BOX=b] [SIZE=s PIX=r,c]
\end{verbatim}

\subsection{Image Masks}

\underbar{Mask Image Regions}:
\begin{verbatim}
   MASK  [R=r1,r2] [C=c1,c2] [BOX=b] [PIX=r,c]
\end{verbatim}

\noindent \underbar{Un-Mask Image Regions}:
\begin{verbatim}
   UNMASK  [R=r1,r2] [C=c1,c2] [BOX=b] [PIX=r,c]
\end{verbatim}

\noindent \underbar{Create an Image to Display Current Image Mask}:
\begin{verbatim}
   MASKTOIM imbuf [BOX=b] [SR=sr] [SC=sc] [NR=nr] [NC=nc]
\end{verbatim}

\noindent \underbar{Store the Current Image Mask on Disk}:
\begin{verbatim}
   SAVE MASK=filename
\end{verbatim}

\noindent \underbar{Retrieve an Old Image Mask from Disk}:
\begin{verbatim}
   GET MASK=filename
\end{verbatim}

\noindent NOTE:  Masks may also be created using the \comm{CLIP} command
(see below).

%-------------------

\section{Image and Spectrum Arithmetic}

\subsection{Basic Arithmetic}

The basic image arithmetic commands are:
\begin{verbatim}
   ADD      image#1 [image#2] [CONST=c] [BOX=B] [DR=dr] [DC=dc]
   SUBTRACT image#1 [image#2] [CONST=c] [BOX=B] [DR=dr] [DC=dc]
   MULTIPLY image#1 [image#2] [CONST=c] [BOX=B] [DR=dr] [DC=dc]
   DIVIDE   image#1 [image#2] [CONST=c] [BOX=B] [DR=dr] [DC=dc]
                    [FLAT]
\end{verbatim}

\noindent The basic two-image operations are:
\begin{verbatim}
   ADD image#1 image#2   implies   image#1 = image#1 + image#2
   SUB image#1 image#2   implies   image#1 = image#1 - image#2
   MUL image#1 image#2   implies   image#1 = (image#1)(image#2)
   DIV image#1 image#2   implies   image#1 = (image#1)/(image#2)
\end{verbatim}

\noindent The basic one-image operations are:
\begin{verbatim}
   ADD image CONST=c      implies   image = image + c
   SUB image CONST=c      implies   image = image - c
   MUL image CONST=c      implies   image = c(image)
   DIV image CONST=c      implies   image = (image)/c
\end{verbatim}

\subsection{Advanced Image Arithmetic}

\begin{center}
   \begin{tabular}{lcl}
      Square Root of an Image:&&\verb+SQRT imbuf [SGN] [SIGN]+\\
      Log (Base 10) of an Image:&&\verb+LOG imbuf+\\
      Exponentiate ($e^x$) an Image:&&\verb+EXP imbuf+\\
      Tangent of an Image (degrees):&&\verb+TAN imbuf+\\
      ArcTan of an Image (returns degrees):&&\verb+ARCTAN imbuf [0to180]+\\
   \end{tabular}
\end{center}

%-------------------

\section{Image Statistics}

\underbar{Mean of the Pixel Values}:
\begin{verbatim}
   MN imbuf [NOBL] [BOX=b] [NOZERO] [MASK] [PIX=p] [SILENT]
               [W=w1,w2]
\end{verbatim}

\noindent \underbar{Full Image Pixel Statistics}:
\begin{verbatim}
   ABX imbuf box1 [box2] ... [ALL] [W=w1,w2] [SILENT]
         [MASK] [TOTAL=var] [MEAN=var] [HIGH=var] [LOW=var]
         [HIGH_ROW=var] [HIGH_COL=var] [LOW_ROW=var]
         [LOW_COL=var] [SIGMA=var] [output redirection]
\end{verbatim}

\noindent Either a list of box numbers or the keyword \comm{ALL} must be
given, otherwise \comm{ABX} will simply terminate without producing output.

\noindent \underbar{Compute ``Sky'' Level of an Image}:
\begin{verbatim}
   SKY imbuf [BOX=b] [SILENT] [CORNERS] [MAX=c]
\end{verbatim}
\comm{SKY} returns the Mode of the pixel intensity distribution.

\noindent \underbar{Plot a Histogram of Image Pixel Intensities}:
\begin{verbatim}
   HISTOGRAM imbuf [BOX=b] [NOLOG] [BIN=bin] [XMIN=xmin]
                   [XMAX=xmax] [YMIN=ymin] [YMAX=ymax] [HARD]
\end{verbatim}

%-------------------

\section{Image Processing}

\subsection{Image Size, Orientation, and Position}

\underbar{Window an Image to a Smaller Size}:
\begin{verbatim}
   WINDOW imbuf BOX=b
\end{verbatim}

\noindent \underbar{Merge Images or Spectra}:
\begin{verbatim}
   MERGE im1 im2 im3 im4 ... [NOMATCH]
\end{verbatim}

\noindent \underbar{Compress an Image or Spectrum}:
\begin{verbatim}
   BIN imbuf [BIN=b] [BINR=br] [BINC=bc] [SR=sr] [SC=sc] [NORM]
\end{verbatim}

\noindent \underbar{Rotate an Image}:
\begin{verbatim}
   ROTATE imbuf [LEFT] [RIGHT] [UD] [PA=degrees] [BOX=b]
\end{verbatim}

\noindent \underbar{Reverse the Order of Rows or Columns of an Image}:
\begin{verbatim}
   FLIP imbuf [ROWS] [COLS]
\end{verbatim}

\noindent \underbar{Shift an Image}:
\begin{verbatim}
   SHIFT imbuf [DC=f] [DR=f] [SINC] [NORM] [RMODEL=i] [CMODEL=i]
               [MEAN]
\end{verbatim}

\subsection{Changing Image Pixel Values}

\underbar{Correct an Image for Baseline Subtraction Noise}:
\begin{verbatim}
   BL imbuf [JUMP]
\end{verbatim}

\noindent \underbar{Changing Individual Pixel Values}:
\begin{verbatim}
   SET X=SETVAL[imbuf,row,col,newvalue]
\end{verbatim}

\noindent \underbar{Replace Pixels Outside a Given Intensity Range}:
\begin{verbatim}
   CLIP imbuf [MAX=f] [MIN=f] [VMAX=f] [VMIN=f] [BOX=b]
              [MASK] [MASKONLY]
\end{verbatim}

\noindent \underbar{Gaussian or Boxcar Smooth an Image or Spectrum}:
\begin{verbatim}
   SMOOTH imbuf [FW=f] [FWC=f] [FWR=f] [BOXCAR]
\end{verbatim}

\noindent \underbar{Image Median Filter and Pixel Zapper}:
\begin{verbatim}
   ZAP imbuf [SIG=f] [SIZE=s] [SIZE=r,c] [BOX=b] [TTY]
\end{verbatim}

\noindent \underbar{Interactive Pixel Zapper}:
\begin{verbatim}
   TVZAP [SIG=f] [SEARCH=s] [TTY]
\end{verbatim}

\noindent \underbar{Compute Median of Several Images}:
\begin{verbatim}
   MEDIAN imbuf im1 im2 im3 [im4 im5 ...] TTY
\end{verbatim}

\noindent \underbar{Fit a Surface to an Image}:
\begin{verbatim}
   SURFACE imbuf [BOX=b] [PLANE] [SUB] [LOAD] [DIV] [MASK]
           [NOZERO] [PIX=N] [output redirection]
\end{verbatim}

\noindent \underbar{Replace an Image by a Spline (Non-Interactive)}:
\begin{verbatim}
   SPLINE imbuf [R=r1,r2,...] [C=c1,c2...] [W=w1,w2,...]
                [AVG=a] [SUB] [DIV]
\end{verbatim}

\noindent \underbar{Cross-Correlate two Images or Spectra}:
\begin{verbatim}
   CROSS imbuf image#1 image#2 [BOX=b] [RAD=r] [RADR=r] [RADC=c]
\end{verbatim}

\noindent \underbar{Interpolate across rows, columns, or masked pixels}:
\begin{verbatim}
   INTERP imbuf [BOX=b1,b2,...] [COL] [ROW] [ORD=n] [AVE=a]
                [MASK]
\end{verbatim}

%-------------------

\section{Spectroscopic Reduction Routines}

Throughout this section, the keyword \comm{spbuf} will be used to refer to a
VISTA buffer which contains (or will contain) a spectrum, and the keyword
\comm{imbuf} will be used to refer to a VISTA buffer which contains (or will
contain) a 2-D image.

\subsection{Spectrum Extraction from 2--D Images}

\underbar{Simple Spectrum Extraction}:
\begin{verbatim}
   MASH spbuf imbuf SP=i1,i2 [BK=b1,b2] [COL] [COL=c1,c2]
                    [ROW=r1,r2] [SKY=s] [NORM] [REFLAT] [SUB]
                    [MASK]
\end{verbatim}
The basic syntax of \comm{MASH} is:
\begin{verbatim}
   MASH <into this spectrum> <from this 2-D image>
\end{verbatim}

\noindent \underbar{Optimal Weighting Scheme Extraction}:
\begin{verbatim}
   EXTRACT spbuf imbuf SP=s1,s2 BK=b1,b2 BK=b3,b4 [SKY=s]
                [VAR=v] [SUB] [SORDER=sord] [PORDER=pord]
                [RONOISE=r] [EPERDN=eperdn]
\end{verbatim}

\noindent \underbar{Extraction Using a Spectrum Centroid Map}:
\begin{verbatim}
   SPECTROID spbuf imbuf [SP=s1,s2] [BK=b1,b2] [SPW=ds] [BKW=db]
                [MODEL=m] [DLOC=d] [LOC=r] [NOSHIFT] [FIT=p1,p2]
                [TRACE] [SELF] [LOAD] [NOMASH] [TAGALONG] [TAG]
                [MOMENTS]
\end{verbatim}

\subsection{Wavelength Calibration}

\underbar{Identify Lines in a Wavelength Calibration Spectrum}:
\begin{verbatim}
   LINEID spbuf [FILE=lineid_file] [ADD] [TTY] [INT]
                [CEN=wavec] [DISP=disp] [output redirection]
\end{verbatim}

\noindent \underbar{Compute Wavelength Calibration from \comm{LINEID} Output}:
\begin{verbatim}
   WSCALE spbuf [ORD=n] [TTY] [INT] [output redirection]
\end{verbatim}

\noindent \underbar{Copy a Wavelength Scale between Spectra}:
\begin{verbatim}
   COPW spbuf source [source2]
\end{verbatim}

\noindent \underbar{Put a Spectrum on a New Wavelength Scale}:
\begin{verbatim}
   ALIGN spbuf DSP=disp W=(lam,pix) [LOG] [LGI] [MS=n] [FLIP]
               [V=f] [Z=z] [DP=dp] [SILENT] [RED=z] [DERED=z]
\end{verbatim}

\noindent \underbar{Correct Wavelength Scale using Night Sky Lines}:
\begin{verbatim}
   SKYLINE sp1 [sp2] [sp3] [sp4] ... [sp15] [INT]
\end{verbatim}

\subsection{Flux Calibration}

\underbar{Compute a Flux Curve using Standard Star Spectra}:
\begin{verbatim}
   FLUXSTAR spbuf [fluxfile] [AVE] [POLY=n] [WT=w] [SYSA] [SYSC]
\end{verbatim}

\noindent \underbar{Flux Calibrate a Spectrum}:
\begin{verbatim}
   FLUX spbuf
\end{verbatim}

\noindent \underbar{Correct for Atmospheric Extinction}:
\begin{verbatim}
   EXTINCT spbuf [CTIO]
\end{verbatim}

\subsection{Spectrum Fitting and Stretching}

\underbar{Fit a Polynomial to a Spectrum}:
\begin{verbatim}
   POLY spbuf ORD=n [SUB] [DIV] [LOAD]
\end{verbatim}

\noindent \underbar{Fit a Spline to a Spectrum}:
\begin{verbatim}
Non-Interactive:
   SPLINE spbuf [R=r1,r2,...] [C=c1,c2...] [W=w1,w2,...]
                [AVG=a] [SUB] [DIV]

Interactive:
   ISPLINE spbuf [XY] [AVG=a] [SUB] [DIV] [HIST]
\end{verbatim}

\noindent \underbar{Stretch a Spectrum into a 2-D Image}:
\begin{verbatim}
   STRETCH imbuf spbuf [VERT] [HORIZ] [SIZE=s] [START=st]
\end{verbatim}

\subsection{Echelle Reduction Routines}

\underbar{Wavelength Calibrate an Echelle Image}:
\begin{verbatim}
   EWAVE imbuf [XOFF=x0] [PORD=nw] [MORD=nm] [REJ=rej]
               [TTY] [TRACE]
\end{verbatim}

\noindent \underbar{Copy an Order of an Echelle Image into a Spectrum}:
\begin{verbatim}
   EXTSPEC spbuf imbuf ORD=nord
\end{verbatim}

%-------------------

\section{Stellar Photometry Routines}

\subsection{Locate Stars on an Image}

\underbar{Locate Stars Interactively}:
\begin{verbatim}
   MARKSTAR [NEW] [RADIUS=r] [NOBOX] [STAR=s1,s2,...] [AUTO]
            [DR=dr] [DC=dc] [RSHIFT=rs] [CSHIFT=cs]
\end{verbatim}

\noindent \underbar{Locate Stars Automatically}:
\begin{verbatim}
   AUTOMARK imbuf [RADIUS=rad] [RANGE=low,high] [REJECT=rej]
                  [BOX=b] [NEW]
\end{verbatim}

\noindent \underbar{Compute RA and DEC for Stars}:
\begin{verbatim}
   COORDS [output redirection]
\end{verbatim}

\subsection{Measure Stellar Brightnesses}

\underbar{Measure Brightnesses by Aperture Photometry}:
\begin{verbatim}
   APERSTAR imbuf STAR=rs SKY=r1,r2 [SKY=NONE]
                  [GAIN=g] [RONOISE=r]
\end{verbatim}

\noindent \underbar{Generate a Point Spread Function Image}:
\begin{verbatim}
   PSF psfbuf [SIZE=n] [SKY=n] [AVERAGE] [RADIUS=r] [BI]
\end{verbatim}

\noindent \underbar{Measure Brightnesses using PSF Fitting}:
\begin{verbatim}
 Interactive:

   FITSTAR psfbuf [FILE] [FLAT] [PLANE] [INTER] [LOCAL] [NOSUB]
                  [RADIUS=r] [SEGMENT]

 Batch:

   BFITSTAR imbuf psfbuf [FILE] [FLAT] [PLANE] [INTER] [LOCAL]
                         [NOSUB] [RADIUS=r] [SEGMENT]
\end{verbatim}

\subsection{Photometry Files}

\underbar{Print Current Photometry File}:
\begin{verbatim}
   PRINT PHOT [BRIEF] [output redirection]
\end{verbatim}

\noindent \underbar{Modify the Entries in the Current Photometry File}:
\begin{verbatim}
   MODPHOT
\end{verbatim}

\noindent \underbar{Store the Current Photometry File on Disk}:
\begin{verbatim}
   SAVE PHOT=filename
\end{verbatim}

\noindent \underbar{Retrieve an Old Photometry File from Disk}:
\begin{verbatim}
   GET PHOT=filename
\end{verbatim}

%-------------------

\section{Extended Object and Surface Photometry Routines}

\underbar{Compute Centroid of an Extended Object}
\begin{verbatim}
   AXES imbuf [BOX=b] [SKY=s] [W=w1,w2] [output redirection]
\end{verbatim}

\subsection{Aperture Photometry}

\underbar{Aperture Photometry of an Extended Object}:
\begin{verbatim}
   APER imbuf [RAD=r1,r2,...,r10] [MAG=M1,M2,...MN] [C=r,c]
              [STEP=size,n] [SCALE=s] [OLD] [INT] [REF]
              [output redirection]
\end{verbatim}

\noindent \underbar{Print Current Aperture Photometry File}:
\begin{verbatim}
   PRINT APER [output redirection]
\end{verbatim}

\noindent \underbar{Store the Current Aperture Photometry File on Disk}:
\begin{verbatim}
   SAVE APER=filename
\end{verbatim}

\noindent \underbar{Retrieve an Old Aperture Photometry File from Disk}:
\begin{verbatim}
   GET APER=filename
\end{verbatim}

\subsection{Surface Photometry by Elliptical Contour Fitting}

\underbar{Surface Brightness Profile by Contour Fitting}:
\begin{verbatim}
   PROFILE spbuf imbuf [N=n] [ITER=n1,n2] [SCALE=s] [CENTER]
                       [PA=f] [INT] [FOUR]
\end{verbatim}

\noindent \underbar{Print Current Surface Photometry File}:
\begin{verbatim}
   PRINT PROFILE [output redirection]
\end{verbatim}

\noindent \underbar{Store the Current Surface Photometry File on Disk}:
\begin{verbatim}
   SAVE PROFILE=filename
\end{verbatim}

\noindent \underbar{Retrieve an Old Surface Photometry File from Disk}:
\begin{verbatim}
   GET PROFILE=filename
\end{verbatim}

\subsection{Radial Brightness Profile Routines}

\underbar{Compute Radial Profile Along a Single Position Angle}:
\begin{verbatim}
   PRAD spbuf imbuf [PA=n] [CEN=n1,n2] [BOTH]
\end{verbatim}

\noindent \underbar{Compute Radial Profile by Azimuthal Averaging}:
\begin{verbatim}
   ANNULUS spbuf imbuf N=n [STEP=dr] [PA=pa] [INC=i] [CEN=r0,c0]
                           [SCALE=s] [FAST] [RAD=r] [PROF]
\end{verbatim}

\subsection{Model 2-D Brightness Profile Generation}

\underbar{Reconstruct a Model Profile from a \comm{SURFACE} Fit}:
\begin{verbatim}
   RECON imbuf [CR=r0] [CC=c0]
\end{verbatim}

\noindent \underbar{Model Image from User's Radial Brightness Profile}:
\begin{verbatim}
   TEMPLATE  imbuf spbuf [PA=n] [PAM=n] [E=n] [FIT=n1,n2] [SUB]
                         [GAUSS] [EXP] [HUB] [DEV]
\end{verbatim}

%-------------------

\section{VISTA Command Procedures}

Note that ANY valid VISTA command (or its unambiguous abbreviation) may be
used in a procedure.

\subsection{Defining Procedures}

\underbar{Edit the Procedure Buffer}:
\begin{verbatim}
   PEDIT
\end{verbatim}

\noindent \underbar{Define a Procedure}:
\begin{verbatim}
   DEF [line_number]
\end{verbatim}

\noindent \underbar{Stop a Procedure Cold}:
\begin{verbatim}
   STOP ['A message']
\end{verbatim}

\noindent \underbar{End a Procedure}:
\begin{verbatim}
   END
\end{verbatim}

\noindent \underbar{Insert Lines into the Procedure Buffer}:
\begin{verbatim}
   IDEF [line_number]
\end{verbatim}

\noindent \underbar{End an \comm{IDEF} and Keep Trailing Lines}:
\begin{verbatim}
   SAME
\end{verbatim}

\noindent \underbar{Remove Lines from the Procedure Buffer}:
\begin{verbatim}
   RDEF [line_number] [LINES=l1,l2]
\end{verbatim}

\noindent \underbar{Print Contents of the Procedure Buffer}:
\begin{verbatim}
   SHOW [output redirection]
\end{verbatim}

\noindent \underbar{Write the Procedure Buffer to Disk}:
\begin{verbatim}
   WP filename
\end{verbatim}

\noindent \underbar{Read an Old Procedure from Disk}:
\begin{verbatim}
   RP filename
\end{verbatim}

\subsection{Executing VISTA Procedures}

\underbar{Start Procedure Execution}:
\begin{verbatim}
   GO [parameter1] [parameter2] ...
\end{verbatim}

\noindent \underbar{Call In and Start a Procedure as a Subroutine}:
\begin{verbatim}
   CALL procedure_filename [parameter1] [parameter2] ...
\end{verbatim}

\noindent \underbar{Return from a {\tt CALL}ed Procedure}:
\begin{verbatim}
   RETURN
\end{verbatim}

\noindent \underbar{Evaluate Parameters Passed to a Procedure}:
\begin{verbatim}
   PARAMETER [var1] [var2] [STRING=string1] ...
\end{verbatim}

\noindent \underbar{Trace Execution of a Procedure}:
\begin{verbatim}
   VERIFY Y   -or-   VERIFY N
\end{verbatim}

\noindent \underbar{Comment Lines in a Procedure}:
\begin{verbatim}
   ! followed by any text, nothing following a ! is executed
\end{verbatim}

\subsection{Variable Input/Output in Procedures}

\underbar{Formatted Output}
\begin{verbatim}
   PRINTF 'Format string' [expressions] [output redirection]
\end{verbatim}

\noindent \underbar{Prompted Input}:
\begin{verbatim}
   ASK  ['An optional prompt in quotes']  var_name
\end{verbatim}

\subsection{Simple Flow Control in Procedures}

\underbar{Pause a Procedure}:
\begin{verbatim}
   PAUSE 'pause message'
\end{verbatim}

\noindent \underbar{Resume a {\tt PAUSE}'d Procedure}:
\begin{verbatim}
   CONTINUE   -or-   C
\end{verbatim}

\noindent \underbar{Jump to a Labelled Line in a Procedure}:
\begin{verbatim}
   GOTO label_name
\end{verbatim}

\noindent \underbar{Label a Line as a \comm{GOTO} Jumping Point}:
\begin{verbatim}
   label_name:
\end{verbatim}

\noindent \underbar{DO Loops}:
\begin{verbatim}
   DO var=from,to,[step]
         <execute these procedure lines>
   END_DO
\end{verbatim}

\noindent NOTE:  No spaces are allowed in the argument list of the DO loop.

\noindent \underbar{Execute on Error}:
\begin{verbatim}
   ERROR  VISTA_command
\end{verbatim}

\noindent \underbar{Execute on End-of-File}:
\begin{verbatim}
   EOF  VISTA_command
\end{verbatim}

\subsection{Logical `IF' Control Statements}

VISTA supports a set of logical operations among VISTA variables to allow
logical comparison (TRUE/FALSE) between two arithmetic expressions.  The
following table lists the supported logical operations:

\begin{center}
   \begin{tabular}{|c|l|ccc|} \hline
      Symbol&\multicolumn{1}{|c|}{Function}&
      \multicolumn{3}{c|}{Examples}\\ \hline
      \verb+>+&greater than&\verb+A>B+&&\verb+A>2.5+\\ \hline
      \verb+<+&less than&\verb+A<B+&&\verb+A<100+\\ \hline
      \verb+==+&equal to&\verb+A==B+&&\verb+A==10+\\ \hline
      \verb+~=+&not equal to&\verb+A~=B+&&\verb+A~=10+\\ \hline
      \verb+<=+&less than or =&\verb+A<=B+&&\verb+A<=5+\\ \hline
      \verb+>=+&greater than or =&\verb+A>=B+&&\verb+A>=3+\\ \hline
   \end{tabular}
\end{center}

\noindent NOTE: as with arithmetic expressions, no spaces must appear anywhere
in a logical expression.

A set of simple \comm{IF} block structures are illustrated below.  In what
follows, the syntax \verb+< >+ is used to denote complicated expressions which
have no general form.  For example, ``\verb+<condition>+'' denotes a logical
expression to be tested.  For example, \verb+<condition>+ might be replaced by
expressions like:
\begin{verbatim}
   A==B
   X~=0
   A*(12.0)>=(B-EXPOS)/4.0
\end{verbatim}
and so forth.  The angle braces (\verb+< >+) are being used to isolate the
conditional statement for easy reading.  They are NEVER typed.

\noindent \underbar{Simple \comm{IF} Block}:
\begin{verbatim}
   IF <condition>
         Execute these lines if <condition> is true.
   END_IF
\end{verbatim}

\noindent \underbar{Simple Two-Level \comm{IF} Block}:
\begin{verbatim}
   IF <condition>
         Execute these lines if <condition> is true.
   ELSE
         Execute these lines if <condition> is false.
   END_IF
\end{verbatim}


\noindent \underbar{Multi-Level \comm{IF} Block}:
\begin{verbatim}
   IF <condition 1>
         lines to be executed if <condition 1> is true.
   ELSE_IF <condition 2>
         lines to be executed when <condition 1> is
         false and <condition 2> is true.
     .
     .
   ELSE_IF <condition N>
         lines to be executed when all conditions
         are false except <condition N>.
   ELSE
         lines to be executed if and only if
         all other conditions are false.
   END_IF
\end{verbatim}

%-------------------

\section{External ASCII Data Files}

\subsection{Opening and Closing Text Files}

Up to 5 ASCII text file may be opened at one time.  VISTA uses user-supplied
``logical names'' to distinguish between them.

\noindent \underbar{Opening an ASCII Text File}:
\begin{verbatim}
   OPEN logical_name file_name
\end{verbatim}

\noindent \underbar{Closing an {\tt OPEN}ed ASCII Text File}:
\begin{verbatim}
   CLOSE logical_name
\end{verbatim}

\subsection{External File Operations}

\underbar{Read the next line of an ASCII Text File}:
\begin{verbatim}
   READ logical_name
\end{verbatim}

\noindent \underbar{Data Syntax for Lines \comm{READ} from a File}:

Each column of an ASCII text file is read into a VISTA variable
named:
\begin{verbatim}
   @LOGNAME.n
\end{verbatim}
Where: \comm{LOGNAME} is the user-assigned logical name (see \comm{OPEN})
and \comm{n} is the column number.

\noindent \underbar{Skip Selected Lines in an ASCII Text File}:
\begin{verbatim}
   SKIP logical_name line [line1,line2]
\end{verbatim}

\noindent \underbar{Reset File to \comm{READ} from First Line}:
\begin{verbatim}
   REWIND logical_name
\end{verbatim}

\noindent \underbar{Get File ``Statistics'' }:
\begin{verbatim}
   STAT variable=MAX[arithmetic expression]
 or
   STAT variable=MIN[arithmetic expression]
 or
   STAT variable=FIRST[arithmetic expression]
 or
   STAT variable=LAST[arithmetic expression]
 or
   STAT variable=COUNT[logical_name]
\end{verbatim}
where ``{\tt arithmetic expression}'' is any valid VISTA arithmetic expression
which must include one (or more) of the ``{\tt @LOGNAME.n}'' variables.

%-------------------

\section{Miscellaneous Commands}

\underbar{Clear the Terminal Screen}:
\begin{verbatim}
   CLEAR [TEXT]
\end{verbatim}

\noindent \underbar{Read the VISTA Welcome Message and Old News}:
\begin{verbatim}
   NEWS
\end{verbatim}

\noindent \underbar{Clock the Execution Time of a Command}:
\begin{verbatim}
   TIME VISTA_command
\end{verbatim}

\noindent \underbar{Turn Prompt Bell On/Off or Ring the Bell}:
\begin{verbatim}
   BELL Y   -or-   BELL N   -or-   BELL R
\end{verbatim}

\noindent \underbar{Execute an Operating System Command from VISTA}:
\begin{verbatim}
   $ Any Valid Operating System Command
\end{verbatim}

\noindent \underbar{Temporarily Return to Operating System Level}:
\begin{verbatim}
   $   with no arguments
\end{verbatim}
To return to VISTA type \comm{LOGOUT}.

%\end{document}
%\end

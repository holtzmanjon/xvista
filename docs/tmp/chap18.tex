\chapter{Miscellaneous Commands}

%
% reformatted and fixed up some, and checked against function in the
% sources.
% rwp/osu 98Jul26
%

\section{\$: Execute a Shell Command from VISTA}
\begin{rawhtml}
<!-- linkto spawn.html -->
<!-- linkto shell.html -->
\end{rawhtml}
\begin{itemize}
  \item[Form: \$ Any valid shell command\hfill]{}
\end{itemize}

You may execute most operating system shell commands directly from within
VISTA by prefixing them with a '\$' sign.  These commands can also be
included as part of a procedure.

If the command line is '\$' only, you go into command mode and stay there
until you type 'exit'.  It is dangerous to run large programs while you are
in this mode.

\section{TIME: Time a Command}
\begin{rawhtml}
<!-- linkto time.html -->
\end{rawhtml}
\begin{itemize}
  \item[Form: TIME cmd\hfill]{}
\end{itemize}
where 'cmd' is any VISTA command.

When you type TIME before any VISTA command, VISTA will start a timer and then
perform the requested command.  After the command is completed, the timer will
be stopped and the elapsed and CPU time of the command will be displayed.

\section{CLOCK: Turn on the Internal Clock for Timing Multiple Commands}
\begin{rawhtml}
<!-- linkto clock.html -->
\end{rawhtml}
\begin{itemize}
  \item[Form: CLOCK\hfill]{}
\end{itemize}
 
When you enter the CLOCK command, an internal clock is started. When the
next CLOCK command is issued, you will get a summary of the elapsed clock
and CPU times between the two CLOCK commands.

\section{BELL: Turn the Prompt Bell ON/OFF or Ring the Bell }
\begin{rawhtml}
<!-- linkto bell.html -->
\end{rawhtml}
\begin{itemize}
  \item[Forms: BELL Y, BELL N, or BELL R\hfill]{}
\end{itemize}

The VISTA prompt can be accompanied by a bell, if desired.
\begin{itemize}
  \item[N\hfill]{prevents the bell from ringing}
  \item[Y\hfill]{restores the bell with the prompt}
  \item[R\hfill]{rings the bell}
\end{itemize}
When you begin running VISTA, the bell does not ring with the prompt. The
'BELL R' command is helpful in procedures for signaling the completion of
some process, or for getting the your attention before occurrences of the
ASK or STRING commands, which accept input.


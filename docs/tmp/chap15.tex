\chapter{Surface Photometry}
\begin{rawhtml}
  <!-- linkto surfphot.html -->
\end{rawhtml}

% 
% minor editing & reformatting, fixed up some routines a fair amount
% to make consistent with function in source code.  keyword situation
% on some commands is not always easy to sort out.  I expect future
% revisions.
% Made the index of surface photometry commands square with
% the contents, and some reordering to put related functions together
% logically (some where scattered somewhat haphazardly through the file)
% moved TEMPLATE to chap 17 (Image Simulation) as that function is closer
% to that than here (it stuck out).
% rwp/osu 98Jul28

The VISTA commands for surface photometry of extended objects are:
\begin{itemize}
  \item[CUT\hfill]{extract a brightness profile cut along any position angle}
  \item[PROFILE\hfill]{compute surface brightness profile by elliptical
       isophote fitting}
  \item[ANNULUS\hfill]{compute a radial profile by azimuthal averaging}
  \item[RECON\hfill]{reconstruct an image from a surface photometry profile}
  \item[APER\hfill]{aperture photometry of an extended object}
  \item[ROUND\hfill]{deproject spirals or elliptical images}
  \item[POLAR\hfill]{regrid an image into polar coordinates}
  \item[EGAL\hfill]{reconstruct a galaxy from its profile}
  \item[SNUC\hfill]{surface photometry of multiple galaxy systems}
  \item[RENUC\hfill]{reconstruct galaxies from multiple galaxy photometry}
  \item[SECTOR\hfill]{surface photometry of star clusters or complex
       objects} 
  \item[AEDIT\hfill]{edit values in selected image regions}
  \item[CLPROF\hfill]{clear the contents of the profile common block}
  \item[CPROF\hfill]{correct the results of a profile calculation}
  \item[EMAG\hfill]{aperture photometry with elliptical apertures}
  \item[EMARK\hfill]{interactively specify ellipse parameters for EMAG} 
  \item[RMARK\hfill]{interactively set maximum radius for EMAG}
  \item[ELLMAG\hfill]{compute elliptical magnitudes for PROFILE ellipses}
  \item[TVPROF\hfill]{display PROFILE results on the image display}
  \item[OPREP\hfill]{open a batch surface photometry preparation file}
  \item[CLPREP\hfill]{close a batch surface photometry preparation file}
  \item[RPREP\hfill]{read a record from a photometry prep-file}
  \item[WPREP\hfill]{write a record into photometry prep-file}
\end{itemize}

%---------------------------------------------------------------------------

\section{CUT: Extract a Brightness Profile Cut along any Position Angle}
\begin{rawhtml}
<!-- linkto cut.html -->
\end{rawhtml}

\index{Surface Photometry!Extract a cut along any PA}
\begin{itemize}
  \item[Form: CUT dest source {[PA=n]} {[C=(r,c)]} {[W=w]} {[NORM]} {[L=h]}
       {[OFF=f]} {[TV]} {[PROF]} {[DPA=dpa]} \hfill]{}
  \item[dest]{(integer or \$ construct) is an image which
       will hold the calculated cut profile,}
  \item[source]{(integer or \$ construct) is the image (object)
       from which the cut is to be measured,}
  \item[PA=n]{sets the position angle of radial slice through the 
       extended object,}
  \item[C=(r,c)]{resets the center of the slice from the AXES values,}
  \item[L=l]{Sets the length of the cut in pixels,}
  \item[W=w]{Sets the cut width in pixels,}
  \item[NORM]{Normalizes by the cut width,}
  \item[OFF=f]{Offsets the cut from the center in the width direction.}
  \item[TV]{Shows the area of the cut on the TV.}
  \item[PROF]{Plots using information from PROFILE common block (center, 
        position angle, sky}
  \item[DPA=dpa]{Using PROF option, plots at an angle dpa degrees from
        PROFILE pa}
\end{itemize}
 
CUT will extract the surface-brightness profile of an object along an
arbitrary position angle.  The center of the object can be loaded with the
AXES command or by the C keyword.  A cut can also be offset from any center
using the OFF keyword, as might be done to take a series of cuts at
different locations along the major axis of a galaxy.  The position angle
is defined from the top of the image.  Adding 180 to any PA will generate a
cut running in the opposite direction.  This can be used to compare or add
profiles reflected about the center of a galaxy, for example.  The default
width is one pixel, and the height or length extends to the image borders.
Both parameters are rounded to the nearest whole pixel. The PLOT keyword
shows the bounds of the cut on the image display.

The cut is loaded into the 'dest' image buffer, and can be treated,
analyzed, processed, saved, etc., as an image with a row dimension of one.
The central column of the cut is always set to occur at column 0.  Sinc
interpolation is used to generated cuts of width 1; bilinear interpolation
is used otherwise.  Pixels are always summed across the width of the cut.
Divide the resulting cut vector by the width in pixels (WIDTH= keyword) to
normalize it after the fact.

See Also: AXES, TEMPLATE


\section{PROFILE: Compute Surface Brightness Profile by Elliptical
         Isophote Fitting}
\begin{rawhtml}
<!-- linkto profile.html -->
\end{rawhtml}
\index{Profile!Measuring profile}
\index{Surface Profile!by fitting elliptical contours}
\begin{itemize}
  \item[Form: PROFILE dest source {[N=n]} {[ITER=n1,n2]} {[SCALE=f]} 
       {[CENTER]} {[PA=f]} \hfill]{}
  \item{{[INT]} {[FOUR]} {[SKY]} {[PSEUDO]} {[GPROF]} {[TTY]} {[NFINE=]} 
       {[RMAX]}}
  \item[dest]{is a spectrum which will hold the calculated profile.}
  \item[source]{is the image which contains the object which is
       having its profile measured.}
  \item[N=n or RMAX]{sets the number of steps in the iteration.}
  \item[ITER=n1,n2]{sets number of iterations for (n1) fast bilinear
       interpolation and (n2), slower sinc interpolation.}
  \item[CENTER]{solves for the contour centers.}
  \item[PA=f]{position angle for the top of the image.}
  \item[INT]{interactively iterate contour solution.}
  \item[FOUR]{include 4-theta terms in solution.}
  \item[SKY]{does sky subtraction using value in VISTA variable SKY}
  \item[NFINE=]{sets the number of contours (radial) for high accuracy 
                          sinc interpolation, defaults to 15}
  \item[PSEUDO]{uses pseudo median rather than median for derivatives}
  \item[GPROF]{sets all parameters to recover Berkeley GPROF routine}
  \item[TTY]{displays output from each iteration }
\end{itemize}

PROFILE computes the radial surface-brightness profile of an object by
fitting it with elliptical contours. The center of the object must first be
calculated with the AXES command.  PROFILE uses this center as the starting
point for its calculations.  The profile is found by sampling the image
with a set of circles with radius 1, 2, 3, ... pixels.  The average value
of the pixels along a circle is the mean surface brightness of that
contour.  Low order sine and cosine transforms are taken along the contour
to derive its center, position angle, and ellipticity.  After these are
found for the entire image in the radius specified, the contours are
adjusted to more exactly fit the isophotes.  The first iteration usually
turns the original circles into ellipses with varying position angles and
eccentricities as a function of major-axis length.

A slightly different version of this command written by G. Djorgovski uses
a slightly modified interpolation scheme, slightly different median
filtering method for computing derivatives, and the inclusion of sky
subtraction in the photometry.  These features may be enabled by using the
GPROF keyword, or may be selectively enabled by using the SKY and PSEUDO
keywords.

High-accuracy sinc interpolation is used to find the values of the pixels
along the inner 15 contours.  This number (15) can be modified with the
NFINE= keyword. Outside of this, either a lower accuracy (but faster)
interpolation can be used, or an even faster bilinear interpolation scheme.
The kind of iteration used is set by the ITER keyword; the first number
specifies the number of iterations using bilinear interpolation, and the
second the number of iterations using a lower accuracy sinc interpolation.
The keywords N=n or RMAX sets the number of contours to be computed, to be
n with the N=n keyword, or to the value in the variable RMAX (set by
command RMARK) if the RMAX keyword is specified.

The result of the profile calculation is stored in a common block for later
use.  Print the contents of this block with PRINT, or save it in a disk file
with SAVE.  The results are also written into the specified spectrum.  Use
PLOT to show that spectrum.

The option FOUR turns on calculations of the profile which have the terms
cos(4*theta) and sin(4*theta), where theta is the position angle in the
calculation.  Normally these are turned off for speed.

Position angles are calculated assuming that the position angle of the
'top' of the image (as seen on the TV) is 0.  This can be changed with the
PA keyword.  Use PA < 0 to indicate that the image has been reversed
right-left (the 'normal' arrangement for an image is north at top, east at
left).

The INT keyword allows interactive control of the iterations.  The user may
change the step size from 1.0 to some fraction, and also solve for contour
centers before the next iteration. Full output from each iteration can be
shown on the terminal if the keyword TTY is specified.

The output of this program can be redirected.

\noindent{Example:}
\begin{itemize}
  \item[PROFILE 11 1 SCALE=0.267 N=20 ITER=2,4 >PROFILE.OUT\hfill]{Create
       spectrum in buffer 11 with the radial profile of the object in image
       1.  The centroid of the object was already calculated with AXES.
       The profile is computed out to 20 pixels, with a scale of 0.267
       arcsec/pixel.  Four iterations will be done, the first two with
       quick bilinear interpolation.  The result is stored in the file
       PROFILE.OUT.}
\end{itemize}

See Also: ANNULUS, SECTOR, RECON, EGAL, TVPROF, CLPROF, CPROF, 
          SAVE, GET, SNUC


\section{ANNULUS: Compute a Radial Profile by Azimuthal Averaging}
\begin{rawhtml}
<!-- linkto annulus.html -->
\end{rawhtml}
\index{Annulus!Measuring azimuthal averages }
\index{Surface Profile!by azimuthal averaging along defined annuli}
\begin{itemize}
  \item[Form: ANNULUS dest source N=n {[STEP=dr]} {[PA=pa]} {[INC=i]} 
       {[CEN=r0,c0]} {[SCALE=s]} {[FAST]} {[RAD=r]} {[PROF]}\hfill]{}
  \item[dest]{Buffer to hold the average radial profile (spectrum)}
  \item[source]{Buffer with the source image.}
  \item[N=n]{Number of concentric annuli to average over.}
  \item[STEP=dr]{(optional) Spacing between annuli in either pixels or
       units of the "SCALE=" keyword.}
  \item[PA=pa]{(optional) Position Angle in degrees of the Major Axis 
       of elliptical annuli.}
  \item[INC=i]{(optional) Inclination Angle in degrees of elliptical 
       Annuli relative to the line-of-sight.}
  \item[CEN=r0,c0]{(optional) Center of annuli (in rows,columns on image).}
  \item[SCALE=s]{(optional) Image scale, typically expressed in units
       of arcseconds/pixel.}
  \item[FAST]{(optional) Use faster bi-linear interpolation in outer annuli.}
  \item[RAD=r]{(optional) Allows the user to compute an average along
       a single annulus.  No profile is calculated}
  \item[PROF]{(optional) Computes averages along isophotes defined by
       the PROFILE command.}
\end{itemize}

Other relevant VISTA commands to be used prior to running ANNULUS:
\begin{itemize}
  \item{AXES: REQUIRED unless the "CEN=" keyword is used.  This provides
        ANNULUS with the object center for the averaging annuli. }

  \item{PROFILE: Only required if you are using the PROF keyword. (see PROF
        below).}

  \item{GET: May be used in place of PROFILE before using the PROF keyword
        to load in a previous PROFILE fit saved in an external file with
        the SAVE command. }
\end{itemize}
 
ANNULUS computes the azimuthally-averaged radial surface-brightness profile
of a 2-D image, storing the profile as a spectrum (1-D image).  The value
at each radius is defined as the azimuthal average of the pixel values
along a suitably defined annulus (ring) of that radius centered on the
object.  High accuracy Sinc interpolation or faster Bilinear interpolation
may be used.

By default, the annuli are concentric circles, but it is possible to define
elliptical annuli to be used as the averaging paths by specifying the major
axis position angle and inclination angle of the ellipses with keywords.

The centers of the annuli are computed before using ANNULUS by running the
AXES command, although this step may be bypassed using a keyword.

It is also possible to forego creating a radial profile, and simply use
ANNULUS to compute the average along a single annulus of a given radius.

Unlike the PROFILE command, which computes the surface brightness profile
of an object by fitting ellipses to the image isophotes, ANNULUS takes
averages along pre-defined annuli without regard to the actual isophotes of
the image.  ANNULUS is more applicable than PROFILE to those cases where
strong non-axisymmetric structure in the object makes fitting ellipses
impractical.  For example, ANNULUS might be used to compute the average
radial surface brightness profile of a Spiral Galaxy, in which the
isophotes, especially in the spiral arms, are manifestly not ellipses.

In cases where isophotes may be defined with PROFILE, reflecting structural
properties like isophotal rotation or variable eccentricity, the results of
a PROFILE calculation may be used as averaging paths in annulus via an
optional keyword.  This is a fairly advanced application of ANNULUS.

ANNULUS produces the radial profile of a 2-D image in the "source"
buffer and stores it as a 1-D image (spectrum) in the "dest" buffer.

The following is a detailed description of each of the keywords:

\noindent{N=n}:\newline
This is the number of concentric annuli to be used.  In general, n > 1.  If
you wish to find the average along only one annulus, then use the "RAD="
keyword.

\noindent{STEP=dr}:\newline
This is the spacing between successive annuli.  The first point of the
profile is always the image center (defined either by AXES or by using the
"CEN=" keyword).  By default a STEP of 1 pixel is used.  In general, STEP
has units of pixels, but if the "SCALE=" keyword is used, then STEP will
have units of length as defined by the images scale (usually arcseconds).

\noindent{PA=pa}:\newline 
This is the Position Angle of the major axis of elliptical annuli in
degrees.  PA is measured from the top of the image counter-clockwise, so
that an ellipse with PA=0.0 will have its long axis oriented along an image
column, while PA=90.0 will have its long axis oriented along an image
row. PA=45.0 is oriented diagonally from the upper left to lower right as
seen on the TV display.

\noindent{INC=i}:\newline
This is the ellipse Inclination Angle in degrees.  It is defined such that:
\begin{verbatim}
  Face-On: i = 0 degrees (circular annuli) 
  Edge-On: i = 90 degrees
\end{verbatim}
INC must always be LESS THAN 90 degrees.  It is related to the ellipse
eccentricity (e) by: e=sin(i)

\noindent{CEN=r0,c0}:\newline
By default, ANNULUS expects that you've run the AXES command on the image
to find the appropriate object centroid (called AXR and AXC by the AXES
command).  You may have ANNULUS ignore the results of the most recent AXES
command by using the "CEN=" keyword to specify the center of each of the
annuli. R0 is the row number of the center, and C0 is the column number. R0
and C0 may be real numbers (e.g., R0=35.4 is allowed).

\noindent{SCALE=s}:\newline
This defines the linear scale of the image.  Typically the scale is in
units of arcseconds/pixel, but any relevant units are OK. The program
printout assumes arcseconds/pixel, but it is only cosmetic.  If the SCALE
and STEP options are used, then the STEP value has the units of SCALE. For
example:\begin{verbatim}
   ANNULUS ... SCALE=0.54 STEP=1
\end{verbatim}
where the units of SCALE are arcsec/pixel implies an annular spacing of 1
arcsecond (or 1.85 pixels).

\noindent{FAST}:\newline
For annular radii smaller than 15 pixels, ANNULUS will use a slow, high
accuracy 2-D Sinc interpolation algorithm to find the intensity at a given
point along the annulus.  Beyond 15 pixels, a somewhat faster Sinc
interpolation scheme is used.  However, using the FAST keyword tells
ANNULUS to use a much faster, but somewhat lower accuracy 2-D bilinear
interpolation scheme beyond 15 pixels radius.  The time savings is quite
noticeable.

\noindent{RAD=r}:\newline
This option allows you to use ANNULUS to find the azimuthal average along a
single, given annulus with radius "r" (and PA and INC as appropriate),
without creating an entire profile.  When using the "RAD=" keyword, the
"N=", "STEP=" are ignored.  RAD has units of Pixels, unless the "SCALE="
keyword is used, at which point it has  units of "SCALE". 

\noindent{The "PROF" keyword:}:\newline
Elliptical annuli used by ANNULUS all share the same center, position
angle, and inclination (eccentricity).  It is possible, using the PROF
keyword, to use a set of annuli in which the position angle, eccentricity,
and even the centers, may vary as a function of radius.  This may be done
by first using the PROFILE command to fit ellipses to the isophotes of an
image, and then using ANNULUS with the PROF keyword to use the ellipse
parameters contained in the PROFILE common block in VISTA to define the
averaging paths.  The PROFILE common block may also be loaded using the GET
command (see GET and SAVE) before running ANNULUS, which suggests all sorts
of possibilities.

\noindent{Other details:}

Output may be redirected to the printer or external files using the '>'
option in VISTA.

For future reference, the FITS header of the spectrum containing the radial
profile generate by ANNULUS contains a number of HISTORY cards recording
the relevant annular parameters (PA, INC, STEP, CEN, etc).  These may be
reviewed by using the "BUF FULL FITS" command, or by using "HEDIT".  The
FITS "STATUS" card is changed to read "Azimuthal Average", and the "CTYPE1"
card reads either Pixels or Arcseconds, depending on whether or not the
"SCALE=" option has been used.

\noindent{Examples:}
\begin{itemize}
  \item[ANNULUS 15 1 N=50\hfill]{A radial profile of the object in buffer 1
       is computed as the azimuthal average over 50 circular annuli spaced
       in radius by 1 pixel and stored in buffer 15.}

  \item[ANNULUS 15 1 N=50 >RADPRF.DAT\hfill]{Same as (1), except that the
       output is redirected to an external file called "RADPRF.DAT" in the
       users current directory.}

  \item[ANNULUS 15 1 N=50 PA=17 INC=59 STEP=1 SCALE=0.5\hfill]{Compute a
       radial profile of the Image in buffer 1 as an average over 50
       concentric ellipses with PA=17 degrees and Inclined 59 degrees
       relative to the line-of-sight.  The annular spacing is 1 arcsecond,
       and the pixel scale is 0.5 "/pixel.  Store the results in buffer
       15.}

  \item[ANNULUS 15 2 RAD=12 CEN=102.3,237.8\hfill]{Compute the azimuthal
       average along a single circular annulus centered at Row=102.3,
       Column=237.8 with a radius of 12 pixels of the image in buffer 2.
       The 15 is a dummy argument, nothing is stored in buffer 15.}
\end{itemize}

An example of a procedure using the PROF keyword:
\begin{verbatim}
  AXES 1 BOX=2
  PROFILE 10 1 N=30 ITER=25 SCALE=0.54
  ANNULUS 12 2 PROF
\end{verbatim}
In this case, PROFILE is run to fit 30 elliptical isophotes to the image in
buffer 1 whose centroid is found in BOX=2 using the AXES command. The
azimuthally averaged radial surface brightness profile is then stored in
buffer 10 (PROFILE also azimuthally averages along the best fit
isophotes). The best fit elliptical isophotes from image 1 are then used to
compute the radial surface brightness profile of the object in image 2
using ANNULUS, and that profile is stored in buffer 12.

\noindent{A Note about selecting values of the "PA=" and "INC=" keywords:}

A circular disk inclined to the line-of-sight of an observer appears in
projection as an ellipse with its major axis rotated by some amount from
vertical.  It is the usual convention in astronomy that position angles on
the sky are measured from North towards the East, with the usual
orientation being North Up, and East to the Left.

The ANNULUS command defines the position angle in a similar fashion.  The
position angle for the "PA=" keyword is the angle between the major axis of
the projected ellipse and a vertical line passing through the center of the
ellipse (i.e., along the direction of a single image column).  The angle is
measured from the top of the image towards the left in a counterclockwise
sense.  If the image is oriented so that North is Up, East is to the Left,
then the position angle for ANNULUS is the same as the conventional
position angle.

The angle of inclination, i, is defined as the angle between the
line-of-sight from the observer to the center of the disk, and a line
perpendicular to the plane of the disk.  For a face-on disk, the
inclination is 0 degrees, and for a perfectly edge-on disk, the inclination
is 90 degrees.

This angle is related to the more easily measured axial ratio of the
ellipse.  If "a" is the length of the major axis of the ellipse, and "b" is
the length of the minor axis, then the inclination "i" is given by
\begin{verbatim}
   cos(i) = b/a
\end{verbatim}
It may also be related to the eccentricity of the ellipse, "e", such that
\begin{verbatim}
   e = \sin(i).
\end{verbatim}
NOTE: For ANNULUS, i must always be LESS than 90 degrees.  For cases where
i = 90 degrees, use the VISTA command CUT instead.

See Also: PROFILE, SECTOR, SAVE/GET, CUT


\section{RECON: Reconstruct an Image from a Surface Photometry Profile}
\begin{rawhtml}
<!-- linkto recon.html -->
\end{rawhtml}
\index{Profile!Reconstruct object from surface profile}
\begin{itemize}
  \item[Form: RECON source {[CR=f]} {[CC=f]}\hfill]{}
  \item[source]{is the image that will contain the reconstructed object.}
  \item[CR=f]{is an optional redefinition of the center row.}
  \item[CC=f]{is an optional redefinition of the center column.}
\end{itemize}

RECON will reconstruct a smooth idealized image from a surface photometry
profile of a galaxy, star, or any other centrally condensed object. The
PROFILE command is used to generate the original profile.  This routine
uses high resolution interpolation to estimate the intensity of the
original object at any pixel in the image array.  The reconstructed image
is faithful to the original brightness, and isophote position angle and
ellipticity profiles, but has perfectly elliptical isophotes.  In some
sense then, the reconstructed object looks like the original object, but
smoothed around a contour.  The reconstructed image can then be compared to
the original to look for faint and subtle features.

The reconstructed image is centered on the same pixel coordinates as the
original image. This can be adjusted with the CR and CC keywords. The
reconstruction extends only as far as the surface photometry profile;
pixels outside of the photometry area are left untouched.

Note that the reconstructed object is placed in an image that already
exists.  Use CREATE to make a new image if one does not exist, or make a
copy of the original image.

The reconstruction is based upon either the most recent surface-brightness
profile computed using PROFILE, or a previously computed fit saved to disk
with the SAVE PROF= command, and read back into memory with the GET PROF=
command.

\noindent{Examples:}
\begin{itemize}
  \item[RECON 3\hfill]{Reconstruct the surface profile in image 3.}

  \item[RECON 3 CR=100 CC=101\hfill]{Do the same as the above example,
       but center the new object at row 100 and column 101.}
\end{itemize}

See Also: PROFILE, SAVE, GET, EGAL


\section{APER: Aperture Photometry of an Extended Object}
\begin{rawhtml}
<!-- linkto aper.html -->
\end{rawhtml}
\index{Aperture Photometry}
\begin{itemize}
  \item[Form: APER source {[RAD=r1,r2,...,r10]} {[MAG=M1,M2,...MN]} 
       {[STEP=size,n]} {[SCALE=f]} {[C=r,c]} {[SCALE=f]}\hfill]{}
  \item{{[GEO=first,fact]} {[OLD]} {[INT]} {[REF]} {[PLOT]} {[STORE]}}
  \item[source]{the image being measured}
  \item[RAD=]{specify radii in pixels (or arcsec)}
  \item[MAG=]{list observed magnitudes for various radii}
  \item[STEP=]{automatically set radii with arithmetic sequence}
  \item[GEO=]{automatically set radii with geometric sequence}
  \item[C]{center of object is at this position}
  \item[SCALE=f]{input scale of image (arcsec / pixel)}
  \item[OLD]{repeat command using old parameters}
  \item[INT]{interactively set radii}
  \item[REF]{inter reference number for object.}
  \item[PLOT]{plots apertures on TV (not all devices)}
  \item[STORE]{stores output in VISTA variables}
\end{itemize}

APER will sum the intensities of pixels falling within circular apertures
centered on an object in the 'source' image.  Using apertures of differing
radii will enable you to characterize the radial intensity distribution of
an object.

Up to 10 aperture radii in arc seconds can be specified with the RAD
keyword. Alternatively, a linearly increasing sequence of apertures can be
specified with the STEP keyword, where 'incr' is the radius increment in
arc seconds, and 'n' is the number of apertures.  A geometric sequence of
apertures can be set up with the GEO= keyword, with the first value being
the first aperture size and the second the multiplier for subsequent
apertures; as many apertures as will fit in the image will be used.  The
pixel scale in arcsec/pixel is specified with the SCALE keyword.  The
centers of the aperture must be calculated ahead of time with the AXES
centroiding command, or loaded with the C keyword. Lastly, the OLD keyword
tells the routine to use any unspecified parameters from the last time.

The routine will produce a list of the apertures, the sum of pixels
interior to them, the average interior surface brightness, and the same
quantities for the rings defined by sequential apertures. The summations
are done to the nearest pixel at the edges of the aperture, that is, no
fractional pixel interpolation is done at the edges.  For this reason,
results obtained with very small apertures may be inaccurate.  The results
are stored in a common block, and can be examined with the PRINT command,
and stored or retrieved with the GET and SAVE commands. With the PLOT
keyword, the apertures will be displayed on the image display (for most
video devices).
 
APER does NOT do sky subtraction.  The background light is included in the
calculation of the brightnesses.  Use SKY to compute the background of the
image, and SUBTRACT to remove that background.

Specifying a list with the MAG word will give observed magnitudes in the
listed apertures. (What does this mean??  good question!  Inspection of
code seems that MAG= keyword allows user to specify magnitude values which
just get output right back.)

The STORE keyword saves some results into VISTA variables. The total
intensity in the first aperture is loaded into variable TOTAL, the area of
the first aperture will be loaded into APAREA, and the average intensity
into variable AVERAGE.

To extract aperture photometry in elliptical apertures, see the EMAG
command.

\noindent{Example:}
\begin{itemize}
  \item[SKY 3; SUBTRACT 3 CONST=SKY; APERTURE 3 RAD=1,3,5,7,9,13
       SCALE=0.267 C=50,35\hfill]{this sequence of commands computes and
       subtracts the mean background level for image 3.  It lists the total
       counts in the object centered on row 50, column 35, in apertures of
       1, 3, 5, 7, 9, and 13 arcsec.  The scale of the image is 0.267
       arcsec/pixel}
\end{itemize}

See Also: EMAG, ELLMAG, PROFILE, SKY


\section{ROUND: Deproject Spirals or Elliptical Images}
\begin{rawhtml}
<!-- linkto round.html -->
\end{rawhtml}
\index{Surface Photometry!Deproject spirals or elliptical images}
\begin{itemize}
  \item[Form: ROUND dest source {[PA=f]} {[E=e]} {[C=(r,c)]} {[NORM]} 
       {[SINC]}\hfill]{}
  \item[dest]{is the destination image buffer. It does not need
       to exist beforehand.}
  \item[source]{is the image buffer to transform}
  \item[PA]{is PA of object's major axis}
  \item[E]{gives its ellipticity = 1-b/a}
  \item[C]{gives its row and column center}
  \item[NORM]{renormalize pixel intensities}
  \item[SINC]{use high-accuracy sinc interpolation.}
\end{itemize}

ROUND will deproject an image by assuming that the true shape of the object
is a circle projected into an ellipse of a given position angle,
ellipticity (1-b/a) and center.  Creates a new image in the `dest' buffer
from the original image in the `source' buffer.

The SINC keyword instructs ROUND to use high-accuracy sinc interpolation
instead of the default bilinear interpolation algorithm.  

The NORM keyword is used to renormalize the pixel intensities (surface
brightness/pixel) in the transformed image.

See Also: POLAR


\section{POLAR: Regrid an Image into Polar Coordinates}
\begin{rawhtml}
<!-- linkto polar.html -->
\end{rawhtml}
\begin{itemize}
  \item[Form: POLAR dest source {[PA=f]} {[E=e]} {[C=(r,c)]} {[NORM]} 
       {[SINC]} {[R=(rmin,rmax)]}\hfill]{}
  \item[dest]{is the destination image buffer. It does not need
       to exist beforehand.}
  \item[source]{is the image buffer to transform}
  \item[PA]{is PA of object's major axis}
  \item[E]{gives its ellipticity = 1-b/a}
  \item[C]{gives its row and column center}
  \item[NORM]{renormalize pixel intensities}
  \item[SINC]{use high-resolution sinc interpolation.}
  \item[R]{gives the radial limits of the new grid}
\end{itemize}

POLAR will transform the image in the `source' buffer from observed
Cartesian coordinates (x,y) to polar coordinates (r,theta) in the `dest'
buffer.  The transformation assumes that the object is intrinsically round
and projected into an ellipse in the sky plane with the given position
angle, ellipticity (1-b/a) and center.

The SINC keyword instructs ROUND to use high-accuracy sinc interpolation
instead of the default bilinear interpolation algorithm.  

The NORM keyword is used to renormalize the pixel intensities (surface
brightness/pixel) in the transformed image.

The R=(rmin,rmax) keyword is used to specify the radial limits of the
r-axis of the polar deprojection.  By default, if R= is not given the
r-axis will run from 0 to the maximum semi-major axis of the inscribed
ellipse.

See Also: ROUND


\section{EGAL: Reconstruct a Galaxy from its PROFILE}
\begin{rawhtml}
<!-- linkto egal.html -->
\end{rawhtml}
\index{Surface Photometry!Reconstruct galaxy from PROFILE}
\begin{itemize}
  \item[Form: EGAL source {[CR=f]} {[CC=f]} {[AVANG]} {[NODEV]}\hfill]{}
  \item[source]{is the image buffer}
  \item[CR=f]{set the row center to f}
  \item[CC=f]{set the column center}
  \item[AVANG]{average isophote PAs together}
  \item[NODEV]{do not fit r\^1/4 law to outer profile}
\end{itemize}

EGAL is used to reconstruct a galaxy image from the high-resolution profile
generated with the PROFILE command.  The reconstructed galaxy will be
smoothed along isophotes and can be used as an ideal model for comparison
to the original.  The reconstruction preserves high spatial frequencies in
the center.  

It is possible to change the center of the reconstructed model galaxy to
any fractional pixel location with the CR and CC keywords.  

Noise in the outer contours may be suppressed with the AVANG keyword, which
azimuthally averages isophotes.  

By the default, the outer contours are replaced with an r\^1/4 fit to
suppress noise.  This feature can be defeated with the NODEV keyword.  This
command can be called at any time provided a profile has been calculated
with PROFILE or loaded from a disk file with the GET command.

See Also: RECON, PROFILE


\section{SNUC: Surface Photometry of Multiple-Galaxy Systems}
\begin{rawhtml}
<!-- linkto snuc.html -->
\end{rawhtml}
\index{Surface Photometry!Photometry of Multiple-Galaxy Systems}
\index{Surface Photometry!Least-squares Profile fitting}
\begin{itemize} 
  \item[Form: SNUC buf {[SKY=f]} {[SCALE=f]} {[OLD]} {[NPASS=n]} 
       {[EX=(n1,n2,...)]}\hfill]{}
  \item{{[RECEN=n]} {[BOX=n]} {[CENTER=(n1,n2,...)]} {[DELT=f]} 
       {[LIST=filespec]} {[MASK]}}
  \item[buf]{Buffer number of image to be processed.}
  \item[SKY=]{DN value of any sky level subtracted from the image}
  \item[SCALE=]{Pixel scale (arcsec/pixel)}
  \item[OLD]{Continue iterating on previous photometry solution}
  \item[NPASS=]{Perform 'n' iterations to find the photometry parameters}
  \item[EX=]{With OLD option, recalulate solution excluding last 'n'
       isophotes from galaxies 1, 2, etc from old solution. If
       'n' is negative, then extend out 'n' more isophotes.}
  \item[RECEN=]{Find centroids of galaxies, using box 'n' pixels in radius}
  \item[BOX=]{Only fit the image within box 'n'}
  \item[CENTER=]{Allow the isophote centers for galaxies 'n1', 'n2', etc
       to be free parameters.}
  \item[DELT=]{Space isophotes geometrically by ratio 1+f}
  \item[LIST=]{Specify file with galaxy centers, radii, processing flags.
       (Format: ROW COL RADIUS FLAG). Full file pathname must be
       provided.}
  \item[MASK]{ignores MASKed pixels in the fit}
\end{itemize}

SNUC will extract the individual surface photometry profiles of overlapping
or interacting galaxies comprising a multiple galaxy system.  Each galaxy
within the system is assumed to have concentric elliptical isophotes with
otherwise arbitrary brightness, position angle, and ellipticity
profiles. The isophotes parameters for all galaxies are solved for
simultaneously by a nonlinear least-squares algorithm that decomposes the
multiple system under the assumption that its morphology can be modeled as
the simple superposition of normal elliptical galaxies.
 
The routine models the image displayed on the TV.  The startup procedure
for the decomposition is to mark the nuclei and extent of the individual
galaxies with the cursor.  Mark any nucleus with (1) to fit the specified
galaxy to the image or {[BOX=n]} border, or (2) to then go on and mark an
outer limiting isophote with (0).  Mark with (0) once all galaxies have
been selected to start the decomposition.  At this time, five galaxies is
the maximum that can be fit.  If desired, of course, this routine will also
fit single isolated galaxies.

The photometry solution is found iteratively.  NPASS iterations are done
for a given run (default is 3).  Further iterations for a given setup can
be run with the OLD keyword.  The EX keyword is used with the OLD keyword
to delete or add more isophotes to the outer extent of any galaxy used in a
previous run.  Isophotes are spaced geometrically by the ratio 1+DELT
(default is 1.25).  Isophote spacing is the same for all galaxies.
Isophote centers are fixed by default and taken to be the nuclear centroid
for each galaxy.  Improved estimates of the centroids can be made during
any pass with the RECEN keyword if the default centroiding is found to be
inaccurate. Isophote centers can be solved for independently using the
CENTER keyword; however, this option does not work well at the present.
Since the decomposition solves for the isophotes by a least squares
procedure, portions of the image contaminated by stars, defects, faint
galaxies can be excluded from the fit.  Pixels set identically to 0.0 are
ignored by the algorithm.  One handy use of this feature is to run a
solution, reconstruct it (see RENUC), subtract it off, and clip any faint
objects or features uncovered in the process to 0.0.  The routine also
generates error estimates for the isophote parameters assuming shot noise
errors in the pixel values.  For accurate error estimates the level of any
sky level subtracted from the image must be given with the SKY keyword.

Photometry results can be saved and recalled with the SAVE and GET
commands, or printed with the PRINT command, using the MULT option in
either of these commands.
\begin{itemize}
  \item[SNUC buf SKY=back SCALE=0.334 RECEN=5\hfill]{Setup a
       decomposition, do three iterations and stop.}

  \item[SNUC buf SKY=back OLD EX=(-2,0,1) NPASS=6\hfill]{Continue this
       solution, but extend the primary galaxy by more isophotes and delete
       the outer isophote of galaxy 3.  Iterate for several passes.}

  \item[SNUC buf SKY=back SCALE=0.334 DELT=0.1\hfill]{Start a solution
       using a finer isophote spacing.}

  \item[SNUC buf SKY=back DELT=0.1 LIST=/mydir/galaxy.lis\hfill]{Use an
       input file to specify galaxy parameters.}
\end{itemize}

See Also: RENUC, PROFILE


\section{RENUC: Reconstruct Galaxies from Multiple-Galaxy Photometry}
\begin{rawhtml}
<!-- linkto renuc.html -->
\end{rawhtml}
\index{Surface Photometry!Reconstruct Multiple Galaxy Systems}
\index{Surface Photometry!Rebuilding Galaxy from SNUC}
\begin{itemize} 
  \item[Form: RENUC source {[BOX=n]} {[EX=(n1,n2,...)]} {[LIM=(n1,n2,...]}
       {[INC=(n1,n2,...)]} {[CENTER=(n1,n2,...)]}\hfill]{}
  \item[source]{The image buffer which will hold the reconstruction}
  \item[BOX=]{Only reconstruct within BOX n.}
  \item[EX=]{Do not reconstruct galaxies n1, n2, etc.}
  \item[INC=]{Only reconstruct galaxies n1, n2, etc.}
  \item[LIM=]{Do not use the last n isophotes from galaxies 1, 2, etc.}
  \item[CENTER=]{Reconstruct using the individual isophote centers for
       galaxies n1, n2, etc.}
\end{itemize} 

RENUC reconstructs a model of a multiple galaxy system or any combination
of its component galaxies from the multiple galaxy system photometry
computed by the SNUC command.

The EX keyword is used to select those galaxies NOT to be reconstructed.
Similarly, the INC keyword can be used to select only those galaxy models
to be included in the reconstruction.

If the quality of the outer isophote solutions for any galaxy is poor, they
can be ignored by the reconstruction algorithm with the LIM keyword.

CENTER tells the routine to use any isophote dependent centers for the
reconstruction that may have been measured with the CENTER option in the
SNUC command (at this time, this option does not work well).

\begin{itemize}
  \item[RENUC 1\hfill]{Reconstruct the entire system fitted by SNUC (into
       image 1).}

  \item[RENUC 1 INC=(1) \hfill]{}

  \item[or RENUC 1 EX=(2,3,4)\hfill]{Only reconstruct galaxy 1 (four
       galaxies were fitted);}

  \item[RENUC 1 LIM=(0,2)\hfill]{Do not use the last isophote of galaxy 2.}
\end{itemize} 

See Also: SNUC


\section{SECTOR: Surface Photometry of Star Clusters or Complex Objects}
\begin{rawhtml}
<!-- linkto sector.html -->
\end{rawhtml}
\index{Surface Photometry!Star clusters or complex objects}
\begin{itemize}
  \item[Form: SECTOR source {[RAD=(lin,geo)]} {[C=(r,c)]} {[SCALE=f]} 
       {[PLOT]}\hfill]{}
  \item[source]{The image buffer}
  \item[RAD=]{Describe the isophote spacing. Isophotes are initially spaced
       linearly every `lin' pixels in radius. Isophotes further out are
       spaced geometrically by `geo' when this step is larger than `lin'.
       Defaults: lin=2 pixels, geo=0.15 (increase radius by 15\% per
       step).}
  \item[C=]{Give the object center.  Default is to use results from
       the AXES or AUTOCEN commands.}
  \item[SCALE=]{Give the scale in arcsec/pixel. Default is 0.55.}
  \item[PLOT]{Plot the isophotes on the TV.}
\end{itemize}

SECTOR measures the brightness distribution of a star cluster or any other
extended object by adding up the flux in a set of circular isophotal rings.
Errors in brightness in any ring are calculated by dividing the ring into
eight angular sectors and intercomparing the individual sector results.
The final profile is saved in the MULT block, and can be saved or gotten
with the MULT keyword in the SAVE or GET commands, printed in the PRINT
command, or plotted with the PLOT command.  The object can also be
reconstructed with the RENUC command.

The center should be found first with the AUTOCEN or AXES command.  The C
keyword can be used to override the default center.  Isophotes increase
linearly from the center at first, and are spaced by the lin entry with the
RAD keyword.  When a geometric increment given by the geo entry gives a
larger spacing, it is used to space the outer isophotes.  The radii used
are at the radii at which the area of the isophotal rings is evenly
divided.  Pixels can be excluded from the calculation by setting them to
zero.

See Also: PROFILE, ANNULUS, AXES

\section{AEDIT: Edit Values in Selected Image Regions}
\begin{rawhtml}
<!-- linkto aedit.html -->
\end{rawhtml}
\begin{itemize}
  \item[Form: AEDIT source {[BOX=n]} {[INT]} {[SET=x]} {[MASK]} {[OLD]} 
       {[filename]}\hfill]{}
  \item[BOX=n]{specifies the BOX to use}
  \item[INT]{BOX will be specified interactively}
  \item[SET=x]{specifies the flag value}
  \item[MASK]{tells AEDIT to create a VISTA logical mask only (pixel values
       are unchanged)}
  \item[OLD]{tells AEDIT to edit pixels specified in the file specified on
       the command line}
  \item[filename]{tells AEDIT to write out the regions it edits into a
       formatted file (if INT is specified) or to read in the regions from
       the file (if OLD is specified)}
\end{itemize}

AEDIT will set all pixels in given image subregions to a constant value and
create a VISTA logical MASK of those pixels.  This can be done either
interactively, through a box number, or through a formatted file containing
a list of box definitions.  If INT is specified, the user is expected to
interactively determine on the display a number of regions to be edited.
INT overrides BOX=n.  

If INT is specified and a file name is given on the command line, AEDIT
will write a list of the boxes it has edited into the specified file.  The
form is SR ER SC EC; the files are formatted (ASCII) files.  If OLD is
specified, AEDIT will read the regions to be edited from the specified
file.  If no file name is given on the command line, the user will be
prompted for it.  OLD overrides INT.

SET=x specifies the flag value to which to set the data (default = -7777).
If the image is not displayed and INT is specified, this routine will
display it.  The keyword MASK tells the routine to create the logical mask
only, and not actually change any pixel values.


\section{CLPROF: Clear the Contents of the PROFILE Buffer}
\begin{rawhtml}
<!-- linkto clprof.html -->
\end{rawhtml}
\begin{itemize}
  \item[Form: CLPROF {[RAD=r]}\hfill]{}
  \item[RAD=r]{Specifies the radius beyond which the prf common
       block is cleared.}
\end{itemize}

CLPROF sets all parameters in the PRF common block to zero. CLPROF should
always be used when the command GET PROF= is used to overwrite the existing
contents of the profile block with new profile parameters stored in a .pro
file. If CLPROF is not run, and if the profile being overwritten is larger
than the new one, the old parameters will remain in the common block for
radii larger than the maximum radius of the new profile.

To retain the existing profile parameters out to a given radius, setting
them to zero beyond this, the optional keyword RAD=r may be given. All
profile parameters corresponding to radii greater than or equal to r are
then set to zero. This is valuable if a PROFILE calculation is seen to have
gone awry beyond a certain radius.

See Also: PROFILE, CPROF


\section{CPROF: Correct the Results of a PROFILE Calculation}
\begin{rawhtml}
<!-- linkto cprof.html -->
\end{rawhtml}
\begin{itemize}
  \item[Form: CPROF {[ORDER=n]} {[MEDIAN]} {[SQWEIGHT]}\hfill]{}
  \item[ORDER=n]{Specifies the order of the polynomial fit. Default is 2.}
  \item[MEDIAN]{Specifies that median surface brightness, rather
       than mean, will be used for exponential fit (if performed)}
  \item[SQWEIGHT]{Specifies that square root weighting, rather than 
       uniform weighting, is to be used during exponential fit}
\end{itemize}

CPROF enables the user to inspect the results of a PROFILE calculation in
graphical form, and to modify them by replacing portions judged to be in
error by the results of a second order polynomial fit to "good" regions.
This often is necessary for spiral galaxies for which the outer contours
can get confused.

The ellipticity, surface brightness, and semi-minor axis values stored in
the PROFILE common block are first plotted on the display.  Frequently one
can find regions of spuriously low ellipticity and surface brightness, with
a corresponding local maximum in the semi-minor axis, known as "excursions";
these represent areas in which the PROFILE iterations have converged
poorly. If such a region is identified, the user is asked to specify the
boundaries of regions on either side of the excursion in which the PROFILE
results are judged to be acceptable; a polynomial is fit to the
ellipticities, position angles, and surface brightnesses in the latter
regions. The results of the fit are then used to recompute ellipticities,
PA's and SB's in the excursion region. The default is to perform a second
order fit, but fits of other order can be used by giving the ORDER=n
keyword.

The process of plotting the profile parameters, identifying glitches, and
replacing with fits continues until the user specifies that the profile is
satisfactory. Another option is to fit only the surface brightnesses in a
region. This is sometimes desirable when the ellipticity undergoes an
abrupt, genuine increase; the surface brightness is then spuriously
enhanced.

CPROF is best used when the galaxy image has been displayed, and the
ellipses overplotted on the image using TVPROF. The success of a CPROF
calculation is best measured by looking at this display.

See Also: PROFILE, TVPROF, CLPROF


\section{EMAG: Aperture Photometry with Elliptical Apertures}
\begin{rawhtml}
<!-- linkto emag.html -->
\end{rawhtml}
\begin{itemize}
  \item[Form: EMAG source {[CENTER=(row,col)]} {[PA=theta]} {[ELL=eps]} {[N=n]}
     {[PROF=p]} {[RMAX]} {[ERAD]} {[APPEND]}\hfill]{}
  \item[source]{is the image EMAG works on}
  \item[CENTER=(row,col)]{allows the user to determine the center
       of the object being worked on.}
  \item[PA=theta]{allows the user to set the position angle of
       the elliptical apertures (theta in degrees.)}
  \item[ELL=eps]{allows the user to set the ellipticity of the
       apertures.}
  \item[N=n]{user-determined number of elliptical apertures;
       the outermost aperture has semimajor axis (n-1) pixels.}
  \item[PROF=p]{tells EMAG to take the ellipse parameters from the p-th
       aperture in the PRF common block--this works only if PROFILE have
       been previously run, or if their results have been read in by
       GPGET.}
  \item[RMAX]{tells EMAG to learn the maximum semimajor axis radius from
       the VISTA variable RMAX (this works only if EMARK or RMARK have been
       previously run.)}
  \item[ERAD]{tells EMAG to take the ellipse parameters from the PRF common
       block, at a radius given by the value of the VISTA variable ERAD,
       set by the routine EMARK.}
  \item[APPEND]{tells EMAG to compute not only elliptical totals, but also
       surface brightnesses in elliptical belts, for radii between the
       previous maximum radius in the PRF common block, and the new
       max. radius determined by the N=n or RMAX keywords; the fixed values
       of ellipticity and PA used for the EMAG calculation are also filled
       in.}
\end{itemize}

EMAG performs aperture photometry within concentric elliptical apertures of
increasing semimajor axis radius but fixed ellipticity and position angle.
The image is assumed not to have been sky subtracted, but the background
level of the image is assumed to be contained in the VISTA variable
SKY. Each pixel then gives a contribution (pixel value - SKY).  The
aperture radii (i.e. semimajor axes) are incremented by one pixel at a
time. The resulting aperture total intensities are recorded in the PRF
common block, replacing the values of the "elliptical totals" computed by
PROFILE.

EMAG attempts to account for the presence of perturbing stars (or any other
glitches) on the image of the galaxy as follows: AEDIT is first used to
mark those pixels which are contaminated; these pixels are given the flag
value -7777. EMAG will not use any pixel with a value less than zero (note
therefore the importance of not performing sky subtraction!); instead, it
will look for that pixel's "symmetric partner", i.e. the one reflected
about the galaxy center. If the partner is not also flagged, it will use
its intensity level as an approximation for the original, flagged pixel; if
it is also flagged, however, neither pixel is used by EMAG.

The user can explicitly specify the center, position angle, and ellipticity
of the apertures, if desired, with the CENTER=(row,col), PA=theta, and
ELL=eps keywords on the command line. It is more common, however, to use
EMAG following the use of PROFILE. In this case, the ellipse parameters may
be determined by accessing the PRF common block. (The ellipse centers,
however, may also be learned from the AX common block, following use of
AXES.) The radius at which the ellipse parameters are taken from the PRF
common block is set either directly, via the PROF=p keyword on the command
line, or by the ERAD keyword, which then gets the radius from the VISTA
variable ERAD, which has been set by the interactive routine EMARK.

The number of apertures for which the computation is performed (or,
equivalently, the maximum radius for photometry), may be set in one of
several ways. The command line keyword N=n sets it explicitly.
Alternatively, the keyword RMAX instructs EMAG to obtain the maximum radius
from the VISTA variable RMAX, which can be set interactively with the
routines RMARK or EMARK. If neither N=n nor RMAX are given on the command
line, EMAG will use the existing maximum radius in the PRF common block
(provided the latter is loaded.)

The command line keyword APPEND instructs EMAG to tack on to the PRF common
block not only the elliptical totals, but also the PA and ellipticity EMAG
works with, and surface brightnesses computed along the way. This will only
work if the radius for photometry determined by the N=n or RMAX keywords is
greater than the previous radius in the PRF common block.

To compute the magnitudes within elliptical apertures with varying
ellipticity and position angle as computed by PROFILE, see the ELLMAG
command.

At present, the results of an EMAG calculation can be viewed by writing the
contents of the PROFILE common block using PRINT PROF.

\noindent{Examples:}
\begin{itemize}
  \item[EMAG 5 CENTER=(243,209) N=81 PA=32.5 ELL=0.71\hfill]{}

  \item[EMAG 5 ERAD RMAX APPEND\hfill]{Gets ellipse parameters from PRF
       common block at radius ERAD, works out to radius RMAX, and tacks on
       ellipticities, position angles, and surface brightnesses.}
\end{itemize}

See Also: EMARK, RMARK, ELLMAG


\section{EMARK: Interactively Specify Ellipse Parameters for 
        Photometry with EMAG}
\begin{rawhtml}
<!-- linkto emark.html -->
\end{rawhtml}
\begin{itemize}
  \item[Form: EMARK\hfill]{}
\end{itemize}

EMARK enables the user to mark with the cursor the radius at which the
ellipse parameters to be used by the routine EMAG are to be taken. The
image must be displayed, and the PROFILE common block loaded, in order for
EMARK to work. It is advisable to have displayed the elliptical isophotes
with TVPROF prior to running EMARK.

EMARK will prompt the user to move the cursor to the position (along the
major axis) of the ellipse whose shape and orientation is the one desired
for use in EMAG photometry. The user then strikes the key "P". A cross will
be displayed at this point, and the user will be given the option of
choosing another position. Once the user is satisfied with the position
chosen, EMARK will compute the distance of this point from the ellipse
center. This distance is then loaded into the VISTA variable ERAD.

When EMAG is run with ERAD as a keyword, values of the PROFILE common block
near the radius specified by EMARK are averaged in order to obtain the
position angles, ellipticities, and centers of the elliptical apertures
used by EMAG.

Following the marking of the radius defining ERAD, EMARK asks the user if
it is desired to mark the maximum radius for photometry. If so, the user is
again prompted to move the cursor to this position and strike the key "P";
again the distance from the center is computed, and loaded into the VISTA
variable RMAX.

See Also: RMARK, EMAG


\section{RMARK: Interactively Set Maximum Radius for EMAG}
\begin{rawhtml}
<!-- linkto rmark.html -->
\end{rawhtml}
\begin{itemize}
  \item[Form: RMARK\hfill]{}
\end{itemize}

RMARK enables the user to interactively set the maximum radius for the
photometry routine EMAG. The image on which photometry is to be performed
must be displayed, and an AXES computation must have been previously
performed, in order for RMARK to work.
 
RMARK will prompt the user to move the cursor to a point at the distance
from the object center to which the photometry is to be performed. The user
then strikes the key "P". A cross will be displayed at this point and the
user will be given the option of choosing another position. Once the user
is satisfied with the position chosen, RMARK will compute the distance of
this point from the object center. This distance is then loaded into the
VISTA variable RMAX.

When RMAX is given as a command line keyword in EMAG, the value of this
variable is retrieved and the photometry proceeds out to this radius.

See Also: EMARK, EMAG


\section{ELLMAG: Compute Elliptical Magnitudes for PROFILE Ellipses}
\begin{rawhtml}
<!-- linkto ellmag.html -->
\end{rawhtml}
\begin{itemize}
  \item[Form: ELLMAG {[MEDIAN]}\hfill]{}
  \item[MEDIAN]{specifies that median, rather than mean, surface
       brightnesses are to be used for computing magnitudes}
\end{itemize}

ELLMAG calculates the elliptical magnitude within the elliptical isophote
fit using the PROFILE command.  It uses the values for the surface
brightness computed previously by PROFILE.  It recomputes the circular
totals using CIRCMAG, so that any results appended to the PROFILE results
with EMAG will have appropriate circular totals all the way out to the
final radius.

This routine differs from EMAG in that the magnitudes within each ellipse
in the PROFILE common block are computed, correctly taking into account the
varying position angles and eccentricities.  EMAG, by contrast, computes
concentric elliptical magnitudes given the parameters for ONE of the
ellipses in the PROFILE common block.

This routine uses the SB results to approximate the elliptical totals
within increasing semi-major axis radii.  For each increase in radius, the
routine approximates the additional ellipse area which needs to be taken
into account for calculating the elliptical totals.  It then multiplies the
SB value at that radius by this area to get an approximation of the extra
number of counts to add to the elliptical total.  The routine does take
into account the possibility of crossing ellipses but, at this time, does
not yet take into account the possibility of changing position angle.

See Also: EMAG, APER, PROFILE


\section{TVPROF: Display PROFILE Results on the Image Display}
\begin{rawhtml}
<!-- linkto tvprof.html -->
\end{rawhtml}
\begin{itemize}
  \item[Form: TVPROF N1=n1 SPACE=n2\hfill]{}
  \item[N1=n1]{first ellipse displayed has semimajor axis radius n1.}
  \item[SPACE=n2]{subsequent ellipse semimajor axis radii are incremented
       by n2 pixels, continuing out to the last entry in the PRF common
       block.}
\end{itemize}

TVPROF will display the ellipses contained in the PRF common block as a
result of a PROFILE or EMAG calculation. The ellipses are drawn on the TV
display. There must be an image displayed in order for TVPROF to work. This
image should be the one for which the model elliptical isophotes were
computed, although TVPROF has no way of knowing what image is displayed, of
course. TVPROF is intended for use in conjunction with CPROF, for the
purpose of inspecting and modifying the results of galaxy surface
photometry calculations.

See Also:  PROFILE, CPROF, EMAG


\section{OPREP: Open a Batch Surface Photometry Preparation File}
\begin{rawhtml}
<!-- linkto oprep.html -->
\end{rawhtml}
\begin{itemize}
  \item[Form: OPREP {[R]} {[W]} {[filename]}\hfill]{}
  \item[R]{open to read}
  \item[W]{open to write}
  \item[filename]{prep-file name.}
\end{itemize}

OPREP opens a "batch photometry preparation file" used for batch-type
surface photometry reductions.  The files contain the following information
(one record per image):
\begin{itemize}
  \item{SKY value}
  \item{RMAX variable (max. radius for photometry)}
  \item{estimated HWHM for the PSF}
  \item{contents of the AXES block}
  \item{image header.}
\end{itemize}

This information is written by WPREP command, and read by RPREP.  This
routine initialized the PREPCON common block, which records the status of
the prep-file (whether open to read or write).  The filename is supplied as
a keyword.  Default directory is V\_DATADIR, with the default extension
``.pre''.  If no name is given on the command line, the default name
PREPFILE is used.  If there is a prep-file already open, this routine closes
it and opens a new one.

NOTE: You must compute the image parameters above ahead of time, either
outside of VISTA or using the AXES command without the CORR option.

See Also:  CLPREP, WPREP, RPREP


\section{CLPREP: Close a Batch Surface Photometry Preparation File}
\begin{rawhtml}
<!-- linkto clprep.html -->
\end{rawhtml}
\begin{itemize}
  \item[Form: CLPREP\hfill]{}
\end{itemize}

CLPREP closes the ``batch surface photometry preparation file" opened with
the OPREP command.

See Also:  OPREP, WPREP, RPREP


\section{RPREP: Read a Record from a Batch Surface Photometry Prep-File}
\begin{rawhtml}
<!-- linkto rprep.html -->
\end{rawhtml}
\begin{itemize}
  \item[Form: RPREP\hfill]{}
\end{itemize}

RPREP reads a record from a batch surface photometry preparation file
opened using OPREP.  These files contain the following information (one
record per image):
\begin{itemize}
  \item{SKY value (into VISTA variable SKY)}
  \item{max. radius for photometry (into variable RMAX)}
  \item{estimated HWHM for the PSF (into variable PSF)}
  \item{contents of the AXES block (into AXES)}
  \item{image header (ignored by this routine).}
\end{itemize}

This information must already have been written into the prep-file using
the WPREP.  OPREP opens the file, which must have already been opened
read-only using the `OPREP R' command

See Also:  OPREP, WPREP, CLPREP


\section{WPREP: Write a Record into a Batch Surface Photometry Prep-File}
\begin{rawhtml}
<!-- linkto wprep.html -->
\end{rawhtml}
\begin{itemize}
  \item[Form: WPREP image\hfill]{}
\end{itemize}

WPREP writes a record into a batch surface photometry preparation file
opened using OPREP.  The data written in each record are as follows:
\begin{itemize}
  \item{SKY value (from VISTA variable SKY)}
  \item{max. radius for photometry (from variable RMAX)}
  \item{estimated HWHM for the PSF (from variable PSF)}
  \item{contents of the AXES block (from AXES)}
  \item{image header.}
\end{itemize}

This information is formatted to be read by RPREP.  The file must have
already been opened with write-access using the `OPREP W' command.

NOTE: You must compute the image parameters above ahead of time, either
outside of VISTA or using the AXES command without the CORR option.

See Also:  OPREP, RPREP, CLPREP

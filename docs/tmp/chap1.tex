\chapter{Introduction}

%
% minor reformatting and editing
% added use of verbatim environment for some examples.
% rwp/osu 98Jul26
%

\section{The VISTA Program}
\begin{rawhtml}
<!-- linkto vista.html -->
\end{rawhtml}

VISTA is an interactive image and spectral reduction and analysis package
developed at the Lick Observatory.  It is a package which is designed to
allow efficient reduction and analysis of astronomical observations.

Many other such packages exist. Some possible advantages of VISTA are that
the routines are primarily memory-based, leading to relatively fast
execution, routines are command-line driven using optional keywords for
easy use, and a convenient display interface. Perhaps most importantly, it
is relatively easy to add new commands/features into VISTA or to modify
existing ones, once one becomes familiar with the programming
structure. VISTA is programmed predominantly in FORTRAN.  An introductory
programmers guide to VISTA is available from
\htmladdnormallink{http://astro.nmsu.edu/~holtz/vista} 
{http://astro.nmsu.edu/~holtz/vista}.

This is the manual for VISTA version 5.0, released July 1998.  For
information on how to obtain the VISTA package and/or how to install it on
your system, check the URL
\htmladdnormallink{http://astro.nmsu.edu/~holtz/vista} 
{http://astro.nmsu.edu/~holtz/vista} or see Appendix A of this manual.

We wish to note that no one officially supports Vista, so we cannot
guarantee any of the results. All problems, comments, etc. should be
addressed to the Vista distributor (holtz@nmsu.edu), and, although we
cannot guarantee a response, we generally try to help.

The manual for VISTA is fairly extensive, with many examples for commands.
Many sections of the manual give examples of procedures, showing how a
particular command can be combined with others to create common
data-reduction operations.

\section{A Brief History of VISTA}
\begin{rawhtml}
<!-- linkto origins.html -->
\end{rawhtml}

VISTA was originally developed by Richard J. Stover and Tod R. Lauer (now
at NOAO) in 1982 at the University of California, Santa Cruz for use on the
UC Lick Observatory VAX/VMS systems.  The original intent was to support
reduction and analysis of data acquired with the first generation of CCD
detectors coming into use at Lick at that time.  Tod Lauer added much of
the first surface photometry routines (based in part on routines developed
by Steve Kent).  The development of VISTA ran roughly in parallel with the
development of the NOAO IRAF package, but the latter effort was supported
on a greater scale and intended to be used by a wider community at the
National Observatories.  VISTA was originally only to support the observing
activities at the University of California at Lick Observatory, but as
students and postdocs from UC began to diffuse out into the community,
VISTA followed.  It has since been modified, augmented, improved and
otherwise messed with continually since then, primarily by former graduate
students and postdocs at UC Santa Cruz/Lick Observatory during the 1980s.

Version 3 was started in 1985 and was primarily the work of Don Terndrup
(now at Ohio State) and Richard Stover (UCO/Lick), with contributions by
Robert ("BF") Goodrich (Keck), Michael De Robertis (York University) ,
Jesus Gonzalez (UNAM), Jon Holtzman (NMSU), and Rick Pogge (Ohio State).
It incorporated most of the spectroscopic reduction routines that are now
part of VISTA, and introduced the first generation of stellar photometry
programs.

Version 4 of VISTA was largely the work of Jon Holtzman, and in addition to
fixing yet more bugs, the primary effort was to port the package to UNIX
operating system.  Several new features were also included, such as the
capacity for better image interaction on Unix workstations using first
SunView then X windows, PostScript image and plot hardcopy, an embedded
version of the DAOPHOT photometry package (written by Peter Stetson at DAO,
who graciously provided the code and allowed its inclusion within VISTA),
Fourier transform routines, some new surface photometry routines, some
routines for computing and applying extinction and transformation
coefficients, etc.  Many of the new routines, including the SUN video
routines and the fast image arithmetic came from Tod Lauer.  John Tonry
(UH) provided basics for implementing the X11 video routines, while Steve
Allen (UCO/Lick) provided the X11 graphics drivers for LickMongo.  Rick
Pogge wrote the PostScript image and graphics drivers.  Jeff Willick
(Stanford) and Stephane Courteau (DAO) provided some new surface photometry
routines.  The LickMongo package was incorporated into a distribution
version, so VISTA no longer needed a separate standalone plotting package
to be installed.  Some features were turned off in Version 4: Many of the
old stellar photometry routines were supplanted by the DAOPHOT versions,
the old LINE and VTEC commands which were device specific have been
discontinued, and VISTA no longer uses the PGPLOT plotting package.

Version 5 incorporates yet more bug fixes and new capabilities. It also has
been modified to run under the Linux operating system on personal
computers. This version of VISTA has been routinely run on Sun, DEC, and
Linux workstations. The support for VMS workstations, however, has been
discontinued, and we only support image and graphical display in the common
X-windows environment.  The manual has been converted to TeX, which allows
exporting it in HTML format for WWW browsing using LaTeX2HTML.

The master version of VISTA is currently maintained by Jon Holtzman at New
Mexico State University and is available via anonymous FTP and the WWW. See
the appendices for details on how to obtain and install the package.

\section{VISTA Basics}
\begin{rawhtml}
<!-- linkto basics.html -->
\end{rawhtml}
\index{Image}
\index{Image Processing}
\index{Spectrum}
\index{Buffer}
\index{Data Files}
\index{Row Numbering}
\index{Column Numbering}

The VISTA program was originally designed to reduce data produced by
astronomical CCD imagers, but has found use for the analysis of any digital
imaging array data, including astronomical IR Arrays, Radio Maps, and X-Ray
images.  Any image array is referred to generically as an ``IMAGE'' in
VISTA.  An image is a 2-D array of floating-point numbers that represent
the data value at a given picture-element or ``pixel'' in the image plane.
The general term for manipulation of these numbers is \textit{image
processing}.  VISTA provides the user with a set of interactive commands
issued either at the keyboard or from a list of commands called a
\textit{procedure script} that can be executed like a mini program in
written in the VISTA interactive command language.  In practice, VISTA can
be used to process any digital images, not just astronomical images,
although many of the routines in VISTA are specialized for astronomical
data analysis (e.g., spectrophotometry or stellar photometry).

A digital image, whatever the source, is coded as a a rectangular array of
'picture elements' or \textit{pixels}.  Each pixel has both a location and
a value; it is the array of values that constitutes an image.  The location
of a pixel is specified by its \textit{row} and \textit{column}
coordinates.  Rows and columns are numbered continuously from some starting
value, usually zero or one. (Thus an image with 400 rows and 350 columns
may have the rows numbered from 1 to 399 and the columns numbered from 1 to
349).  The starting row or column may start at some other value, for
example, the numbering may run from 100 to 499.  When an image is
displayed, a convention is adopted for how rows and columns run; generally
columns increase from left to right, but various packages differ in how
they display rows.  By default, the VISTA convention is for the origin to
refer to the displayed \textit{upper left} corner of the image; you can use
the FLIP keyword with the TV command to have the origin fall in the lower
left. Neither way guarantees that your image will appear in ``sky
orientation''; this depends on the optics in the system and the way that
the CCD is read out at the telescope.

It is convenient to refer to images by a label or name, rather than having
to refer to each of the pixels individually.  For this reason, the VISTA
program has several \textit{buffers} (reserved blocks of memory) for the
storage of images.  VISTA allows the use of up to one hundred image
buffers, although the actual number of buffers that may be filled at any
given time depends on the sizes of the images and the amount of memory
available on your system.  An image is referred to by the number of the
buffer that holds it, with buffer numbers running from 1 to 100.

Additional information about each image is stored in an associated
\textit{header}, also stored in memory.  The header is structured on
the FITS header format, and contains a table of values describing the
properties of the image, including the image dimensions and data format,
observing information, etc.  These are always associated with an image and
are automatically copied or adjusted if the image itself is copied or
adjusted.  

A \textit{spectrum} is a one-dimensional array of numbers representing the
amount of light at each wavelength interval from an astronomical object.  A
CCD used for spectroscopy does not produce a spectrum directly; rather, it
records a two-dimensional image of the spread-out light behind the slit and
grating.  This image is later converted (''mashed'') into a spectrum.
Spectra are stored in the same buffers that are used for images.  A
spectrum is an image with one row.

Various VISTA programs produce data sets that are neither images nor
spectra.  The buffers for the storing of these data are not referred to by
number, usually because there is only one of each type.  These data can be
saved on the disk (with the SAVE command), retrieved (with GET) or
displayed (PRINT).

A very important feature of VISTA is \textit{procedures}.  These are lists
of commands that are executed as a group.  There is a special buffer for
storing procedures, and they may be saved on the disk for repeated
use. Various conditional and looping ability is available in procedures,
allowing repetitive tasks to be easily executed.

VISTA also has \textit{variables}, which are symbolic names for numerical
values or character strings.  Using variables in procedures greatly expands
the power of VISTA, by allowing the user to write complex
programs. However, VISTA is not a fully flexible programming language; it
does not have full array capabilities (although the image buffers
themselves can be used as one or two dimensional arrays). The command
parsing is also not highly sophisticated.

VISTA has adopted the Flexible Image Transport System (FITS) format for all
images (although we can read a variety of related image formats, FITS is
preferred).  A complete description of the FITS format may be found at the
FITS Support Office Home Page at the NASA Goddard Space Flight Center
(\htmladdnormallink{http://fits.gsfc.nasa.gov/}{http://fits.gsfc.nasa.gov/}).

\section{VISTA Command Syntax}
\begin{rawhtml}
<!-- linkto commands.html -->
\end{rawhtml}
\index{Commands!Types}
\index{Commands!Syntax}
\index{Commands!Keywords}
\index{Keywords!Defined}

VISTA works by calling various subroutines from a set of libraries. You
tell VISTA what to do by entering commands.

Commands come in several types.  The simplest commands, such as 'QUIT'
(which halts VISTA) work on no data and only operate in one way. Commands
that manipulate images or spectra have an 'object' that the command
operates on.  You should always specify the object immediately after the
command.  An example of this is 'ZAP 1' which performs the ZAP operation
(q.v.) on image number 1. More complicated commands have 'keywords' that
control their operation.  Keywords allow you to tailor the operation of the
command to suit your needs. An example of a command with keywords is
\begin{itemize}
   \item{BOX 2 CC=100 CR=123 NC=100 NR=25}
\end{itemize}
Keywords can appear in any order on the command line. The allowed keywords
for any command can be found by entering the command name followed by a
'?'; e.g.  BOX ? will list all the allowed keywords for the BOX command.

All of the objects and keywords for a command must appear on the same line
as the command.  The command name ALWAYS comes first.  Several commands may
be placed on a single line, separated by semicolons.  An example of this
is:
\begin{itemize}
   \item{COPY 3 1; TV 3 BOX=1; CONTOUR 3 BOX=1}
\end{itemize}

Command lines are split by the VISTA parser into individual words.  The
parsing uses blank spaces to separate words, with the consequence that you
\textbf{cannot} put blank spaces within a word/keyword. For example, the
simple arithmetic statement:
\begin{itemize}
   \item{a=b+3}
\end{itemize}
will not work if there are any embedded blank spaces. Strings can have
embedded blank spaces if the entire string in enclosed in single quotation
(') marks.

A command can be extended over more than one line by using the character
$|$ at the end of the line.  Type HELP EXTEND if you are at a terminal for
more information about this feature.  This information is also in section 2
of this manual.

Any character can appear on the command line, but characters after the
first occurrence of ! in a command are ignored.  The ! is the VISTA
'comment character', providing a way to label individual commands.  Here is
an example of a command with a comment:
\begin{itemize}
   \item{SUBTRACT 1 CONST=40.3  ! Subtract 40.3 from image 1.}
\end{itemize}
Commenting commands is very handy in procedures. 

Buffer numbers always take the form of integer constants.  These are used
to denote buffers that store the images and spectra.  They must not have a
decimal point: for example, '1' is an integer constant, while '1.' is not.
A very important exception to this rule is the '\$' construction which
allows you to use variables to denote objects.  See OBJECT (type HELP
OBJECT if you are on a terminal) for information.

Keywords come in several types:
\begin{enumerate}
   \item REAL CONSTANTS: These are numbers containing a decimal point
         (e.g., '3.14'), and are used as input to programs that require a
         numeric value as a parameter.  An example of a command with a real
         constant in it is
	 \begin{verbatim}
	   TV 2 1024.0
	 \end{verbatim}

\item WORDS: These are single or multiple words without an
      '=' sign in them.  Most words turn on or off options in the programs.
      An example is:
      \begin{verbatim}
         CONTOUR 1 DASH NOERASE
      \end{verbatim}
      IMPORTANT: Sometimes you will want to consider several words as a
      unit.  Multiple words used in this way must be enclosed in single
      quotes.  An example is the CHANGE command, which changes the name of
      an image.  If you wanted the new name to be a single word, you could
      say:
      \begin{verbatim}
         CHANGE 3 NGC5128
      \end{verbatim}
      but if the new name has several words in it, you would need quotes,
      so:
      \begin{verbatim}
         CHANGE 3 'NGC5128 - through clouds '
      \end{verbatim}

\item KEYWORD VALUES.  These are combinations of keywords and
      arithmetic constants, and are used to pass values to VISTA
      subroutines, where needed.  Keyword values have the form
      'WORD=value': i.e., a keyword immediately followed by an equal sign,
      immediately followed by a constant.  The constant can be either
      explicit or symbolic, for example, these two command lines:
      \begin{verbatim}
         SHIFT 1 DR=2.3
         SHIFT 1 DR=INCR
      \end{verbatim}
      do the same thing, provided INCR has the value 2.3.  You can also use
      arithmetic expressions following the '=' sign, as in 
      \begin{verbatim}
         SHIFT 1 DR=INCR+3.2
      \end{verbatim}

\end{enumerate}

VISTA takes the command line, and splits it up into several lists: a list
of integers, a list of floating-point numbers, and a list of character
strings.  The programs in VISTA examine these lists to decide what to do.

Most commands check the input line to make sure that every word you give is
something the command knows about.  If you type an unknown command, you
will receive an error message, and the command that you are trying to
execute will not run.

More information about command syntax can be found in section 2 of this
manual.

\section{VISTA and Disk Files}
\begin{rawhtml}
<!-- linkto files.html -->
\end{rawhtml}

\index{Directories,Specifying directories for images, etc.}
\index{Procedure,Startup procedure}

Many VISTA commands read from or write to the disk, and therefore require
filenames. The program has defined several default directories and
extensions for the various types of files that VISTA uses: for example,
images are stored in one directory, spectra in a second, procedures in a
third, etc. Unless you specify otherwise, VISTA will automatically search
in certain directories for images, spectra, etc., and will assume that the
files have standard extensions.  You can view these default directories
with the VISTA command PRINT DIRECTORIES.

An example of a default directory is this: The WD command writes an image
to disk with a filename specified in the command.  Assume the default
directory for images is /vista/ccd/ and the default file extension for
images is .CCD.  The command
\begin{itemize}
   \item[WD 2 m15\hfill]{}
\end{itemize}
writes the image in buffer 2 to the file /vista/ccd/m15.fits You can
override the default locations any time: for example
\begin{itemize}
   \item[WD 2 /demo/m15\hfill]{writes to /demo/m15/m15.fits}
   \item[WD 2 m15.xyz\hfill]{writes to: /vista/ccd/m15.xyz}
   \item[WD 2 /demo/m15.xyz\hfill]{writes to: /demo/m15.xyz}
\end{itemize}

The default directories are established when the VISTA program is run by a
subroutine called INITIAL.  That subroutine has a standard list of
directories and extensions that it loads into a common block for use by
other subroutines. The default directories can be changed by the user in
two ways:

\begin{enumerate}
   \item Use the SETDIR command from within VISTA

   \item Every user can set up his/her own set of default directories using
         environment variables. The appropriate environment variables to
         modify are:\newline
\begin{tabular}{ll}
   Variable    & Directory for \\ \hline
   V\_CCDIR    & Images \\
   V\_PRODIR   & Procedures \\
   V\_SPECDIR  & Spectra \\
   V\_FLUXDIR  & Flux calibrations \\
   V\_LAMBDIR  & Wavelength calibrations \\
   V\_COLORDIR & TV color files \\
   V\_DATADIR  & Data files. \\
   V\_DAODIR   & DAOPHOT files. \\ \hline
\end{tabular}

An example is:
\begin{verbatim}
   setenv V\_CCDIR /demo/ccd/
\end{verbatim}
Remember, you must execute these setenv commands BEFORE starting VISTA.

\end{enumerate}

\section{Running VISTA}
\begin{rawhtml}
<!-- linkto startv.html -->
\end{rawhtml}

\index{VISTA,Running VISTA}

The VISTA program is generally stared with the command
\begin{itemize}
   \item[xvista\hfill]{}
\end{itemize}
assuming that it has been installed somewhere which is found in your
command path.

VISTA will respond with a 'welcome' message and a prompt. The welcome
message, if it is long, comes out a screen at a time.  Hit 'return' at the
end of each page to keep reading a message.  Hit anything else to go on.
The program will then give you a prompt ( GO: ). The prompt tells you that
the program is ready to execute a command. Type a command you want, and hit
RETURN.  When the command is completed, the prompt will reappear.  If you
type a command that VISTA does not recognize, it will say so.

VISTA is a rather large program, you may need access to a reasonably large
amount of physical and virtual memory.

On startup, VISTA can also be instructed to run a designated procedure.
This is a convenient feature which allows you to customize your VISTA
environment.  Use the environment variable V\_STARTUP to specify the name
of the file containing the procedure you want executed as the program
begins.  Typically this procedure will define aliases or set the values of
variables.  Read the section on procedures (HELP PROCEDURE) for more
information.

Some additional symbols which also can (or need to ) be defined in
environment variables for use on startup:
\begin{itemize}
   \item{V\_HELPDIR     (directory with helpfiles)}
   \item{V\_LATITUDE (latitude in degrees)}
   \item{V\_LONGITUDE (longitude in degrees)}
   \item{V\_SYSTEM (where to find dosystem file needed to spawn UNIX commands)}
\end{itemize}

\section{HELP: The VISTA Online Help Facility}
\begin{rawhtml}
<!-- linkto help.html -->
\end{rawhtml}

\begin{itemize}
   \item[\textbf{Form: } HELP {[subjects]} {[output redirection]}\hfill]{}
   \item[subjects]{are the subjects for which information is requested}
\end{itemize}

HELP is used to get information about commands and topics, or to produce a
hardcopy manual for reference.  A command is, of course, a process that can
be executed directly by VISTA.  A topic is a set of information about some
aspect of VISTA.  For example, this information appears under the command
'HELP', while introductory material about procedures is found under the
topic 'Procedures'.

The simplest form of the command is HELP with no arguments.  This form of
the command will produce on your terminal a list of the the major
categories of VISTA commands (e.g., INTRODUCTION, IMAGE PROCESSING,
SPECTROSCOPY, etc.)  The program then will display the names of the
commands under each category.  After that, you can get help on a particular
command or topic by typing its name -- the names are shown on the terminal.

To get information on a particular command or topic, type HELP followed by
the name of the command or topic.  If you want help on several subjects,
type them all on the command line.  The words should be separated with
blanks, as is usual with VISTA commands. Three examples are:
\begin{itemize}
   \item[HELP SUBTRACT \hfill]{information on the command SUBTRACT}
   \item[HELP ZAP FITSTAR \hfill]{information on ZAP, AND FITSTAR.}
   \item[HELP Photometry \hfill]{information on photometry.}
\end{itemize}

The first line of information about commands contains a line beginning with
'Form: ', which spells out the syntax of that command in detail.  Following
this are more detailed explanations of the workings of that command.  These
paragraphs form an example of an entry readable by the HELP command.
Important: Keywords listed in square brackets (as [ALL], above) are
OPTIONAL; they modify the operation of the command, if desired, but need
not be specified under all circumstances.

The output of this program can be redirected. (The output redirection
mechanism is described in the section 'REDIRECT' -- type HELP REDIRECT if
you are on a terminal.)  

New to Version 4.2 is the capability to run the HELP program standalone.
This is particularly useful in a workstation environment where you may wish
to have one window dedicated to the VISTA HELP. This program is named
vistahelp; check with your local VISTA manager for its location.


\noindent{Examples}
\begin{itemize}
   \item[HELP MASH\hfill]{Sends the information on 'MASH'
        to your terminal.}
   \item[HELP MASH $>$MASH.XXX\hfill]{Sends the information on 'MASH'
        to the file MASH.XXX}
\end{itemize}

If there are several commands or topics which begin with a pattern given on
the command line, the helpfiles for all those commands will be printed.
For example,
\begin{itemize}
   \item{HELP CO}
\end{itemize}
will list the help files for COPY, COPW, CONTINUE, CONTOUR, and any other
commands which begin with the letters 'CO'.  Exception: If there are
several commands which begin with the pattern you give, but the pattern
matches exactly the name of one command or topic, only the match will be
printed.  For example, HELP PRINT will give you information on PRINT only,
even though there is another command (PRINTF) which begins with the same
pattern.

The character '?' can be used as a abbreviation for the HELP command.  See
the section '?' in the manual (or type 'HELP ?' if you are on a terminal.)

\section{Useful Commands for the Beginner}
\begin{rawhtml}
<!-- linkto simpcmd.html -->
\end{rawhtml}

\index{Commands, Useful commands for the beginner}
\index{Introduction, Useful commands for the beginner}

Here are some commands often used in image processing.  Read these sections
of the HELP manual first:

\begin{itemize}
   \item[WD\hfill]{write an image to disk}
   \item[RD\hfill]{read an image from disk}
   \item[BUF\hfill]{show which images are in the buffers}
   \item[MN\hfill]{compute average of an image}
   \item[TV\hfill]{display an image on the television}
   \item[PLOT\hfill]{plot a row, column or spectrum}
   \item[CONTOUR\hfill]{produce a contour plot of an image}
   \item[MASH\hfill]{extract a spectrum from an image}
\end{itemize}

\section{QUIT: Stopping VISTA}
\begin{rawhtml}
<!-- linkto quit.html -->
\end{rawhtml}

\index{VISTA,Stop}
\index{Stopping VISTA}
\begin{itemize}
   \item[\textbf{Form: } QUIT\hfill]{}
\end{itemize}

This command stops VISTA, returning the user to command level.  This
command is executed immediately. The contents of all image buffers are
lost.

If you are running VISTA as part of a command procedure, make sure you put
QUIT as the last command line before the next command.

\section{VISTA News}
\begin{rawhtml}
<!-- linkto oldnews.html -->
\end{rawhtml}
\index{Help, List recent changes to program or other news}

\begin{itemize}
  \item[\textbf{Form: } NEWS\hfill]{}
\end{itemize}

NEWS prints the current message of the day, and all the accumulated
messages of the day during the recent past.  The output is displayed a
screen at a time.  At the end of each page, hit 'return' to continue, or
anything else to stop.

\section{TERM: Specify or Change the Graphics Terminal Type}
\begin{rawhtml}
<!-- linkto term.html -->
\end{rawhtml}
\index{Terminal, Specifying default graphics terminal type}
\index{Display, Specifying graphics terminal type}

\begin{itemize}
   \item[\textbf{Form: } TERM {[TERMINAL=vterm]} {[HARDCOPY=vhard]}\hfill]{}

   \item[(none)]{Prompt user for graphics display terminal type to use.
        Hardcopy devices must be changed explicitly using HARDCOPY= below.}

   \item[TERMINAL=vterm]{Makes default graphics display terminal for plots
        the device corresponding to LickMongo "vterm" device code}

   \item[HARDCOPY=vhard]{Makes default hardcopy device for plots the device
        corresponding to on of the VHARD device codes.  Typing HARDCOPY=0
        will give the user the hardcopy code menu and prompt for the
        desired option.}
\end{itemize}

The default VISTA graphics device is an X11 window that will be created
automatically when the first graphics commands are used.  VISTA uses the
LickMongo package for all vector graphics (line plots, contour maps, etc.)
and PostScript hardcopy.

If you are using VISTA from a terminal that does not support X11 graphics
(e.g., from an old VT100 terminal or over a modem from a PC running a
Tektronix emulator), you can change the default terminal type used
for screen graphics using the TERM command.

Issuing TERM without an argument will print a menu and prompt the
user for the device type.

TERM is called by the VISTA program on startup and sets the default
graphics device for the session.  This may be changed at any time
during a session with the TERM command, but the default is hardwired
into the program at compile time, and any changes will be forgotten
during the next VISTA session unless you put a TERM command with
the necessary keywords in your startup procedure file.

Hardcopy devices must be changed explicitly with the HARDCOPY= keyword.  If
no command line argument is given, then the user will be allowed to change
the terminal type only.

The default Hardcopy device for VISTA is Adobe PostScript files.  All PS
output from VISTA conforms to the PostScript standard and will produce
output on any PostScript device.  Encapsulated PS output is produced by
some commands (usually via an EPS keyword), and conforms to the Adobe EPS
Level 2.0 structuring conventions.


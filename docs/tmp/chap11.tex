\chapter{Basic Image Analysis}
\begin{rawhtml}
<!-- linkto imbasic.html -->
\end{rawhtml}

%
% All routines checked for keyword and function against current sources
% HISTOGRAM help block moved to Chapter 8 (display & plotting)
%
% rwp/osu 1998 Aug 13 & 25
%

VISTA Provides the following commands for doing basic statistical
analysis of pixel values:
\begin{itemize}
  \item[MN\hfill]{Compute the Mean of the Pixel Values}
  \item[SKY\hfill]{Compute the Modal Sky/Background Level in an Image}
  \item[ABX\hfill]{Analyze Pixel Statistics in an Image}
  \item[AXES\hfill]{Compute the Centroid of an Object in an Image}
\end{itemize}
See also the HISTOGRAM command.

\section{MN: Compute the Mean of the Pixel Values}
\begin{rawhtml}
<!-- linkto mn.html -->
\end{rawhtml}
\index{Image!Mean}
\begin{itemize}
  \item[\textbf{Form:}MN source {[NOBL]} {[BOX=b]} {[NOZERO]} {[MASK]}
       {[PIX=p]} {[SILENT]} {[W=w1,w2]}\hfill]{}
  \item[source]{the image or spectrum for which the mean
       is to be computed}
  \item[NOBL]{ignore baseline (last) column in an image}
  \item[BOX=b]{find the mean in box b}
  \item[NOZERO]{ignore pixels with value zero}
  \item[MASK]{ignore masked pixels}
  \item[PIX=p]{use every p'th pixel (for speed)}
  \item[W=w1,w2]{find the mean in a wavelength calibrated
       spectrum in the interval from wavelength w1 to w2.}
  \item[SILENT]{do not print the results of the calculation.}
\end{itemize}

The mean of all the pixels in the image contained in buffer 'source' is
computed.  Use MASK or NOZERO to ignore masked or zero-valued pixels in the
computation.  PIX=p computes the mean using every p'th pixel in every p'th
row, for speed.

The computed mean value is printed at the terminal (unless the SILENT word
is given) and is also loaded into the VISTA variable Mn, where 'n' is the
image buffer number. (ex: the command MN 1 loads the variable M1.  The
command MN 31 loads the variable MN31).  The value of the mean is also
loaded into the variable MEAN.

The mean is used by other commands such as the TV command for a default
display range, or the DIVIDE command for rescaling images after flat-field
divisions.

See MASK for the operation of the pixel mask.

\noindent{Examples:}
\begin{itemize}
  \item[MN 4\hfill]{finds the mean of buffer 4.  The
  value of the mean is printed on the terminal and is loaded into the 
  variables MEAN and M4.}
  \item[MN 2 PIX=5\hfill]{computes the mean using every 5th 
  pixel in every 5th row.  This makes the computation go very fast.}
  \item[MN 5 W=4000,5000\hfill]{finds the mean of the wavelength-
  calibrated spectrum in buffer 5 in the interval from 4000 to 5000 Angstroms.}
  \item[MN \$Q MASK\hfill]{finds the mean of the object in buffer
  Q (Q a variable), ignoring masked pixels.}
  \item[MN 4 BOX=6\hfill]{finds the mean of the object in
  buffer 4 using those pixels in box 6.}
\end{itemize}

\section{SKY: Compute the Modal Sky/Background Level in an Image}
\begin{rawhtml}
<!-- linkto sky.html -->
\end{rawhtml}
\index{Image!Background level}
\begin{itemize}
  \item[\textbf{Form:}SKY source {[BOX=n]} {[SILENT]} {[CORNERS]}
       {[MAX=c]}\hfill]{}
  \item[source]{is the number of the image that SKY works on.}
  \item[BOX=n]{tells SKY to work only box=n.}
  \item[SILENT]{do not print the result of the calculation.}
  \item[CORNERS]{tells SKY to use the corners of the image.}
  \item[MAX=c]{tells SKY to ignore pixels greater than 'c'.}
\end{itemize}

This routine finds the sky background level of an image under the
assumption that the most common pixel intensity in the image is the level
of the 'sky' or background.  This is a nonlinear algorithm which is largely
insensitive to bright objects in the image. The routine calculates the mean
of the image 'source', and builds a histogram of pixel intensities about
the mean. The region of the peak value is located in the histogram, and is
fit with a parabola to find its precise location. This center of the
parabola is defined to be the sky value, and is loaded into the VISTA
variable 'SKY' for access by other commands.  The width of the histogram is
determined by a simple linear interpolation.  The value loaded into the
VISTA variable 'SKYSIG' is a 1-sigma uncertainty estimate derived from the
full width half max, and assumes the histogram is Gaussian.

The use of a box may be helpful if a large fraction of the image is
occupied by stars or an extended object, causing the routine to measure a
sky level systematically higher than the true level.  The box can in this
case be used to select a region of the image that does not have bright
objects in it.  The CORNERS keyword tells SKY to find the sky value in the
four corners of the image.  Each corner area consists of a region whose
sides are one fourth the dimensions of the full image.  The minimum sky
value found in the four corners is assumed to be the true sky value.  The
MAX=c keyword tells SKY to ignore all pixel values greater than 'c'.  This
is normally not needed unless there are many bright stars in the field.

This version of SKY is new to Vista Version 4.2. It was supplied by Tod
Lauer, and has been tested to be more accurate. The old VISTA SKY routine
can still be accessed using the command OLDSKY.

\noindent{Examples:}
\begin{itemize}
  \item[SKY 5\hfill]{ finds the background in image 5, loading 
  its value into the variable SKY.}
  \item[SKY \$W BOX=5\hfill]{finds the sky level in image W
  (W a variable) in box 5.}
  \item[SKY 2 CORNERS SILENT\hfill]{finds the sky level in image 2 using
  only the corners of the image.  The value is loaded into SKY, but nothing
  is printed on the terminal.}
\end{itemize}

\section{ABX: Analyze Pixel Statistics in an Image}
\begin{rawhtml}
<!-- linkto abx.html -->
\end{rawhtml}
\index{Image!Statistics in selected regions}
\index{Box!Statistics of image or spectrum in}
\index{Image!High pixel}
\index{Image!Low pixel}
\index{Image!Mean}
\index{Image!Standard deviation}
\begin{itemize}
  \item[\textbf{Form:}ABX source {[boxes]} {[ALL]} {[W=w1,w2]} {[SILENT]} 
       {[MASK]}\hfill]{}
  \item{{[TOTAL=var]} {[MEAN=var]} {[HIGH=var]}}
  \item{{[LOW=var]} {[HIGH\_ROW=var]} {[HIGH\_COL=var]}}
  \item{{[LOW\_ROW=var]} {[LOW\_COL=var]} {[SIGMA=var]} (redirection)}
  \item{{[AREA=farea]} {[P=var]}}
  \item[source]{specifies the object.}
  \item[boxes or BOX=b1,b2...]{list boxes to be used in the analysis.}
  \item[ALL]{tells the program to analyze the entire image or spectrum.}
  \item[W=w1,w2]{limits the analysis to the wavelength interval w1 to w2 for 
       wavelength-calibrated spectra.}
  \item[SILENT]{do not print output.} 
  \item[MASK]{ignore masked pixels.}
  \item[AREA=farea]{Determine the pixel position where the total reaches
       the value farea.}
  \item[var]{the name of a variable.}
\end{itemize}

ABX analyzes the statistics of an image or spectrum.  It computes the total,
mean, and standard deviation about the mean.  It also finds the locations
and values of the highest and lowest pixels.  The output of this program
can be redirected.

To analyze the object in regions defined by boxes, give the NUMBERS of the
boxes on the command line.  If you give no numbers, the entire image will
be analyzed. Alternatively, specify any number of boxes using the
BOX=b1,b2,... keyword. Examples:
\begin{itemize}
  \item[ABX 3\hfill]{Finds the properties of object 3 in the entire image}
  \item[ABX 3 1\hfill]{Finds the properties of object 3 in box 1}
  \item[ABX 2 3 4 5 6\hfill]{Analyzes object 2 in boxes 3, 4, 5, and 6}
  \item[ABX 2 BOX=3,4,5,6 \hfill]{Analyzes object 2 in boxes 3, 4, 5, and 6}
\end{itemize}

Note that in ABX, boxes are specified by integers, and not by BOX= keyword,
as is usual in other commands.  See BOX to see how to define boxes.

To analyze the entire image or spectrum, use the keyword ALL or simply do
not specify a box number. Any box specifiers on the command line are
ignored.  To analyze part of a wavelength-calibrated spectrum, use W= to
specify the part in Angstroms.

ABX finds the image mean, total count over all pixels, values of the
highest and lowest pixels, location of the highest and lowest pixels, and
the standard deviation of the counts in the pixels about the mean. A table
of the results is printed on the output device.

ABX will store the values for the various properties of the image in
variables if certain keywords are included on the command line:
\begin{itemize}
  \item[TOTAL=var\hfill]{Stores the total count of all the pixels in 'var'}
  \item[MEAN=var\hfill]{Stores the average of the image}
  \item[HIGH=var\hfill]{Stores the VALUE of the pixel with the highest count}
  \item[LOW=var\hfill]{Stores the VALUE of the pixel with the lowest count}
  \item[HIGH\_ROW=var\hfill]{Stores the row number in which the highest-valued
       pixel is located}
  \item[HIGH\_COL=var\hfill]{Stores the column number in which the 
       highest-valued pixel is located}
  \item[LOW\_ROW=var\hfill]{Stores the row number in which the lowest-valued
       pixel is located.}
  \item[LOW\_COL=var\hfill]{Stores the column number in which the lowest-valued
       pixel is located.}
  \item[SIGMA=var\hfill]{Stores the standard deviation of the pixel values
       about the mean.}
  \item[P=var \hfill]{ Stores the pixel where the total reaches farea
       (using AREA=farea) keyword.}
\end{itemize}

\noindent{Examples of storing information in variables are:}
\begin{itemize}
  \item[ABX 1 3 MEAN=M3 SIGMA=SIG3\hfill]{Analyzes image 1 in box 3, storing
       the mean in variable M3 and the standard deviation in SIG3.}
  \item[ABX 2 7 HIGH\_ROW=HR HIGH\_COL=HC\hfill]{ Analyzes image 2 in box 7, 
       storing the location of the highest-valued pixel in HR and HC}
\end{itemize}
The use of the variable-setting keywords should be used only when you
analyze an image one box at a time, as the values will be loaded into the
variables only for the last box analyzed.

SILENT is used to prevent printing of the output.  This is helpful in
procedures when variables are set.

\section{HISTOGRAM: Display a Histogram of Image Pixel Values}
\begin{rawhtml}
<!-- linkto histogram.html -->
\end{rawhtml}
\begin{itemize}
  \item[\textbf{Form:}   HISTOGRAM source {[BOX=b]} {[NOLOG]} {[BIN=n]} {[XMIN=x1]} {[XMAX=x2]}\hfill]{}
  \item{{[YMIN=y1]} {[YMAX=y2]} {[HARD]} {[WIND=n]} {[BUF=buf]}}
  \item{{[NOERASE]} {[PORT]}}
  \item[source\hfill]{   specifies the image.}
  \item[BOX=b\hfill]{limits the computation to those pixels in box 'b'.}
  \item[NOLOG\hfill]{displays the number of pixels at each intensity,
rather than the logarithm.}
  \item[BIN=n\hfill]{   bins the image values by the specified factor.}
  \item[XMIN, XMAX\hfill]{  limits the computation to those pixels with values
between x1 and x2, inclusive.}
  \item[YMIN, YMAX   \hfill]{   limits the display of the histogram on the Y-axis
to be from y1 to y2.}
  \item[HARD\hfill]{sends the plot to the hardcopy device.}
  \item[WIND=n\hfill]{put the plot in window n of a 2x2 grid}
  \item[BUF=buf\hfill]{  load the histogram data into image buffer 'buf'}
  \item[   NOERASE   \hfill]{don't erase screen before plotting.}
  \item[PORT\hfill]{make hardcopy output in portrait mode (default: landscape)}
\end{itemize}

This program displays a histogram of an image, plotting the logarithm of
the number of pixels at each value for the image 'source'.

Use the BIN word to specify how wide the intensity intervals show in the
plot is to be.  Normally the binning factor is 1, meaning that the plot
displayed is the logarithm of the number of pixel values at each intensity
(the image values are converted to integers).  If the bin factor is
non-zero, the display is the log of the number in larger bins.  For
example, if the bin was 5, and the minimum value in the image is 0, then
the plot shows the number of pixels with intensity 0 - 4, 5 - 9, 10 - 14,
15 - 19, etc. If there is a large range in intensities, the BIN word should
be used to keep the plot from having so many points that it looks like
hash. If the number of points in the histogram is larger than 2048, the
program will increase the bin size to reduce the number of points in the
plot.

The BOX word limits the calculation to those pixels in the specified box.
The XMIN and XMAX words limit the calculation to those pixels in the
specified intensity range.  If XMIN is not given, the lower limit will be
the minimum pixel value in the image.  If XMAX is not given, the upper
limit will be the maximum pixel value in the image.

The YMIN and YMAX words, by contrast, limit the DISPLAY of the histogram so
that the Y-axis runs over the given range.  These words do not change the
calculation in any way.  If YMIN is not used, the lower limit of the
display will be the smallest number of pixels in the image that have a
given value (often this is zero pixels at many intensities).  If YMAX is
not used, the upper limit of the display will be the largest number of
pixels which have a certain intensity.
       
NOLOG makes the plot show the actual number of pixels at each intensity,
rather than the logarithm.  When the logarithm is computed in the default
option, intensities with no pixels are given the value 0, so you cannot
distinguish an intensity with 1 pixel and an intensity with 0 pixels unless
you use the NOLOG word.

The NOERASE keyword suppresses the erasure of the screen before plotting.
Up to 4 histograms may be plotted on a single output page using the WIND=n
keyword, where n=(1,2,3,4), with windows numbered from left-to-right,
bottom-to-top in the conventional MONGO order.  Hardcopy is flushed to the
printer only after WIND=4 is used.  The plot may be printed in portrait
mode (long axis of the paper vertical) using the PORT keyword.

The histogram for an image may be loaded into an image buffer as a
"spectrum" (1-D image) for further analysis or storage for use with
external programs using the BUF=buf keyword.

\noindent{Examples:}
\begin{itemize}
  \item[HISTOGRAM 4 \hfill]{Shows the histogram for image 4}
  \item[HISTOGRAM \$Q BOX=3 \hfill]{Shows the histogram for image Q 
       (where Q is a variable) using only the pixels in box 3}
  \item[HISTOGRAM 2 XMIN=1000 XMAX=1999 \hfill]{
       Shows the log of number of pixels with values between 1000 and 1999.}
  \item[HISTOGRAM 4 NOLOG \hfill]{
       Shows the number of pixels (not the log) at each value in image 4.}
  \item[HISTOGRAM 3 HARD\hfill]{
       computes a histogram for image 3, sending it to hardcopy printer.}
  \item[HISTOGRAM 2 XMIN=1000 XMAX=1999 BUF=10\hfill]{ same as \#3 above,
       except that instead of plotting the histogram, it is loaded into image
       buffer 10 as a spectrum.}
\end{itemize}

\section{AXES: Compute the Centroid of an Object in an Image}
\begin{rawhtml}
<!-- linkto axes.html -->
\end{rawhtml}
\index{Image!Centroid of object in}
\index{Centroid!Finding centroid of object in.}
\begin{itemize}
  \item[\textbf{Form:}AXES source {[BOX=n]} {[SKY=s]} {[W=w1,w2]} (redirection) {[SILENT]}\hfill]{}
  \item[source\hfill]{is the image that AXES uses.}
  \item[BOX=n\hfill]{specifies the section of the image used.}
  \item[SKY=s\hfill]{specifies the sky value used in the calculation.}
  \item[W=\hfill]{finds the centroid in a specified interval
of a wavelength-calibrated spectrum (for use
with emission spectra, NOT absorption spectra.)}
  \item[SILENT \hfill]{suppresses terminal output}
\end{itemize}

This command will find the centroid of an object within the specified box
in the source image, or in the entire image if no box specifier is
given. The centroid coordinates are loaded into VISTA variables AXR and
AXC, and are also held in a common block for use by other routines.  If the
W= keyword is used, the wavelength of the centroid of the spectrum in the
given interval is loaded into the variable WAVE.

The routine uses the highest pixel value on the box perimeter as a
threshold value for the centroid, unless the SKY word is given, in which
case the argument of the SKY word is taken as the threshold.  The threshold
is subtracted from each pixel in the box during the calculation.

The output of AXES can be redirected.

\noindent{Examples:}
\begin{itemize}
  \item[AXES 3\hfill]{finds the centroid of the object in buffer 3.}
  \item[AXES 4 BOX=5\hfill]{finds the centroid of the part of
       object 4 that is in box 5.}
  \item[AXES 1 W=6550,6800\hfill]{finds the centroid of the part of
       the wavelength calibrated spectrum that lies between wavelengths 6550
       and 6800 Angstroms.}
  \item[AXES 3 >AXES.OUT\hfill]{does the same as the first example,
       but prints the result of the calculation on the file AXES.OUT, 
       instead of on the terminal.}
  \item[AXES 3 SILENT\hfill]{does the same as the first example,
       but does not print output.}
\end{itemize}

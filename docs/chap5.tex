\chapter{Flow Control in Procedures}

% minor editing and reformatting
% used verbatim for most script examples
% rwp/osu 98Jul26

\section{PAUSE: Pause Procedure Execution}
\begin{rawhtml}
<!-- linkto pause.html -->
\end{rawhtml}

\index{Procedure!Pausing during}
\index{Pausing!During procedure}
\index{Control-C!To halt procedure}
\begin{command}
\item[\textbf{Form: } PAUSE 'prompt message'\hfill]{}
\item[ctrl-C\hfill]{}
\end{command}

When the PAUSE command is encountered in a procedure, VISTA prints the
prompt 'PAUSE ' followed by the rest of the (optional) prompt which appears
on the command line.  The execution of the procedure is then stopped.
While paused, you can enter any commands in the normal immediate mode of
execution.  To resume the procedure where it paused type the command
CONTINUE.  You do not need to type a command to permanently halt the
procedure.  If you give the command GO any previous pause state will be
canceled, and VISTA will start the program from the beginning.

WARNING: ALWAYS place a PAUSE instruction on its own line in a procedure.
A procedure which looks like
\begin{verbatim}
  command 1
  command 2; PAUSE; command 3
  command 4
\end{verbatim}
will pause properly, but CONTINUE will start the procedure with 
command 4, not command 3.

\section{CONTINUE: Resume a Paused Procedure}
\begin{rawhtml}
<!-- linkto continue.html -->
\end{rawhtml}

\begin{command}
  \item[\textbf{Form: } CONTINUE\hfill]{}
\end{command}
Re-starts a procedure stopped by PAUSE. See PAUSE for more
information. 

\section{GOTO: Jump to a Specified Place in a Procedure}
\begin{rawhtml}
<!-- linkto goto.html -->
\end{rawhtml}

\index{Procedure!Jump to specified line}
\index{Jump!In procedures}
\begin{command}
  \item[\textbf{Form: } GOTO label\_name\hfill]{}
  \item[label\_name]{is a label defined somewhere else in the procedure.}
\end{command}
GOTO tells VISTA to jump to a line in the procedure buffer beginning with
``label\_name:''.  The label\_name can be any alphanumeric string.  You can
jump out of, but not into, a procedure DO loop or IF block.  In the latter
cases, VISTA will trigger an error condition and abort the procedure.  The
jump can be in either direction in the procedure buffer (either before or
after the GOTO command in the procedure).

The jump-point label may be any valid string terminated with a ':' (color)
immediately following the label.  No other commands can appear following
the label on the same line.  Note that the ':' is omitted in the GOTO
statement.

\noindent{Example 1:}
\begin{verbatim}
   GOTO WHEREVER
     Any number of procedure lines...
   WHEREVER:
   The next commands to be executed...
\end{verbatim}
This jumps forward over the intervening lines to the label ``WHEREVER:''.

\noindent{Example 2:}
\begin{verbatim}
   NOWHERE:
     Any number of procedure lines...
   GOTO NOWHERE
\end{verbatim}
This exectues the lines, then jumps backwards to the label NOWHERE: and
executes them again.

\section{DO/END\_DO: Using Do-Loops in Procedures}
\begin{rawhtml}
<!-- linkto do.html -->
\end{rawhtml}

\index{Procedure!DO loops}
\index{Procedure!Repeating segments of}
\begin{command}
  \item[\textbf{Form: } DO var=N1,N2,{[N3]}\hfill]{}
  \item[\textbf{Form: } \{any vista commands\}\hfill]{}
  \item[\textbf{Form: } END\_DO\hfill]{}
  \item[var]{is a variable name,}
  \item[N1]{is the initial value of the variable,}
  \item[N2]{is the final value,}
  \item[N3]{is the increment by which N1 is adjusted in
       each pass through the DO loop.}
\end{command}

The DO/END\_DO commands enable you set set up repeatable groups of commands
within the procedure buffer.  The VISTA DO-LOOP is very similar to the
Fortran DO-LOOP.

The variable 'var' is initially set equal to the starting value N1.  When
the END\_DO statement is encountered the value is changed by an amount
equal to N3. The value of N3 can be either positive or negative. If N3 is
positive then the looping terminates when N1 becomes greater than N2. If N3
is negative then looping terminates when N1 becomes less than N2.  If N3 is
not specified then it defaults to +1.0 if N2 is greater than N1 or to -1.0
if N2 is less than N1.

N1, N2, and N3 can all be arithmetic expressions as described in the SET
command.  The value of 'var' can be changed within the loop without
affecting the do-loop operation.  However, VISTA will reset 'var' to its
appropriate loop value at the beginning of each loop.

The underline is required in END\_DO because VISTA requires commands to be
one word long.  Up to 20 do-loops can be nested.  Do-loops are recognized
only within procedures. The GOTO command can be used to jump out of a DO
loop, but VISTA will not permit jumping into one. Further, DO loops must
contain or be contained completely within any IF blocks.

\noindent{Examples:}
\begin{enumerate}
  \item{
  \begin{verbatim}
     DO I=1,3
        Any number of procedure lines.  These lines are executed 3 times.
     END_DO
  \end{verbatim}
}
  \item{
  \begin{verbatim}
     DO Q=1,N
        Any number of procedure lines.  These lines are executed N times.
        Here N is a variable that has had its value set by the SET command.
     END_DO
  \end{verbatim}
}
  \item{
  \begin{verbatim}
     DO B=D+I,N-J,-1
        Any number of procedure lines.  The counter
        decrements from D+I to N-J.
     END_DO
  \end{verbatim}
}
\end{enumerate}

\section{IF, ELSE\_IF, ELSE, END\_IF: Conditional (IF) Flow Control}
\begin{rawhtml}
<!-- linkto if.html -->
\end{rawhtml}

\index{Procedure!Conditional branching}

VISTA procedures allow testing of variables and branching based on the
results of those tests.  This capability greatly expands the usefulness of
procedures.

The simplest use of IF is to mark a section of a procedure that is executed
only if come condition is true.  It has the form:
\begin{verbatim}
   IF condition
      Procedure lines (any number) that are executed if the
      specified condition is met.
   END_IF
\end{verbatim}

You can also have two level branching:
\begin{verbatim}
   IF condition
      Procedure lines to be executed if the condition is true.
   ELSE
      Procedure lines to be executed if the condition is false.
   END_IF
\end{verbatim}

These may be strung together:
\begin{verbatim}
   IF condition_1
      Procedure lines to be executed if condition_1 is true.
   ELSE_IF condition_2
      Procedure lines to be executed when condition_1 is
      false and condition_2 true.
      ...
   ELSE_IF condition_N
      Procedure lines to be executed when all conditions
      are false except condition_N.
   ELSE
      Procedure lines to be executed if and only if
      all other conditions are false.
   END_IF
\end{verbatim}

The conditions tested by the IF and ELSE\_IF statements are really just
VISTA arithmetic expressions.  An expression is considered to be true if it
evaluates to be non-zero, and false otherwise.  VISTA arithmetic supports
various logical operators whose value is either 1 or 0 depending on whether
the logical test is true or false.  The logical operators are the
following, where A and B can represent single VISTA variables or algebraic
expressions.
\begin{example}
  \item[IF A$>$B\hfill]{Test A greater than B}
  \item[IF A$>$=B\hfill]{Test A greater than or equal to B}
  \item[IF A==B\hfill]{Test A equal to B}
  \item[IF A~=B\hfill]{Test A not equal to B}
  \item[IF A$<$=B\hfill]{Test A less than or equal to B}
  \item[IF A$<$B\hfill]{Test A less than B}
\end{example}

There are two logical conjunctions \& (and) and | (or) which can be used to
join several of the above tests. Examples of the conjunctions are below:
\begin{example}
  \item[IF (A$>$B)\&(A==C)\hfill]{Test A $>$ B and A = C}
  \item[IF ((A==B)|(C$<$D))\&(C==B)\hfill]{Test (A = B or C $<$ D) and C = B}
\end{example}

The syntax of the IF statements is designed to look similar to the
FORTRAN-77 IF block structures.  Each IF block must begin with an IF
command and end with the END\_IF command.  An algebraic statement to be
tested must follow the IF on the same line.  If the relation is true, then
the procedure commands following the IF command are executed. If the
relation is false, VISTA looks for any ELSE\_IF tests, any final ELSE
statement, or jumps to the procedure lines following the END\_IF statement.

The ELSE\_IF command also must have a condition to be tested on the same
line.  ELSE\_IF's are optional, but permit you to test other conditions and
execute other blocks of the procedure buffer in the event that the initial
IF or any preceding ELSE\_IF's are false.  In this way you can allow VISTA
to 'trickle down' through several tests looking for one that is true.

The ELSE statement is also optional and marks a set of procedure lines for
VISTA to execute if and only if the initial IF and any following ELSE\_IF's
all test out false. Basically, the IF, any ELSE\_IF's, or any ELSE
statements all mark out various blocks of the procedure to be executed
under different conditions.  After the execution of any block, VISTA
transfers control to the procedure lines following the END\_IF statement.

IF blocks can be nested within other IF blocks up to 15 levels deep.  IF
blocks can be jumped out of, but not into, by the GOTO command. IF blocks
must contain or be contained within DO loops completely. Some examples of
IF blocks are given below:

\begin{enumerate}
  \item Simple Single Test:
  \begin{verbatim}
    IF X>Y
       Do these procedure lines if X is greater than Y
    END_IF
  \end{verbatim}

  \item IF/ELSE Test:
  \begin{verbatim}
    IF (X>Y)&(X<Z)
       Do these procedure lines if X is greater than Y but less than Z.
    ELSE
       Otherwise jump to these procedure lines.
    END_IF
  \end{verbatim}

  \item IF/ELSE\_IF Test:
  \begin{verbatim}
    IF SKY-LIMIT>BACKGRND
       Do these procedure lines if IF test true.
    ELSE_IF BACKGRND==0
       Do these procedure lines if IF test is false, 
       but the ELSE_IF condition is true.
    END_IF
  \end{verbatim}

  \item Test on 1 or 0 for an expression:
  \begin{verbatim}
    IF IMAGE-1
      Do these procedure lines if IMAGE is not equal
      to 1 (which would make the expression evaluate to 0)
    END_IF
  \end{verbatim}

\end{enumerate}

\section{ERROR: Execute On Error}
\begin{rawhtml}
<!-- linkto error.html -->
\end{rawhtml}

\index{Procedure!Execute on error}
\begin{command}
  \item[\textbf{Form: } ERROR  VISTA\_command\hfill]{}
  \item[VISTA\_command]{is any valid VISTA command.}
\end{command}
ERROR tells VISTA to execute the given VISTA command whenever an execution
error occurs.  The command can be any VISTA command but is quite often a
GOTO command.  VISTA\_command can not contain multiple commands separated
by semicolons.

\begin{enumerate}
  \item{
\begin{verbatim}
  ERROR GOTO WHEREVER
  Any number of procedure lines...
  WHEREVER:
  The next commands to be executed after an error...
\end{verbatim}
}
  \item{
\begin{verbatim}
  ERROR ERRFLAG=1
  Sets the VISTA variable ERRFLAG to 1 upon an error.
\end{verbatim}
}
\end{enumerate}

\section{EOF: Execute On End-Of-File (EOF)}
\begin{rawhtml}
<!-- linkto eof.html -->
\end{rawhtml}

\index{Procedure!Execute on end of file}
\index{ASCII files!Execute on end of file}
\begin{command}
  \item[\textbf{Form: } EOF  VISTA\_command\hfill]{}
  \item[VISTA\_command]{is any valid VISTA command.}
\end{command}
EOF tells VISTA to execute the given VISTA command whenever an end of file
is encountered in an OPEN'ed ASCII data file. See the OPEN and READ
commands for information on how to use ASCII data files. The command can be
any VISTA command but is quite often a GOTO command.  VISTA\_command can
not contain multiple commands separated by semicolons.

\begin{enumerate}
  \item{
\begin{verbatim}
  EOF GOTO WHEREVER
  Any number of procedure lines...
  WHEREVER:
  The next commands to be executed after an end-of-file.
\end{verbatim}
}
  \item{
\begin{verbatim}
  EOF EOF_FLAG=1
  Sets the VISTA variable EOF_FLAG to 1 upon an end-of-file.
\end{verbatim}
}
\end{enumerate}

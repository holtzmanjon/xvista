\chapter{Image Display, Plotting, and Interaction}
\begin{rawhtml}
<!-- linkto display.html -->
\end{rawhtml}

%
% Help block for HISTOGRAM moved here from Chapter 11
% Added TEXT help pages
% All commands checked for keyword lists and functionality against
% the most current source code
%
% rwp/osu 1998Aug10
%

VISTA provides a number of commands for the graphical display of data.
These include image display programs like TV, and a range of plotting
programs:
\begin{example} 
  \item[TV\hfill]{Load and image into the display}
  \item[IMPOST\hfill]{Make a PostScript hardcopy of an image}
  \item[TVRPLOT\hfill]{Make a Radial Intensity Plot of an object on the 
       TV Display}
  \item[TVBOX\hfill]{Overlay a VISTA box on the image display.}
  \item[TVPLOT \hfill]{Draw stuff like lines, boxes, text, and so on onto the 
       TV display}
  \item[PLOT\hfill]{Plot a row, column or spectrum}
  \item[RPLOT\hfill]{Make a radial intensity plot of an object in an image}
  \item[CONTOUR\hfill]{Make a contour plot of an image.}
  \item[PLOT3D\hfill]{Make a 3-d perspective (mesh) plot}
  \item[OVERLAY\hfill]{Draw contours over a greyscale image (PostScript)}
  \item[HISTOGRAM\hfill]{Display a Histogram of Image Pixel Values}
  \item[TEXT\hfill]{Embed Permanent Text on an Image}
\end{example} 
Each of these is described in separate help files below.  VISTA makes use
of native X11 graphics, the LickMongo plotting package, and PostScript for
all of its graphics.

\section{Introduction to the X11 Image Display}
\index{Display!X11 Image Display Window}
\index{Display!Overview}
\index{Display!Interactive commands}
\index{Display!X11 Resource Variables}
\index{Display!Marking pixels}
\index{Display!Scroll color map}
\index{Display!Zooming and panning}
\index{Color map!Defined}
\begin{rawhtml}
<!-- linkto x11.html -->
\end{rawhtml}

VISTA provides a set of basic image display and interaction commands for
X11 windows.  The following commands setup and communicate with the display
window.  For more information see their individual help files.
\begin{example} 
  \item[TV\hfill]{Load and image into the display}
  \item[COLOR\hfill]{Reload the color map or define a new one.}
  \item[ITV\hfill]{Synchronize procedures and display interaction, mark
       features.}
  \item[TVBOX\hfill]{Overlay a VISTA box on the image display.}
  \item[TVPLOT \hfill]{Draw stuff like lines, boxes, text, and so on onto the 
       TV display}
\end{example} 
A number of other commands (e.g., MARKSTAR) will also interact with the
display for specific purposes, such as measuring image profiles.

The first time the TV command is given, VISTA opens an Image Display
window on the console, and loads a color map as well as displaying the
image. A "Zoom" window is also opened that shows a magnified section of
the image under the current cursor location (a kind of "magnifying
glass").
 
Once an image has been displayed, the Display window will accept
interaction asynchronously of commands in the VISTA command window,
provided that a wait for input, or any other I/O is not pending.  To
interact with the image, simply move the mouse onto the display window.
The current pixel location of the cursor will be displayed in a frame at
the base of the image display along with the pixel intensity.  The arrow
keys are used for find control (one pixel at a time) of the cursor
position.
 
The following mouse buttons and keyboard keys are active while the mouse
is located on the image display:
 
\begin{center}
{\bf Mouse Buttons}\\
\begin{tabular}{ll}
\hline
Button & Function\\
\hline
LEFT  &ZOOM IN, centered on the cursor\\
MIDDLE&ZOOM OUT, centered on the cursor\\
RIGHT &PAN, move the pixel under the cursor to the center\\
\hline
\end{tabular}
\end{center}
 
\begin{center}
{\bf Keyboard Commands}\\
\begin{tabular}{cl}
\hline
Key & Function\\
\hline
 R &RESTORE image to the original zoom/pan\\
 + &BLINK Forwards through the last 4 images.\\
 - &BLINK Backwards through the last 4 images.\\
 P &Find the PEAK pixel near the cursor \& jump the cursor there\\
 V &Find the LOWEST pixel ("Valley") near the cursor \& jump the cursor there\\
 \# &"Power Zoom" zoom at the cursor to the maximum zoom factor\\
 H &Toggle between small and full-screen cross-hairs\\
 F &FREEZE/UNFREEZE the Zoom window.\\
0-9&Mark the pixel coordinates and store in VISTA\\
   &variables Rn and Cn, where "n" is the number key hit\\
 ] &Clear boxes and stuff off the image display\\
\hline
\end{tabular}
\end{center}

Other commands may use additional keys, but this set is always available.

The ITV command is still useful in procedures for making VISTA wait for
pixels to marked on the screen.  If ITV or any other VISTA routine is
waiting for screen input, the the INPUT status light is lit (which
otherwise says ASYNC).  ITV can also be used to print out pixel values in
the VISTA window or for marking up features on the display with boxes or
circles.

When VISTA is CPU bound (busy doing an intensive calculation), the display
will be put into a wait state before any interactive commands to the
display or ``damage'' from other windows can be serviced.  A note for
programmers: CPU intensive routines should have the LOOPDISP call inserted
into them to permit occasional servicing of display window events.  See the
VISTA Programmer's Guide for details.

\noindent{\bf Color Bar Adjustment:}

If you place the mouse on the color bar, these commands are available
to adjust the contrast of the image:
\begin{example}
  \item[LOW CONTRAST]{Hold down the LEFT Mouse button, drag the left
       end of the color bar.}

  \item[HIGH CONTRAST]{Hold down the RIGHT Mouse button, drag the right
       end of the color bar.}

  \item[ROLL COLOR MAP]{Hold the MIDDLE Mouse button, "roll" the
       color bar left or right.}
\end{example} 
The position of the mouse cursor displays the range of intensities
represented by that color.
 
Pressing the R key while the mouse is on the color bar restores the
original color map (undoing any change of the contrast or "roll" changes
made with the mouse buttons).
 
\noindent{\bf X-Windows Resources:}
 
The properties of the TV display window are defined by Resource variables
that can be customized by making appropriate entries in your .Xdefaults
file.  The resource variables are of the form:
\begin{verbatim}
   xvista.variable:        value
\end{verbatim}
where "xvista" is the resource name.

\begin{center}
{\bf Vista Display Resource Variables}\\
\begin{tabular}{lll}
\hline
Resource & Description & Default \\
\hline
xvista.maxwidth   & Max width of displayed images      &1024 \\
xvista.maxheight  & Max height of displayed images     & 865 \\
xvista.resize     & Auto-resize the window on display  &   1 \\
                  & of large images up to the max size.&\\
                  & 1=auto-resize [default]            &\\
                  & 0=no auto-resize                   &\\
xvista.autozoomout& Automatically zoom out images larger& 0\\
                  & than the max size so that the whole &\\
                  & image is visible.                   &\\
                  & 0 = no zoom out [default]           &\\
                  & 1 = auto zoom out                   &\\
xvista.zoomsample & Selects the de-zoom algorithm        &1 \\
                  & 1 = fast pixel-sampling [default]   &\\
                  & 0 = slower pixel-binning            &\\
\hline
\end{tabular}
\end{center}
 
The following are recommended values to place in your .Xdefaults file:
\begin{verbatim}
   xvista.maxwidth:        800
   xvista.maxheight:       800
   xvista.resize:            0
   xvista.autozoomout:       1
   xvista.zoomsample         1
\end{verbatim}
These will keep the TV window from obscuring too much of the screen, and
enable the fastest auto zoom mode.
 
Note that for changes to the .Xdefaults file to take effect, you either
have to log out and log back in again (brute force), or load the changes to
your .Xdefaults file with the Unix "xrdb -load" command.
 
\section{TV: Display an Image}
\index{Television!Load image in}
\index{Image!Display in TV}
\index{Blink comparison!Loading the images}
\index{Television!Loading color map}
\begin{rawhtml}
<!-- linkto tv.html -->
\end{rawhtml}
\begin{command}
  \item[\textbf{Form: }TV buf {[span]} {[zero]} {[L=span]} {[Z=zero]} 
       {[BOX=n]} {[CF=xxx]} {[NOERASE]} {[BW]}\hfill]{}
  \item[{[CLIP]} {[FLIP]} {[OLD]} {[NCOLOR=]}\hfill]{}
  \item[buf]{is the image to be displayed,}
  \item[span or L=]{set the span level for the color map,}
  \item[zero or Z=]{set the zero level,}
  \item[OLD]{use zero and span levels from previous display}
  \item[BOX=n]{displays the part of the image in box 'n',}
  \item[CF]{specifies the color map,}
  \item[NOERASE]{prevents the TV from erasing the previous image
       when displaying a new one (used for blinking),}
  \item[BW]{displays the image in black and white.}
  \item[CLIP]{prevents roll-over of the color map.}
  \item[FLIP]{display origin in lower left rather than upper left}
  \item[NCOLOR=]{limit the number of colors displayed (X11)}
\end{command}

For more general TV help, type HELP X11.

TV displays the image in 'buf' on the X11 image display.  Only those pixels
with data numbers between Z and Z+L are displayed.  Data values greater
than the maximum are all shown as one color, although the interactive
cursor will still be able to display the data associated with the "clipped"
pixels.
 
If no intensity scaling is given on the command line, TV will assume Z=0
and L=3 times the image mean intensity.  If the mean has not been
calculated, TV will do a quick calculation using every 9th pixel.  A peak
intensity of 4x the mean is a reasonable values to use for many
astronomical images with a large dynamic range.  It is not so good,
however, for viewing flat fields, skies, or dark images that have a narrow
range of data values.  In the latter case, you should specify the display
levels explicitly on the command line.
 
BOX=n tells TV to only display that part of the image within box 'n'
(see BOX command).
 
OLD tells TV to use the same display limits as the previous image.
 
CF=filename loads the color map ("look-up table") used to translate
pixel values into colors.  'BW' loads a black and white color map.  Once
the map is loaded, it remains in effect until changed by either the
COLOR or subsequent TV commands.
 
The basic set of color maps is:
\begin{example}
  \item[CF=BW\hfill]{black and white}
  \item[CF=IBW\hfill]{inverse black and white}
  \item[CF=RAIN\hfill]{the color distribution in a rainbow}
  \item[CF=WRMB\hfill]{thought by many to be the most useful}
\end{example}
More are available, see the COLOR command for specifics.

The NCOLOR= keyword can be used to limit the number of colors displayed in
X windows.  By default, 256 colors are displayed.  A limited color map is
useful when VISTA is being run at the same time as other color-hungry
applications (e.g., SAOimage, or web browsers like Netscape).  If VISTA
asks for too many colors, the X server can refuse to open the image
display.  An alternative is to shut down the unneeded applications when
running VISTA.

The TV display uses a 4-place "ring buffer" to hold the 4 most recently
displayed images in memory.  You can page through these images using the
$+$ and $-$ keys when the mouse is on the TV display window.  Blinking may
be accomplished by displaying two images sequentially, and then using the
$+/-$ keys to blink between them.
 
If the blink buffers are getting filled with a confusing mix of images, the
CLEAR IMAGES command may be used to clean them out (except for the
currently displayed image) and allow you to start over again.
 
The default image coordinates are as follows:
\begin{verbatim} 
  (SC,SR)
        +-------------+
        |             |
     |  |             |
     |  |             |
     R  |             |
     |  |             |
     V  |             |
        |             |
        +-------------+
            ---C-->   (EC,ER)
\end{verbatim}

The origin is at the upper left-hand corner, with columns increasing to the
left and rows increasing from top to bottom.  This is the "natural" order
in which to directly map images from memory onto the display, and allows us
to display images as fast as possible without remapping.
 
The FLIP keyword is provided to invert the rows axis so as to increase
vertically from bottom to top, i.e., in ``Cartesian'' coordinate order.
This will give you an orientation that is more ``natural'' from the user's
point of view, but which incurs some performance penalty.
 
The properties of the display (size, whether or not to auto-dezoom, etc.)
can be customized using X resources variables set in the user's .Xdefaults
file.  See HELP X11 for an overview.
 
Whenever an image is in the display, a set of basic cursor keys are
available to provide simple interactive capabilities.  See HELP X11 for
details.
 
\noindent{Examples:}
\begin{enumerate}
  \item{
\begin{example}
  \item[MN 1\hfill]{}
  \item[TV 1 CF=RAIN\hfill]{loads image 1 into the display, setting the
       span to be 4 times the image mean.}
\end{example}
}
  \item{
\begin{example}
  \item[TV 3 100. 30.0\hfill]{loads image 3 into the display, setting the
       span to be 100 and the zero to be 30.}
\end{example}
}
  \item{
\begin{example}
  \item[TV 3 L=100. Z=30.\hfill]{does the same thing as example 2.}
\end{example}
}
  \item{
\begin{example}
  \item[TV 3 BOX=1\hfill]{Displays that part of the image
       which is in box 1.}
\end{example}
}
\end{enumerate}

\section{ITV: Interact with the Image on the TV}
\index{Television!Interacting with image in}
\index{Pixel!Examining value}
\begin{rawhtml}
<!-- linkto itv.html -->
\end{rawhtml}
\begin{command}
  \item[\textbf{Form: }ITV\hfill]{}
\end{command}

ITV enables you to interactively examine an image displayed on the video
display.  The mouse buttons and several keyboard keys are activated to
perform functions when pressed:

\begin{center}
{\bf Mouse Buttons}\\
\begin{tabular}{ll}
\hline
Button & Function\\
\hline
LEFT  &ZOOM IN, centered on the cursor\\
MIDDLE&ZOOM OUT, centered on the cursor\\
RIGHT &PAN, move the pixel under the cursor to the center\\
\hline
\end{tabular}
\end{center}

\begin{center}
{\bf Keyboard Commands}\\
\begin{tabular}{cl}
\hline
Key & Function\\
\hline
 E &EXIT ITV and return to VISTA (or resume procedure)\\
 R &RESTORE image to the original zoom/pan\\
 + &BLINK Forwards through the last 4 images.\\
 - &BLINK Backwards through the last 4 images.\\
 P &Find the PEAK pixel near the cursor \& jump the cursor there\\
 V &Find the LOWEST pixel ("Valley") near the cursor \& \\
   &jump the cursor there\\
 \# &"Power Zoom" zoom at the cursor to the maximum zoom factor\\
 H &Toggle between small and full-screen cross-hairs\\
 F &FREEZE/UNFREEZE the Zoom window.\\
0-9&Mark the pixel coordinates and store in VISTA\\
   &variables Rn and Cn, where "n" is the number key hit\\
 D &Print selected pixel row, column, and value.\\
   &Load variables R and C with the location.\\
 ] &Clear boxes and stuff off the image display\\
\hline
\end{tabular}
\end{center}

The D key also loads the coordinates of the most recently selected pixel
into VISTA variables R and C; the 0--9 keys load Rn and Cn, where n is the
key struck.  If the image had to be compressed to fit in the display window
(auto or manual dezoom), the main display will not show every pixel,
although the magnifying glass window will, so choose the pixel in that
window.

You also have control of the color map (upper \& lower contrast drag and
color-map roll), see the X11 help page for details.

The ITV command is very useful in procedures for letting you use the D or
0--9 keys to extract position information from the image of interest needed
elsewhere within the procedure.  Recall that variables (or any arithmetic
expression) may be used in place of a number in words of the form
word=value.  Here is an example of a procedure which draws a contour map of
a section of an image located by the cursor.  It draws at most a 51x51 box,
but will draw a smaller section of an image if the cursor is placed near
the edge of the frame.

\begin{verbatim}
   ASK 'DISPLAY WHICH IMAGE >> ' BUF        ! Get number of image
   MN $BUF; TV $BUF                         ! Display the image
   SROW=SR[BUF] SCOL=SC[BUF]                ! Note start row/col
   EROW=SR[BUF]+NR[BUF]-1                   ! Note end row
   ECOL=SC[BUF]+NC[BUF]-1                   ! Note end column
   PRINTF 'PLACE CURSOR ON CENTER OF THE REGION TO MAP'
   PRINTF 'THEN HIT THE D KEY, FOLLOWED BY E'
   ITV                                      ! Define center R and C
   NRSHOW=51 NCSHOW=51 DEL=IFIX[NRSHOW/2]   ! Default size
   LOROW=R-DEL HIROW=R+DEL                  ! Limits to box
   LOCOL=C+DEL HICOL=C+DEL
   IF LOROW<SROW                            ! Edge adjustment
      LOROW=SROW
   END_IF
   IF HIROW>EROW
      HIROW=EROW
   END_IF
   IF LOCOL<SCOL
      LOCOL=SCOL
   END_IF
   IF HICOL>ECOL
      HICOL=ECOL
   END_IF
   NRBOX=HIROW-LOROW+1 NCBOX=HICOL-LOCOL+1  ! Size of box
   BOX 1 SR=LOROW SC=LOCOL NR=NRBOX NC=NCBOX
   CONTOUR $BUF BOX=1
   END
\end{verbatim}

\section{COLOR: Load a Color Map into the Image Display}
\index{TV!Loading color map}
\index{Color map!Loading}
\index{Color map!Make new map}
\begin{rawhtml}
<!-- linkto color.html -->
\end{rawhtml}
\begin{command}
  \item[\textbf{Form: }COLOR {[CF=filename]} {[BW]} {[INV]}\hfill]{}
  \item[CF=]{loads an already-defined map into the AED,}
  \item[BW]{loads a black and white map, and}
  \item[INV]{inverts the ordering of the map.}
\end{command}

COLOR is used to load a new color look-up table from a file, or to create a
new color map.  Color maps (look-up tables) are used to translating pixel
data values into displayed colors.  The new look-up table will be loaded
into the display without needing to reloading the image.

To load a black and white map, type 
\begin{hanging}
  \item{COLOR BW}
\end{hanging}
To invert the order of the colors, type
\begin{hanging}
  \item{COLOR INV}
\end{hanging}

Color maps are assumed to be in the color directory. (Type PRINT
DIRECTORIES to display the directories).  Many color maps are available
with the default VISTA distribution.  Get a directory listing of the color
directory to see the names of the distribution color maps.  You can specify
color files that are in specific directories with the usual manner.  For
example:
\begin{example}
  \item[COLOR CF=WRMB\hfill]{looks for WRMB.CLR in the color directory}
  \item[COLOR CF=WRMB.XXX\hfill]{looks for WRMB.XXX in the color directory}
  \item[COLOR CF=./WRMB\hfill]{looks for ./WRMB.CLR}
\end{example}

A color map is a list of length 256, each entry of which holds three
numbers for the mapping to Red, Green, and Blue, respectively.  These
entries, or 'levels', run from 0-255 for each color.  The 256 entries in
the table give the proportions of each of these colors to use to display
each of the 256 intervals of intensity (8-bits).  COLOR defines these
numbers, either by loading them from a pre-determined list, or by creating
a new list.

If you wish to create a new list, type the command 'COLOR' with no
arguments.  The program will ask you to define the proportions of red,
green, and blue intensity to use for each of the entries in the color map.
The entries are numbered from 0 to 255.  The proportions of each color are
linear functions of entry number.  You define a starting entry number and
value, and an ending entry number and value. You may chain several linear
segments together to produce various effects in the color map.  You must
define all 256 entries in the map.  Typing in color maps by hand is tricky
and tedious, and not advised for most users.

The last thing the COLOR program will ask you is to specify the color of
the label and tick marks.

As an example of defining a color map, consider the black and white map.
For each level of intensity, the proportions of red, green, and blue are
equal.  We want increasing brightness to correspond to higher intensity, so
each color should have zero intensity at the bottom of the map, and full
intensity (255) at the top of the map.  The sequence of responses for
defining this color map would be:
\begin{verbatim}
   COLOR(begins the sequence)
   Red: starting level 0 enter initial value:   0
   Go to level                                255
   with intensity                             255
   Green: starting level 0 enter initial value: 0
   Go to level                                255
   with intensity                             255
   Blue: starting level 0 enter initial value:  0
   Go to level                                255
   with intensity                             255
   Define label intensities: Red:             200
                             Green:           230
                             Blue:             50
   Save color file:                        Y or N
\end{verbatim}

In this example, all three colors have zero intensity at the bottom level
of the color file, and full intensity at the top level.  The label color is
green with slightly less red, with a tinge of blue.

\section{TVBOX: Draw a Box on the Image Display}
\index{Box!Display boundaries on TV}
\index{Television!Display box on}
\begin{rawhtml}
<!-- linkto tvbox.html -->
\end{rawhtml}
\begin{command}
  \item[\textbf{Form: } TVBOX {[BOX=b]} {[SIZE=s PIX=r,c]}\hfill]{}
  \item[BOX=b]{tells the program to show the bounds
       of box 'b' on the television.}
  \item[SIZE=s PIX=r,c]{draws a box of size 's' centered at row 
       'r' and column 'c'}
\end{command}

TVBOX is used to show the boundaries of a box on the television.  You can
show a box defined in the command BOX, or show a region centered on a given
point.  You must specify EITHER the box number with BOX=, or the size AND
the location with SIZE= and PIX=.

There must be an image already in the TV for this command to work. The
displayed box will be clipped at the edge of the image if the box extends
over the bounds of the image.

\noindent{Examples:}
\begin{example}
  \item[TVBOX BOX=3\hfill]{display the boundaries of box 3
       on the television.}
  \item[TVBOX SIZE=51 PIX=123,234\hfill]{display a box of size 51
       pixels, centered at row 123 and column 234.}
\end{example}

\section{TVPLOT: Draw Text and Graphics on the Image Display}
\index{Display!Marking image with graphics and text}
\begin{rawhtml}
<!-- linkto tvplot.html -->
\end{rawhtml}
\begin{command}
  \item[\textbf{Form: }TVPLOT {[with keywords below as needed]}\hfill]{}
  \item[P=(r,c)]{         Enter a position (use twice as needed)}
  \item[L=l]{             Enter line length}
  \item[C=(r,c)]{         Enter center (row, column)}
  \item[TEXT=s]{          Plot text in quotes or string variable}
  \item[BOX=n]{           Plot image subsection box number n}
  \item[BOX]{             Plot boxes other than predefined image boxes}
  \item[W=w]{             Enter width of box}
  \item[H=h]{             Enter height of box}
  \item[CROSS]{           Plot cross}
  \item[CIRC=rad]{        Plot a circle}
  \item[COMPASS=rad]{     Plot a compass}
  \item[TICKS]{           Put ticks marks on line}
  \item[SCALE=s]{         Use pixel scale for all lengths}
  \item[PA=f]{            Position angle}
  \item[OFF=d]{           Offset of graphics from center}
\end{command}

TVPLOT is used to markup, label, annotate, or otherwise deface the image
display.  It has loads of options and clever use of defaults, so you don't
have to say everything.  The origin of the plots is from a predefined
center loaded by the AXES command or is given by the C= keyword.

In simple use TVPLOT will draw lines.  Use just L=l, or just P= for the
default center, or two P=(r,c) endpoints to ignore the default origin.  Use
SCALE if you mean arcseconds for L and not pixels, PA to rotate, or OFF to
give an offset in whatever direction the X axis is for any PA
specification.  TICKS will give nice scale marks on the line.

Show the image subsection boxes (see BOX) with BOX=n.  You can also make up
any boxes just for plotting by giving the BOX keyword.  This tells TVPLOT
to think about BOX plotting instead of lines.  Use W and H with the default
center or C= keyword, two P= keywords to give opposite corners, P and C
together for corners, any of L, W, or H by themselves for square boxes, and
so on.  Use PA, OFF, SCALE to help out.

Other symetrical plots.  Here you have CROSS, CIRC=r, COMPASS=r, all of
which follow the AXES, C=, or P= centers. Use PA, OFF, SCALE here, too.

Now you can make any label you want with the TEXT option -- except you can't
rotate it with PA (sorry -- it uses system fonts that I didn't want to muck
with).  Use P=, or C= to position it -- where you give where you want the
center of the string to be.  Right now VISTA has a little trouble with
multiword strings enterred directly on the TVPLOT command line, but they
work fine if you load them first with the STRING command.

\noindent{Examples:}
\begin{example}
  \item[AXES 1 BOX=7\hfill]{Find the center of the object within
       the BOX 7 subsection of image 1.}
  \item[TVPLOT BOX=7\hfill]{Plot this same box on the TV.}
  \item[TVPLOT CIRC=10\hfill]{Draw a 10-pixel radius circle around it.}
  \item[TVPLOT CROSS\hfill]{Mark it with a cross.}
  \item[TVPLOT COMPASS=20\hfill]{a really cool map compass}
  \item[STRING BIGDEAL  'This is the object in BOX 7']{}
  \item[TVPLOT TEXT=BIGDEAL\hfill]{Label the stupid thing.}
\end{example}

O.K. so now you want real vector graphics.
\begin{example}
  \item[TVPLOT P=(101,237)\hfill]{This will draw a line from where ever
       the center is to row=101, col=237.}
  \item[TVPLOT P=(r1,c1) P=(r2,c2)\hfill]{OK, this is your standard line
       between any two points.}
  \item[TVPLOT BOX P=(r1,c1) P=(r2,c2)\hfill]{Makes a box instead of a line.}
  \item[TVPLOT TEXT=BIGDEAL P=(r,c)\hfill]{Stick a label anywhere.}
\end{example}
Use procedures to make a neat grid at a funny angle.
\begin{verbatim}
   YDIM=25 
   XDIM=15 
   FUNNY=37 
   STEP=5
   DO Y=-YDIM,YDIM,STEP
      TVPLOT P=(-XDIM,Y) P=(XDIM,Y) PA=FUNNY
   END_DO
   DO X=-XDIM,XDIM,STEP
      TVPLOT P=(X,-YDIM) P=(X,YDIM) PA=FUNNY
   END_DO
\end{verbatim}

\section{IMPOST: Make a PostScript Hardcopy of an Image}
\index{Image!Hardcopy display}
\begin{rawhtml}
<!-- linkto impost.html -->
\end{rawhtml}
\begin{command}
  \item[\textbf{Form: } IMPOST imbuf {[BOX=b]} {[Z=zero]} {[L=span]} {[CLIP]} 
       {[POSITIVE]} {[TITLE]}\hfill]{}
  \item[{[HIST=xxx]} {[LAND]} {[FOUR]} {[AXES]} {[COMMENT]} {[COMMENT=xxx]}
        {[FILE=xxx]} {[OUT=xxx]}]{}
  \item[{[SCALE=s]} {[CEN=r,c]} {[FLIP]} {[BAR=xxx]} {[NOBAR]} {[WIND=w,h]} 
        {[ORIGIN=x,y]} {[PAGE=L,S]}]{}
  \item[{[COPIES=n]} {[LARGE]} {[INFO]} {[INT]} {[MACRO=]} {[EPS]}]{}
  \item[imbuf]{The buffer containing the image to be printed.}
  \item[BOX=b]{Only print the region within the given box.}
  \item[Z=zero ]{Zero point of the intensity mapping.}
  \item[L=span ]{Span of the intensity mapping.  }
  \item[CLIP]{Prevent roll-over of the intensity mapping.}
  \item[POSITIVE ]{Print the image intensities as White-on-Black.
                   The default mapping is Black-on-White.}
  \item[TITLE ]{Write the OBJECT name on the print (from the header).}
  \item[HIST ]{Map image intensities using flat histogram equalization.}
  \item[HIST=xxx]{Map image intensities using other histogram scalings
                  Options: (FLAT,SQRT,LOG).  "HIST" = "HIST=FLAT"}
  \item[LAND ]{Output will appear in LANDSCAPE mode}
  \item[FOUR ]{Use a 4 bit (16 level) gray scale rather than the
               default 8 bit (256 level) gray scale.}
  \item[AXES ]{Draw coordinates axes around the image.}
  \item[COMMENT='xxx']{Add a comment line to the print.  Comments may be
                       up to 64 characters long. }
  \item[COMMENT ]{Prompt the user interactively for the comment line.}
  \item[FILE=psfile]{Direct the output into a file named "psfile".}
  \item[OUT=psfile]{(same as FILE=psfile)}
  \item[SCALE=s ]{specify the pixel scale in units/pixel.}
  \item[CEN=(r,c)]{specify the image fiducial center in pixels for
                  scaled coordinate axes (see SCALE=).}
  \item[FLIP ]{flips image in rows (bottom-to-top raster order).}
  \item[BAR='xxx' ]{Label the gray scale intensity bar with the string
                    given.  Default bar label is "Intensity"}
  \item[NOBAR ]{Do not draw a gray scale intensity bar on the print.}
  \item[INFO ]{Write in a line of auxiliary info on the print.}
  \item[INT  ]{Puts user into interactive Mongo mode for labelling.}
  \item[MACRO=macrofile]{ Executes Mongo macro file macrofile after drawing}
  \item[Advanced Page Control\hfill]{}
  \item[EPS]{Make an Encapsulated PostScript file.}
  \item[COPIES=n ]{Specify the number of copies to print if more than 1.}
  \item[WIND=(w,h)]{Specify the maximum size in inches of the window in 
                    which IMPOST will try to print the image.}
  \item[LARGE ]{Alternative to WIND=, will make the window as large as 
                possible for the given paper size and orientation.}
  \item[ORIGIN=(x,y)]{Specify the origin of the plotting window in inches 
                      from the lower left-hand corner of the paper.}
  \item[PAGE=(L,S) ]{specify the physical size of the paper in inches.
                     "L" = long dimension, "S" = short dimension.}
\end{command}
\begin{itemize}
  \item[\textbf{Form: } IMPOST imbuf {[BOX=b]} {[Z=zero]} {[L=span]} {[CLIP]} 
       {[POSITIVE]} {[TITLE]}\hfill]{}
  \item[{[HIST=xxx]} {[LAND]} {[FOUR]} {[AXES]} {[COMMENT]} {[COMMENT=xxx]}
        {[FILE=xxx]} {[OUT=xxx]}]{}
  \item[{[SCALE=s]} {[CEN=r,c]} {[FLIP]} {[BAR=xxx]} {[NOBAR]} {[WIND=w,h]} 
        {[ORIGIN=x,y]} {[PAGE=L,S]}]{}
  \item[{[COPIES=n]} {[LARGE]} {[INFO]} {[INT]} {[MACRO=]} {[EPS]}]{}
  \item[imbuf]{The buffer containing the image to be printed.}
  \item[BOX=b]{Only print the region within the given box.}
  \item[Z=zero ]{Zero point of the intensity mapping.}
  \item[L=span ]{Span of the intensity mapping.  }
  \item[CLIP]{Prevent roll-over of the intensity mapping.}
  \item[POSITIVE ]{Print the image intensities as White-on-Black.
                   The default mapping is Black-on-White.}
  \item[TITLE ]{Write the OBJECT name on the print (from the header).}
  \item[HIST ]{Map image intensities using flat histogram equalization.}
  \item[HIST=xxx]{Map image intensities using other histogram scalings
                  Options: (FLAT,SQRT,LOG).  "HIST" = "HIST=FLAT"}
  \item[LAND ]{Output will appear in LANDSCAPE mode}
  \item[FOUR ]{Use a 4 bit (16 level) gray scale rather than the
               default 8 bit (256 level) gray scale.}
  \item[AXES ]{Draw coordinates axes around the image.}
  \item[COMMENT='xxx']{Add a comment line to the print.  Comments may be
                       up to 64 characters long. }
  \item[COMMENT ]{Prompt the user interactively for the comment line.}
  \item[FILE=psfile]{Direct the output into a file named "psfile".}
  \item[OUT=psfile]{(same as FILE=psfile)}
  \item[SCALE=s ]{specify the pixel scale in units/pixel.}
  \item[CEN=(r,c)]{specify the image fiducial center in pixels for
                  scaled coordinate axes (see SCALE=).}
  \item[FLIP ]{flips image in rows (bottom-to-top raster order).}
  \item[BAR='xxx' ]{Label the gray scale intensity bar with the string
                    given.  Default bar label is "Intensity"}
  \item[NOBAR ]{Do not draw a gray scale intensity bar on the print.}
  \item[INFO ]{Write in a line of auxiliary info on the print.}
  \item[INT  ]{Puts user into interactive Mongo mode for labelling.}
  \item[MACRO=macrofile]{ Executes Mongo macro file macrofile after drawing}
  \item[Advanced Page Control\hfill]{}
  \item[EPS]{Make an Encapsulated PostScript file.}
  \item[COPIES=n ]{Specify the number of copies to print if more than 1.}
  \item[WIND=(w,h)]{Specify the maximum size in inches of the window in 
                    which IMPOST will try to print the image.}
  \item[LARGE ]{Alternative to WIND=, will make the window as large as 
                possible for the given paper size and orientation.}
  \item[ORIGIN=(x,y)]{Specify the origin of the plotting window in inches 
                      from the lower left-hand corner of the paper.}
  \item[PAGE=(L,S) ]{specify the physical size of the paper in inches.
                     "L" = long dimension, "S" = short dimension.}
\end{itemize}

IMPOST makes a hardcopy print of an image in PostScript format.  It
is designed to be used just like the TV command, with very similar
keywords. You can include extra information on the picture with the
TITLE, AXES, COMMENT, INFO and related keywords.  These are discussed
below.
 
IMPOST creates a PostScript file on disk.  This file will be located in the
current directory (either where you started VISTA or under Unix versions of
VISTA, the last directory you CD'd to).  By default this file will take the
name "image.ps".  You can specify your own name for the file by using the
FILE= or OUT= keywords.  Currently, this file is not automatically printed.
From outside of VISTA (or from a spawned subprocess or C-shell), submit
this to your printer as you would submit any other PostScript file (e.g.,:
"lpr image.ps" on a Unix system, or "\$ print/queue=pslaser image.ps" on a
VMS system).

Multiple copies can be made using the COPIES=n keyword.  The POSITIVE
keyword makes white images on a black background; the default is to make
black images on a white background.

Output from IMPOST, if prepared with care, is of the highest quality.
IMPOST generated prints made on good quality paper or special "stat" paper
have been accepted for publication without re-photography required.

\noindent{Intensity Scaling}

IMPOST produces a hardcopy of an image by representing pixel intensities as
an 8-bit halftone gray scale.  The optional keyword "FOUR" will use a 4-bit
(16-level) gray scale instead.  Although an 8-bit gray scale has 256 levels
from white to black, on standard 300 dpi laser printers, only 32 levels are
distinguishable.  The output is a PostScript file which may be sent to any
standard PostScript hardcopy device.


The simplest intensity scaling is through use of the Z= and L= keywords
like in the TV command.  These set the zero point (Z=ZERO) and intensity
span (L=SPAN) of a "linear" mapping from the actual image intensity in each
pixel, IMG(r,c), to a displayed gray scale value between 0 and 255,
INTENS(r,c), following the formula:
\begin{hanging}
  \item{          INTENS(r,c) = {[IMG(r,c) - ZERO]}*(255/SPAN)}
\end{hanging}
With the FOUR option, the scale factor is 15 instead of 255, and
INTENS(r,c) will take values from 0 to 15.

In this simple linear mapping, pixels with IMG(r,c) <= ZERO will set
INTENS(r,c)=0 (white) and pixels with IMG(r,c)=ZERO+SPAN will set
INTENS(r,c)=255 (black).  If the CLIP keyword is used, then all values of
IMG > ZERO+SPAN are set black, otherwise, the color scale "wraps" and above
IMG=ZERO+SPAN, INTENS starts again at 0.  This continues modulo (255/SPAN)
until all pixel intensity values are mapped.

The default mapping is an inverse (or "negative") scale, where the image is
drawn as black against a white background.  If the optional keyword
POSITIVE is given, then the intensities are mapped in a more "normal" or
"positive" scale in which the image is displayed as white against a black
background, though the results with most laser printers is less than
optimal.

The second scaling option is to map image intensities into an 8-bit gray
scale using the distribution of pixel intensities in the image to provide a
scaling in which all gray levels are equally present.  Such a scaling is
called a "Histogram Equalization", and is used to bring up faint features
without sacrificing more high-intensity features in the same image.  This
is done using the HIST keyword.  Histogram equalization is a common
technique in image processing, most notably it is used for making the
high-contrast images that came from the Voyager and Viking space probes.
Note that a simple one-to-one mapping of pixel intensities and gray level
is lost, and such mapping should be used with caution.

If the Z= and L= keywords are used with HIST, then Z= is taken to be the
lowest intensity value to be considered, and L= tells hist to ignore all
intensities above Z+L when determining the histogram mapping.  The CLIP
keyword is implicit with any histogram scaling option -- no rollover is
permitted.

Additional histogram scaling options are available via the HIST=LOG and
HIST=SQRT keywords.  HIST=LOG will map intensities by equalizing over
logarithmic intervals.  This is useful when there is a very large range of
intensities in the image, and you wish to emphasize the brighter regions
without washing out fainter features.  The HIST=SQRT option will map
intensities by equalizing over square-root intervals.  This represents a
middle ground between the "FLAT" histogram mapping described above (which
enhances fainter regions) and the "LOG" histogram mapping.

Some experimentation to determine the best option to use will undoubtably
be required.  Images which are mostly background sky will usually not
benefit from any form of histogram equalization, while images of extended
objects in which the frame is more-or-less filled with "object" will
usually benefit.  By setting the zero point (Z=) to be above the image sky
level, you can have the advantages of the histogram equalization scalings
while still showing surrounding objects.

\noindent{Labeling the Output}

By default, IMPOST will only draw a box around the image and display a bar
with the gray scale intensity range (0-255 or 0-15) in a box below the
image.  You can add axes and labels using a variety of keywords provided.
\begin{example}
  \item[TITLE]{puts the object name of the image (stored in the header in
          the OBJECT FITS card).  The title appears above the upper
          left-hand corner of the image in large type.  Standard 
          Mongo escape sequences can be used to print special characters
          (like a greek alpha), or super- and sub-scripts.}
  \item[COMMENT]{appends a line of text below the image title in a slightly
          smaller font.  Comments may be specified on the command line
          by using the COMMENT='comment string' keyword.  You may also
          have IMPOST prompt the terminal for the comment by using
          the COMMENT keyword without an ='s sign.  Comments may be 
          single or multiple words up to 64 characters long.  Note that
          Mongo special characters may also be used, in addition to 
          super- and sub-scripts, but the extra escape sequence characters
          count against the 64 maximum.}
  \item[AXES]{In its simplest form, AXES will draw coordinate axes around
          the image and label the axes in pixel units.  Using the
          SCALE= and/or CEN= keywords (see below) will allow you
          to scale the axes in arcseconds.}
  \item[SCALE= ]{Sets the pixel scale in arcseconds/pixel for the AXES 
          keyword (above).  When used with AXES, the center of
          the image is taken to be the origin (0,0), and the axes
          are labeled accordingly.  To set a particular pixel to be
          the center, use the CEN= keyword.}
  \item[CEN=]{Sets the origin of the image to be the (r,c) location given.
          Two (2) arguments must be given, and need not lie inside the
          region actually plotted (useful if you wish to show a zoom of
          a region in a larger image while still preserving the original
          coordinate system).  If CEN is not used with SCALE, then the
          center of the region displayed is assumed to be the origin.
          (r,c) may be fractional pixel locations (say the centroid of
          a galaxy or planetary nebula central star).}
  \item[NOBAR]{Tells IMPOST not to draw in the gray intensity scale bar
          on the image. }
  \item[BAR=]{By default, IMPOST will label the gray intensity scale
          bar with the word "Intensity", but you may set it to 
          anything you like with the BAR= keyword.  For example,
          BAR='Log H\\ga Flux' will label the intensity bar as
          indicating the logarithm of the H-alpha flux.  Any valid
          Mongo special character escape sequence (including 
          super- and sub-scripts is allowed).}
  \item[INFO]{Puts a line of auxiliary information across the bottom of
          the print.  This provides a hardcopy of the parameters
          that went into the picture (zero point, span, clipping, etc).}
  \item[INT]{Puts the user into interactive Mongo after drawing image. The
          limits are the row and column limits. Here you can do your
          own labelling, overlays of contours, etc.}
  \item[MACRO=file]{Executes the Mongo macro file after drawing image. Like
          INT but uses predefined Mongo macro.}
  \item[FLIP]{This keyword "flips" the image orientation in the window
          vertically (about rows) so that the origin of the image is
          in the lower left-hand corner of the page rather than the
          upper left-hand corner as is default for VISTA.  This is
          the orientation used by IRAF and STSDAS.}
\end{example}

\noindent{Advanced Page Control}

A number of optional keywords are provided which affect the physical size
of the image as it appears on the page, its orientation, the number of
copies printed etc.  These options may be used if IMPOST prints are
intended for use in papers or dissertations, where precise specification of
margins and the location and orientation of the image on the page are
essential.  These keywords are described below:

\begin{example}
  \item[LAND]{This keyword will cause the print to be oriented with the
         horizontal axis aligned with the long side of the paper 
         (so-called "landscape" mode).  By default, IMPOST prints are 
         made in "portrait" mode.}
  \item[COPIES=n]{If more than one copy is desired, then set the COPIES keyword.
         This option is very economical of processing time, as the
         printer only has to "draw" the image once and then knock off
         as many copies as required.  This is far more efficient than 
         printing the same file N-times.}
  \item[WIND=(w,h)]{This keyword specifies the maximum possible size of the
         plotting window, in inches.  The window is the largest
         region in which IMPOST will try to fit the image.
         By default, the plotting window is:
             Portrait (default):  6" x 6"
               Landscape (LAND):  8" x 5"
         Dimensions are specified as "width","height" in units of inches.
         These dimensions must be smaller than the paper dimensions 
         (see PAGE keyword).}
  \item[LARGE]{An alternative to WIND= is the "LARGE" keyword.  Will attempt
         to make the window as large as possible on the page, leaving only
         a 0.25-inch margin around the edge.  LARGE is equivalent to
            ORIGIN=0.25,0.25  WIND=8.0,10.5  PAGE=11.0,8.5}
  \item[ORIGIN=(dh,dv)]{Specifies the origin of the plotting window
         in inches measured from the lower left-hand corner of the 
         paper.  DH is the horizontal offset, and DV is the vertical
         offset.  By default, IMPOST tries to center the plotting 
         window on the page, and adds a 30pt "binding margin" to 
         Portrait mode prints.  ORIGIN overrides this binding margin.}
  \item[PAGE=(lng,sh)]{This keyword specifies the physical size of the paper in
         inches.  "lng" = long dimension, "sh" = short dimension.
         Note that which is horizontal or vertical depends on
         whether or not the LAND keyword was used.  For example, 
         PAGE=(11,8.5) is equivalent to the IMPOST default, and
         corresponds to the dimensions of standard U.S. letter
         paper.  PAGE should only be used if you have non-standard 
         sized paper in your local PostScript printer and do not
         wish to change the hardcoded values in the IMPOST fortran
         code itself.  European VISTA users should have their
         local custodians hardwire IMPOST to the appropriate local
         paper size.}
\end{example}

\noindent{Advanced Notes:}

IMPOST uses the LickMongo PostScript drivers to make the axes and labels.
As such, while PostScript instructions generated by this program may be
used to produce the image hardcopy on any PostScript device via an
appropriate server, they do not fully conform to to the "Encapsulated
PostScript" structuring conventions described in the 'PostScript Language
Program Design' Manual (Adobe Systems Incorporated 1988), or following the
Version 2.0 EPS Specifications (Adobe Systems Inc. January 1989).  In
future releases of LickMongo, this should be corrected (it is an artifact
of the fact that the LickMongo PostScript drivers were written well before
the standards were published).  Even so, with minor modifications, it
should be possible to import images created with IMPOST into other
encapsulated PostScript files (like those generated by TeX).

Images created with IMPOST have been displayed successfully on the X11/NeWS
window system, on displayed PostScript devices, and transmitted
electronically via both email and tcp/ip with no difficulties.

'PostScript' is a registered trademark of Adobe Systems Incorporated.

\noindent{A Note About the Name:}

IMPOST is the command formerly known as POSTIT until 1997 Nov 21, when the
author (Rick Pogge) was made aware of a letter to his department chairman
from the law firm of Arnold, White, \& Durkee.  This firm represents the
Minnesota Mining \& Manufacturing Company (3M) in trademark and other
intellectual property rights matters.  It seems after coming across a web
version of the Prospero manual (a VISTA derivative written at OSU that
incorporates POSTIT because it was written by Rick Pogge), they deemed our
use of the word POSTIT as an infringment of the trademark of 3M for its
famous yellow sticky notes, aka Post-It(tm) Notes that might potentially
confuse people encountering the web version of our help pages.

While we disagree that anyone could possibly confuse the ubquitous yellow
sticky notes manufactured by 3M with an obscure function embedded within an
even more obscure advanced astronomical image processing package, the
current legal climate of excessive litigiousness that serves only to
trivialize intellectual property rights requires that we change the name.

\section{PLOT: Plot a Row or Column of an Image or Spectrum}
\index{Image!Row display}
\index{Image!Column display}
\index{Spectrum!Plot}
\begin{rawhtml}
<!-- linkto plot.html -->
\end{rawhtml}
\begin{command}
  \item[\textbf{Form: } PLOT b1 b2 b3 .. {[R=n1,n2]} {[C=n1,n2]}
       {[RS=r1,r2]} {[CS=c1,c2]}\hfill]{}
  \item{{[XS=f]} {[XE=f]} {[MIN=f]} {[MAX=f]} {[XMIN=]} {[XMAX=]} 
        {[YMIN=]} {[YMAX=]}}
  \item{{[SEMILOG]} {[LOG]} {[R4]} {[NH=nh]} {[NV=nv]} {[OLD]}}
  \item{{[HIST]} {[POINTS=psty]} {[USER]} {[PIXEL]} {[GRID]} }
  \item{{[NOLABEL]} {[NOERASE]} {[INFO]} {[PROFILE]} {[MULT=]} }
  \item{{[INT]} {[MACRO=file]} {[HARD]} {[PORT]} {[PSFILE=file]} {[EPS]} 
       {[WAIT]} }
  \item[b1,b2,b3..]{the image buffers which will be plotted (up to 15).}
  \item[BOX=b]{limits the image 'buf' to box 'b'.}
  \item[R=n1,n2, C=n1,n2]{plots the selected row(s) or column(s).}
  \item[RS=r1,r2]{plots sum of selected rows.}
  \item[CS=c1,c2]{plots sum of selected columns.}
  \item[XS, XE]{select limits for x-axis of plot (also XMIN \& XMAX).}
  \item[MIN, MAX]{select limits for y-axis of plot (also YMIN \& YMAX).}
  \item[SEMILOG]{plots log(y) against x.}
  \item[LOG]{plots log(y) against log(x).}
  \item[R4      ]{plot x**0.25 instead of x.}
  \item[NH=nh, NV=nv ]{number of horizontal and vertical windows
       for multiple plots in multiple windows}
  \item[WIND=nx,ny,iw]{sets desired plotting window}
  \item[OLD]{uses parameters from previous plot.}
  \item[HIST]{produces a histogram style (step) plot instead of a line plot.}
  \item[POINTS=psty]{Plot points in the given Mongo style instead of a line
                     plot.}
  \item[USER]{uses axis scaling and labels stored in the FITS header.}
  \item[PIXEL]{force plot in pixels of wavelength calibrated spectra.}
  \item[GRID]{produces a full plot grid.}
  \item[NOLABEL]{suppresses all labelling of the plot.}
  \item[NOERASE]{do not erase the previous plot to allow overlapping
       plots (see WAIT for hardcopy).}
  \item[INFO]{puts additional information about the data on the plot.}
  \item[PROFILE ]{plots results from the PROFILE common block}
  \item[MULT=n ]{plots the nth galaxy from the internal multiple
        nucleus common block}
  \item[INT]{places the user in interactive Mongo after plotting.}
  \item[MACRO=file ]{executes the Mongo macro file 'file' after plotting.}
  \item[HARD]{sends the output to the hardcopy device specified by
       the TERM command.}
  \item[PORT]{make an portrait mode plot (default: landscape).}
  \item[WAIT]{keep the hardcopy channel open so subsequent plots will
              appear on the same image (or PostScript file).}
  \item[PSFILE=file]{directs PostScript output to the specified file,
       rather than sending it to the printer}
  \item[EPS]{make an Encapsulated PostScript file.}
\end{command}

PLOT produces a plot of a selected rows, columns, or spectra for up to 15
different images, using the LickMongo plotting package. All the specified
buffers will come out on the same plot, or each successive specified buffer
can come out in a series of windows (with the number of horizontal and
vertical windows specified by the NH= and NV= keywords). If desired, all
the plots can come out in a single subwindow using the WIND= keyword.
PLOT also allows the possibility of hardcopy on your hardcopy device
(as specified by the TERM HARCOPY= command).

The index 'buf' specifies which of the image buffers contains the image for
plotting a row or column. No 'buf' is needed to plot results from the
PROFILE or the MULT common block, which contain profiles determined with
the PROFILE or SNUC command.  The optional index 'box' specifies one of the
10 available boxes (see BOX command) to be used to select a sub-section of
an image. Row 'n' is displayed by using the option R=n, column 'n' by using
the C=n option, and spectrum 'n' by using no option. The RS and CS keywords
are used to display the sum of the selected rows or columns in an image.

Normally the endpoints of the x- and y-axes are selected in this way: The
x-axis runs over the entire row, column, or spectrum selected. The minimum
and maximum y-values define the corresponding limits for the y-axis.

XS and XE override this autoscaling and allow you to set the limits along
the X-axis.  XS and XE refer to the starting and ending X values in pixels
(or Angstroms for a spectrum with a wavelength scale determined.)  If the
USER option is defined, XS and XE are in the user-defined coordinates.
MIN and MAX override the autoscaling and allow the user to set the limits
along the Y-axis.

The keyword GRID will produce a coordinate grid on the plot.  This is very
useful if you are trying to make measurements from a hard copy plot.  The
keyword INFO will put some additional information onto the plot.

LOG and SEMILOG will produce Log--Log or Log--Linear plots as desired.  The
R4 keyword will plot $x^1/4$ along the x-axis, particularly useful when
plotting the output image from PROFILE. NOLABEL prevents the drawing of
axis labels and the name of the image or spectrum.  Use this with NOERASE
to overplot several graphs.  The OLD keyword will produce a plot using the
last used set of parameters.

If multiple window are desired, use the NH= and NV= keywords to set the
number of windows in the horizontal and vertical directions,
respectively. The plots of b1, b2, b3 ... will appear in the window
starting in the upper left and progressing right across the page and then
down. If limits are specified with MIN and MAX, e.g., all windows will have
these limits. If no limits are specified, each window will autoscale
individually.

If you wish all plots to be sent to a subwindow, use the WIND= keyword,
with three parameters following the normal LickMongo syntax; number of
windows in the x direction, number of windows in the ydirection, and
a specification of which window to draw into (e.g. WIND=2,3,5 will
draw in the upper left window of a 2x3 layout).

HARD will send the plot to the hardcopy device as specified by the TERM
command (or the default device). If you use the PSFILE=file option,
PostScript output is sent to the specified file, rather than directly to
the printer.  The PORT keyword will make a Portrait mode plot (long side of
the paper vertical), and EPS will make an Encapsulated PostScript file.
EPS should be used in conjunction with the PSFILE= keyword to create the
EPS file.

The MACRO=file keyword can be used to execute the Mongo macro file 'file'
after each plot is made.  This provides a way to add additional annotation
or data to a plot (especially for preparation of figures for papers or
talks).

INT is an alternative to MACRO= that invokes an interactive LickMongo shell
allowing users to enter plot annotations and other commands interactively.
Note that if this is done on a screen plot you need to use the WRITE ALL
... command in LickMongo to save the macro, then when you make a hardcopy
you can use the MACRO= keyword to execute the LickMongo commands.)  If
MACRO= and INT are both specified, the macro file will be executed before
the user is put into the interactive LickMongo shell.

USER will scale and label the axes according to the user-defined
coordinates instead of pixels.  This is done using the CRVALn, CRPIXn,
CDELTm, CTYPEn and BUNIT cards in the image (or spectrum) FITS header.  The
conversion between user coordinates and pixels is given by:
\begin{hanging}
  \item{X = CRVAL1 + CDELT1 * FLOAT(COL - CRPIX1)}
\end{hanging}
for Columns, or
\begin{hanging}
  \item{X = CRVAL2 + CDELT2 * FLOAT(ROW - CRPIX2)}
\end{hanging}
for Rows, where
\begin{itemize}
  \item[]{CRPIXn = INTEGER equal to the starting pixel of a Column (n=1)
         or a Row (n=2) in an image.  For spectra, CRPIX1=1 typically.}
  \item[]{CRVALn = REAL value of user coordinates at pixel=CRPIXn.  For
         spectra, CRVAL1 = the starting wavelength (of pixel=1 typically).}
  \item[]{CDELTn = REAL value scaling factor between rows (n=2) or columns
         (n=1) and user coordinates.  For spectra, CDELT1 = linear dispersion.}
\end{itemize}

The X-axis label is taken from the CTYPEn card, which is a CHARACTER string
surrounded by single quote (') marks.  The Y-axis label is taken from the
BUNIT card, which is a CHARACTER string surrounded by single quote (')
marks containing the appropriate units for the "Intensity" units of the
pixels (e.g. counts/sec or flux).  These header cards may be set using the
HEDIT command.  If a particular FITS card is not found, appropriate
defaults are chosen, but be warned: older images and spectra from the Lick
Data Taking programs do not conform to the FITS standard and may give
strange results.

\noindent{Examples:}
\begin{example}
  \item[PLOT 1 R=100]{Plots row 100 of image 1.}
  \item[PLOT 1 RS=100,120 MIN=0. MAX=100.]{  Plots sum of rows 100 to
       120 of image 1, with Y-axis running from 0. to 100.}
  \item[PLOT 4 XS=20 XE=45 R=4]{Plots row 4 of image 4,
       where the X-axis runs over columns 20 through 45.}
  \item[PLOT 4 OLD SEMILOG]{Plot the old graph (whatever it was)
       in semi-log format.}
  \item[PLOT 1]{Plots the spectrum in buffer 1.  No
       row or column specifers are necessary.}
  \item[PLOT 4 INT ]{After plotting the spectrum (note that 4
       must be a spectrum as no row or col was
       specified!) enter interactive Mongo mode
       to label and finish the plot.}
  \item[PLOT 3 R=50 HIST]{Draws a plot of row 50 in image 3.
       The plot is d}one in histogram style.
  \item[PLOT 3 MIN=0.0 MAX=1000. HIST]{Does the same as example
       8, but sets the Y-axis to run from 0.0 to 1000.0.}
  \item[PLOT 2 R=50 HARD USER]{Plots R=50 of image 2 on the Imagen laser
       printer and scales and labels the axes with
       the user coordinate info stored in the FITS header of image 2.}
\end{example}

\section{RPLOT: Radial Intensity Plot of an Object in an Image}
\begin{rawhtml}
<!-- linkto rplot.html -->
\end{rawhtml}
\begin{command}
  \item[\textbf{Form: } RPLOT imbuf {[RAD=r{[,c]}]} {[CEN=r0,c0]} 
       {[SCALE=s]} {[HARD]}\hfill]{}
  \item{{[MIN=ymin]} {[MAX=ymax]} {[COLOR=c]} {[NOERASE]} {[LOG]}}
  \item[imbuf]{Specify the image buffer.  If no image buffer
       is given, the currently displayed image is assumed.}
  \item[RAD=r{[,c]}]{ Set the radius of the box in pixels in both rows 
        and columns.  A second argument sets a rectangular 
        box with the radius in Rows and Columns separately 
        defined.  (Default: RAD=5)}
  \item[CEN=r0,c0]{ Define the central pixel for the radial intensity
       plot.}
  \item[SCALE=s]{Define the pixel scale in Arcsec/Pixel for the plot
       (does not apply to the RAD= keyword values)}
  \item[HARD]{Make a hardcopy of the radial intensity plot }
  \item[MIN=, MAX=]{ Sets y-axis limits for plot}
  \item[COLOR=]{ Set the color to be used for the plot}
  \item[NOERASE]{ Do not erase between plots}
  \item[LOG]{ Plot intensities on a logarithmic scale}
\end{command}

This program makes a simple radial intensity plot of a section of an image.
If the object is a star, it shows a mean point-spread function profile.  An
interactive version that works off the TV display is available via the
TVRPLOT command.

The radial intensity plot is formed by finding the radius to all points
within a given box around the specified location and plotting intensity
versus radius.  Each point making up the plot is from a single pixel; no
interpolation is done.  The effect is to produce an azimuthally wrapped
(not averaged) plot of intensity.  For example, if doing RPLOT on the image
of a star, if that star is round, then a fairly clean radial point-spread
function profile, will be plotted.  If the star is elliptical (due to bad
guiding or focus), the radial brightness profile will be a fuzzy cloud of
points.

The center of the radial plot may be specified in one of two ways:
\begin{enumerate}
  \item{The center is as defined in the AXES common block.  You
        must have run the AXES command on the image before invoking
        RPLOT.}
  \item{The center is specified on the command line using the CEN= keyword.}
\end{enumerate}

Centers may be integer or fractional pixel values, but the size of the
search region will be truncated to integer pixel intervals.

The interactive routine, TVRPLOT, allows the user to select the center of
the radial plot with the interactive display cursor on the currently
displayed image (see TVRPLOT).  A hardcopy plot may be generated using the
HARD keyword.

NOERASE suppresses erasing of the plotting window, allowing you to
superimpose subsequent radial plots on top of each other.  

LOG will plot the logarithm of the intensity as a function of radius.

COLOR=c changes
the color of the points plotted.  The default color is white (1).  At
present only the 7 "primary" graphics colors are available:

\begin{center}
{\bf Color Codes}\\
\begin{tabular}{cccc}
\hline
Code & Color & Code & Color\\
\hline
0 & background & 4 & Blue\\    
1 & foreground & 5 & Yellow\\  
2 & Red        & 6 & Magenta\\ 
3 & Green      & 7 & Cyan\\    
\hline
\end{tabular}
\end{center}

Colors may be used in conjunction with NOERASE to overlay two different
radial profiles, using the colors to distinguish them.

\noindent{Examples:}
\begin{example}
  \item[RPLOT 1 RAD=10 CEN=51,25\hfill]{
   Makes a radial intensity profile of the image in buffer 1 centered on
   the pixel at Row=51, Column=25, using all pixels within a 21x21 pixel
   wide box surrounding the center pixel.}
  \item[RPLOT 1 RAD=10 CEN=51.5,25.2\hfill]{
   Same as above, but the center of the plot is now at fractional pixel
   location (51.5,25.2)}
  \item[RPLOT\hfill]{
   Makes a radial intensity plot of the currently displayed image using
   the center computed by the previous AXES command (AXR and AXC variables).
   The box is 11x11 pixels (RAD=5) centered on (AXR,AXC).}
  \item[RPLOT SCALE=0.54 HARD\hfill]{
   Same as above, but scales the radii plotted to the pixel scale of 0.54
   arcseconds/pixel, and sends the plot to the hardcopy device.  The size
   of the box is still 11x11 pixels.}
\end{example}

\section{TVRPLOT: Radial Intensity Plot of an Object on the Image Display}
\begin{rawhtml}
<!-- linkto tvrplot.html -->
\end{rawhtml}
\begin{command}
  \item[\textbf{Form: } TVRPLOT {[RAD=r{[,c]}]} {[SCALE=s]} {[MIN=ymin]} 
       {[MAX=ymax]}\hfill]{}
  \item[RAD=r{[,c]}]{ Set the radius of the box in pixels in both rows 
       and columns.  A second argument sets a rectangular 
       box with the radius in Rows and Columns separately 
       defined.  (Default: RAD=5)}
  \item[SCALE=s]{   Define the pixel scale in Arcsec/Pixel for the plot
       (does not apply to the RAD= keyword values)}
  \item[MIN=, MAX=]{ Sets y-axis limits for plot}
  \item[COLOR=]{ Set the color to be used for the plot}
  \item[NOERASE]{ Do not erase between plots}
  \item[LOG]{ Plot intensities on a logarithmic scale}
\end{command}

This program makes a simple radial intensity plot of a section of an image.
If the object is a star, it shows a mean point-spread function profile.  An
non-interactive version is available via the RPLOT command.
 
The radial intensity plot is formed by finding the radius to all points
within a given box around the specified location and plotting intensity
versus radius.  Each point making up the plot is from a single pixel; no
interpolation is done.  The effect is to produce an azimuthally wrapped
(not averaged) plot of intensity.  For example, if doing TVRPLOT on the
image of a star, if that star is round, then a fairly clean radial
point-spread function profile, will be plotted.  If the star is elliptical
(due to bad guiding or focus), the radial brightness profile will be a
fuzzy cloud of points.

The center of the radial plot may be specified in one of two ways:
\begin{enumerate}
  \item{"C": The center is the intensity centroid nearest the 
             cursor, computed using the same algorithm used by MARKSTAR and
             AUTOMARK.}
  \item{"X": The center is the pixel where the display cursor
             is located.  No centroid is computed in this case.}
\end{enumerate}
The radius axis on the plot may be scaled to arcseconds using the SCALE=
keyword.  Note that using SCALE will not affect the size of the search
radius; this is always expressed in pixel units.

NOERASE suppresses erasing of the plotting window, allowing you to
superimpose subsequent radial plots on top of each other.  

LOG will plot the logarithm of the intensity as a function of radius.

COLOR=c changes
the color of the points plotted.  The default color is white (1).  At
present only the 7 "primary" graphics colors are available:

\begin{center}
{\bf Color Codes}\\
\begin{tabular}{cccc}
\hline
Code & Color & Code & Color\\
\hline
0 & background & 4 & Blue\\    
1 & foreground & 5 & Yellow\\  
2 & Red        & 6 & Magenta\\ 
3 & Green      & 7 & Cyan\\    
\hline
\end{tabular}
\end{center}

Colors may be used in conjunction with NOERASE to overlay two different
radial profiles, using the colors to distinguish them.

See RPLOT for the non-interactive (batch) version of this program.

\section{CONTOUR: Make Contour Maps Images}
\index{Plot!contour}
\begin{rawhtml}
<!-- linkto contour.html -->
\end{rawhtml}
\begin{command} 
  \item[\textbf{Form:} CONTOUR im1 {[im2...]} {[BOX=b]} 
       {[LEVELS=(L1,L2,...)]} {[LOW=l]} {[RATIO=r]} ]{}
  \item[{[DIFF=d]} {[FID=l]} {[SCALE=s]} {[USER]} 
        {[TR=(X0,X1,X2,Y0,Y1,Y2)]} ]{}
  \item[{[TITLE]} {[DASH]} {[EXACT]} {[NOERASE]} {[NOLABEL]} {[NOAXES]}]{}
  \item[{[LWEIGHT=w]} {[LTYPE=n1,n2,...]} {[COLOR=c1,c2,...]} {[NV=n]} 
        {[NH=n]} {[NW=n]} {[SUBMAR=sx,sy]}]{}
  \item[{[HARD]} {[LAND]} {[FILE=psfile]} {[EPS]} {[NOPRINT]} {[FULL]} ]{}
  \item[im1 im2 ...]{buffer(s) with the images to be plotted (max of 15)}
  \item[BOX=b]{plots only the part of the image in box 'b'.}
  \item[\textbf{Contour Level Control}\hfill]{}
  \item[LEVELS=]{sets the levels of the plot explicitly.}
  \item[LOW=l]{gives the lowest contour level.}
  \item[RATIO=r]{says that contours above the lowest level will increase
       each time by the given ratio.}
  \item[DIFF=d]{says that the contours above the lowest level will
       increase each time by the given difference.}
  \item[FID=l]{establishes a fiducial contour level for the plot.  The
       fiducial contour is drawn in a heavier line style.
       Contours greater than this level are drawn as a lighter
       continuous line, and those below it are drawn as broken
       lines.  The default value is 0.0.}
  \item[\textbf{Contour Plot Scaling Keywords}\hfill]{}
  \item[SCALE=s]{Sets the image scale to "s" arcseconds/pixel on BOTH axes.}
  \item[USER]{Scales the contour axes based on the CRPIXn, CRVALn,
       and CDELTn FITS header cards, and labels the axes using
       the CTYPEn FITS cards.}
  \item[TR=()]{specify a general linear transformation matrix to scale
       the contour map (see below).}
  \item[\textbf{Contour Style Control}\hfill]{}
  \item[DASH]{draw contours as broken (dashed) lines.}
  \item[EXACT]{Uses a slower contour "following" algorithm than the
       default routine.}
  \item[LWEIGHT=w]{Sets the thickness of all line segments}
  \item[LTYPE=n1,n2...]{determines the line type of im1,im2,...}
  \item[COLOR=c1,c2...]{determines the color of im1,im2,...}
  \item[\textbf{Plot Title and Output Control}\hfill]{}
  \item[TITLE]{put object label on plot}
  \item[NOERASE]{do not erase screen.}
  \item[NOLABEL ]{do not label axex.}
  \item[FULL ]{makes contour plot fill up entire page, rather
       than preserving correct aspect ratio}
  \item[\textbf{Multiwindow Plots}\hfill]{}
  \item[NH=h]{Divide the plotting window into H windows horizontally}
  \item[NV=v]{Divide the plotting window into V windows vertically}
  \item[NW=n]{Put the contour plot into window N for single images 
       (mongo order)}
  \item[SUBMAR=dx,dy]{Change the default spacing (submargins) between windows}
  \item[\textbf{Hardcopy Output Control}\hfill]{}
  \item[HARD]{sends output to the default hardcopy device}
  \item[FILE=psfile]{Send output to the PostScript file named "psfile"}
  \item[EPS]{Make an Encapsulated PostScript file.}
  \item[NOPRINT]{do not send hardcopy to the printer (PostScript only).}
  \item[LAND]{Make the hardcopy in landscape mode (default: portrait)}
\end{command}
 
CONTOUR produces a contour map of an image or set of images either on the
graphics terminal screen or on a hardcopy device.  Contour levels are user
specified or suitable defaults are applied.  Options for coordinate
transformations between the image pixels and real-world coordinates are
possible via a simple linear transformation matrix defined by the user.  It
is possible to overlap several images onto a single contour plot by
specifying multiple image buffers on the command line, or to plot them into
individual windows on the plot using special windowing keywords.
 
\noindent{Setting Contour Levels:}
 
Contour levels for the plot may specified in one of two ways.  In the
first, the levels are explicitly set by the LEVELS keyword.  Up to 40
levels may be specified in this way.  In the second method, the keyword LOW
is used to set the lowest level of the plot, and the DIFF or RATIO keywords
are used to set the other levels.  DIFF means that successive contours will
be separated by the given difference; RATIO means that successive contours
will be separated by the given ratio.
 
When multiple images are specified on the command line, use of these
keywords sets the contour levels for ALL of the images.  A technique for
plotting multiple contours on the screen or in hardcopy plots with
\textit{different} contouring levels is described under Advanced Techniques
below.

The FID keyword establishes a "fiducial" contour level for the plot. All
contours above this fiducial level are plotted as normal, unbroken lines
and all contours below this level are plotted as broken lines.  The
fiducial contour itself is plotted with a heavy solid line to distinguish
it.  The default value is FID=0.0, so that all negative contours will be
dashed by default following the convention established by radio
astronomers.
 
If the contour levels are not specified, the program computes its own in
the following way.  If multiple images are given, the the program computes
levels for each image \textit{independently}.
 
The lowest contour is the mean of the image in the area being plotted.
Subsequent contours are spaced 0.5 magnitudes (a factor of 1.583) apart, or
5 contour levels per decade of intensity.  This turns out to be a useful
default for a wide variety of astronomical applications. The mean is
computed using every fifth pixel to save time; for small images with great
variations in intensity across the image, the fifth-pixel mean will not be
a good representation of the true mean, and so the default contours may not
produce a good plot.  Use the MN command to find the true mean and set the
contours by hand.
 
\noindent{Contour Map Scaling:}
 
By default, CONTOUR will assume nothing about the image pixel scale, and
plot the axes of the contour map in pixel space, labeling the axes in rows
and columns with orientation matching the appearance on the color image
display.
 
The simplest scaling option is to specify the image pixel scale.  The SCALE
keyword sets the scale of both axes in arcsec/pixel, and sets the origin to
be the center of the portion of the image being mapped, or to the center
defined by the CEN keyword.  If SCALE is omitted, the axes are displayed in
pixel units.  The orientation of the axes is this: the units increase from
right to left on the bottom of the graph, and from bottom to top along the
left side.  This will be in the usual astronomical convention if north is
at the top of the graph and east is at the left.  Use FLIP and/or ROTATE to
get the image in this orientation.
 
A more sophisticated scaling option is available via the USER keyword. The
USER keyword sets the scaling and labelling of each axis independently
using the contents of the FITS header.  The axes are scaled according to:
\begin{hanging} 
  \item{X = CRVAL1 + CDELT1*(C-CRPIX1)}
  \item{Y = CRVAL2 + CDELT2*(R-CRPIX2).}
\end{hanging} 
X and Y axis labels are taken from the CTYPE1 and CTYPE2 cards
respectively.  Any of these FITS cards may be changed (or defined) using
the FITS or HEDIT commands.
 
The most general coordinate tranformation between the image's pixel grid
(Cols,Rows) and "world coordinates" (X,Y) may be performed using the "TR="
option.  This allows the user to specify a general transformation matrix
which includes possible rotation and shear terms.  The "TR=" keyword
requires 6 transformation matrix elements, given in the form:
\begin{hanging} 
  \item{TR=(X0,X1,X2,Y0,Y1.Y2)}
\end{hanging}
where:
\begin{hanging} 
  \item{X = X0 + X1*C + X2*R}
  \item{Y = Y0 + Y1*C + Y2*R}
\end{hanging}
are the transformation equations.  X and Y are the "world" coordinates in
which you want the plot to appear, and C and R are the image Column and Row
numbers.
 
The default scaling is equivalent to:  "TR=(0.,1.,0.,0.,0.,1.)"
 
The SCALE=s keyword is equivalent to: "TR=(C0,s,0.0,R0,0.0,-s)" where
(C0,R0) is the image central pixel.
 
The USER keyword is equivalent to:
"TR=(CRVAL1-CDELT1*CRPIX1,CDELT1,0.0,CRVAL2-CDELT2*CRPIX2,0.0,CDELT2)"
where the CXXXn are the relevant FITS cards (see USER above).
 
Be advised that this is an option for experts.

\noindent{Contour Formatting Options:}

Below are a number of options for changing the appearance of the contour
lines themselves, but not affecting the actual contour levels.
 
EXACT tells the program to use a slower contour "following" algorithm
rather than the default "fast" raster drawing algorithm.  This will result
in somewhat rougher (though perhaps truer) contours, and it handles dashed
lines far better.  In noisy cases, use of EXACT is not advised.
 
DASH draws contours for all images given as dashed lines (see LTYPE= below)
 
LWEIGHT=w: Changes the weight (line width) of the contours to W.  For
Screen plots, W is taken to be a multiple of the default pen width.  For
PostScript hardcopy plots, W is the width of the graphics pen in units of
printers points.  Default is 1.0 for screen plots, and 0.5 for hardcopy
plots.
 
LTYPE=l1,l2,...: Changes the line style (solid, dashed, etc) of the
contours for a given image.  Multiple arguments are used to set different
styles for each of the images on the command line (l1 for the first image,
l2 for the second, etc).  There are 7 line styles available:

\begin{center}
{\bf Line-Type Codes}\\
\begin{tabular}{cccc}
\hline
LTYPE & Style \\
\hline
0 & Solid    \\   
1 & Dotted     \\       
2 & Short Dash   \\     
3 & Long Dash\\
4 & Dot / Short Dash\\
5 & Dot / Long Dash\\
6 & Short Dash / Long Dash\\
\hline
\end{tabular}
\end{center}

COLOR=c1,c1,...: Changes the line color of the contours for a given image,
by analogy with the LTYPE keyword above.  The default color is white (1).
At present only the 7 "primary" graphics colors are available:

\begin{center}
{\bf Color Codes}\\
\begin{tabular}{cccc}
\hline
Code & Color & Code & Color\\
\hline
0 & background & 4 & Blue\\    
1 & foreground & 5 & Yellow\\  
2 & Red        & 6 & Magenta\\ 
3 & Green      & 7 & Cyan\\    
\hline
\end{tabular}
\end{center}

\noindent{Contour Plot Axis Formatting Options:}
 
The following keywords allow you to change the format of the axes
surrounding the contour map.
 
TITLE puts the object title (from the image header) over the contour map.
This can be changed using either CH or HEDIT to suit your needs.
 
NOLABEL draws the axes with tick marks, but does not label the axes.
 
NOAXES draws the axes, but does not draw ticks or labels (e.g., plain box)
 
NOERASE prevents erasure of the screen before beginning a plot. This may be
used to superimpose two contour plots for comparison.  NOERASE is ignored
for hardcopy, but there are comparable tricks for overplotting different
contours with different levels discussed under Advanced Techniques below.
 
\noindent{Large Size and Multi-Window Contour Plots:}
 
The following command can be used to change the position and size of the
contour plots on the graphics area.  By default, a single contour axis
window is drawn, with scaling set so as to draw the largest window on the
screen (or paper) as possible and still preserve a 1:1 aspect ratio between
the X and Y axes (e.g., it assumes square image pixels).
 
FULL draws a single plot, but without preserving a 1:1 aspect ratio.  This
keyword is useful for plotting contours of long-slit spectra, for example.
 
NH=h, NV=v: These two keywords allow the user to divide the plotting area
in to HxV windows, each of which can take a separate contour plot.  The
window convention is the same as used in the LickMongo package.  For
example: "NH=2 NV=2" will divide the screen/paper in to 4 windows, 2x2 on
the page.
 
NW=n (Single Image Contour Plotting Only): This keyword tells the contour
program to put the current image into window number N, where windows are
number in LickMongo fashion running left-to-right, and bottom-to-top
starting in the lower left-hand corner of the graphics screen/page.  This
keyword is only acknowledged if ONE image was given on the command line.
To build up a multi-window plot with NW=, use the NOERASE or NOPRINT
keywords (for screen \& hardcopy respectively -- see Advanced Techniqes
below)
 
SUBMAR=sx,sy: By default, the windows specified with NH= and NV= have
margins of white space between them to accommodate axis labels, numbers,
titles, etc.  The amount of margin allowed is fixed automatically by the
LickMongo plotting package, but the user can choose the margins with the
SUBMAR command.  This works the same as the LickMongo "SUBMARGINS"
command/subroutine.  SX and SY are scaling factors of the default margin,
thus SUBMAR=1,1 is the default state.
 
SUBMAR=0,0 will make the windows but together, with no margin between the
axes.  Thus one can make a continuous grid of windows using this keyword.
For example, a 2x3 grid of 9 windows with no independent axes would be
drawn using: "NH=2 NV=3 SUMBAR=0,0 NOLABEL" Where the NOLABEL keyword
suppresses all axis labels, but draws tick marks.  Ticks could be
suppressed by invoking "NOAXES" in place of "NOLABEL"
 
When multiple images have been specified on the command line, using the NH=
and NV= keywords (without NW=) will have the following effects:
\begin{enumerate} 
  \item If the number of images equals the number of windows (N=NH*NV),
   then the first image will go in window 1, the second in window 2, and so
   forth.  Remember that window 1 is in the lower LEFT corner, not the
   upper left as with the Mongo WINDOW command.  The NW= keyword is ignored
   All contours will have the same levels.  To have different levels is a
   bit tricker, see Advanced Techniques below.
 
  \item If the number of images is less than the number of windows, then a
  warning to that effect is issued for your information, and the windows
  are filled up until there are no more images.
 
  \item If there are more images than windows, an error message is issued.
\end{enumerate}
 
\noindent{Contour Map Hardcopy:}
 
The LickMongo plotting routines support a wide number of hardcopy devices.
The default hardcopy device for your VISTA installation is indicated at
startup, and may be changed (as appropriate) using the TERM command.  At
most installations, this will be a PostScript laser printer (like the Apple
LaserWriter).
 
HARD instructs the program to draw the contour map on a hardcopy device
instead of on the graphics screen.  No plot appears on the terminal.  If
used by itself without either the NOPRINT or FILE= keywords, the hardcopy
plot will be directed to the default printer upon completion of the plot
and no "plotting file" will be retained on disk.
 
The following keywords ONLY work when used in conjunction with the HARD
keyword:
 
LAND creates a hardcopy plot oriented in "Landscape" mode (X-axis along the
long-axis of the paper).  By default, contour plot hardcopy are given the
"Portrait" orientation.
 
FILE=psfile: By default, HARD alone creates a scratch plotting file in a
scratch disk area, and then deletes this file as soon as it is printed.
Since PostScript files might be useful for embedding contour map figures in
LaTeX or other wordprocessing program, or to send a contour map to others
via email, this keyword creates a PostScript file named "psfile" in the
user's current working directory.  When the plot is completed, if the
NOPRINT keyword is not given, the PostScript file will be closed and ready
for printing.  If NOPRINT is given, the PostScript file is not closed.
 
NOPRINT suppresses closing and printing of the hardcopy file.  This allows
the user to append successive plots the hardcopy file to facilitate
building up a composite contour map (for example: multiple contour plots,
with a different contour level set in each window).  NOPRINT is the
hardcopy equivalent of the NOERASE keyword.

\noindent{Advanced Techniques:}
 
There are two main motivations behind the redesign of CONTOUR.  The first
is to allow users to overlay 2 or more contour maps on each other in both
screen and hardcopy mode.  The second is to allow the user to make a
multi-window plot in which each image is in a separate window.  Limitations
on the VISTA command-line make it difficult to give each image in the
keyword list a different set of contour levels.  Thus, when multiple images
are plotted (either in overlay or multi-window modes), all images are given
the same contour levels.  To plot images with different contour levels
requires that CONTOUR be executed multiple times.
 
For screen plots, this is straightforward.  The first contour is plotted
using the CONTOUR program with only one image specified.  This erases the
screen and draws the axes and contours.  Then for each subsequent image,
CONTOUR is called with the NOERASE keyword so the previous contour map is
not erased from the screen.  If a multi-window plot is being generated, the
NW= keyword is used to make the next window on the screen.
 
For hardcopy, things are a bit trickier, requiring different actions for
the first, middle, and last plots.  The first plot is made using the
keywords HARD NOPRINT FILE=psfile, which opens and initializes the
PostScript file name "psfile" in the current working directory.  Subsequent
plots, EXCEPT THE LAST PLOT, are also made using the same HARD NOPRINT
FILE=psfile keywords.  For the LAST plot, only "HARD FILE=psfile" is
issued, so that when the last plot is finished, the PostScript file
"psfile" is closed and now ready to be printed at the user's leisure by
hand.
 
An example is a 2x3 (HxV) 6-panel contour plot, in which each of the 6
images uses a different contour scaling.  We want to make a PostScript file
named "montage.ps" to contain the figure, so we can print it again later
without having to regenerate it with CONTOUR each time (e.g., for a figure
for a journal article).  The images are in buffers 1-6.  We want no axis
labels or ticks, and for there be to be no white space between the plots.
We also want the images to be oriented on the page so that the 3 vertical
rows of 2 windows will be along the long-axis of the paper ("Portrait
mode").  It would go like this:

\begin{verbatim} 
   CONTOUR 1 NH=2 NV=3 SUBMAR=0,0 HARD NOAXES FILE=montage.ps |
        LOW=10 DIFF=20 NW=1 NOPRINT
   CONTOUR 2 NH=2 NV=3 SUBMAR=0,0 HARD NOAXES FILE=montage.ps |
        LOW=3.5 DIFF=10 NW=2 NOPRINT
   CONTOUR 3 NH=2 NV=3 SUBMAR=0,0 HARD NOAXES FILE=montage.ps |
        LOW=1. RATIO=1.5 NW=3 NOPRINT
   CONTOUR 4 NH=2 NV=3 SUBMAR=0,0 HARD NOAXES FILE=montage.ps |
        LEVELS=(.01,.02,.03,.05,.07,.1,.2,.3,.5,.7,.9) NW=4 NOPRINT
   CONTOUR 5 NH=2 NV=3 SUBMAR=0,0 HARD NOAXES FILE=montage.ps |
        LOW=22 RATIO=1.583 NW=5 NOPRINT
   CONTOUR 6 NH=2 NV=3 SUBMAR=0,0 HARD NOAXES FILE=montage.ps |
        LEVELS=(10,20,100,1000,10000) NW=6
\end{verbatim} 
Notice that the last invocation of CONTOUR dropped the NOPRINT keyword.
Also notice that the sequence of CONTOUR calls was uninterrupted.  This
avoids the possibility that another graphics call will intervene, causing
VISTA to forget that the file "montage.ps" is already opened and
initialized.
 
This same sequence of commands, without the "HARD FILE=montage.ps" keyword
and using NOERASE for "CONTOUR 2 ..." through "CONTOUR 6 ..." will allow
you to "preview" this hardcopy plot on the screen before making the file.
 
\noindent{Examples:}
\begin{example}
  \item[CONTOUR 2\hfill]{
Plots all of image 2.  The contour levels are set in the program.}
  \item[CONTOUR 2 HARD\hfill]{
Does the same as example 1, but sends the output to the default
        hardcopy device}
  \item[CONTOUR 2 BOX=3\hfill]{
Does the same as example 1, but plots only the part of the image in
box 3.}
  \item[CONTOUR 1 LEVELS=10,20,30,40,50\hfill]{
An example of setting the levels explicitly.}
  \item[CONTOUR 1 LOW=10.0 DIFF=20.0\hfill]{
Plots image 1 with contours at 10, 30, 50, 70, ...}
  \item[CONTOUR 1 LOW=2 RATIO=2\hfill]{
Plots image 1 with contours levels of 2, 4, 8, 16, 32, ...}
  \item[CONTOUR 1 LOW=2. RATIO=2.5 HARD USER\hfill]{
Plots image 1 with contours spaced at 1 magnitude intervals on the
default hardcopy device, and scales and labels the axes using the
FITS header of image 1.}
  \item[CONTOUR 3 LOW=20. DIFF=10. TITLE\hfill]{
Plots image 3 with contours at levels of 20, 30, 40, ... , and puts
the object name at the top.}
  \item[CONTOUR 1 2 LOW=20. DIFF=10. FID=50. HARD\hfill]{
        Plots image 1 \& 2 over each other with contours at levels of 20, 30, 
        40, ..., where the levels below 50 are plotted as dashed lines, 
        levels above 50 as normal solid lines, and level 50 as a heavier 
        solid line. The result is sent the default hardcopy device.}
  \item[CONTOUR 1 2 LEVELS=1000,5000 LTYPE=0,1 \hfill]{
        Plots images 1 and 2 over each other with contour levels 1000 \& 5000.
        Image 1 is drawn as a solid line (LTYPE=0), and image 2 as a dotted 
        line (LTYPE=1).}
  \item[CONTOUR 1 2 LOW=20. DIFF=10. COLOR=1,2\hfill]{
Plots images 1 and 2 over each other, with contour levesl 20,30,...
        Image 1 is plotted as a white line, while images 2 is plotted as
        a red line (color 2) on a color device, otherwise both appear white.}
  \item[CONTOUR 1 2 LOW=20. DIFF=10. NH=2 NV=1\hfill]{
Divides the plotting window in half horizontally, plotting image
        1 in the first (left-hand) window, and image 2 in the second 
        (right-hand) window.  Both contours have levels 20,30,...}
  \item[CONTOUR 1 2 3 4 LOW=10 RATIO=2.5 NH=2 NV=2 HARD SUBMAR=0,0 NOAXES HARD PORT\hfill]{
Divides the the page in quarters (2 windows horizontal and 2 windows
        vertical), plotting images 1-4 in separate windows, starting from
        the lower left-hand window and ending up in the upper right-hand
        window.  Only the axis box is plotted without ticks or labels, and
        the margins are set to 0,0, so all windows are joined together.
        The plot is sent to the hardcopy device as a Portrait mode plot.
All contour maps have the SAME contour levels: 10,25,62.5,165.25....}
\end{example}

%---------------------------------------------------------------------------

\section{OVERLAY: Draw contours over a Greyscale Image (PostScript)}
\begin{rawhtml}
<!-- linkto overlay.html -->
\end{rawhtml}
\begin{command}
  \item[\textbf{Form:} OVERLAY ibuf {[IBOX=b]} {[cbuf]} {[CBOX=b]} {[PROF]} 
       {[PROF=n]}\hfill]{}
  \item[{[Z=zero]} {[L=span]} {[CLIP]} {[POSITIVE]} {[NOBIN]}
        {[LEVELS=(c1,c2,c3,...)]} {[LOW=f]} {[RATIO=f]} {[DIFF=f]}]{}
  \item[{[NC=n]} {[FID=f]} {[CTHRESH=f]} {[DASH]} {[LTYPE=n]} {[LWEIGHT=f]} 
        {[COLOR=c]} {[COLOR=r,g,b]}]{}
  \item[{[EXACT]} {[TITLE]} {[BAR=xxx]} {[COMMENT]} {[COMMENT=xxx]}
        {[SCALE=s]} {[CEN=r,c]} {[LAND]} {[MACRO=xxx]} {[FILE=xxx]}]{}
  \item[{[NOAXES]} {[NOBAR]} {[LARGE]} {[INFO]} {[WIND=w,h]} {[ORIGIN=x,y]} 
        {[PAGE=L,S]} {[COPIES=n]}]{}
  \item[ibuf ]{the buffer containing the grayscale image}
  \item[IBOX=]{only consider the pixels of the grayscale image within IBOX}
  \item[cbuf]{ the buffer containing the contour image}
  \item[CBOX=]{only consider the pixels of the contour image within CBOX}
  \item[PROF]{ instead of a contour image, draw the current best fit 
isophotes in the PROFILE common block}
  \item[PROF=n]{Only draw every Nth isophote in the PROFILE common block}
  \item[Keywords controlling the grayscale image appearance:\hfill]{}
  \item[Z=zero ]{Zero point of the intensity mapping.  Default value is 0.0}
  \item[L=span]{Span of the intensity mapping.  If none is specified, the 
                default value will be taken to be 4 times the image mean.}
  \item[CLIP]{Prevent roll-over of the intensity mapping.  Default is 
              no clipping.}
  \item[POSITIVE]{Make the hardcopy White-on-Black background.  Default is 
                  Black-on-White (conventional photonegative).}
  \item[NOBIN]{Suppresses autobinning if the image is larger than about
               512x512 pixels.  Default is autobinning.}
  \item[Keywords controlling the contour map \hfill]{}
  \item[LEVELS=(c1,c2,c3,...)]{set the levels explicitly.  Up to 40 contours 
       may be given on the command line.}
  \item[LOW=c0]{Lowest contour to be drawn}
  \item[DIFF=f]{Difference between contour levels (equally spaced contours)}
  \item[RATIO=f]{Ratio between contours (for log-spaced contours).  RATIO
                 and DIFF are mutually exclusive}
  \item[NC=n]{Limit the number of contours drawn using LOW/DIFF or 
                  LOW/RATIO to N (up to a maximum of 100).}
  \item[FID=f]{Define a fiducial contour.  Levels above this value are
                  drawn with the default line type, below are dashed, and
                  the fiducial contour itself is drawn with a heavier
                  line weight.  Default fiducial is the zero contour.}
  \item[CTHRESH=f]{All contours with intensity above this level will be
                  drawn in reverse color (default: white).  Provides a means
                  for rudimentary contour "visibility" against saturated
                  colors on the grayscale.  Default is uniform color.}
  \item[DASH]{All contours are to be drawn as dashed (not just those below
              the fiducial level).  Default is solid lines above fiducial.}
  \item[LTYPE=n ]{Draw contours with LickMongo line type N (see table below)}
  \item[LWEIGHT=f]{Draw contours with line weight F (Default is 0.5)}
  \item[COLOR=c]{Draw contours with color C.  Default is black.  A color table 
                 is given below.}
  \item[COLOR=r,g,b]{ Draw contours with the RGB color given.  0 <= RGB <= 1.  
               Allows for arbitrary colors (or shades of gray).}
  \item[EXACT]{Contours are drawn using a slow contour-following algorithm.
               Default: fast rastering algorithm.}
  \item[Keywords affecting the axes of the plot\hfill]{}
  \item[TITLE  ]{Put a title (in FITS header OBJECT card for the grayscale 
                 image) on the plot.}
  \item[BAR=XXX ]{Label the intensity scaling bar with string "xxx" to 
                  indicate the units. Default label is "Intensity".}
  \item[COMMENT ]{Print a comment line on the plot.  Comment lines may}
  \item[COMMENT= ]{be up to 64 characters long. }
  \item[SCALE=s ]{Pixel scale in units/pixel for the image.  The axis ticks 
                  will be in these units rather than pixels (default).  
                  The origin is assumed to be the image center unless the 
                  CEN=(r,c) keyword is used.}
  \item[CEN=(r,c) ]{The center of the image in pixels for use with the SCALE=s 
                  keyword.  Default center is the image array center, unless 
                  PROF or PROF= was called, then the isophote center is used }
  \item[MACRO=xxx]{External file with LickMongo commands to be executed 
                  after the plot is completed.  Provides a facility for 
                  customizing axes and adding annotation.}
  \item[LAND ]{Orient the plot in landscape mode.  Default is Portrait.}
  \item[FILE=xxx]{Direct the PostScript Image into a file named xxx.  By 
                  default, OVERLAY writes the image into IMAGE.PS in the 
                  current working directory.}
  \item[NOAXES ]{Supress plotting of coordinates axes on the image.}
  \item[NOBAR ]{Supress plotting of the intensity scaling bar}
  \item[LARGE]{Expand the plot as large as possible for the given paper size
               while still preserving the 1:1 axis aspect ratio.}
  \item[INFO]{Write in a line of auxilliary info along the bottom of 
              the page in small type.}
  \item[Advanced Output Format Control\hfill]{}
  \item[WIND=(w,h)]{The maximum possible size of the plotting window in inches.
                  This defines the largest region that OVERLAY will try to 
                  fit the image into.  Defaults:\newline
                         Portrait:  6" x 6"\newline
                         Landscape: 8" x 5"\newline
                  Must be smaller than the paper dimensions (see PAGE).}
  \item[ORIGIN=(x,y)]{origin of the plotting window in inches, measured from 
                  the lower left-hand corner of the page.  X is horizontal, 
                  and Y is vertical.  Default centers the window.}
  \item[PAGE=(L,S)]{physical size of the paper in inches.  "L" = long 
                  dimension, "S" = short dimension.  Default: (11,8.5)
                  Used if making a plot for an odd-sized paper printer.}
  \item[COPIES=n]{Number of copies to print.  Default is 1 copy.  Makes the
                  generation of multiple 1st generation copies efficient.}
\end{command}

The OVERLAY command creates a PostScript file displaying one image as a
grayscale map with a contour map of a second image superimposed upon it.
Contours can be from an image in a VISTA buffer or the output of one of the
VISTA isophote fitting routines (e.g., PROFILE).  OVERLAY has many
keywords, most an amalgamation of the basic keywords used by the POSTIT and
CONTOUR commands.  The PostScript file is written into the user's current
working directory, named either "image.ps" or the name given by the user
with the FILE= keyword.  PostScript files generated by OVERLAY conform
closely to both the general and "encapsulated" PostScript standards, and so
may be printed on most PostScript laser printers (both monochrome and
color), or included in documents or other programs that can import
PostScript image files.  We suggest using a workstation PostScript
"previewer" program (of which the GhostView program provided public domain
by the Free Software Foundation is an excellent choice) will save much time
and wasted paper.

One barrier to making this routine fully general is a lack of a fully
developed standard for specifying the mapping from pixels to some kind of
world coordinate system (e.g., RA and Dec) in FITS format images.  To get
around this problem for the present time, OVERLAY makes some restrictive
assumptions about the relatives spatial scaling and registration of the two
images.  For argument, we will call the buffer that contains the image to
be rendered as a grayscale map as the "gray image", and the buffer with the
image to be rendered as a contour map as the "contour image".  Eventually,
the FITS WCS conventions will become common enough to implement them here.

The first assumption is that while the gray and contour images do not have
to have the same sizes in pixels, they are assumed to map into the exact
same region in world coordinates.  Thus, given a 300x300 pixel gray image
that spans 30-arcseconds on a side, if the contour image is 512x512 pixels
in size, it will be assumed that 512 pixels maps into 30-arcseconds as well
(or, put another way, will assume that the pixel scale of the contour
buffer is 300/512 times that of the gray image).

The second assumption is that the two images have the same aspect ratio.
This is trivial if both images are square (buffer sizes of 300x300 and
512x512 pixels, respectively, for example).  If they are NOT square, then
OVERLAY assumes that the gray image will define the box, and that the
contour image fits snugly into it.  In other words, the contour image is
not allowed to overflow the gray image.  When there is an argument about
sizes, the gray image is the reference (and wins by default).

The third assumption is that the two buffers are perfectly aligned (or
registered) spatially.  Thus if the gray image is a B-band image and the
contour image is a K-band image, it assumes the stars will line up
perfectly when the arrays are overlaid.

These assumptions can be controlled by a judicious choice of boxes.  The
IBOX= keyword specifies the box to use for the gray image, and the CBOX=
keyword is used for the contour image.  A further trick is to pick one
image as the fiducial and shift the other image to line up.  For example,
if overlaying a radio map with a strong nuclear peak on an optical image of
a galaxy, shift both images so that the centroid falls exactly on the
center of a pixel.  Remember that the registration of the images by this
program is assumed to be on an integer grid.

Black contours drawn against a black pixels are not too effective.  To
provide a first-cut at getting around this problem the user can set the
CTHRESH= keyword to an intensity value in the contour buffer above which
the color of the contours is reversed (i.e., drawn as white).  This
improves the contour "visibility" against a dark background.  At present,
the LickMongo contour "image model" does not make it easy to compare the
two images and choose the color dynamically (volunteers?).  All contours
with value above the indicated threshold are drawn in reverse color, while
contours below this value are drawn as black.

A further aid to improving visibility is the NC= keyword that is used to
limit the number of contours drawn when the LOW= and DIFF=/RATIO= keyword
pairs are used to specify the contour levels.  This suppresses the drawing
of many high-level contours (e.g., for bright stars when the primary
contouring object is some faint extended source).  Such contours can crowd
together, blotting out any underlying features.  At present, the ultimate
limit is 100 contours.  Restricting the number of contours drawn can also
reduce the size of the final PostScript file by eliminating unnecessary
high levels.

\noindent{PROFILE Plotting}

Using OVERLAY with the PROF or PROF=n keywords is the PostScript analog of
the TVPROF command.  In this case, PROF is used in place of a buffer with a
contour image (note that PROF and a contour image buffer are mutually
exclusive).  The profile parameters to be plotted as elliptical isophotes
are taken from the PROFILE common block, either loaded using one of the
fitting programs like PROFILE, or loaded from a SAVE'd profile computation
with the GET command.

The optional argument of the PROF keyword tells OVERLAY to only plot every
Nth isophote for clarity.  If the CTHRESH= keyword is given, the color of
the isophote will be reversed if the value of the mean surface brightness
of that counter is greater than or equal to CTHRESH.  This allows the user
to plot inner contours that would otherwise get lost against the dark inner
galaxy as white lines, making them "visible".  A fancier contour visibility
algorithm will take more time to come up with (volunteers?).


\noindent{Axis Plotting}

A number of keywords are provided to allow customization of the plotting
axes.  These are identical in function to those familiar from the POSTIT
and CONTOUR commands.  By default, the pixel coordinates of the gray image
are used.  The SCALE= and CEN= keywords can be used to specify an image
scale (arcseconds/pixel) and fidicual image center (coordinates in (R,C)
corresponding to (0,0) in arcseconds), again referenced to the gray image.
If the PROF keyword is used to overlay best-fit isophotes, the CEN= keyword
will be overridden by the isophote centers stored in the PROFILE common
block, if and only if the SCALE= keyword is given.  The NOAXIS keyword will
suppress plotting of any axes, only drawing a box around the gray image.

A more sophisticated option for customizing the axes, and for adding labels
or other annotation is the MACRO= keyword that executes LickMongo plotting
commands stored in an external file.  See the help file for the POSTIT
command for some details on how this is used.


Additional commands are provided for changing the size and orientation of
the plot on the paper.  These are identical to the keywords given in the
POSTIT command, and described in detail there.

\noindent{Special Goodies}

Users with access to color PostScript printers can define contours in
colors, either using the COLOR=n keyword to select a preset color, or
COLOR=r,g,b to roll your own custom color.  Users without this facility can
cheat and use the COLOR=r,g,b keyword to make contours a shade of gray
different then black, enhancing their visibility against a dark background
(within limits imposed by your printer engine).  The preset colors are as
follows (same table as CONTOUR):

\begin{center}
{\bf Color Codes}\\
\begin{tabular}{cccc}
\hline
Code & Color & Code & Color\\
\hline
0 & background & 4 & Blue\\    
1 & foreground & 5 & Yellow\\  
2 & Red        & 6 & Magenta\\ 
3 & Green      & 7 & Cyan\\    
\hline
\end{tabular}
\end{center}

RGB colors are specified with 0<=R,G,B<=1, and to make a shade of gray, all
three color levels are specified the same.  For example, a 50\% gray color
is COLOR=0.5,0.5,0.5, and 50\% red would be COLOR=0.5,0.,0.

Different line types or line widths may be chosen using the LTYPE= and
LWEIGHT= keywords.  The default line width is LWEIGHT=0.5, whereas
LWEIGHT=1.0 is equivalent to the weight of the axes drawn around the
image box.  The LTYPE code is a standard LickMongo line style code:

\begin{center}
{\bf Line-Type Codes}\\
\begin{tabular}{cccc}
\hline
LTYPE & Style \\
\hline
0 & Solid    \\   
1 & Dotted     \\       
2 & Short Dash   \\     
3 & Long Dash\\
4 & Dot / Short Dash\\
5 & Dot / Long Dash\\
6 & Short Dash / Long Dash\\
\hline
\end{tabular}
\end{center}

\noindent{Examples:}
\begin{example}
  \item[OVERLAY 1 2 Z=0. L=100. CLIP LOW=30. RATIO=1.5 SCALE=0.32 FILE=n1068.ps\hfill]{
   Plots the image in buffer 1 as a grayscale map with values running from
   0. to 100.0 rendered as white-to-black, and draws on top of that figure
   the image in buffer 2 as a contour map with contours running from 30.0
   an spaced geometrically every factor of 1.5 in intensity.  The axes
   are scaled as 0.32 arcseconds/pixel referenced to buffer 1, with the
   origin (0,0) assumed to be the center of buffer 1.  The output is
   directed to a file called n1068.ps in the current working directory
   for later printing at the user's discretion (there is no autoprinting).}
  \item[OVERLAY 1 1 Z=0. L=100. CLIP LOW=20. DIFF=20. SCALE=0.32 FILE=graycon.ps\hfill]{
   This plots the image as a grayscale, with the given Z and L, and then
   plots on top of this the image intensity contours, starting at 20 and
   spaced every 20 intensity units.  The scale is 0.32 arcsec/pixel and
   the origin (0,0) is the image center.  Output is sent to a file
   called graycon.ps  (old AIPS users will recognize this kind of figure)}
  \item[OVERLAY 1 PROF=3 Z=0. L=1000. CLIP FILE=isofit.ps\hfill]{
   The image in buffer 1 is rendered as a grayscale plot between
   intensities (0:1000), and the current best-fit isophotes stored in
   the PROFILE common block are superimposed.  For clarity, only every 
   third isophote is  plotted (including the first and last isophote).
   The output file is called "isofit.ps".}
  \item[OVERLAY 1 PROF=3 Z=0. L=1000. CLIP CTHRESH=1000. FILE=isofit.ps\hfill]{}
  \item[\hfill]{Same as (3) above, but now all best-fit isophotes with a mean
   surface brightness >= 1000. are plotted as white lines to make
   them visible against the black image pixels in the center.}
  \item[OVERLAY 1 IBOX=2 4 CBOX=1 Z=0. L=1000. 
        CLIP LEVELS=(.1,.2,.3,.5,.7,.9)\hfill]{}
  \item[SCALE=0.51 CEN=(AXR,AXC) FILE=hacon.ps\hfill]{
   The parts of buffer 1 within box 2 are rendered as a grayscale map
   between intensities (0:1000), with the contour levels of the parts
   of buffer 4 within box 1 are superimposed.  The axes are scaled to 0.51
   arcseconds/pixel, with the image center given by the results of a recent
   AXES calculation stored in the AXR and AXC variables (see AXES).}
  \item[OVERLAY 1 IBOX=2 4 CBOX=1 Z=0. L=1000. 
        CLIP LEVELS=(.1,.2,.3,.5,.7,.9)\hfill]{}
  \item[NOAXES MACRO=hacon.mgo FILE=hacon.ps\hfill]{
   Same as (5) except now no axes are drawn by OVERLAY, and instead the
   external LickMongo macro file "hacon.mgo" in the current working 
   directory is used to draw the axes and put on custom labels.  Output
   is directed to the PostScript file hacon.ps.}
  \item[OVERLAY 1 PROF=3 Z=0. L=1000. CLIP FILE=isofit.ps COLOR=3\hfill]{}
  \item[LTYPE=2\hfill]{
   Same as (3) above, but this user has a color PostScript device, and
   is plotting the best-fit isophotes as dashed green lines (no accounting
   for taste).  }
  \item[OVERLAY 1 PROF=3 Z=0. L=1000. CLIP FILE=isofit.ps 
        COLOR=.7,.7,.7\hfill]{} 
  \item[LTYPE=2\hfill]{
   Having the plot made in (7) thrown back by his co-authors, the user
   now draws the best-fit isophotes as 70\% gray dashed lines.}
\end{example}

\section{HISTOGRAM: Display a Histogram of Image Pixel Values}
\begin{rawhtml}
<!-- linkto histogram.html -->
\end{rawhtml}
\begin{command}
  \item[\textbf{Form:}   HISTOGRAM source {[BOX=b]} {[NOLOG]} {[BIN=n]} {[XMIN=x1]} {[XMAX=x2]}\hfill]{}
  \item{{[YMIN=y1]} {[YMAX=y2]} {[HARD]} {[WIND=n]} {[BUF=buf]}}
  \item{{[NOERASE]} {[PORT]}}
  \item[source\hfill]{   specifies the image.}
  \item[BOX=b\hfill]{limits the computation to those pixels in box 'b'.}
  \item[NOLOG\hfill]{displays the number of pixels at each intensity,
rather than the logarithm.}
  \item[BIN=n\hfill]{   bins the image values by the specified factor.}
  \item[XMIN, XMAX\hfill]{  limits the computation to those pixels with values
between x1 and x2, inclusive.}
  \item[YMIN, YMAX   \hfill]{   limits the display of the histogram on the Y-axis
to be from y1 to y2.}
  \item[HARD\hfill]{sends the plot to the hardcopy device.}
  \item[WIND=n\hfill]{put the plot in window n of a 2x2 grid}
  \item[BUF=buf\hfill]{  load the histogram data into image buffer 'buf'}
  \item[   NOERASE   \hfill]{don't erase screen before plotting.}
  \item[PORT\hfill]{make hardcopy output in portrait mode (default: landscape)}
\end{command}

This program displays a histogram of an image, plotting the logarithm of
the number of pixels at each value for the image 'source'.

Use the BIN word to specify how wide the intensity intervals show in the
plot is to be.  Normally the binning factor is 1, meaning that the plot
displayed is the logarithm of the number of pixel values at each intensity
(the image values are converted to integers).  If the bin factor is
non-zero, the display is the log of the number in larger bins.  For
example, if the bin was 5, and the minimum value in the image is 0, then
the plot shows the number of pixels with intensity 0 - 4, 5 - 9, 10 - 14,
15 - 19, etc. If there is a large range in intensities, the BIN word should
be used to keep the plot from having so many points that it looks like
hash. If the number of points in the histogram is larger than 2048, the
program will increase the bin size to reduce the number of points in the
plot.

The BOX word limits the calculation to those pixels in the specified box.
The XMIN and XMAX words limit the calculation to those pixels in the
specified intensity range.  If XMIN is not given, the lower limit will be
the minimum pixel value in the image.  If XMAX is not given, the upper
limit will be the maximum pixel value in the image.

The YMIN and YMAX words, by contrast, limit the DISPLAY of the histogram so
that the Y-axis runs over the given range.  These words do not change the
calculation in any way.  If YMIN is not used, the lower limit of the
display will be the smallest number of pixels in the image that have a
given value (often this is zero pixels at many intensities).  If YMAX is
not used, the upper limit of the display will be the largest number of
pixels which have a certain intensity.
       
NOLOG makes the plot show the actual number of pixels at each intensity,
rather than the logarithm.  When the logarithm is computed in the default
option, intensities with no pixels are given the value 0, so you cannot
distinguish an intensity with 1 pixel and an intensity with 0 pixels unless
you use the NOLOG word.

The NOERASE keyword suppresses the erasure of the screen before plotting.
Up to 4 histograms may be plotted on a single output page using the WIND=n
keyword, where n=(1,2,3,4), with windows numbered from left-to-right,
bottom-to-top in the conventional MONGO order.  Hardcopy is flushed to the
printer only after WIND=4 is used.  The plot may be printed in portrait
mode (long axis of the paper vertical) using the PORT keyword.

The histogram for an image may be loaded into an image buffer as a
"spectrum" (1-D image) for further analysis or storage for use with
external programs using the BUF=buf keyword.

\noindent{Examples:}
\begin{example}
  \item[HISTOGRAM 4 \hfill]{Shows the histogram for image 4}
  \item[HISTOGRAM \$Q BOX=3 \hfill]{Shows the histogram for image Q 
       (where Q is a variable) using only the pixels in box 3}
  \item[HISTOGRAM 2 XMIN=1000 XMAX=1999 \hfill]{
       Shows the log of number of pixels with values between 1000 and 1999.}
  \item[HISTOGRAM 4 NOLOG \hfill]{
       Shows the number of pixels (not the log) at each value in image 4.}
  \item[HISTOGRAM 3 HARD\hfill]{
       computes a histogram for image 3, sending it to hardcopy printer.}
  \item[HISTOGRAM 2 XMIN=1000 XMAX=1999 BUF=10\hfill]{ same as \#3 above,
       except that instead of plotting the histogram, it is loaded into image
       buffer 10 as a spectrum.}
\end{example}

\section{PLOT3D: Make a 3-D Perspective (mesh) Plot}
\begin{rawhtml}
<!-- linkto plot3d.html -->
\end{rawhtml}
\begin{command}
  \item[\textbf{Form: }PLOT3D image {[BOX=b]} ALT={[alt]} AZ={[az]} 
   ZFAC={[zfac]}\hfill]{}
  \item{ZOFF={[zoff]} NOERASE HARD}
  \item[image]{buffer to be plotted}
  \item[BOX=b]{only plot the image section within the given box}
  \item[ALT=]{viewing altitude}
  \item[AZ=]{viewing azimuth}
  \item[ZFAC=,ZOFF=]{Z-axis scale factor and offset}
  \item[SCALE=]{scale factor to apply to ZFAC and ZOFF}
  \item[NOERASE]{doesn't clear screen first}
  \item[HARD]{generates hardcopy}
\end{command}

PLOT3D (also THREED or 3D) makes a 3 dimensional perspective plot of the
specified buffer. The viewing altitude and aziumuth can be specified with
the ALT= and AZ= terminal; if unspecified, they default to 45 and 225
degrees, respectively. The vertical scale is done automatically, but can be
overridden by use of the ZFAC and ZOFF keywords. See the Mongo manual for
more details. NOERASE prevents erasure of the screen before making the
plot. HARD generates hardcopy.

The 3-D plots that LickMongo makes can take a long time to draw.  Once the
plot starts, there is no way to stop it until it is completed.
Consequently, we suggest that you only use it on small sections of images,
smaller than 100 x 100 pixels. The BOX= keyword makes it easy to select
small image sections. If you enter an image larger than 100x100, the
program will confirm that you really wish to continue before doing so.

Some control over the vertical relief plotted is possible using the ZFAC
and ZOFF keywords.  These affect how the minimum and maximum data values
are fit inside the virtual cube being plotted.  By changing these by hand,
you can enhance or depress the vertical stretch of the plot (ZFAC) and move
the plot up and down vertically within the plotting window (ZOFF).

If these are too arcane for you, then SCALE can be used instead to tweak
the defaults computed by the routine from the data values, where SCALE=1 is
equivalent to accepting the defaults, SCALE$>$1 will increase relief, and
SCALE$<$1 will depress relief.


\section{CLEAR: Clear the Terminal or Image Display}
\index{Clear screen}
\index{Terminal!Clear screen}
\begin{rawhtml}
<!-- linkto clear.html -->
\end{rawhtml}
\begin{command}
  \item[\textbf{Form: } CLEAR {[TEXT]} {[VEC]} {[IMAGES]}\hfill]{}
\end{command}

With no keywork or with the TEXT keyword, this command clears the command
terminal screen.

With an X11 display window, the VEC keyword will clear the vector (overlay)
list and redisplay the current images. The IMAGES keyword will clear the
image list which is accessible with the $+/-$ keys, resetting the list so
that only the current image in included.


\section{TEXT: Embed Permanent Text on an Image}
\index{Display!Embed permanent text on an image}
\begin{rawhtml}
<!-- linkto imtext.html -->
\end{rawhtml}
\begin{command}
  \item[\textbf{Form: }TEXT  imbuf {[COL=c ROW=r]} {[X0=x Y0=y]} 
       {[TEXT=s]}\hfill]{}
  \item[]{{[SCALE=s]} {[ROT=ang]} {[FILL=f]}}
  \item[imbuf]{   buffer with the image to be tagged.}
  \item[X0=,Y0=]{coordinates of the text string (upper left corner of
       the text box) in pixels (X0 \& Y0 are columsn \& rows, respectively).}
  \item[TEXT=s]{Text string to embed (can be a string variable)}
  \item[SCALE=f]{Text scale factor (default=1)}
  \item[ROT=ang]{Text rotation angle in degrees (default=0)}
  \item[FILL=]  {Text Fill (default=0=nofill, 1=fill with zeros)}
\end{command}

TEXT is used to add a permanent annotation text to an image by ``poking''
pixel values to render the text.

The coordinates (X0=, Y0=) give the location in pixels of the upper
left-hand corner of the text box.  You can compute centering of a string
using the cell size and the number of characters in the string.  The TEXT=
keyword is used to give the text string to embed, but if omitted the
program will prompt for the input string.  The text may include any valid
ASCII keyboard characters.

Text is embedded on the image as a string of 8x11-pixel ASCII characters by
changing the data values of the pixels to 32766.  The height of an
uppercase M is 10 pixels height of a lowercase x is 6 pixels.  Each
character cell has 1 pixel of white space between it and the surrounding
characters, and one pixel of white space below for descenders (e.g., on y
and g).  The size of the text may be scaled larger using the SCALE=
keyword.

Data pixels under the characters proper are lost, but data between
character pixels is untouched.  If FILL=1 is given, however, the inter-text
pixels are set to 0.0 This provides a way of ``boxing and filling'' the
text so it does not conflict with background data.

The ROT keyword is used to rotate the text, in increments of 90 degrees.
Since we are using a simple text rendering, odd angles do not work
satisfactorily.

Note that TEXT makes \textbf{PERMANENT} changes to the data values in the
image being annoted.   To make only temporary labels (drawn on the image
display's ``vector'' plane, not by changing image pixels proper), see the
TVPLOT command.


\chapter{Advanced VISTA Command Syntax}
\begin{rawhtml}
<!-- linkto command.html -->
\end{rawhtml}

%
% minor reformatting and editing
% used verbatim environment for some examples (will process as <PRE> tags
% in HTML
% rwp/osu 98Jul25
%

\section{Abbreviating VISTA Commands}
\begin{rawhtml}
<!-- linkto abbrev.html -->
\end{rawhtml}
\index{Command Abbreviation}

You do not need to type the full command name to execute a command -- all
that is needed is the shortest unambiguous string which defines a command.
For example, the command FITSTAR can be executed with FITSTAR, FITSTA,
FITST, ... etc.  If it turns out that more than one command begins with the
command that you type, all those commands will be listed, and none
executed.

The question mark command can be used to list those commands which begin
with a certain pattern.  Type '?' followed by the pattern you are
examining.  There should be no spaces between the ? and the pattern.
Examples:

\begin{example}
  \item[?A\hfill]{List all commands beginning with A}
  \item[?MA\hfill]{List all commands beginning with MA}
\end{example}

Keywords and command arguments CANNOT be abbreviated.

The ? operator may also be used as an abbreviation for the HELP command.
To receive help on a given topic, type ?, followed by a space, followed by
the topic about which you wan help.  Some examples:

\begin{example}
  \item[? MASH\hfill]{Print help on the command MASH}
  \item[? HELP\hfill]{Print help on the command HELP}
  \item[?\hfill]{Help on everything.}
\end{example}

\section{Output Redirection}
\begin{rawhtml}
<!-- linkto redirect.html -->
\end{rawhtml}

\index{Data,Printing in file}
\index{Data,Printing on lineprinter}
\index{Output Redirection Mechanism}

Many (but not all!) programs that produce large amounts of information may
have their output REDIRECTED by the user.  The output from these programs
normally goes to the terminal, but instead can be written to a file or to
the line-printer.

To redirect the output, you use the '$>$' or '$>>$' constructions at the
end of a valid command.  A single $>$ will write the output to a new file
(overwriting one if it already exists).  A $>>$ will append to an existing
file, or create a new file if the named file does not exist. If you use the
output redirection '$>$lp:', then the output will be printed on your
printer (using the lpr command to the default printer).

Recall that you cannot have any spaces in a word on the command line; this
applies to output direction as well. Do not put a space after the $>$ and
before your file name!

\noindent{Examples:}

\begin{example}
  \item[PRINT PHOT $>$first.lis\hfill]{Prints the contents of a 
       photometry file into the file 'first.lis'.  The file will be
       located in the current directory.}
 
   \item[HELP MASH $>>$help.xxx\hfill]{Appends the help information for
        MASH to the end of file 'help.xxx', if it exists.  If it does not,
        the file is created.}
\end{example}

\section{HISTORY: Recall the Last Several Commands }
\begin{rawhtml}
<!-- linkto history.html -->
\end{rawhtml}
\index{Commands!Show list of recent commands}

\begin{command}
  \item[\textbf{Form: }HISTORY (output redirection)]{}
\end{command}

The command HISTORY will show on your terminal the last 50 commands that
you have executed.  Use this to find commands that you want to repeat with
the \% substitution character.

The output from HISTORY can be redirected.  As an example:
\begin{hanging}
  \item{HISTORY $>$history.txt}
\end{hanging}
prints the recent command history to the file history.txt in your
current working directory.

\section{\%: Repeating/Editing a Previous Command}
\begin{rawhtml}
<!-- linkto repeat.html -->
\end{rawhtml}
\index{Commands,Repeat previous command}

You can repeat a command in VISTA without typing it over by using the \%
character.  The \% character comes before the command you want to repeat.
It is also possible to modify a command while you repeat it.

The \% command comes in several forms:
\begin{itemize}
  \item{Typing '\%' by itself repeats the last command that you executed. }

  \item{Typing '%n' where 'n' is a number repeats command n.  The numbers
   of the command can be found by the HISTORY command.}

  \item{Typing '\%string' repeats the last command beginning with 'string'.}
\end{itemize}

As an example, suppose the HISTORY command gives the output:
\begin{verbatim}
   10  RD 1 [MYDIR]HD183143
   11  MN 1
   12  ADD 2 1
   13  PLOT 1 R=342 MIN=100.0 MAX=300.0
   14  MASH 4 1 SP=100,103
   15  PLOT 4
\end{verbatim}
Then
\begin{example}
   \item[\%\hfill]{repeats command 15 (the last command):  PLOT 4}
   \item[\%10\hfill]{repeats command 10:  RD 1 [MYDIR]HD183143}
   \item[\%MASH \hfill]{repeats command 14 (the last MASH command): 
       MASH 4 1 SP=100,103}
   \item[\%M\hfill]{repeats command 14:  MASH 4 1 SP=100,103}
\end{example}

Note the last example: The history mechanism will repeat the last command
which begins with 'M'.  If there are several commands which begin with the
same letter, you have to supply enough of the command name after the \% to
uniquely specify the command to be repeated.

It is possible to modify a previous command using the \% construction
before executing the command again. There are several ways in which this
modification can be done:

\begin{enumerate}
  \item{You can ADD words to the command by typing them after you enter
       '\%command', provided that these words are not keyword values
       (EXPRESSION=VALUE) which are already present in the previous
       command.}

  \item{You can modify keyword values which already exist in the previous
       command by simply repeating the keyword with a new value.  For
       instance, if the old command was "PLOT 1 R=200 XS=200 XE=300
       MIN=400", and you re-ran the command with "\%PLOT R=234", the new
       command would be "PLOT 1 R=234 XS=200 XE=300 MIN=400"}

  \item{You can force a keyword value to be added to or deleted from the
       previous command (and not substituted as described above) by
       preceding the keyword with a plus (+) or minus (-) sign.}
\end{enumerate}

As an examples of command modification with \%, assume you wanted to repeat
in various ways the command
\begin{hanging}
  \item{PLOT 1 R=50}
\end{hanging}
which, we will assume is number 10 in the history list.
\begin{example}
  \item[\%10 HARD\hfill]{does PLOT 1 R=50 HARD}
  \item[\%10 R=55\hfill]{does PLOT R=55}
  \item[\%10 R=55 HARD\hfill]{does PLOT R=55 HARD}
  \item[\%10 XS=100 XE=200\hfill]{does PLOT 1 R=50 XS=100 XE=100}
\end{example}
Now suppose that command number 58 was
\begin{hanging}
  \item{PLOT 3 PIXEL HARD XS=100 XE=300 MAX=4096.0}
\end{hanging}
Then
\begin{example}
  \item[\%58 -PIXEL -HARD\hfill]{does PLOT 3 XS=100 XE=300 MAX=4096.0}
\end{example}

Note: When using the '-' sign to delete words, you must supply an exact
match to the word being deleted.

\section{Extending Commands Past One Line}
\begin{rawhtml}
<!-- linkto extend.html -->
\end{rawhtml}
\index{Commands,Extending a command}

A VISTA command is ordinarily terminated by typing a RETURN, thus limiting
the normal command length to one line of 80 characters.  However, you can
continue a command onto the next line by ending the current line of
characters with the $|$ character.  The last non-blank character must be a
$|$ for your command to be extended.  You will be prompted for the rest of
the command with a colon on the next line. Example:

\begin{hanging}
  \item{ RD 1 /mydirectory/$|$}
  \item{really\_long\_file\_name SPEC}
\end{hanging}

There is a limit to the length of a command permitted even when using the
$|$ character, and that limit is 500 characters.

\section{ALIAS/UNALIAS: Define/Delete Command Aliases}
\begin{rawhtml}
<!-- linkto alias.html -->
<!-- linkto unalias.html -->
\end{rawhtml}
\index{Commands,Defining synonyms}
\index{Commands,Removing synonyms}
\index{Synonyms,Defining or deleting synonyms for commands}

\begin{command}
  \item[\textbf{Form: }ALIAS {[synonym]} {[command]} {[output
       redirection]}\hfill]{} 
  \item[\textbf{Form: }UNALIAS {[synonym]}\hfill]{}
\end{command}

The ALIAS and UNALIAS commands are used to define and delete synonyms
(``aliases'') for commands, respectively.  The aliasing mechanism can save
you typing if you have several commands that have to be used repeatedly.

ALIAS (with no arguments) shows a list of synonyms of your terminal.  The
list can be loaded into a file or sent to the printer with the output
redirection mechanism.

To define an alias, use the full syntax:
\begin{verbatim}
  ALIAS SYNONYM COMMAND
\end{verbatim}
For example:
\begin{hanging}
  \item{ALIAS T 'TV 1 1234.0 CF=NEWTHREE'}
\end{hanging}
defines a new command, T, which executes 'TV 1 ...' . As is usual, if the
command for which you are defining an alias is composed of more than one
word, the entire command must be enclosed in quotes.

If you give only the new alias and no not give the command, the program
will ask for it.  You can redefine a synonym at any time.

To execute the command that you have defined with ALIAS, type the synonym
just as you would type any other command.  You can add keywords to the
command at the time of execution by typing them after the synonym.  Using
the example above, the command
\begin{hanging}
  \item{T BOX=1}
\end{hanging}
executes TV 1 1234.0 CF=NEWTHREE BOX=1.  The command 'T' goes on the
history list. 

The aliases that you define need not be commands that would actually run;
they MUST contain a command name and MAY contain keywords, but these are
the only restrictions.  For example:
\begin{hanging}
  \item{ALIAS H 'HISTOGRAM'}
\end{hanging}
is a perfectly valid alias.  Typing H by itself as a command would not
work, since HISTOGRAM is not a complete command (it needs an image number).
You can, however, add arguments to an alias exactly the same as with
commands, hence:
\begin{hanging}
  \item{H 3\hfill}
\end{hanging}
will execute the HISTOGRAM command on image 3.

To delete an alias, use the UNALIAS command.  Type UNALIAS followed by the
name of the synonym that you want to remove from the list.  As an example,
UNALIAS T will remove the definition of T that was defined above.

It is not possible to define a alias that includes other aliases.  For
example, the sequence
\begin{verbatim}
  ALIAS A  'TV 1 1234.0 CF=NEWTHREE'
  ALIAS AA 'A BOX=1'
\end{verbatim}
correctly defines the command alias 'A', but AA will be invalid because it
contains the alias 'A'.

Similarly, you cannot define a single alias made up of several commands
chained together by a semicolon (;).  For example:
\begin{hanging}
  \item{ALIAS ALL 'RD 1 ./a.fits; MN 1; PLOT 1 R=50; BUF'}
\end{hanging}
is not allowed: \textbf{one command per alias}.

You can have aliases defined automatically on startup by putting ALIAS
commands in the startup procedure.  See the sections FILES and PROCEDURE
for information about the startup procedure.

\section{EDIT: Editing the Last Command}
\begin{rawhtml}
<!-- linkto edit.html -->
\end{rawhtml}

\index{Command,Edit last command}

\begin{command}
  \item[\textbf{Form: }EDIT\hfill]{}
\end{command}

The EDIT command loads the last command into a temporary file in your
current directory, and then puts you into the default editor.  On Unix
systems, the default editor is vi.  If you leave the editor with saving
changes, the edited command is immediately executed.  If abort the editor
(exiting without saving changes), the command is not executed. Only the
first line of the temporary file is read; do NOT use EDIT to create several
lines, thinking that you are making a procedure.  Procedures are created
with PEDIT.

Do not put more than one command in the temporary file you are editing.  Do
not chain several commands together with semicolons.

For EDIT and other commands which invoke an editor (HEDIT, WEDIT, PEDIT),
the default is to execute the editor vi. If you wish to change this, you
can do so by setting the environment variable VISUAL to be the editor that
you wish to use. For example, to use the emacs editor, execute the
following statement BEFORE starting up VISTA (e.g., in your .cshrc file):
\begin{verbatim} 
  setenv VISUAL emacs
\end{verbatim}
In many cases, you may have to provide the full path to an editor, for
example:
\begin{verbatim} 
  setenv VISUAL /opt/local/bin/emacs
\end{verbatim}
in order to be able to execute it from within VISTA.  It is always a good
idea to verify the full path (using the Unix which command) before setting
an environment variable.

\section{SETDIR: Set the VISTA Default Directories and File Extensions}
\index{Image!Set default directory}
\index{Spectra!Set default directory}
\index{Directories!Set default directories}
\index{Extensions!Set default for files}
\index{Files!Default directories and extensions}
\index{Procedures!Set default directory}
\begin{rawhtml}
<!-- linkto setdir.html -->
\end{rawhtml}
\begin{command}
  \item[\textbf{Form: } SETDIR code {[DIR=directory\_name]} 
       {[EXT=extension]}\hfill]{}
  \item[code]{specifies which directory is being set or changed}
  \item[DIR= ]{   specifies a directory for the type of object
       indicated by the code.}
  \item[EXT=]{gives the extension for files in the default directory}
\end{command}

SETDIR sets the default directories and extensions of files storing images,
spectra, color maps, etc.  You can see the default values with the command
PRINT DIRECTORIES.  See the section FILES (type HELP FILES if you're on a
terminal) for information about default directories and extensions.  See
also the command CD to change the current working directory (./) on a UNIX
system.

The DIR word gives the default directory for the type of object specified
by the code, and the EXT word gives the extension for that type of object.
An example of a default extension is that for '.fits' for images.  An
example of a default directory is 'ccd/spec' for spectra.  You must specify
either the directory or the extension or both with SETDIR. If the extension
is not blank, it must include a period as its first character: For example,
'.xyz' is a valid extension, while 'flk' is not.

The 'code' gives the directory which is being set or changed. The code is
derived from the type of object in the directory you are specifying.  You
must type at least the first two letters of the code:

\begin{center}
\begin{tabular}{llll}
Directory&Code&Abbrev&Notes\\
\hline
Images&IMAGES&IM&(user defined, usually ./)\\
Spectra&SPECTRA&SP&(archaic, use IMAGES)\\
Procedures&PROCEDURES&PR&(user defined)\\
Data Files&DATA&DA&(user defined, usually ./)\\
DAOPHOT files&PHOTOMETRY&PH&(user defined)\\
Flux calibration files&FLUX&FL&(assigned by system)\\
Wavelength files&WAVE&WA&(assigned by system)\\
Color maps&COLOR&CO&(assigned by system)\\
\hline
\end{tabular}
\end{center}

Examples: Suppose you see with PRINT DIR that the default directory for CCD
spectra is ccd/spec and the default extension is '.fits'
\begin{example}
  \item[SETDIR SP DIR=mydir/spec\hfill]{changes the default directory
       to mydir/spec}
  \item[SETDIR SP EXT=.xyz\hfill]{changes the default extension to '.xyz'}
  \item[SETDIR SP EXT=.XYZ DIR=mydir/spec\hfill]{changes both the
       directory and extension at one time.}
\end{example}

\section{CD: Change the Current Working Directory}
\begin{rawhtml}
<!-- linkto cd.html -->
\end{rawhtml}
\begin{command} 
  \item[\textbf{Form: } CD path\_name\hfill]{}
  \item[path\_name]{any valid Unix directory path}
\end{command}
 
CD will change the current working directory (./) of VISTA to be
path\_name.  This working directory will remain current until either you
issue CD again, or you exit VISTA.  On exiting VISTA, you will be in the
*original* working directory from which you executed VISTA.
 
CD is most useful when you have defined the default image directory to the
current working directory, (./), allowing you to navigate among different
directories containing data of interest without having to redefine the
default image directory each time with the SETDIR command.  It also defines
the current working directory for shell commands (\$).
 
